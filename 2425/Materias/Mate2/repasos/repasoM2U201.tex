\documentclass[12pt,addpoints]{repaso}
\grado{2}
\nivel{Secundaria}
\cicloescolar{2022-2023}
\materia{Matemáticas}
\unidad{2}
\title{Repaso para el examen de la Unidad}
\aprendizajes{
    \item Resuelve problemas de proporcionalidad directa e inversa y de reparto proporcional.
    \item Obtiene la expresión algebraica y construye gráficas de una situación de
    proporcionalidad directa e inversa.
    \item Construye polígonos regulares a partir de algunas
    medidas (lados, apotema, diagonales, etcétera).
    \item Descompone figuras en otras para calcular su área.
    \item Calcula el perímetro y el área de polígonos regulares y del círculo a partir de diferentes datos.
}
\author{Melchor Pinto, J.C.}
\begin{document}
\INFO%
\begin{multicols}{2}
    \include*{../blocks/block002}
    \include*{../blocks/block003}
    \include*{../blocks/block000}
\end{multicols}
\ejemplosboxed[\include*{../questions/question009}]
\begin{questions}
    \questionboxed[10]{\include*{../questions/question011}}
    \questionboxed[10]{\include*{../questions/question004}}
    \ejemplosboxed[\include*{../questions/question001}]
    \questionboxed[10]{\include*{../questions/question005}}
    \questionboxed[10]{\include*{../questions/question002}}
    \questionboxed[10]{\include*{../questions/question003}}
    \questionboxed[10]{\include*{../questions/question012}}
    \questionboxed[10]{\include*{../questions/question010}}
    \ejemplosboxed[\include*{../questions/question013}]
    \ejemplosboxed[\include*{../questions/question014}]
    \questionboxed[10]{\include*{../questions/question006}}
    \questionboxed[10]{\include*{../questions/question008a}}
    \questionboxed[10]{\include*{../questions/question015}}
\end{questions}
\end{document}