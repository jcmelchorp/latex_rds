Construí un fuerte conectando dos cajas.
La primera caja mide 5 metros de largo, 9 metros de ancho y 9 metros de altura.
La segunda caja mide 3 metros de largo, 8 metros de ancho y 2 metros de altura.
\textbf{¿Cuántos metros cúbicos de espacio tiene mi fuerte?}



% \begin{minipage}{0.35\linewidth}
\fullwidth{
    \begin{oneparchoices}
        \choice 36 m$^3$
        \choice 372 m$^3$
        \choice 96 m$^3$
        \CorrectChoice 453 m$^3$
    \end{oneparchoices}
}
% \end{minipage}\hfill
% \begin{minipage}{0.65\linewidth}
%     \begin{solutionbox}{6cm}\scriptsize
%         Primero calculemos el volumen de la primera caja.
%         \[ V =5\times 9\times 9 =405\]
%         Después, calculamos el volumen de la segunda caja.
%         \[ V =3\times 8\times 2 =48\]
%         Para calcular el volumen total del furte, sumamos los volumenes anteriores.
%         \[405+48=453\]
%         El volumen del fuerte es de 453 metros cúbicos.
%     \end{solutionbox}
% \end{minipage}