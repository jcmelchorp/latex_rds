\documentclass[12pt,addpoints]{guia}
\grado{2$^\circ$ de Secundaria}
\cicloescolar{2022-2023}
\materia{Matemáticas 2}
\guia{33}
\unidad{3}
\title{Equivalencia de expresiones algebraicas}
\aprendizajes{\item Formula expresiones de primer grado para representar propiedades (perímetros y áreas)
        de figuras geométricas y verifica equivalencia de expresiones, tanto algebraica como
        geométricamente (análisis de las figuras).}
\author{JC Melchor Pinto}
\begin{document}
\INFO%
\ejemplosboxed[\include*{../questions/question078b}]
\begin{questions}
    \questionboxed[10]{\include*{../questions/question078c}}
    \questionboxed[5]{\include*{../questions/question078d}}
    \questionboxed[10]{\include*{../questions/question078e}}
    \questionboxed[5]{\include*{../questions/question078f}}
    \ejemplosboxed[\include*{../questions/question032}]
    \questionboxed[10]{\include*{../questions/question076b}}
    \ejemplosboxed[\include*{../questions/question100}]
    \questionboxed[18]{\include*{../questions/question073}}
    \questionboxed[16]{\include*{../questions/question076a}}
    \ejemplosboxed[\include*{../questions/question075b}]
    \questionboxed[16]{\include*{../questions/question075a}}
    \questionboxed[10]{\include*{../questions/question078a}}
\end{questions}
\end{document}