\documentclass[12pt,addpoints]{guia}
\grado{2$^\circ$ de Secundaria}
\cicloescolar{2022-2023}
\materia{Matemáticas 2}
\guia{30}
\unidad{3}
\title{Series y sucesiones aritméticas}
\aprendizajes{\item Verifica algebraicamente la equivalencia de expresiones de primer grado, formuladas a partir de sucesiones.}
\author{JC Melchor Pinto}
\begin{document}
\INFO%
\begin{multicols}{2}
    \include*{../blocks/block030a}
    \include*{../blocks/block030d}
    \include*{../blocks/block030b}
    \include*{../blocks/block030c}
\end{multicols}
\begin{importantbox}
    La \textbf{regla de recurrencia} de una sucesión es una expresión algebraica que permite calcular el valor de cada término con sólo saber su posición en la serie ($n$).
\end{importantbox}
\begin{questions}
    \questionboxed[15]{\include*{../questions/question020}}
    \questionboxed[10]{\include*{../questions/question021}}
    \ejemplosboxed[\include*{../questions/question026}]
    \questionboxed[10]{\include*{../questions/question024}}
    \questionboxed[15]{\include*{../questions/question027}}
    \ejemplosboxed[\include*{../questions/question028}]
    \questionboxed[10]{\include*{../questions/question023}}
    \ejemplosboxed[\include*{../questions/question029}]
    \questionboxed[10]{\include*{../questions/question031}}%
    \questionboxed[15]{\include*{../questions/question025}}
    \ejemplosboxed[\include*{../questions/question030}]
    \questionboxed[15]{\include*{../questions/question022}}
\end{questions}
\end{document}