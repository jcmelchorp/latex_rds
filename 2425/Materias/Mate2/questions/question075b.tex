Elige la(s) respuesta(s).
\begin{multicols}{2}
    \begin{parts}
        % \part El perímetro de un polígono regular es 8 u, si las longitudes de sus lados son enteros ¿de qué polígono se trata?
        % \begin{choices}
        %     \choice Triángulo
        %     \choice Cuadrado
        %     \choice Pentágono
        %     \choice Hexágono
        %     \choice Ninguna de las respuestas anteriores
        % \end{choices}

        % \part ¿En qué tipo de triángulo no puedes usar la multiplicación para simplificar el cálculo del perímetro?
        % \begin{choices}
        %     \choice Equilátero
        %     \choice Escaleno
        %     \choice Isósceles
        %     \choice Regular
        %     \choice Ninguna
        % \end{choices}

        % \part ¿Cuál o cuáles de las siguientes expresiones permiten calcular el perímetro de un hexágono regular tal que uno de sus lados mide $\frac{3}{2}$?
        % \begin{choices}
        %     \choice $5\left(\frac{3}{2}\right)$
        %     \choice $\frac{3}{2}+\frac{3}{2}+\frac{3}{2}+\frac{3}{2}+\frac{3}{2}+\frac{3}{2}$
        %     \choice $6\left(\frac{3}{2}\right)$
        %     \choice $2(3)\left(\frac{3}{2}\right)$
        %     \choice Ninguna de las respuestas anteriores
        % \end{choices}

        \part La expresión $2(3.6) + 2(5.4)$ resultó de considerar las longitudes de los lados de un cuadrilátero para calcular su perímetro, ¿de qué tipo de cuadrilátero se trata?
        \begin{choices}
            \choice Cuadrado
            \choice Rectángulo
            \choice Trapecio
            \choice Rombo
            \choice Ninguna
        \end{choices}

        \part El área de un triángulo es $3\cdot 5 \text{ u}^2$, si las longitudes de la base y la altura son enteros, ¿cuál es la longitud posible de la base?
        \begin{choices}
            \choice $3$
            \choice $5$
            \choice $6$
            \choice $10$
            \choice Ninguna
        \end{choices}

        % \part La expresión $\frac{(11+5)(7)}{2}$ u$^2$ permite calcular el área de un trapecio, ¿cuál o cuáles de las expresiones también permite calcular dicha área?
        % \begin{choices}
        %     \choice $(11+5)\left(\frac{7}{2}\right)$ u$^2$
        %     \choice $\frac{11+5}{2}\left(7\right)$ u$^2$
        %     \choice $\left(\frac{11}{2}+5\right)\left(7\right)$ u$^2$
        %     \choice $\left(11+\frac{5}{2}\right)\left(7\right)$ u$^2$
        %     \choice Ninguna de las respuestas anteriores
        % \end{choices}

        % \part La expresión $\frac{(1347)(7489)}{2}$ u$^2$ resultó de sustituir la base y la altura de un triángulo para calcular el área de éste. ¿Cuál o cuáles de las siguientes expresiones también generan el área del triángulo?
        % \begin{choices}
        %     \choice $\left(\frac{1347}{2}\right)\left(7489\right)$ u$^2$
        %     \choice $\left(1347\right)\left(\frac{7489}{2}\right)$ u$^2$
        %     \choice $\left(1347\right)\left(7489\right)\left(\frac{1}{2}\right)$ u$^2$
        %     \choice $\left(\frac{1347}{2}\right)\left(\frac{7489}{2}\right)$ u$^2$
        %     \choice Ninguna de las respuestas anteriores
        % \end{choices}

        % El área de un pentágono se calcula con la expresión $\frac{(3+3+3+3+3)(2)}{2}$ ¿Cuál o cuáles de las siguientes expresiones también generan el área del pentágono?
        % \begin{choices}
        %     \choice $3+3+3+3+3$
        %     \choice $\frac{(3(2)+3(2)+3(2)+3(2)+3(2))}{2}$
        %     \choice $\left(\frac{3}{2}+\frac{3}{2}+\frac{3}{2}+\frac{3}{2}+\frac{3}{2}\right)\left(2\right)$
        %     \choice $(5)(3)$
        %     \choice Ninguna de las respuestas anteriores
        % \end{choices}
    \end{parts}
\end{multicols}