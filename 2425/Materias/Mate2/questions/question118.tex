Un padre repartirá 2700 dólares entre sus cinco hijos y decidió que cada hijo reciba un monto de dinero distinto de tal forma que la diferencia de montos entre los hijos sea la misma.
Si le dará 780 dólares al hijo mayor,
\textbf{¿cuánto dinero recibirán los demas hijos?}

\begin{solutionbox}{8cm}
    \begin{multicols}{2}
        Sabemos que la suma de la serie que representa la reparticion del dinero es:
        \[s_{5}=\dfrac{5(780+a_{5})}{2}=2,700\]
        despejando $a_5$:
        \begin{align*}
            \dfrac{5(780+a_{5})}{2} & =2,700     \\
            5(780+a_{5})            & =5,400     \\
            780+a_{5}               & = 1,080    \\
            a_{5}                   & =1,080-780 \\
            a_{5}                   & =300
        \end{align*}
        El hijo menor recibirón 300 dólares; calculando la regla de recurrencia:
        \[a_{5}=d(5-1)+780=300\]
        despejando $d$:
        \begin{align*}
            d(5-1)+ 780 & =300     \\
            d(5-1)      & =300-780 \\
            d           & =-120
        \end{align*}
        Por lo tanto, el hijo mayor (1) recibe 780 dólares, el segundo recibe 660 dólares, el tercero recibe 540 dólares, el cuarto recibe 420 dólares, y el menor (5) recibe 300 dólares.
    \end{multicols}
\end{solutionbox}