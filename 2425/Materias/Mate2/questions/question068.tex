Se quieren fabricar jarras con formas de prisma recto cuya base sea un
polígono regular. Las jarras son de dos tipos, las que tienen por base un
octágono regular y las de menor capacidad, que tienen por base un
hexágono regular.

\begin{parts}
    ¿Qué altura debe tener la jarra con base de hexágono regular para que pueda
    contener 1.1 L, si el lado del hexágono es de 4.6 cm y la apotema es de 4 cm?

    \begin{solutionbox}{2.2cm}
        1.1 L = 1.1 dm$^3$ = 1 100 cm$^3$.

        Luego, el área del hexágono es 55.2 cm.
        Así, la altura es $\dfrac{1100}{55.2} = 19.93$ cm.
    \end{solutionbox}

    ¿Qué capacidad, en litros, tiene la jarra cuya base es un octágono regular de lado 4.2 cm, apotema 5 cm y altura de 21 cm?

    \begin{solutionbox}{1.2cm}
        El área de la base es 84 cm$^2$, el volumen es de 1764 cm$^3$, esta cantidad es
        equivalente a 1.764 L.
    \end{solutionbox}

    ¿Cuál es la altura de la jarra cuya base es un octágono regular con las medidas
    anteriores del polígono para contener $2\dfrac{1}{2}$?

    \begin{solutionbox}{1.7cm}
        Debe tener la misma área de 84 cm$^2$. Como la capacidad equivale a un
        volumen de 2500 cm$^3$, la altura debe ser de 29.76 cm.
    \end{solutionbox}
\end{parts}