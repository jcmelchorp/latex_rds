Convierte las siguientes expresiones gramaticales en expresiones algebraicas.

\begin{parts}
    Escriban una expresión algebraica que describa cada oración.

    \begin{subparts}
        \subpart En un salón de clases hay cierto número de estudiantes de los cuales las dos terceras partes son niñas.

        \begin{solutionbox}{1.6cm}
            \[\dfrac{2}{3}x\]
        \end{solutionbox}

        \subpart Un número natural multiplicado por su consecutivo.

        \begin{solutionbox}{1.6cm}
            \[n\left(n+1\right)\]
        \end{solutionbox}

        \subpart Un número al cuadrado menos su quinta parte.

        \begin{solutionbox}{1.6cm}
            \[x^2-\dfrac{x}{5} \qquad\text{ ó }\qquad x^2-\dfrac{1}{5}x\]
        \end{solutionbox}

        \subpart La suma entre el cubo de un número, la mitad de su cuadrado y la cuarta parte del mismo número.

        \begin{solutionbox}{1.6cm}
            \[x^3-\dfrac{x^2}{2}+\dfrac{x}{4}\]
        \end{solutionbox}

    \end{subparts}

    ¿Qué característica tienen en común las expresiones que escribieron?

    \begin{solutionbox}{2cm}
        Todas las expresiones son de una sola variable o literal representando,
        en cada caso, el total de elementos de un grupo o un número desconocido.
    \end{solutionbox}
\end{parts}