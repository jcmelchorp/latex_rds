Convierte las siguientes expresiones gramaticales en expresiones algebraicas.

\begin{parts}

    Escriban una expresión algebraica que describa cada oración.

    \begin{subparts}

        \subpart En un salón de clases hay 28 niñas que son las dos terceras partes del total de alumnos.

        \begin{solutionbox}{1.6cm}
            \[\dfrac{2}{3}x=28\]
        \end{solutionbox}

        \subpart Un número natural multiplicado por su consecutivo es igual a 42.

        \begin{solutionbox}{1.6cm}
            \[n\left(n+1\right)=42\]
        \end{solutionbox}

        \subpart Un número al cuadrado menos su quinta parte es igual a $\dfrac{1}{2}$.

        \begin{solutionbox}{1.6cm}
            \[x^2-\dfrac{x}{5}=\dfrac{1}{2} \qquad\text{ ó }\qquad x^2-\dfrac{1}{5}x=\dfrac{1}{2}\]
        \end{solutionbox}

        \subpart La suma del cubo de un número, la mitad de su cuadrado y la cuarta parte del mismo número es igual a cero.

        \begin{solutionbox}{1.6cm}
            \[x^3-\dfrac{x^2}{2}+\dfrac{x}{4}=0\]
        \end{solutionbox}

    \end{subparts}

    ¿Qué característica tienen en común las expresiones que escribieron?

    \begin{solutionbox}{2cm}
        Todas las ecuaciones representan una igualdad con la que es posible obtener los valores de la variable o literal.
    \end{solutionbox}
\end{parts}