\documentclass[12pt,addpoints]{guia}
\grado{3$^\circ$ de Secundaria}
\cicloescolar{2022-2023}
\materia{Matemáticas 3}
\guia{36}
\unidad{3}
\title{Usa el teorema de Pitágoras para calcular el perímetro}
\aprendizajes{\item Formula, justifica y usa el teorema de Pitágoras.
    }
\author{JC Melchor Pinto}
\begin{document}
\INFO%
\begin{multicols}{2}%
    \include*{../blocks/block034a}
    \include*{../blocks/block034c}
    \include*{../blocks/block034d}
    \columnbreak
    \include*{../blocks/block034b}
    \include*{../blocks/block034e}
\end{multicols}
\ejemplosboxed[\include*{../questions/question039a}]
\begin{questions}
    \questionboxed[10]{\include*{../questions/question039b}}
    \questionboxed[10]{\include*{../questions/question039c}}
    \ejemplosboxed[\include*{../questions/question039d}]
    \questionboxed[15]{\include*{../questions/question039e}}
    \ejemplosboxed[\include*{../questions/question039f}]
    \questionboxed[10]{\include*{../questions/question039g}}
    \questionboxed[15]{\include*{../questions/question039h}}
    \ejemplosboxed[\include*{../questions/question039i}]
    \questionboxed[15]{\include*{../questions/question039j}}
    \questionboxed[10]{\include*{../questions/question039k}}
    \questionboxed[15]{\include*{../questions/question039l}}
\end{questions}
\end{document}