\documentclass[12pt,addpoints,answers]{guia}
\grado{3$^\circ$ de Secundaria}
\cicloescolar{2022-2023}
\materia{Matemáticas 3}
\guia{39}
\unidad{3}
\title{Utiliza áreas de cuadrados para visualizar el teorema de Pitágoras}
\aprendizajes{\item Formula, justifica y usa el teorema de Pitágoras.
    }
\author{JC Melchor Pinto}
\begin{document}
\INFO%
\begin{multicols}{2}
    \include*{../blocks/block034b}
    \include*{../blocks/block034a}
    \include*{../blocks/block034c}
\end{multicols}
\ejemplosboxed[\include*{../questions/question043c}]
\begin{questions}
    \questionboxed[20]{\include*{../questions/question043b}}
    \questionboxed[20]{\include*{../questions/question043c}}
    \ejemplosboxed[\include*{../questions/question043g}]
    \questionboxed[20]{\include*{../questions/question043d}}
    \ejemplosboxed[\include*{../questions/question043e}]
    \questionboxed[20]{\include*{../questions/question043f}}
    \ejemplosboxed[\include*{../questions/question043h}]
    \questionboxed[20]{\include*{../questions/question043i}}
\end{questions}
\end{document}