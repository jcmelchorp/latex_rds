\begin{defcard}
    \textbf{Álgebra} $\rightarrow$ representación simbólica de cantidades.\\
    \textbf{Cociente} $\rightarrow$ el resultado de una división.\\
    \textbf{Coeficiente} $\rightarrow$ número que multiplica a una literal; ejemplo: $a$, $b$, $c$ son coeficientes de $ax^2+bx+c$.\\
    \textbf{Diferencia} $\rightarrow$ resta de dos numeros (el mayor menos el menor).\\
    \textbf{Exponente} $\rightarrow$ número en superíndice que indica la cantidad de veces que un nuúmero se multiplica por si mismo.\\
    \textbf{Factor} $\rightarrow$ aquello que se multiplica.\\
    \textbf{Factorizar} $\rightarrow$ convertir una expresión algebráica en un producto.\\
    \textbf{Fórmula} $\rightarrow$ ecuación con más de dos variables o incógnitas.\\
    \textbf{Miembro} $\rightarrow$ son las expresiones que aparecen a cada lado del signo igual en una ecuación o identidad.\\
    \textbf{Producto} $\rightarrow$ el resultado de una multiplicación.\\
    \textbf{Resolver} $\rightarrow$ encontrar el valor de la incógnita.\\
    \textbf{Solución} $\rightarrow$ el valor de una incógnita.\\
    \textbf{Término} $\rightarrow$ los monomios de cada miembro.\\
\end{defcard}