\documentclass[12pt,addpoints]{repaso}
\grado{3}
\nivel{Secundaria}
\cicloescolar{2024-2025}
\materia{Matemáticas}
\unidad{3}
\title{Practica la Unidad}
\aprendizajes{
    \item Analiza y compara diversos tipos de variación a partir de sus representaciones tabular, gráfica y algebraica, que resultan de modelar situaciones y fenómenos de la física y de otros contextos.
    \item Diferencia las expresiones algebraicas de las funciones y de las ecuaciones.
    \item Comprende los criterios de congruencia de triángulos y los utiliza para determinar triángulos congruentes.
    \item Formula, justifica y usa el teorema de Pitágoras.
    }
\author{Melchor Pinto, J.C.}
\begin{document}
\INFO%
\begin{multicols}{2}
    \include*{../blocks/block034d}
    \include*{../blocks/block034e}
    \include*{../blocks/block034b}
\end{multicols}
\ejemplosboxed[\include*{../questions/question037c}]
\begin{questions}
    \questionboxed[10]{\include*{../questions/question037a}}
    \ejemplosboxed[\include*{../questions/question043c}]
    \questionboxed[10]{\include*{../questions/question043a}}
    \ejemplosboxed[\include*{../questions/question032m}]
    \questionboxed[10]{\include*{../questions/question032n}}
    \ejemplosboxed[\include*{../questions/question038k}]
    \questionboxed[15]{\include*{../questions/question038l}}
    \ejemplosboxed[\include*{../questions/question040c}]%
    \questionboxed[15]{\include*{../questions/question040a}}
    \ejemplosboxed[\include*{../questions/question039i}]
    \questionboxed[15]{\include*{../questions/question039j}}
    \ejemplosboxed[\include*{../questions/question036f}]
    \questionboxed[15]{\include*{../questions/question036g}}
    \ejemplosboxed[\include*{../questions/question030k}]
    \questionboxed[10]{\include*{../questions/question030i}}
\end{questions}
\end{document}