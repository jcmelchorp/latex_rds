\documentclass[12pt,addpoints]{evalua}
\grado{3$^\circ$ de Secundaria}
\cicloescolar{2024-2025}
\materia{Matemáticas 3}
\unidad{2}
\title{Examen de {\color{brown}recuperación} de la Unidad}
\aprendizajes{
      \item Resuelve problemas mediante la formulación y solución algebraica de ecuaciones lineales.
      \item Calcula el perímetro de polígonos y del círculo, y áreas de triángulos y cuadriláteros desarrollando y aplicando fórmulas.
      \item Calcula el volumen de prismas y cilindros rectos.
      }
\author{Prof.: Julio César Melchor Pinto}
\begin{document}%
\begin{questions}
	\question[8]{Determina las medidas de tendencia central en los siguientes conjuntos de datos. \textit{(De ser necesario redondea tu respuesta a la decima más cercana)}:

		\begin{multicols}{2}
			\begin{parts}
				\part Los puntajes obtenidos en un juego son: \\54, 55, 59, 61, 77, 58, 55, 71, 59, 55, 60, 53, 56 y 60. \\[1em]
				La media es:

				\begin{solutionbox}{2cm}
					$\dfrac{54+55+59+\ldots+56+60}{14}=    \dfrac{823}{14}=59.5$
				\end{solutionbox}

				La mediana es: \fillin[$58.5$][0in]\\
				La moda es: \fillin[$55$][0in] \\
				La desviación media es: \fillin[$4.5$][0in].

				\begin{solutionbox}{3cm}
					Para calcular la desviación media:\\
					\[\dfrac{|54-59.5|+\ldots+|60-59.5|}{14}=4.5\]
				\end{solutionbox}

				\part 22, 25, 21, 23, 29, 30, 28, 27, 23, 26. \\[1em]
				La media es:

				\begin{solutionbox}{2cm}
					$\dfrac{22+25+21+\ldots+23+26}{10}=\dfrac{254}{10}=25.4$
				\end{solutionbox}

				La mediana es: \fillin[$25.5$][0in]\\
				La moda es: \fillin[$23$][0in]\\
				La desviación media es: \fillin[$2.6$][0in].

				\begin{solutionbox}{3cm}
					Para calcular la desviación media:\\
					\[\frac{|22-25.4|+\ldots+|26-25.4|}{10}=2.6\]
				\end{solutionbox}
			\end{parts}
		\end{multicols}
	}


	% \subsection*{Eventos mutuamente excluyentes}
	\question[10]{Resuelve los siguientes problemas:

		\begin{parts}
			\part En una urna hay 8 pelotas moradas, 12 naranjas, 7 rojas, 11 azules y 7 blancas. Calcula la probabilidad de sacar una pelota roja o azul.

			\begin{solutionbox}{2.8cm}
				Para calcular la probabilidad de sacar una pelota roja o azul, hay que calcular la probabilidad de sacar una pelota roja, que es de $\dfrac{7}{45}$ y la probabilidad de sacar una pelota azul, que es de $\dfrac{11}{45}$. Por lo tanto, la probabilidad de sacar una pelota roja o azul es de $\dfrac{7}{45}+\dfrac{11}{45}=\dfrac{18}{45}=\dfrac{2}{5}$
			\end{solutionbox}

			\part En un salón hay 24 niñas, de las cuales 8 son extranjeras y 16 son mexicanas y hay 22 niños, de los cuales 18 son mexicanos y 4 son extranjeros. Calcula la probabilidad de elegir a un niño extranjero.

			\begin{solutionbox}{2.8cm}
				Para calcular la probabilidad de elegir a un niño extranjero, hay que calcular la probabilidad de elegir a un niño, que es de $\dfrac{22}{46}$ y la probabilidad de elegir a un extranjero, que es de $\dfrac{4}{22}$. Por lo tanto, la probabilidad de elegir a un niño extranjero es de $\dfrac{4}{46}=\dfrac{2}{23}$
			\end{solutionbox}

			% \part Ricardo quiere poner una barda alrededor de un terreno pentagonal que mide 15 metros por lado. ¿Cuánta barda necesitará Ricardo para poner barda en todo el terreno?

			% \begin{solutionbox}{1.2cm}
			% \end{solutionbox}


		\end{parts}
	}

	% \section*{ Figuras y cuerpos geométricos}
	% \subsection*{Perímetro y Área}
	\question[8]{Encuentra el perímetro y el área de las siguientes figuras:

		\begin{multicols}{2}

			\begin{parts}
				\part \includegraphics[width=0.7\linewidth]{mex_0018.png} \\
				Perímetro: \fillin[$31\times 3=93$][0in] \\[0.5em] Área: \fillin[$\dfrac{31\times 24}{2}=372$][0in]

				\part \includegraphics[width=0.7\linewidth]{mex_0012.png} \\
				Perímetro: \fillin[$3.14 \times 88 =276.32$][0in] \\[0.5em]  Área: \fillin[$3.14 \times 44^2 =6079.04$][0in]
			\end{parts}
		\end{multicols}
	}

	% \newpage

	\question[6]{Selecciona la respuesta correcta:

		\begin{multicols}{2}
			\begin{parts}
				\part El punto A$(1,0)$, ¿está ubicado sobre el eje $x$?

				\begin{oneparcheckboxes}
					\CorrectChoice Verdadero
					\choice Falso
				\end{oneparcheckboxes}

				\part El punto A$(2,0)$, ¿está ubicado sobre el eje $y$?

				\begin{oneparcheckboxes}
					\choice Verdadero
					\CorrectChoice Falso
				\end{oneparcheckboxes}

				\part El punto A$(0,-5.9)$, ¿está ubicado sobre el eje $x$?

				\begin{oneparcheckboxes}
					\choice Verdadero
					\CorrectChoice Falso
				\end{oneparcheckboxes}

				\part El punto A$(0,8.24)$, ¿está ubicado sobre el eje $y$?

				\begin{oneparcheckboxes}
					\CorrectChoice Verdadero
					\choice Falso
				\end{oneparcheckboxes}

				\part El punto A$(-1.5,0)$, ¿está ubicado sobre el eje $x$?

				\begin{oneparcheckboxes}
					\CorrectChoice Verdadero
					\choice Falso
				\end{oneparcheckboxes}

				\part El punto A$(0,-10)$, ¿está ubicado sobre el eje $x$?

				\begin{oneparcheckboxes}
					\choice Verdadero
					\CorrectChoice Falso
				\end{oneparcheckboxes}

			\end{parts}
		\end{multicols}
	}

	% \subsection*{Resolución de problemas}
	% \question[4]{Resuelve los siguientes problemas:
	%       \begin{multicols}{2}
	%             \begin{parts}\footnotesize%
	%                   \part Calcula la altura de un prisma que tiene como área de la base 6 m$^2$ y 66 m$^3$ de capacidad.

	%                   \begin{solutionbox}{1cm}
	%                   \end{solutionbox}



	%                   \part ¿Cuál es el perímetro de un campo de fútbol que mide 95.12 metros de largo y 45.27 metros de ancho?

	%                   \begin{solutionbox}{1cm}
	%                   \end{solutionbox}
	%             \end{parts}
	%       \end{multicols}
	% }

	% \subsection*{Área lateral, Área total y Volumen}
	\question[10]{Calcula el volumen, el área lateral y el área total de las siguientes figuras:

		\begin{multicols}{2}
			\begin{parts}
				\part \includegraphics[width=0.8\linewidth]{mex_0022.png} \\
				Pirámide hexagonal cuyos lados "l" de la base miden 8 cm, su apotema mide 7 cm y la altura mide 21 cm.

				\begin{solutionbox}{6cm}
					Volumen:
					\[V=\dfrac{1}{3}A_b\cdot h= \dfrac{1}{3}\left(\dfrac{nla}{2}\right)h= \dfrac{6(8)7}{6}(21)=1176\]
					A. Lateral:
					\[A_L=n\dfrac{lh}{2}=6\cdot 8\cdot 21=1008\]
					A. Total:
					\[A_T=A_L+\dfrac{nla}{2}=840+64=904\]
				\end{solutionbox}

				% \part \includegraphics[width=\linewidth]{mex_0023.png}
				% Pirámide cuyos lados "l" de la base miden 13 cm y la altura "h" mide 42 cm.\\
				% Volumen: \fillin[$$ u$^3$][0.4in] \\A. Lateral: \fillin[$$ u][0.4in] \\ A. Total: \fillin[$$ u$^2$][0.4in]

				\part \includegraphics[width=0.8\linewidth]{mex_0024.png} \\
				Cilindro con altura $h=17$ cm y un radio $r=4$ cm.

				\begin{solutionbox}{6cm}
					Volumen:
					\[V=\pi r^2h=(3.14) 4^2\cdot 17= 857.12\]
					A. Lateral:
					\[A_L=2\pi rh=2(3.14) 4\cdot 17= 2(3.14) 68=428.48\]
					A. Total:
					\[A_T=A_L+2\pi r^2=428.48+2(3.14) 16=528.96\]
				\end{solutionbox}
			\end{parts}
		\end{multicols}
	}


	\newpage


	\question[10]{Escribe la ecuación de las recta para dada uno de los siguientes incisos:

		\begin{multicols}{2}
			\begin{parts}
				\part Escribe la ecuación de la recta que pasa por los puntos$ A(1,6)$ y $B(2,1)$

				\begin{solutionbox}{5cm}\
					Para obtener la ecuación necesitamos calcular la pendiente de la recta, que es: \[m=\dfrac{y_2-y_1}{x_2-x_1}= \dfrac{1-6}{2-1}=\dfrac{-5}{1}=-5\], y la ordenada al origen, que es: $b=y-mx=6-5(1)=6-5=1$. \\
					Por lo tanto, la ecuación de la recta es: \\$y=-5x+1$.
				\end{solutionbox}

				\part Escribe la ecuación de la recta que pasa por los puntos $A(-2,3)$ y $B(1,0)$

				\begin{solutionbox}{5cm}
					Para obtener la ecuación necesitamos calcular la pendiente de la recta, que es: \[m=\dfrac{y_2-y_1}{x_2-x_1}= \dfrac{0-3}{1-(-2)}=\dfrac{-3}{3}=-1\], y la ordenada al origen, que es: $b=y-mx=3-(-1)(-2)=3+2=5$. \\
					Por lo tanto, la ecuación de la recta es: \\$y=-x+5$.
				\end{solutionbox}
			\end{parts}
		\end{multicols}
	}

	% \subsection*{Cuadrantes en el plano cartesiano}

	% \subsection*{Pendiente y ordenada}
	% \question[5]{Identifica la pendiente y ordenada de las siguientes rectas:
	%       \begin{parts}
	%             \begin{multicols}{3}
	%                   \part $y=-2x+1$ \\[1em]
	%                   Pendiente = \fillin[$-2$][0in] \\  Ordenada = \fillin[$1$][0in]

	%                   \part $y=\dfrac{1}{2}x-3$ \\[1em]
	%                   Pendiente = \fillin[$\dfrac{1}{2}$][0in]  \\ Ordenada = \fillin[$-3$][0in]

	%                   \part $y=-3x+3$ \\[1em]
	%                   Pendiente = \fillin[$-3$][0in] \\  Ordenada = \fillin[$3$][0in]
	%             \end{multicols}

	%             \begin{multicols}{2}
	%                   \part \includegraphics[width=0.8\linewidth]{mex_0047.png}\\
	%                   Pendiente = \fillin[$-2$][0in] \qquad \qquad Ordenada = \fillin[$1$][0in]

	%                   \part \includegraphics[width=0.8\linewidth]{mex_0051.png}\\
	%                   Pendiente = \fillin[$-2$][0in]  \qquad \qquad Ordenada = \fillin[$1$][0in]
	%             \end{multicols}

	%       \end{parts}
	% }



	% \subsection*{Pendiente dados dos puntos}
	% \question[7]{Calcula la pendiente en cada uno de los siguientes incisos:
	%       \begin{multicols}{2}

	%             \begin{parts}
	%                   \part Calcula la pendiente de la recta que pasa por los puntos A(0,-3) y B(5,1).   \\[1em]
	%                   $m=$ \fillin[$\dfrac{4}{5}$][0in]
	%                   \part Calcula la pendiente de la recta que pasa por los puntos A(-8,6) y B(-3,8).  \\[1em]
	%                   $m=$ \fillin[$\dfrac{2}{5}$][0in]
	%                   \part Calcula la pendiente de la recta que pasa por los puntos A(1,1) y B(5,-3).   \\[1em]
	%                   $m=$ \fillin[$-1$][0in]
	%                   \part Calcula la pendiente de la recta que pasa por los puntos A(-7,-3) y B(6,10). \\[1em]
	%                   $m=$ \fillin[$1$][0in]
	%                   \part Calcula la pendiente de la recta que pasa por los puntos A(-7,-3) y B(-5,7). \\[1em]
	%                   $m=$ \fillin[$5$][0in]

	%                   \part Calcula la pendiente de la siguiente recta:\\
	%                   \includegraphics[width=0.8\linewidth]{mex_0050.png} $m=$ \fillin[$-1$][0in]

	%                   \part Calcula la pendiente de la siguiente recta:\\
	%                   \includegraphics[width=0.8\linewidth]{mex_0048.png} $m=$ \fillin[$\dfrac{4}{5}$][0in]
	%             \end{parts}
	%       \end{multicols}
	% }

	% \section*{Ecuación lineal}

	% \subsection*{Ecuaciones lineales}
	\question[10]{Resuelve las siguientes ecuaciones lineales

		\begin{multicols}{2}
			\begin{parts}
				\part $-\dfrac{1}{2}x-\dfrac{1}{4}x=\dfrac{5}{6}$

				\begin{solutionbox}{3.5cm}\footnotesize%
					\[ -\frac{2}{4}x-\frac{1}{4}x=\frac{5}{6} \] \[ -\frac{3}{4}x=\frac{5}{6} \] \[ x=\frac{5}{6}\divisionsymbol -\frac{3}{4} \]  \[ x=-\frac{10}{9} \]
				\end{solutionbox}

				\part $-\dfrac{x}{6}=\dfrac{7}{54}$

				\begin{solutionbox}{3.2cm}\footnotesize%
					\[ -\frac{x}{6}=\frac{7}{54} \] \[ -\frac{54}{6}x=7 \] \[ -9x=7 \]  \[ x=-\frac{7}{9} \]
				\end{solutionbox}
			\end{parts}
		\end{multicols}
	}

	% \subsection*{Lenguaje algebraico}
	\question[2]{Escribe la expresión algebraica correcta para los siguientes enunciados

		\begin{multicols}{2}
			\begin{parts}
				\part El cubo de un número cualquiera aumentado en 10.

				\begin{solutionbox}{1cm}
					$x^3+10$
				\end{solutionbox}

				\part El cuadrado de la suma de dos números cualquiera.

				\begin{solutionbox}{1cm}
					$(x+y)^2$
				\end{solutionbox}

				% \part El recíproco de un número cualquiera.

				% \begin{solutionbox}{1cm}
				%       $\dfrac{1}{x}$
				% \end{solutionbox}



				% \part La mitad del cubo de la suma de dos números cualquiera.

				% \begin{solutionbox}{1cm}
				%       $\dfrac{1}{2}(x+y)^3$
				% \end{solutionbox}


			\end{parts}
		\end{multicols}
	}


	% \subsection*{Resolución de problemas}
	\question[6]{Resuelve los siguientes problemas de ecuaciones lineales
		
      \begin{parts}
			\part La suma de dos números es 215 y el mayor excede al menor en 31 unidades. ¿Cuáles son estos dos números?

			\begin{solutionbox}{3cm}
				\[x+(x+31)=215\] \[2x+31=215\] \[2x=184\] \[x=92\]
			\end{solutionbox}
		\end{parts}

	}

	% \subsection*{Ecuaciones lineales con fracciones}
	% \question[10]{Resuelve las siguientes ecuaciones lineales con fracciones

	%       \begin{multicols}{2}
	%             \begin{parts}
	%                   \part $\dfrac{1}{2}x-\dfrac{1}{4}x=\dfrac{5}{6}$

	%                   \begin{solutionbox}{1.5cm}
	%                   \end{solutionbox}

	%                   \part $-\dfrac{x}{6}=\dfrac{7}{54}$
	%                   \begin{solutionbox}{1.5cm}
	%                   \end{solutionbox}

	%             \end{parts}
	%       \end{multicols}

	% }

	\newpage

	% \section*{Sistemas de ecuaciones}
	\setlength{\columnsep}{1cm}
	\begin{multicols}{2}
		\question[10]{Utilizando el m\'etodo de tu preferencia, encuentra el valor de $x$ y $y$ para
			el siguiente sistema de ecuaciones lineales:


			\begin{eqnarray}
				13x-6y & = & 22 \\
				x & = & y+6
			\end{eqnarray}

			\setcounter{equation}{0}

			\begin{solutionbox}{7cm}
				Usando el método de sustitución, sustituimos la ecuación (4) en la ecuación (5) para obtener $x$:
				\begin{eqnarray}
					13(y+6)-6y & = & 22 \nonumber\\
					13y+78-6y & = & 22 \nonumber\\
					7y & = & -56 \nonumber\\
					y & = & -8 \nonumber
				\end{eqnarray}
				Sustituimos el valor de $y$ en la ecuación (5) para obtener $x$:
				\begin{eqnarray}
					x & = & -8+6 \nonumber\\
					x & = & -2 \nonumber
				\end{eqnarray}
			\end{solutionbox}
		}

		\question[10]{Resuelve el siguiente sistema de ecuaciones lineales con fracciones:
			\begin{eqnarray}
				2x-y & = & 3 \\
				3x-y & = & 3
			\end{eqnarray}

			\setcounter{equation}{0}

			\begin{solutionbox}{8cm}
				Usando el método de eliminación, multiplicamos la ecuación (1)| por -1 para obtener  $x$:
				\begin{eqnarray}
					-2x+y & = & -3 \nonumber\\
					3x-y & = & 3 \nonumber
				\end{eqnarray}
				Sumamos las ecuaciones (2) y (3) para obtener $x$:
				\begin{eqnarray}
					x & = & 0 \nonumber
				\end{eqnarray}
				Sustituimos el valor de $x$ en la ecuación (1) para obtener $y$:
				\begin{eqnarray}
					2(0)-y & = & 3 \nonumber\\
					-y & = & 3 \nonumber\\
					y & = & -3 \nonumber
				\end{eqnarray}
			\end{solutionbox}
		}

	\end{multicols}

	% \question[15]{Numera correctamente los pasos para resolver un sistema de dos ecuaciones con dos inc\'ognitas por los m'etodos a continuaci\'on:
	%       \begin{choices}


	%             \choice M\'etodo de suma-resta:
	%             \begin{itemize}
	%                   \item[\rule{1cm}{0.2mm}] Sumar o restar las ecuaciones para eliminar una de las inc\'ognitas.
	%                   \item[\rule{1cm}{0.2mm}] Multiplicar una o ambas ecuaciones por los n\'umeros necesarios para realizar la eliminaci\'on bajo la suma o resta.
	%                   \item[\rule{1cm}{0.2mm}] Resolver la ecuaci\'on resultante.
	%                   \item[\rule{1cm}{0.2mm}] Sustituir los valores en las ecuaciones originales para comprobar que son la soluci\'on.
	%                   \item[\rule{1cm}{0.2mm}] Sustituir el valor obtenido en una de las ecuaciones iniciales y resolverla.
	%             \end{itemize}

	%             \choice M\'etodo de sustitución:
	%             \begin{itemize}
	%                   \item[\rule{1cm}{0.2mm}] Resolver la ecuaci\'on resultante.
	%                   \item[\rule{1cm}{0.2mm}] Despejar una inc\'ognita en una de las ecuaciones.
	%                   \item[\rule{1cm}{0.2mm}] Sustituir la expresi\'on de esta inc\'ognita en la otra ecuaci\'on para obtener una ecuaci\'on con una sola inc\'ognita.
	%                   \item[\rule{1cm}{0.2mm}] Sustituir el valor obtenido en la ecuaci\'on en la que aparec\'ia la inc\'ognita despejada.
	%                   \item[\rule{1cm}{0.2mm}] Sustituir los valores en las ecuaciones originales para comprobar que son la soluci\'on.
	%             \end{itemize}

	%             \choice M\'etodo de igualaci\'on:
	%             \begin{itemize}
	%                   \item[\rule{1cm}{0.2mm}] Resolver la ecuaci\'on resultante.
	%                   \item[\rule{1cm}{0.2mm}] Igualar las expresiones para obtener una ecuaci\'on con una inc\'ognita.
	%                   \item[\rule{1cm}{0.2mm}] Despejar la misma inc\'ognita en ambas ecuaciones.
	%                   \item[\rule{1cm}{0.2mm}] Sustituir los valores en las ecuaciones originales para comprobar que son la soluci\'on.
	%                   \item[\rule{1cm}{0.2mm}] Sustituir el valor obtenido en cualquiera de las dos expresiones en las que aparec\'ia despejada la otra inc\'ognita.
	%             \end{itemize}
	%       \end{choices}
	% }

	\question[10]{Resuelve el siguiente sistema de ecuaciones lineales:
		\begin{eqnarray}
			x+y+z&=&2 \\
			x+2y-z&=&9\\
			3x-y+z&=&-2
		\end{eqnarray}
		\begin{solutionbox}{8cm}
                  \footnotesize
			\begin{multicols}{2}
				Para resolver el sistema de ecuaciones lineales, sumamos las ecuaciones (1) y (2) para eliminar a $z$ y obtener una ecuación (4):
				\begin{eqnarray}
					x+y+z&=&2 \nonumber\\
					x+2y-z&=&9 \nonumber\\ \hline
					2x+3y&=&11
				\end{eqnarray}
				Después, se suman las ecuaciones (2) y (3) para obtener una ecuación (5).
				\begin{eqnarray}
					x+2y-z&=&9 \nonumber\\
					3x-y+z&=&-2 \nonumber\\ \hline
					4x+y&=&7
				\end{eqnarray}
				Ahora se resuelve el sistema conformado por las ecuaiones (4) y (5). Para ello multiplicamos la ecuación (4) por -1 y la ecuación (5) por 3 para eliminar a $y$:
				\begin{eqnarray}
					-2x-3y&=&-11 \nonumber\\
					12x+3y&=&21 \nonumber\\ \hline
					10x&=&10 \nonumber\\
					x&=&1 \nonumber
				\end{eqnarray}
				Sustituimos el valor de $x$ en la ecuación (5) para obtener el valor de $y$:
				\begin{eqnarray}
					4(1)+y&=&7 \nonumber\\
					4+y&=&7 \nonumber\\
					y&=&3 \nonumber
				\end{eqnarray}
				Finalmente, sustituimos los valores de $x$ y $y$ en la ecuación (1) para obtener el valor de $z$:
				\begin{eqnarray}
					1+3+z&=&2 \nonumber\\
					4+z&=&2 \nonumber\\
					z&=&-2 \nonumber
				\end{eqnarray}
			\end{multicols}
		\end{solutionbox}
	}
	% \subsection*{Sistema de ecuaciones con fracciones}

\end{questions}
\end{document}