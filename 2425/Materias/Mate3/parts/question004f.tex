El área de un rectángulo es 20 cm$^2$. Su altura es 4 cm más que el triple del ancho.
Sea $x$ el ancho del rectángulo.

\begin{subparts}
    \subpart \textbf{¿Cuál de las siguientes ecuaciones cuadráticas satisface $x$?}

    \begin{oneparchoices}
        \choice $3x^2+4x+20=0$
        \choice $3x^2-4x-20=0$
        \CorrectChoice $3x^2+4x-20=0$
        \choice $3x^2-4x+20=0$
    \end{oneparchoices}

    \begin{solutionbox}{3.5cm}
        Sea $x$ el ancho del rectángulo, entonces su altura es $3x+4$. Y el producto es:
        \begin{align*}
            x(3x+4)     & =20 \\
            3x^2+4x     & =20 \\
            3x^2+4x -20 & =0
        \end{align*}
    \end{solutionbox}

    \subpart \textbf{Determina el ancho del rectángulo $x$.}

    \begin{solutionbox}{14.8cm}
        Para encontrar $x$, se debe resolver la ecuación:
        \[
            3x^2+4x -20   =0
        \]
        De acuerdo con la forma estándar $ax^2 + bx + c = 0$ de una ecuación cuadrática, sus coeficientes son:
        \[            a=3, \quad b=4 \quad y \quad c=-20\]
        Sustituyendo los coeficientes en la fórmula cuadrática:
        \[x  = \dfrac{-b\pm\sqrt{b^2-4ac}}{2a} \]
        Se obtiene:
        \begin{align*}
            x   & = \dfrac{-4\pm\sqrt{b^2-4ac}}{2a}                \\[1.2em]
            x   & = \dfrac{-4\pm\sqrt{4^2-4(3)(-20)}}{2(3)}        \\[1.2em]
            x   & = \dfrac{-4\pm\sqrt{256}}{6}                     \\[1.2em]
            x   & = \dfrac{-4\pm 16}{6}                            \\[1.2em]
            x_1 & = \dfrac{-4-16}{6}=\dfrac{-20}{6}=\dfrac{-10}{3} \\[1.2em]
            x_2 & = \dfrac{-4+16}{6}=\dfrac{12}{6}= 2
        \end{align*}
        Ya que las medidas no pueden ser negativas:
        $\therefore x=2$
    \end{solutionbox}
\end{subparts}