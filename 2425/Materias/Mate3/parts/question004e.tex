Antoine se encuentra en un balcón y lanza una pelota a su perro, que está a nivel del suelo.
La altura $h(t)$ de la pelota (en metros sobre el suelo), $t$ segundos después de que Antoine la lanzó, está modelada por:
\[h(t)=-2t^2+4t+16\]
\textbf{¿Cuántos segundos después de ser lanzada la pelota llegará al suelo?}

\begin{solutionbox}{18cm}
    %\begin{multicols}{3}
    Para conocer el tiempo en que la pelota llega al suelo (donde la altura es cero), se debe resolver la ecuación:
    \[
        -2t^2+4t+16  =0
    \]
    De acuerdo con la Forma estándar $at^2 + bt + c = 0$ de una ecuación cuadrática, sus coeficientes son:
    \begin{align*}
        a & =-2 \\
        b & =4  \\
        c & =16 \\
    \end{align*}
    Sustituyendo los coeficientes en la fórmula cuadrática:
    \[t  = \dfrac{-b\pm\sqrt{b^2-4ac}}{2a} \]
    Se obtiene:
    \begin{align*}
        t   & = \dfrac{-4\pm\sqrt{4^2-4(-2)(16)}}{2(-2)} \\[1.5em]
        t   & = \dfrac{-4\pm\sqrt{144}}{-4}              \\[1.5em]
        t   & = \dfrac{-4\pm 12}{-4}                     \\[1.5em]
        t_1 & = \dfrac{-4-12}{-4}=\dfrac{-16}{-4}=4      \\[1.5em]
        t_2 & = \dfrac{-4+12}{-4}=\dfrac{8}{-4}=-2       \\[1.5em]
    \end{align*}
    Ya que el tiempo no puede ser negaivo, la solución que tiene sentido en este problema es:
    \[\therefore t=4\]
\end{solutionbox}
