\documentclass[12pt,addpoints,answers]{guia}
\grado{3$^\circ$ de Secundaria}
\cicloescolar{2022-2023}
\materia{Ciencias y Tecnología: Química}
\guia{23}
\unidad{3}
\title{Moléculas de importancia para la vida}
\aprendizajes{\item Identifica componentes químicos importantes que participan en la estructura y funciones del cuerpo humano. 
        \item Representa y diferencia elementos y compuestos, así como átomos y moléculas.
        \item Explica y predice propiedades físicas de los materiales con base en modelos submicroscópicos.
    }
\author{JC Melchor Pinto}
\begin{document}
\pagestyle{headandfoot}
%\thispagestyle{plain}

\INFO
%\printanswers
%\pagestyle{headandfoot}

\begin{startInfo}[¿Qué moléculas nos constituyen?]
    {
        Casi 99 \% de nuestra masa corporal está compuesta por seis elementos:
        carbono (C), hidrógeno (H), oxígeno (O), nitrógeno (N), fósforo
        (P) y azufre (S) (figura \ref{fig:masabody}). El 1  \% restante incluye muchos
        de los elementos de la tabla periódica, pero se estima que
        sólo 23 de ellos son esenciales para la vida. Estos se conocen
        como bioelementos y se dividen en:
        \begin{itemize}
            \item \textbf{Elementos principales o mayoritarios:} se presentan en cantidades superiores a 0.1 \% del peso del organismo: oxígeno (O),
                  carbono (C), hidrógeno (H), nitrógeno (N), calcio (Ca), fósforo
                  (P), azufre (S), cloro (Cl) y sodio (Na).
            \item \textbf{Elementos traza:} representan entre 0.1 \% y 0.0001 \% en peso.
                  Algunos de ellos son el hierro (Fe) y cinc (Zn).
            \item \textbf{ElemenstartInfotos ultratraza:} se presentan en cantidades menores a
                  0.0001 \% en peso; por ejemplo, yodo (I) y manganeso (Mn).
        \end{itemize}
    }{../images/cuerpo-humano-elementos}{Porcentaje en masa de los elementos
        en el cuerpo humano. Los nombres de los elementos están en inglés.}
\end{startInfo}
\begin{questions}
    \include*{../questions/question001}
    \begin{boxG}
        Los bioelementos primarios, son los elementos indispensables para formar las biomoléculas
        orgánicas (carbohidratos, lípidos, proteínas y ácidos nucleicos), constituyen aproximadamente
        el 96 \% de la materia seca, sin contar el \% de agua. Son seis los bioelementos primarios más
        abundantes en la materia viva: el C, H, O, N, P y el S.
    \end{boxG}
    \include*{../questions/question002}

    \begin{boxG}
        La \textbf{fotosíntesis} es un proceso anabólico cuya función es convertir la energía luminosa en energía química, que se emplea para sintetizar moléculas
        orgánicas a partir de compuestos inorgánicos. Como subproducto, se desprende oxígeno. Intentemos estudiar estos aspectos
        básicos del funcionamiento de la química de la vida, fundamentales para entender la fotosíntesis, analizando una reacción metabólica clásica: el catabolismo total de la glucosa.
        \begin{center}
            C$_6$H$_{12}$O$_6$ + 6O$_2$ $\Rightarrow$ 6CO$_2$ + 6H$_2$O + Energía química (ATP)
        \end{center}
        En primer lugar salta a la vista algo que tú ya conocías: la materia orgánica, incluida la glucosa, está constituida principalmente
        por carbono e hidrógeno. Sin embargo, el producto carbonado de esta degradación es el dióxido de carbono, constituido únicamente por carbono y oxígeno. Echa mano de tus conocimientos de química e intenta pensar qué reacción química ha tenido
        lugar en esta transformación. ¿Tendrá esto que ver con el catabolismo y la liberación
        de energía que éste lleva asociada?
    \end{boxG}

    \include*{../questions/question003}

\end{questions}

%\vfill
%\puntuacion

\end{document}