\begin{warncard}[adjusted title={Ley de la conservación de la materia}]
    En 1789, \textbf{Antoine Laurent Lavoisier (1743-1794)} publicó el Tratado elemental
    de química que incluía una ley general denominada \comillas{de la conservación de la
        masa}, la cual establecía que
    \begin{center}\centering\bfseries\sffamily\color{colorrds}
        \emph{\comillas{La materia no se crea, ni se se destruye, sólo se transforma}.}
    \end{center}
    Lavoisier consideraba que la masa de las sustancias era una medida
    directa de la cantidad de materia presentes de ellas. Por eso los resultados se
    conocen como \textbf{Ley de la conservación de la materia o de la masa}.
\end{warncard}