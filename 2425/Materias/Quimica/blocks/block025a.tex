\begin{defcard}
    \begin{description}
        \item[Calor (\ce{\Delta})] Energía térmica que se transfiere entre dos cuerpos a diferentes temperaturas.
        \item[Efervecencia (\ce{^})] Fenómeno químico que se produce cuando se desprenden gases de una sustancia.
        \item[Metátesis] Del griego \emph{meta}, preposición inseparable, significa después, de otro modo y del griego \emph{thesis} colocación. Otro modo de colocación.
        \item[Precipitado (\ce{v})] Sustancia que se forma al disolverse una sustancia en otra hasta el punto de saturación.
        \item[Producto] Lo que se obtiene después de ocurrida la reacción química.
        \item[Reacción (\ce{->})] Proceso químico que transforma una o más sustancias en otras.
        \item[Reactivo] Sustancia que participa en una reacción química.
    \end{description}
\end{defcard}