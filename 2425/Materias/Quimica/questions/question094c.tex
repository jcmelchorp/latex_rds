[10] Observa los siguientes modelos y escribe en cuál ocurre una colisión efectiva y en cuál no.

\begin{figure}[H]
    \centering
    \includegraphics[width=0.8\textwidth]{Images/SINQU_U3_AC94_IMG1.png}
    \caption{En la tabla se incluyen los alimentos que consume Beatriz en el desayuno.}
\end{figure}
\begin{solution}
    En la primera imagen (arriba) no hay una colisión efectiva entre las partículas, ya que al final del proceso se quedan las mismas sustancias que al inicio. Por otro lado, en la segunda imagen (abajo) hay una colisión efectiva, pues se observa la formación de nuevas sustancias al final de la reacción.
\end{solution}