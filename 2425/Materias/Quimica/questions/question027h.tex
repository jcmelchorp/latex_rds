Una muestra de 26.0 g de un compuesto contiene 6.0 g de carbono, 1.0 g de hidrógeno y 19.0 g de fluor.
\textbf{¿Cuál es la fórmula que corresponde a este compuesto?}

\begin{oneparchoices}
    \choice  \ce{C2H2F2}
    \choice  \ce{CHF}
    \CorrectChoice  \ce{CH2F2}
    \choice  \ce{C2H4F4}
\end{oneparchoices}

\begin{solutionbox}{7.5cm}
    La fórmula empírica de un compuesto es la proporción en números enteros más simple de elementos en el compuesto.
    Primero, usemos las masas molares de \ce{C}, \ce{H} y \ce{F} para determinar cuantos moles de cada elemento están presentes en la muestra:

    \[ 6.0 \text{ g \ce{C}} \times \frac{1 \text{ mol \ce{C}}}{12.01 \text{ g \ce{C}}} \approx 0.5 \text{ mol \ce{C}}  \qquad 1.0 \text{ g \ce{H}} \times \frac{1 \text{ mol \ce{H}}}{1.008 \text{ g \ce{H}}} \approx 1.0 \text{ mol \ce{H}}  \qquad  19.0 \text{ g \ce{F}} \times \frac{1 \text{ mol \ce{F}}}{19.00 \text{ g \ce{F}}} \approx 1.0 \text{ mol \ce{F}} \]

    Por lo tanto, los 26.0 g de muestra contienen aproximadamente 0.5 moles de átomos de carbono, 1 mol de átomos de hidrógeno y 1 mol de átomos de fluor.
    A continuación, encontremos la relación en números enteros más simple de átomos en el compuesto dividiendo cada uno de los valores molares del carbono, hidrógeno y fluor entre el menor de estos tres valores:

    \[ \frac{0.5 \text{ mol \ce{C}}}{0.5 \text{ mol \ce{C}}} = 1 \qquad \frac{1.0 \text{ mol \ce{H}}}{0.5 \text{ mol \ce{C}}} = 2 \qquad \frac{1.0 \text{ mol \ce{F}}}{0.5 \text{ mol \ce{C}}} = 2 \]

    Entonces, hay 1 átomo de carbono, 2 átomos de hidrógeno y 2 átomos de fluor por cada 1 átomo de carbono en el compuesto.
    La fórmula empírica del compuesto es \ce{CH2F2}.
\end{solutionbox}