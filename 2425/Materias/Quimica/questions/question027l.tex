Una muestra de un compuesto que contiene únicamente átomos de carbono e hidrógeno se combustiona completamente produciendo 110.0 g de \ce{CO2} y 27.0 g de \ce{H2O}.
\textbf{¿Cuál es la fórmula que corresponde a este compuesto?}

\begin{oneparchoices}
    \choice  \ce{CH}
    \choice  \ce{C3H3}
    \CorrectChoice  \ce{C5H3}
    \choice  \ce{C5H6}
\end{oneparchoices}

\begin{solutionbox}{8cm}
    Cuando un compuesto que contiene únicamente carbono e hidrógeno se combustiona completamente, todos los átomos de carbono terminan formando \ce{CO2} y todos los átomos de hidrógeno terminan formando \ce{H2O}.
    Primero, usemos las masas molares del \ce{CO2} y \ce{H2O} para determinar cuantos moles de carbono e hidrógeno había en la muestra del compuesto antes de su combustión:

    \[ 110.0 g \text{ \ce{CO2}} \times \frac{1 \text{ mol \ce{CO2}}}{44.01 \text{ g \ce{CO2}}} \times \dfrac{ 1 \text{ mol \ce{C}}}{1 \text{ mol \ce{CO2}}} \approx 2.5 \text{ mol \ce{C}}  \]
    \[ 27.0 g \text{ \ce{H2O}} \times \frac{1 \text{ mol \ce{H2O}}}{18.02 \text{ g \ce{H2O}}}  \times \dfrac{ 1 \text{ mol \ce{H}}}{1 \text{ mol \ce{H2O}}} \approx 1.5 \text{ mol \ce{H}} \]

    Por lo tanto, la muestra contienen aproximadamente 2.5 moles de átomos de carbono y 1.5 moles de átomos de hidrógeno.
    A continuación, encontremos la relación en números enteros más simple de átomos en el compuesto dividiendo cada uno de los valores molares del carbono e hidrógeno entre el menor de estos dos valores:

    \[ \frac{2.5 \text{ mol \ce{C}}}{1.5 \text{ mol \ce{H}}} = 1.67 \qquad \frac{1.5 \text{ mol \ce{H}}}{1.5 \text{ mol \ce{H}}} = 1 \]
    Entonces, por cada 5 átomos de carbono en el compuesto hay 3 átomos de hidrógeno.
    La fórmula empírica del compuesto es \ce{C5H3}.
\end{solutionbox}