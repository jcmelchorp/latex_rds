\begin{multicols}{2}
Usando la información de la tabla \ref{tab:q003},
\textbf{Calcula el número de moles en una muestra de 2.03 kg de ácido cítrico (\ce{C6H8O7}).}\\
\emph{Escribe tu respuesta usando tres cifras significativas.}

    \begin{table}[H]
        \centering
        \caption{Masa molar de algunos elementos.}
        \label{tab:q003}
        \begin{tabular}{c|p{2.2cm}}
            \textbf{Elemento} & \textbf{Masa molar (g/mol)} \\\midrule
            H                 & 1.008                       \\\hline
            C                 & 12.01                       \\\hline
            O                 & 16.00                       \\\hline
            \bottomrule
        \end{tabular}
    \end{table}
\end{multicols}

\begin{solutionbox}{8cm}
    Podemos usar la masa molar de una sustancia para convertir gramos a moles de sustancia. Con esto en mente, primero calculemos la masa molar de \ce{C6H8O7}, usando la tabla \ref{tab:q003}:

    \begin{align*}
        6 \text{ mol de C} & = 6 \times 12.01  = 72.06 \text{g}           \\
        8 \text{ mol de H} & = 8 \times 1.008 = 8.064 \text{g}            \\
        7 \text{ mol de O} & = 7 \times 16.00 = 112.0 \text{g}            \\
        \text{Masa molar} & = 72.06 + 8.064 + 112.0 = 192.1 \text{g/mol}
    \end{align*}

    Por lo tanto, 1 mol de \ce{C6H8O7} tiene una masa molar de 192.1 g/mol.
    Después, usemos la masa molar de \ce{C6H8O7} para convertir gramos a moles de \ce{C6H8O7}.
    Ya que nos estan dando la masa de \ce{C6H8O7} en kilogramos, también necesitaremos incluir el factor de conversión de kilogramos a gramos:
    \[2.03 \text{kg} \times \frac{1000 \text{g}}{1 \text{kg}} \times \frac{1 \text{mol}}{192.1 \text{g}} = 10.6 \text{mol}\]

    Por lo tanto, hay 10.6 moles de \ce{C6H8O7} en 2.03 kg de \ce{C6H8O7}.
\end{solutionbox}