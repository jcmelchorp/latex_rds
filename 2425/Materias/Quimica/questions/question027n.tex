La masa molar del silicio (Si) es 28,09 g/mol.
\textbf{Calcula el número de átomos en una muestra de 92.8 mg de Si.}\\
\emph{Escribe tu respuesta en notación científica usando tres cifras significativas.}

\begin{solutionbox}{7cm}
    Podemos usar la masa molar de la sustancia para convertir gramos a moles de sustancia. Después, podemos usar el número de Avogadro,  $6.023 \times 10^{23}$, para convertir moles a  partículas representativas (como átomos, moléculas o unidades de fórmula).
    Primero, usemos la masa molar de \ce{Si} para convertir de gramos de \ce{Si} a moles de \ce{Si}. Ya que nos están dando la masa de \ce{Si} en miligramos, también necesitaremos incluir el factor de conversión de miligramos a gramos:
    \[92.8 \text{ mg \ce{Si}} \times \frac{1 \text{ g \ce{Si}}}{1000 \text{ mg \ce{Si}}} \times \frac{1 \text{ mol \ce{Si}}}{28.09 \text{ g \ce{Si}}} \approx 3.30 \times 10^{-3} \text{ mol \ce{Si}} \]
    Después, usemos el número de Avogadro para convertir de moles de \ce{Si} a átomos de \ce{Si}:
    \[3.30 \times 10^{-3} \text{ mol \ce{Si}} \times \frac{6.023 \times 10^{23} \text{ átomos \ce{Si}}}{1 \text{ mol \ce{Si}}} \approx 1.99 \times 10^{21} \text{ átomos \ce{Si}} \]
    Por lo tanto, una muestra de 92.8 mg de \ce{Si} tiene 1.99$\times 10^{21}$ átomos.
\end{solutionbox}