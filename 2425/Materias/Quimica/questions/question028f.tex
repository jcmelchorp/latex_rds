El peso molecular de la glucosa, \ce{C6H12O6}, es 180 g/mol.
\textbf{¿Cuántos moles de glucosa hay en 19.1 g de glucosa?}
\emph{Expresa la respuesta con 3 cifras significativas.}

\begin{solutionbox}{4cm}
    Podemos encontrar los moles de glucosa dividiendo los gramos de glucosa entre el peso molecular. Las unidades de gramos se cancelan, lo que significa que la respuesta estará en moles.
    \[ n =  19.1 \text{ g} \times \frac{1 \text{ mol}}{180 \text{ g}} = 0.106 \text{ mol} \]
\end{solutionbox}