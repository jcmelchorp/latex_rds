Con base en la información de la tabla \ref{tab:q01}, \textbf{¿cuál de los siguientes compuestos contiene el mayor porcentaje de sodio por masa?}

\begin{multicols}{2}
    \begin{oneparchoices}
        \choice   \ce{NaCN}
        \choice          \ce{NaN3}
        \CorrectChoice          \ce{NaOH}
        \choice          \ce{NaCl}
    \end{oneparchoices}

    \begin{table}[H]
        \centering
        \caption{Compuestos que contienen sodio}
        \label{tab:q02}
        \begin{tabular}{r|p{2.2cm}|p{2.4cm}}
            \textbf{Compuesto} & \textbf{Masa molar (g/mol)} & \textbf{Porcentaje de sodio (\%)} \\ \midrule
            \ce{NaCN}          & 49                          & \ifprintanswers{46.4\%}\fi        \\ \hline
            \ce{NaN3}          & 65                          & \ifprintanswers{35.4\%}\fi        \\ \hline
            \ce{NaOH}          & 40                          & \ifprintanswers{57.5\%}\fi        \\ \hline
            \ce{NaCl}          & 58.4                        & \ifprintanswers{39.2\%}\fi        \\ \hline
            \bottomrule
        \end{tabular}
    \end{table}

    \columnbreak

    \begin{solutionbox}{7cm}
        Ya que el peso atómico del sodio es de 22.99, el porcentaje de sodio en cada compuesto se puede calcular como:
        \[100\% \times \dfrac{\ce{Na}}{\ce{NaCN}}=100\% \times \dfrac{22.99}{49}  =46.4\% \]
        \[100\% \times \dfrac{\ce{Na}}{\ce{NaN3}}=100\% \times \dfrac{22.99}{65}  =35.4\% \]
        \[100\% \times \dfrac{\ce{Na}}{\ce{NaOH}}=100\% \times \dfrac{22.99}{40}  =57.5\% \]
        \[100\% \times \dfrac{\ce{Na}}{\ce{NaCl}}=100\% \times \dfrac{22.99}{58.4}=39.2\% \]
    \end{solutionbox}
\end{multicols}