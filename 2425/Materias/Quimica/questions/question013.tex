Completa la siguiente tabla determinando para cada especie, la cantidad de protones {\tiny\circled{\large\bf+}}, neutrones {\tiny\circled{\large{n}}} y electrones {\tiny\circled{\large\bf-}}.

\renewcommand{\arraystretch}{1.2}
\begin{table}[H]
    \centering
    \begin{tabular}{p{2.4cm}|c>{\columncolor{BurntOrange!40!white}}cc>{\columncolor{blue!40!white}}c}
        Especie                   & Símbolo & {\tiny\circled{\large\bf+}} & {\tiny\circled{\large{n}}} & {\tiny\circled{\large\bf-}} \\ \hline%& Masa atómica \\ \hline
        % Plutonio                  &         &          &           &            &              \\    \hline
        % Ión positivo de Estaño    &         &          &           &            &              \\    \hline
        % Niobio                 &         &          &           &            &              \\    \hline
        Xenón                     &         &                             &                            &                             \\ \hline%& \\    \hline
        Ión negativo de Antimonio &         &                             &                            &                             \\ \hline%& \\    \hline
        % Uranio                    &         &          &           &            &              \\    \hline
        % Ión positivo de Plata  &         &          &           &            &              \\    \hline
        % Tecnesio                  &         &          &           &            &              \\    \hline
        % Ión positivo de Litio     &         &          &           &            &              \\    \hline
        Fósforo                   &         &                             &                            &                             \\ \hline%& \\    \hline
        % Circonio                  &         &          &           &            &              \\    \hline
        Ión negativo de Azúfre    &         &                             &                            &                             \\ \hline%& \\    \hline
        % Ión negativo de Cloro   &         &          &           &            &               \\    \hline
        % Nitrógeno               &         &          &           &            &               \\    \hline
        Ión positivo de Silicio   &         &                             &                            &                             \\ \hline%& \\    \hline
        % Cobalto                &         &          &           &            &              \\    \hline
        % Curio                     &         &          &           &            &              \\    \hline
        % Oro                       &         &          &           &            &              \\    \hline
        % Ión negativo de Fósforo   &         &          &           &            &              \\    \hline
        % Iridio                    &         &          &           &            &              \\    \hline
        % Torio                     &         &          &           &            &              \\    \hline
    \end{tabular}
\end{table}