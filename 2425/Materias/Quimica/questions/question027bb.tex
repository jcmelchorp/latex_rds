En un recipiente se introducen 15 g de dióxido de carbono, \ce{CO2}.\\
\textbf{Calcula}:

\begin{parts}
    \part Los moles de sustancia introducidos.

    \begin{solutionbox}{5cm}
        Calculamos la masa molecular del dióxido de carbono, \ce{CO2}:
        \[ m_m(\ce{CO2}) = m(\ce{C}) + 2 \times m(\ce{O}) = 12 + 16 + 16 = 44 \text{ UMA} \]
        Entonces, la masa molar es:
        \[ M(\ce{CO2}) = 44 \text{ g mol}^{-1} \]
        El número de moles de \ce{CO2} se calcula con la ecuación (\ref{eq:masa_molar}), de la siguiente forma:
        \[ n(\ce{CO2}) = \frac{m(\ce{CO2})}{M(\ce{CO2})} = \frac{15 \text{ g}}{44 \text{ g mol}^{-1}} = 0.34 \text{ mol} \]
    \end{solutionbox}

    \part ¿Cuántas moléculas de \ce{CO2} y átomos de carbono y de oxígeno hay en el recipiente?

    \begin{solutionbox}{2cm}
        Del inciso anterior, sabemos que hay 0.34 moles de \ce{CO2}. Entonces, el número de moléculas de \ce{CO2} es:
        \[ 0.34 \text{ mol} \times 6.023 \times 10^{23} \text{ moléculas} = 2.05 \times 10^{23} \text{ moléculas} \]
    \end{solutionbox}
\end{parts}
