Usando la información de la tabla \ref{tab:q006},
\textbf{Calcula el número de unidades fórmula en una muestra de 33.8 g de fluoruro de estroncio (\ce{SrF2}).}\\
\emph{Escribe tu respuesta usando tres cifras significativas.}

\begin{multicols}{2}
    \begin{table}[H]
        \centering
        \caption{Masa molar de algunos elementos.}
        \label{tab:q006}
        \begin{tabular}{c|p{2.2cm}}
            \textbf{Elemento} & \textbf{Masa molar (g/mol)} \\\midrule
            F                 & 18.99                       \\\hline
            Sr                & 87.62                       \\\hline
            \bottomrule
        \end{tabular}
    \end{table}

    \columnbreak

    \begin{solutionbox}{7cm}
        Podemos usar la masa molar de la sustancia para convertir gramos a moles de sustancia. Después podemos usar el número de Avogadro, $6.023\times 10^{23}$, para convertir moles a partículas representativas (como átomos, moléculas o unidades de fórmula).
        Con esto en mente, primero calculemos la masa molar de \ce{SrF2} usando la tabla \ref{tab:q006}:
        \[ 1 \text{ mol \ce{Sr}} \times 87.62 \text{ g/mol \ce{Sr}} = 87.62 \text{ g \ce{Sr}} \]
        \[ 2 \text{ mol \ce{F}} \times 18.99 \text{ g/mol \ce{F}} = 37.98 \text{ g \ce{F}} \]
        \[ 87.62 \text{ g \ce{Sr}} + 37.98 \text{ g \ce{F}} = 125.6 \text{ g \ce{SrF2}} \]
        Por lo tanto, la masa molar de \ce{SrF2} es 125.6 g/mol.
        Después, usemos la masa molar de \ce{SrF2} para convertir gramos de \ce{SrF2} a moles de \ce{SrF2}:
        \[33.8 \text{ g \ce{SrF2}} \times \frac{1 \text{ mol}}{125.6 \text{ g \ce{SrF2}}} = 0.269 \text{ mol \ce{SrF2}} \]
        Finalmente, usemos el número de Avogadro para convertir moles de \ce{SrF2} a unidades de fórmula de \ce{SrF2}:
        \[0.269 \text{ mol \ce{SrF2}} \times \frac{6.023 \times 10^{23} \text{ unidades de fórmula}}{1 \text{ mol \ce{SrF2}}} = 1.62 \times 10^{23} \text{ unidades de fórmula} \]
        Por lo tanto, una muestra de 33.8 g de \ce{SrF2} contiene $1.62 \times 10^{23}$ unidades de fórmula de \ce{SrF2}.
    \end{solutionbox}
\end{multicols}
