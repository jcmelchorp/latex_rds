Calcula el número de partículas, la masa molecular o átómica ($m_m$ ó $m_a$) y la masa molar $M$ para las siguientes cantidades de sustancia:

\begin{multicols}{2}
    \begin{parts}
        \part 1 mol de átomos de \ce{N}

        \begin{solutionbox}{5cm}
            En 1 mol de \ce{N} hay $6.023\times 10^{23}$ átomos.\\
            Ya que es un sólo átomo, la masa atómica es (ver tabla periódica):
            \[ m_a(N)=14 \text{ UMA}\]
            La masa molar es:
            \[ M(N)=14 \text{ g/mol}\]
        \end{solutionbox}

        \part 1 mol de moléculas de \ce{N2}

        \begin{solutionbox}{4.5cm}
            En 1 mol de \ce{N2} hay $6.023\times 10^{23}$ moléculas.\\
            Ya que es una molécula, la masa molecular es:
            \[ m_m(\ce{N2})=2\times m_a(\ce{N})=2 \times 14 = 28 \text{ UMA}\]
            La masa molar es:
            \[ M(\ce{N2})=28 \text{ g/mol}\]
        \end{solutionbox}

        \part 1.5 mol de \ce{H2SO4}.

        \begin{solutionbox}{7cm}
            En 1 mol de \ce{H2SO4} hay $6.023\times 10^{23}$ átomos. Por lo tanto, en 1.5 mol de \ce{H2SO4} hay:
            \[ 1.5\times 6.023\times 10^{23} = 9.0345\times 10^{23} \text{ moléculas}\]
            Ya que es una molécula, la masa molecular es:
            \begin{align*}
                m_m(\ce{H2SO4}) & =m_a(\ce{H})+m_a(\ce{S})+2\times m_a(\ce{O}) \\
                                & =1+32+2\times 16                             \\
                                & = 65 \text{ UMA}
            \end{align*}

            \columnbreak

            La masa molar es:
            \[ M(\ce{H2SO4})=65 \text{ g/mol}\]
        \end{solutionbox}
    \end{parts}
\end{multicols}