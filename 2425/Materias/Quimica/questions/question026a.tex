Balancea la siguiente ecuación química:

\[
    \ce{Mg(OH)2 + HCl -> MgCl2 + H2O}
\]

\begin{solutionbox}{6cm}
    Hay 1 Mg en los productos y 1 en los reactivos, por lo que el Mg está balanceado.
    Hay 2 H en los productos y 1 en los reactivos, entonces multiplicamos por 2 al \ce{HCl}.
    \[
        \ce{Mg(OH)2 + 2HCl -> MgCl2 + H2O}
    \]
    Ahora hay 4 H en los reactivos y 2 en los productos, por lo que multiplicamos por 2 al \ce{H2O}.
    \[
        \ce{Mg(OH)2 + 2HCl -> MgCl2 + 2H2O}
    \]
    Ahora también O está balanceado, por lo que la ecuación balanceada es:
    \[
        \ce{Mg(OH)2 + 2HCl -> MgCl2 + 2H2O}
    \]
\end{solutionbox}