Elige la o las palabras que completan las afirmaciones.
\begin{center}
    \fbox{rompen} \quad \fbox{forman} \quad \fbox{emitida} \quad
    \fbox{absorbida} \quad \fbox{rompen} \quad \fbox{forman} \quad
    \fbox{emitida} \quad
    \fbox{absorbida} \quad
    \fbox{requerida} \quad \fbox{producida} \quad \fbox{igual} \quad
    \fbox{mayor} \quad \fbox{menor} \quad \fbox{requerida} \quad
    \fbox{producida} \quad \fbox{igual} \quad
    \fbox{mayor} \quad \fbox{menor} \quad \fbox{fuertes} \quad
    \fbox{d\'ebiles} \quad \fbox{igual} \quad
    \fbox{mayor} \quad \fbox{menor} \quad
\end{center}
\begin{parts}
    \part[2] Cuando los enlaces de una sustancia se \rule{3cm}{0.2mm}
    en una reacción endotérmica, la energía en forma de calor es \rule{3cm}{0.2mm}.

    \part[2] Cuando los enlaces de una sustancia se \rule{3cm}{0.2mm}
    en una reacción exotérmica, la energía en forma de calor es \rule{3cm}{0.2mm}


    \part[2] En procesos endotérmicos, la energía \rule{3cm}{0.2mm}
    para romper enlaces en los reactivos es \rule{3cm}{0.2mm}
    que la energía \rule{3cm}{0.2mm}
    al formar nuevos productos; en procesos exotérmicos, la energía es \rule{3cm}{0.2mm}


    \part[2] Los enlaces entre átomos del mismo tipo son más \rule{3cm}{0.2mm}
    que los enlaces entre átomos de distintos tipos, por lo requieren \rule{3cm}{0.2mm}
    energía para romperse.
\end{parts}