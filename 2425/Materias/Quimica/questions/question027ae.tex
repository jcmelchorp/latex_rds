Un estudiante determina que 9.8 g de una mezcla de \ce{MgCl2}(s) y \ce{NaNO3}(s) contiene 0.050 mol de \ce{MgCl2}(s).\\
\textbf{Con base en los resultados del estudiante, ¿cuál es el porcentaje de masa de Cl en la mezcla?}\\

\begin{oneparchoices}
    \choice  18\%
    \CorrectChoice  36\%
    \choice  48\%
    \choice  72\%
\end{oneparchoices}

\begin{solutionbox}{6cm}
    El porcentaje de masa de una sustancia en una mezcla se puede determinar por la comparación de la masa de la sustancia en la mezcla contra la masa total de la mezcla.
    Primero, determinemos la masa de \ce{cl} en la mezcla. Utilizando los resultados del estudiante y la masa molar del \ce{Cl}, podemos encontrar el número de moles de \ce{Cl} en la mezcla, y después convertir moles de \ce{Cl} a gramos de \ce{Cl}:
    \[0.050\text{mol \ce{Cl}} \times \frac{35.45\text{g \ce{Cl}}}{1\text{mol \ce{Cl}}} = 1.77 \text{g \ce{Cl}}\]
    Posteriormente, utilizando la masa calculada de \ce{Cl} y la masa total de la mezcla, podemos calcular el porcentaje de masa de \ce{Cl} en la mezcla: 
    \[3.5\text{g \ce{Cl}} \times \frac{100\%}{9.8\text{g mezcla}} = 36\%\]
    El porcentaje de masa de \ce{Cl} en la mezcla es 36\%.
    \end{solutionbox}
