Señala en cada uno de los enunciados si la sentencia es falsa o verdadera.

\begin{multicols}{2}
    \begin{parts}\footnotesize
        \part  La tabla periódica se encuentra
        constituida por filas (períodos) y
        columnas (grupos).

        \begin{oneparcheckboxes}
            \CorrectChoice Verdadero
            \choice Falso
        \end{oneparcheckboxes}

        % \part  A los elementos del subgrupo B,
        % se les denomina representativos.
        % Los elementos de transición se
        % ubican en el subgrupo A.

        % \begin{oneparcheckboxes}
        %     \CorrectChoice Verdadero
        %     \choice Falso
        % \end{oneparcheckboxes}

        % \part  La tabla periódica actual se puede
        % dividir en 5 bloques; s, p, d, f y g.

        % \begin{oneparcheckboxes}
        %     \choice Verdadero
        %     \choice Falso
        % \end{oneparcheckboxes}

        % \part  El bloque s está constituido por los grupos I y II A.

        % \begin{oneparcheckboxes}
        %     \choice Verdadero
        %     \choice Falso
        % \end{oneparcheckboxes}

        \part  Los electrones de valencia se encuentran siempre en el último
        nivel de energía.

        \begin{oneparcheckboxes}
            \CorrectChoice Verdadero
            \choice Falso
        \end{oneparcheckboxes}

        % \part Si la configuración electrónica de
        % un elemento termina en s o en p,
        % pertenece al subgrupo B.

        % \begin{oneparcheckboxes}
        %     \choice Verdadero
        %     \choice Falso
        % \end{oneparcheckboxes}

        \part El oxígeno y el nitrógeno son dos gases nobles de gran importancia.

        \begin{oneparcheckboxes}
            \choice Verdadero
            \CorrectChoice Falso
        \end{oneparcheckboxes}

        \part El mercurio es un elemento líquido.

        \begin{oneparcheckboxes}
            \CorrectChoice Verdadero
            \choice Falso
        \end{oneparcheckboxes}

        \part Los metales se ubican a la derecha y al centro de la tabla periódica.

        \begin{oneparcheckboxes}
            \CorrectChoice Verdadero
            \choice Falso
        \end{oneparcheckboxes}



        %  Las mejores fuentes de hierro
        % son las visceras, el hígado, los
        % quelites, acelgas y espinacas.

        % \begin{oneparcheckboxes}
        %     \CorrectChoice Verdadero
        %     \choice Falso
        % \end{oneparcheckboxes}





        %   El magnesio es el constituyente
        % esencial de la clorofila en las plantas verdes.

        % \begin{oneparcheckboxes}
        %     \CorrectChoice Verdadero
        %     \choice Falso
        % \end{oneparcheckboxes}

        %   La deficiencia de yodo es la causa
        % del bocio en los humanos.

        % \begin{oneparcheckboxes}
        %     \CorrectChoice Verdadero
        %     \choice Falso
        % \end{oneparcheckboxes}


        \part Los metales son maleables, dúctiles y buenos conductores del calor y la electricidad.

        \begin{oneparcheckboxes}
            \CorrectChoice Verdadero
            \choice Falso
        \end{oneparcheckboxes}


        %   Los metaloides se ubican arriba y
        % abajo de la línea diagonal que divde a los metales de los no metales.

        % \begin{oneparcheckboxes}
        %     \CorrectChoice Verdadero
        %     \choice Falso
        % \end{oneparcheckboxes}

        % El número de oxidación de un elemento combinado con otro, es cero.

        % \begin{oneparcheckboxes}
        %     \choice Verdadero
        %     \CorrectChoice Falso
        % \end{oneparcheckboxes}
        % {
        % \printanswers

        \part La fórmula H$_2$O expresa que la molécula de agua está constituida por dos átomos
        de oxígeno y uno de hidrógeno.

        \begin{oneparcheckboxes}
            \choice Verdadero
            \CorrectChoice Falso
        \end{oneparcheckboxes}
        % }

        \part En la fórmula de la Taurina, 4C$_{2}$H$_{7}$NO$_3$S, el número 4 indica que hay 4 átomos de carbono.

        \begin{oneparcheckboxes}
            \CorrectChoice Verdadero
            \choice Falso
        \end{oneparcheckboxes}

        \part Al número entero positivo, negativo o cero que se asigna a cada elemento en
        un compuesto, se denomina número de oxidación.

        \begin{oneparcheckboxes}
            \choice Verdadero
            \CorrectChoice Falso
        \end{oneparcheckboxes}

        \part En la construcción de una fórmula química se escribe primero la parte positiva y enseguida la negativa.

        \begin{oneparcheckboxes}
            \CorrectChoice Verdadero
            \choice Falso
        \end{oneparcheckboxes}

        \part Los subíndices expresan el número de átomos de los elementos presentes en
        una molécula o unidad fórmula.

        \begin{oneparcheckboxes}
            \CorrectChoice Verdadero
            \choice Falso
        \end{oneparcheckboxes}

        \part El símbolo Cl$^-$ indica que el átomo de cloro ha tenido una reducción o pérdida de
        electrones.

        \begin{oneparcheckboxes}
            \choice Verdadero
            \CorrectChoice Falso
        \end{oneparcheckboxes}

        \part  Una fórmula química sólo expresa la composición cualitativa de una sustancia.

        \begin{oneparcheckboxes}
            \choice Verdadero
            \CorrectChoice Falso
        \end{oneparcheckboxes}
        % {
        % \printanswers

        \part En una fórmula química, los coeficientes indican el número de moléculas o unidades fórmula; así como también el número de moles presentes de la sustancia.

        \begin{oneparcheckboxes}
            \CorrectChoice Verdadero
            \choice Falso
        \end{oneparcheckboxes}

        % }
        %  En una fórmula química la suma de cargas negativas y positivas siempre es
        % mayor de cero.
        % El electrón es una partícula subatómica que se encuentra ubicada en el núcleo atómico.

        % \begin{oneparcheckboxes}
        %     \choice Verdadero
        %     \CorrectChoice Falso
        % \end{oneparcheckboxes}
        % {
        % \printanswers
        %  La carga eléctrica del electrón es -1.602$\times$10$^{-19}$ Coulombs.

        % \begin{oneparcheckboxes}
        %     \CorrectChoice Verdadero
        %     \choice Falso
        % \end{oneparcheckboxes}
        % }
        %  El electrón tiene una masa que es aproximadamente 1836 veces menor con respecto a la del neutrón.

        % \begin{oneparcheckboxes}
        %     \CorrectChoice Verdadero
        %     \choice Falso
        % \end{oneparcheckboxes}
        % {
        % \printanswers

        \part El neutrón es una partícula subatómica que se encuentra girando alrededor del núcleo atómico.

        \begin{oneparcheckboxes}
            \choice Verdadero
            \CorrectChoice Falso
        \end{oneparcheckboxes}
        % }

        \part La masa de un neutrón es similar a la del protón.

        \begin{oneparcheckboxes}
            \CorrectChoice Verdadero
            \choice Falso
        \end{oneparcheckboxes}

        \part Las únicas partículas elementales en el núcleo, son los protones y neutrones.

        \begin{oneparcheckboxes}
            \choice Verdadero
            \CorrectChoice Falso
        \end{oneparcheckboxes}

        \part El número de masa representa la suma de protones y neutrones.

        \begin{oneparcheckboxes}
            \CorrectChoice Verdadero
            \choice Falso
        \end{oneparcheckboxes}

        \part El número total de electrones en un átomo lo determina el grupo al que pertenece.

        \begin{oneparcheckboxes}
            \choice Verdadero
            \CorrectChoice Falso
        \end{oneparcheckboxes}

        \part Los protones y neutrones son partículas constituidas por quarks.

        \begin{oneparcheckboxes}
            \CorrectChoice Verdadero
            \choice Falso
        \end{oneparcheckboxes}
    \end{parts}
\end{multicols}
