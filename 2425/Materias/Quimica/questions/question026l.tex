Balancea la siguiente ecuación química:

\begin{table}[H]
    \centering
    \begin{tabular}{cccc}
        \ce{H2O}                                                 & + \ce{->} & \ce{H2}                                                  & \ce{O2 }                                                 \\
        \includegraphics[height=0.5cm]{../images/20230415003551} &           & \includegraphics[height=0.5cm]{../images/20230415002057} & \includegraphics[height=0.5cm]{../images/20230415003542}
    \end{tabular}
\end{table}

\begin{solutionbox}{13cm}
    % \begin{multicols}{2}
    Si representamos la ecuación química con átomos de distintos colores para cada elemento, tenemos:
    \begin{table}[H]
        \centering
        \begin{tabular}{cccc}
            \ce{H2O}                                                 & + \ce{->} & \ce{H2}                                                  & \ce{O2 }                                                 \\
            \includegraphics[height=0.5cm]{../images/20230415003551} &           & \includegraphics[height=0.5cm]{../images/20230415002057} & \includegraphics[height=0.5cm]{../images/20230415003542}
        \end{tabular}
    \end{table}
    Hay 2 O en los productos y 1 O en los reactivos, por lo que hay que multiplicar por 2 al \ce{H2O}.
    \begin{table}[H]
        \centering
        \begin{tabular}{cccc}
            \ce{2H2O}                                                & + \ce{->} & \ce{H2}                                                  & \ce{O2 }                                                 \\
            \includegraphics[height=0.5cm]{../images/20230415003551} &           & \includegraphics[height=0.5cm]{../images/20230415002057} & \includegraphics[height=0.5cm]{../images/20230415003542} \\[-0.5em]
            \includegraphics[height=0.5cm]{../images/20230415003551} &           &                                                          &
        \end{tabular}
    \end{table}

    Ahora, hay 4 H en los reactivos y 2 H en los productos, por lo que hay que multiplicar por 2 al \ce{H2}.
    \begin{table}[H]
        \centering
        \begin{tabular}{cccc}
            \ce{2H2O}                                                & + \ce{->} & \ce{2H2}                                                 & \ce{O2 }                                                 \\
            \includegraphics[height=0.5cm]{../images/20230415003551} &           & \includegraphics[height=0.5cm]{../images/20230415002057} & \includegraphics[height=0.5cm]{../images/20230415003542} \\[-0.5em]
            \includegraphics[height=0.5cm]{../images/20230415003551} &           & \includegraphics[height=0.5cm]{../images/20230415002057} &
        \end{tabular}
    \end{table}
    Por lo tanto, la ecuación química balanceada es:
    \[
        \ce{2H2O -> 2H2 + O2}
    \]
    % \end{multicols}
\end{solutionbox}