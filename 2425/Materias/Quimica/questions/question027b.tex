Con base en la información de la tabla \ref{tab:q02}, \textbf{¿cuál de los siguientes compuestos contiene el menor porcentaje de carbono por masa?}

\begin{multicols}{2}
    \begin{oneparchoices}
        \choice         \ce{CH4}
        \choice  \ce{CH2O}
        \CorrectChoice          \ce{CO2}
        \choice          \ce{CO}
    \end{oneparchoices}

    \begin{table}[H]
        \centering
        \caption{Compuestos que contienen carbono}
        \label{tab:q02}
        \begin{tabular}{r|p{2.2cm}|p{2.4cm}}
            \textbf{Compuesto} & \textbf{Masa molar (g/mol)} & \textbf{Porcentaje de carbono (\%)} \\ \midrule
            \ce{CH4}           & 16                          & \ifprintanswers{75\%  }\fi          \\ \hline
            \ce{CH2O}          & 30                          & \ifprintanswers{40\%  }\fi          \\ \hline
            \ce{CO}            & 28                          & \ifprintanswers{42.9\%}\fi          \\ \hline
            \ce{CO2}           & 44                          & \ifprintanswers{27.3\%}\fi          \\ \hline
            \bottomrule
        \end{tabular}
    \end{table}

    \columnbreak

    \begin{solutionbox}{7cm}
        Ya que el peso atómico del carbono es 12.01, el porcentaje de carbono en cada compuesto se puede calcular como:
        \[100\% \times \dfrac{\ce{C}}{\ce{CH4} }=100\% \times \dfrac{12.01}{16}=75\%   \]
        \[100\% \times \dfrac{\ce{C}}{\ce{CH2O}}=100\% \times \dfrac{12.01}{30}=40\%   \]
        \[100\% \times \dfrac{\ce{C}}{\ce{CO}  }=100\% \times \dfrac{12.01}{28}=42.9\% \]
        \[100\% \times \dfrac{\ce{C}}{\ce{CO2} }=100\% \times \dfrac{12.01}{44}=27.3\% \]
    \end{solutionbox}
\end{multicols}


