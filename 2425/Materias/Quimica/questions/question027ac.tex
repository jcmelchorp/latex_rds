Se encuentra que una tableta de 1.60 g contiene 0.0133 mol del aminoácido glicina (\ce{C2H5NO2}). (La masa molar de \ce{C2H5NO2} es 75.07 g/mol.)\\
\textbf{¿Cuál es el porcentaje de masa de \ce{C2H5NO2} en la tableta?}\\
\emph{Escribe tu respuesta usando tres cifras significativas.}\\
\begin{solutionbox}{6cm}
    El porcentaje de masa de una sustancia en una mezcla se puede determinar por la comparación de la masa de la sustancia en la mezcla contra la masa total de la mezcla.
    Primero, calculemos la masa de \ce{C2H5NO2} en la tableta. Utilizando la masa molar de \ce{C2H5NO2}, podemos convertir moles de \ce{C2H5NO2} a gramos de \ce{C2H5NO2}:

    \[0.0133\text{mol \ce{C2H5NO2}} \times \frac{75.07\text{g \ce{C2H5NO2}}}{1\text{mol \ce{C2H5NO2}}} = 0.998 \text{g \ce{C2H5NO2}}\]

    Posteriormente, utilizando la masa calculada de \ce{C2H5NO2} y la masa total de la tableta, podemos calcular el porcentaje de masa de \ce{C2H5NO2} en la tableta:
    \[0.998\text{g \ce{C2H5NO2}} \times \frac{100}{1.60\text{g tableta}} = 62.0\%\]
    El porcentaje de masa de \ce{C2H5NO2} en la tableta es 62.0\%.
\end{solutionbox}