Halla la masa de ozono \ce{O3}, que contiene $1\times 10^{25}$ átomos de oxígeno.

\begin{solutionbox}{6.5cm}
    Calculamos la masa molecular del ozono, \ce{O3}:
    \[ m_m(\ce{O3}) = 3 \times m(\ce{O}) = 3 \times 16 = 48 \text{ UMA} \]
    Entonces, la masa molar es:
    \[ M(\ce{O3}) = 48 \text{ g mol}^{-1} \]
    Por otro lado, sabemos que la cantidad de moles de \ce{O3} es:
    \[ n(\ce{O3}) = \frac{1\times 10^{25}}{6.023\times 10^{23}} = 16.60 \text{ mol} \]
    Por lo tanto, la masa de ozono es:
    \[ m(\ce{O3}) = n(\ce{O3}) \times M(\ce{O3}) = 16.60 \times 48 = 797 \text{ g} \]
\end{solutionbox}