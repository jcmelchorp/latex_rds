Una muestra pura de un compuesto contiene 80\% de azufre y 20\% de oxígeno por masa.
\textbf{¿Cuál es la fórmula que corresponde a este compuesto?}

\begin{oneparchoices}
    \choice \ce{SO}
    \choice \ce{SO2}
    \CorrectChoice \ce{S2O}
    \choice \ce{S2O2}
\end{oneparchoices}

\begin{solutionbox}{8cm}
    La fórmula empírica de un compuesto es la proporción en números enteros más simple de elementos en el compuesto.
    Si un compuesto es 80 \% azufre y 20\% oxígeno por masa, entonces 100 g del compuesto contendrían 80 g de azufre y 20 g de oxígeno. Utilizando estos valores y las masas molares del azufre y del oxígeno podemos determinar el número de moles de cada elemento en 100 g del compuesto:

    \[80 \text{ g \ce{S}} \times \frac{1 \text{ mol \ce{S}}}{32.07 \text{ g \ce{S}}} = 2.5 \text{ mol \ce{S}} \]
    \[20 \text{ g \ce{O}} \times \frac{1 \text{ mol \ce{O}}}{16.00 \text{ g \ce{O}}} = 1.25 \text{ mol \ce{O}} \]

    Por lo tanto, la muestra contienen aproximadamente
    2.5 moles de átomos de azufre y 1.25 moles de átomos de oxígeno.
    A continuación, podemos encontrar la relación en números enteros más simple de átomos en el compuesto dividiendo cada uno de los valores molares del azufre y del oxígeno entre el menor de estos dos valores:
    \[ \frac{2.5 \text{ mol \ce{S}}}{1.25 \text{ mol \ce{O}}} = 2 \qquad \frac{1.25 \text{ mol \ce{O}}}{1.25 \text{ mol \ce{O}}} = 1 \]
    Entonces, hay 2 átomos de azufre por cada átomo de oxígeno en el compuesto.
    La fórmula empírica del compuesto es \ce{S2O}.
\end{solutionbox}