\documentclass[12pt,addpoints]{evalua}
\grado{1$^\circ$ de Secundaria}
\cicloescolar{2023-2024}
\materia{Matemáticas 1}
\unidad{}
\title{Examen Extraordinario}
\aprendizajes{
    \item Determina y usa la jerarquía de operaciones y los paréntesis en operaciones con números naturales, enteros y decimales (para multiplicación y división, sólo números positivos).
    \item Resuelve problemas mediante la formulación y solución algebraica de ecuaciones lineales.
    \item Analiza y compara situaciones de variación lineal a partir de sus representaciones tabular, gráfica y algebraica. Interpreta y resuelve problemas que se modelan con estos tipos de variación.
    \item Calcula valores faltantes en problemas de proporcionalidad directa, con constante natural, fracción o decimal (incluyendo tablas de variación).
    \item Resuelve problemas de cálculo de porcentajes, de tanto por ciento y de la cantidad base.
    \item Calcula el perímetro de polígonos y del círculo, y áreas de triángulos y cuadriláteros desarrollando y aplicando fórmulas.
   }
\author{Prof.: Julio César Melchor Pinto}
\begin{document}
\begin{questions}
    \question[10] \include*{../questions/question004}
    \newpage
    \question[10] \include*{../questions/question001}
    % \newpage
    % \question[10] \include*{../questions/question077a}
    % \question[20] \include*{../questions/question067a}
    % \newpage
    \question[20] \include*{../questions/question050d}
    \newpage
    \question[10] \include*{../questions/question073a}
    % \question[10] \include*{../questions/question074a}
    % \question[25] \include*{../questions/question050b}
    \newpage
    \question[20] \include*{../questions/question003}

    \question[15] \include*{../questions/question050e}
    \begin{multicols}{2}
        \question[15] \include*{../questions/question011}
    \end{multicols}
\end{questions}
\end{document}