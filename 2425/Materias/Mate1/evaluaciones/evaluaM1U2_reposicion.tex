\documentclass[12pt,addpoints]{evalua}
\grado{1$^\circ$ de Secundaria}
\cicloescolar{2023-2024}
\materia{Matemáticas 1}
\unidad{2}
\title{Examen de {\color{brown}recuperación} de la Unidad}
\aprendizajes{
    \item Determina y usa la jerarquía de operaciones y los paréntesis en
    operaciones con números naturales, enteros y
    decimales (para multiplicación y división, sólo números
    positivos).
    \item Resuelve problemas de cálculo de porcentajes, de tanto por
    ciento y de la cantidad base.
}
\author{Prof.: Julio César Melchor Pinto}
\begin{document}
\begin{questions}
    % \section*{\ifprintanswers{Operaciones con decimales}\else{}\fi}
    \question[8]Realiza las siguientes operaciones de decimales:

        \begin{multicols}{2}
            \begin{parts}\large
                \part
				\ifprintanswers{\opadd[hfactor=decimal,resultstyle=\color{red},carryadd=true]{241.81}{23.48}\\[1em]}
				\else{\opadd[hfactor=decimal,resultstyle=\color{white},carryadd=false]{241.81}{23.48}\\[1.5em]}\fi

                \part
				\ifprintanswers{\opsub[hfactor=decimal,resultstyle=\color{red},carrysub=true]{5.234}{2.347}}
				\else{\opsub[hfactor=decimal,resultstyle=\color{white},carrysub=false]{5.234}{2.347}}\fi\\[1em]

                \columnbreak%

                \part
				\ifprintanswers{\opmul[resultstyle=\color{red},displayintermediary=all]{738.4}{12.2}\\[1em]}
				\else{\opmul[resultstyle=\color{white},displayintermediary=None]{738.4}{12.2}\\[3em]}\fi

                \part
				\ifprintanswers{\opdiv[style=text,resultstyle=\color{red}]{187.772}{3.14}}
				\else{${187.772}\divisionsymbol{3.14}=$}\fi\\[1em]
            \end{parts}
        \end{multicols}

    % \subsection*{\ifprintanswers{Resolución de problemas}\else{}\fi}

    \question[10] Resuelve los siguientes problemas:
 
    \begin{parts}
        \part El precio de 385 artículos comerciales es de 1,232 pesos. ¿Cuál es el precio unitario de cada artículo?

        \begin{solutionbox}{2.5cm}
            \opdiv[style=text]{1232}{385}
        \end{solutionbox}
    \end{parts}

    \newpage
    % \section*{\ifprintanswers{Operaciones con fracciones}\else{}\fi}

    % \subsection*{\ifprintanswers{Operaciones con fracciones}\else{}\fi}

    \question[6] Realiza las siguientes operaciones con fracciones:

    \begin{multicols}{3}

        \begin{parts}\large
            \part $\dfrac{7}{8}-\dfrac{3}{8}=$ \fillin[$\dfrac{4}{8}=\dfrac{1}{2}$][0in]\\[0.5em]

            \part $\dfrac{3}{5} \divisionsymbol\dfrac{2}{3}=$ \fillin[$\dfrac{9}{10}$][0in]  \\[1em]

            \columnbreak%

            \part $\dfrac{7}{8}-\dfrac{3}{4}=$ \fillin[$\dfrac{7}{8}-\dfrac{6}{8}=\dfrac{1}{8}$][0in] \\[3em]

            \part $\dfrac{9}{5}\times\dfrac{15}{4}=$ \fillin[$\dfrac{135}{20}=\dfrac{27}{4}$][0in]\\[0.5em]

            \columnbreak%

            \part $\dfrac{5}{8}+\dfrac{3}{4}=$ \fillin[$\dfrac{5}{8}+\dfrac{6}{8}=\dfrac{11}{8}$][0in]  \\[3em]

            \part $\dfrac{7}{12}\divisionsymbol\dfrac{2}{3}=$ \fillin[$\dfrac{21}{24}=\dfrac{7}{8}$][0in]\\[1em]

        \end{parts}
    \end{multicols}


    % \subsection*{\ifprintanswers{Resolución de problemas}\else{}\fi}

    \question[10] Resuelve los siguientes problemas:

    \begin{parts}
        \part Un granjero siembra 2/5 de su granja con maíz y 3/10 con soya, ¿qué cantidad de su granja queda por sembrar?

        \begin{solutionbox}{2.5cm}
            Para conocer la cantidad de su granja que queda por sembrar, se debe restar 2/5 y 3/10 a 1; entonces:
            \[1-\dfrac{2}{5}-\dfrac{3}{10}=\dfrac{10}{10}-\dfrac{4}{10}-\dfrac{3}{10}=\dfrac{3}{10}\]

        \end{solutionbox}

        % \part Un reloj se adelanta 3/7 de minuto cada hora. ¿Cuánto se adelantará en 5 horas?

        % \begin{solutionbox}{2.5cm}
        %     Para conocer cuánto se adelantará en 5 horas, se debe multiplicar 3/7 por 5; entonces:
        %     \[\dfrac{3}{7}\times 5=\dfrac{15}{7}\]
        % \end{solutionbox}
    \end{parts}

    % \section*{\ifprintanswers{Porcentajes}\else{}\fi}
    % \subsection*{\ifprintanswers{Porcentajes a decimal}\else{}\fi}

    \question[6] Escribe como decimal los siguientes porcentajes:

    \begin{multicols}{3}
        \begin{parts}\large
            \part 10\% = \fillin[$\dfrac{10\%}{100\%}=0.1$][0in]\\[0.25em]
            \part 45\% = \fillin[$\dfrac{45\%}{100\%}=0.45$][0in]\\[0.25em]
            \part 5\% = \fillin[$\dfrac{5\%}{100\%}=0.05$][0in]\\[0.25em]
            \part 120\% = \fillin[$\dfrac{120\%}{100\%}=1.20$][0in]\\[0.25em]
            \part 30.9\% = \fillin[$\dfrac{30.9\%}{100\%}=0.309$][0in]\\[0.25em]
            \part 0.5\% = \fillin[$\dfrac{0.5\%}{100\%}=0.005$][0in]\\[0.25em]
        \end{parts}
    \end{multicols}

    % \newpage
    % \subsection*{\ifprintanswers{Decimal a porcentaje}\else{}\fi}

    \question[6] Escribe como porcentaje los siguientes decimales:

    \begin{multicols}{3}
        \begin{parts}\large
            \part $0.12=$  \fillin[$0.12 \times 100\% = 12\%$][0in]
            \part $0.103=$  \fillin[$0.09 \times 100\%= 9\%$][0in]
            \part $0.001$   \fillin[$0.001 \times 100\% = 0.1\%$][0in]
            \part $0.52=$  \fillin[$0.52 \times 100\% = 52\%$][0in]
				\part $0.09=$  \fillin[$0.09 \times 100\%= 9\%$][0in]
				\part $1.5=$   \fillin[$1.5 \times 100\% = 150\%$][0in]
        \end{parts}
    \end{multicols}

    % \subsection*{\ifprintanswers{Porcentaje de cantidades}\else{}\fi}

    \question[10] Calcula el porcentaje de las siguientes cantidades:

    \begin{multicols}{3}
    \begin{parts}
        % \part 60\% de 360 = \MULTIPLY{6}{36}{\sol} \fillin[$0.60\times 360=\sol$][0in]
        \part 16\% de 900 = \MULTIPLY{16}{9}{\sol} \fillin[$0.16\times 900=\sol$][0in]
        % \part 30\% de 600 = \MULTIPLY{30}{6}{\sol} \fillin[$0.30\times 600=\sol$][0in]
        \part 3\% de 1200 = \MULTIPLY{3}{12}{\sol} \fillin[$0.03\times 1200=\sol$][0in]
        \part 5\% de 7100 = \MULTIPLY{5}{71}{\sol} \fillin[$0.05\times 7100=\sol$][0in]
        \part 45\% de 800 = \MULTIPLY{45}{8}{\sol} \fillin[$0.45\times 800=\sol$][0in]

        \columnbreak%

        \part Si se sabe que 210 es el 21\% de cierta cantidad, ¿cuál es esta cantidad?
        
        \begin{solutionbox}{3cm}
            Para conocer la cantidad, se debe dividir 210 entre 21\%; entonces:\[\dfrac{100\%\times 210}{21\%}=1000\]
        \end{solutionbox}

        % \part Si se sabe que 200 es el 250\% de cierta cantidad, ¿cuál es esta cantidad?
    
        % \begin{solutionbox}{2cm}
        % 	Para conocer la cantidad, se debe dividir 200 entre 250; entonces:
        % 	\[100\times\dfrac{200}{250}=80\]
        % \end{solutionbox}

        \part Si se sabe que 120 es el 96\% de cierta cantidad, ¿cuál es esta cantidad?
    
        \begin{solutionbox}{3cm}
            Para conocer la cantidad, se debe dividir 120 entre 96\%; entonces:\[\dfrac{100\%\times 120}{96\%}=125\]
        \end{solutionbox}

    \end{parts}
\end{multicols}

    % \subsection*{\ifprintanswers{Resolución de problemas}\else{}\fi}

    \question[8] Resuelve los siguientes problemas:

    % \begin{multicols}{2}
        \begin{parts}
            \part El costo de una computadora es de \$12220 pesos, si la tasa de impuesto es del 16\%. ¿Cuánto será el total a pagar por la computadora?
			\begin{solutionbox}{3cm}
				Para conocer el total a pagar por la computadora, se debe multiplicar \$12220 por 16\%; entonces:
				\[\$12220\times 116\%= \$14175.20\]
				Por lo tanto, el total a pagar por la computadora es de \$14175.20 pesos.
			\end{solutionbox}

			\part El 24\% de los habitantes de un pueblo tienen menos de 30 años. ¿Cuántos habitantes tiene el pueblo si hay 120 jóvenes menores de 30 años?
			\begin{solutionbox}{3cm}
				Para conocer el total de habitantes del pueblo, se debe dividir 120 entre 24\%; entonces:
				\[\dfrac{100\%\times 120}{24\%}=500\]
				Por lo tanto, el pueblo tiene 500 habitantes.
			\end{solutionbox}
        \end{parts}
    % \end{multicols}


    % \section*{\ifprintanswers{Potencias y raíces}\else{}\fi}
    % \subsection*{\ifprintanswers{Potenciación}\else{}\fi}

    \question[6] Realiza las siguientes potencias:

    \begin{multicols}{3}
        \begin{parts}\large
            \part $3^4=$ \POWER{3}{4}{\sol} \fillin[$3\times 3 \times 3 \times 3=\sol$][0in] \\
          \part $\left(\dfrac{6}{2}\right)^3=$ \fillin[$\dfrac{216}{8}=27$][0in]
          \part $2^6=$ \POWER{2}{6}{\sol} \fillin[$2\times 2\times 2\times 2\times 2\times 2=\sol$][0in]\\
          \part $\left(\dfrac{10}{5}\right)^4=$ \fillin[$\dfrac{10000}{625}=16$][0in]
          \part $10^3=$ \POWER{10}{3}{\sol} \fillin[$10\times 10\times 10=\sol$][0in]\\
          \part $\left(\dfrac{5}{9}\right)^2=$ \fillin[$\dfrac{25}{81}$][0in]
        \end{parts}
    \end{multicols}

    % \subsection*{\ifprintanswers{Notación científica}\else{}\fi}

    \question[6]{Escribe la forma desarrollada de los siguientes números:

        \begin{multicols}{3}
            \begin{parts}\large
                \part $1.025\times10^{2}=$  \fillin[$102.5$][0in] \\
				\part $3.94\times10^{5}=$   \fillin[$394000$][0in]
				\part $12\times10^{8}=$     \fillin[$1200000000$][0in] \\
				\part $4\times10^{-2}=$      \fillin[$0.04$][0in]
				\part $2.08\times10^{-6}=$   \fillin[$0.00000208$][0in] \\
				\part $0.5\times10^{-3}=$    \fillin[$0.0005$][0in]
            \end{parts}
        \end{multicols}
    }

    \question[6]{Escribe con notación científica los siguientes números:

        \begin{multicols}{3}
            \begin{parts}\large
                \part $76000=$			\fillin[$7.6\times 10^{4}$][0in]
				\part $0.0104=$			\fillin[$1.04\times 10^{-2}$][0in]
				\part $83000000=$		 \fillin[$8.3\times 10^{7}$][0in]
				\part $0.00009=$		\fillin[$9\times 10^{-5}$][0in]
				\part $5000000000000=$	 \fillin[$5\times 10^{12}$][0in]
				\part $0.0000000002=$	\fillin[$2\times 10^{-10}$][0in]
            \end{parts}
        \end{multicols}
    }


    % \subsection*{\ifprintanswers{Raíces}\else{}\fi}
    \question[6]{Calcula las siguientes raíces cuadradas:
        \begin{multicols}{3}
            \begin{parts}\large
                \part $\sqrt{6.25}=$  \SQUAREROOT{6.25}{\sol}    \fillin[$\sol$][0in] \\
                \part $\sqrt{0.36}=$   						     \fillin[$0.6$][0in]
                \part $\sqrt{900}=$   \SQUAREROOT{900}{\sol}     \fillin[$\sol$][0in] \\
                \part $\sqrt{400}=$   \SQUAREROOT{400}{\sol}     \fillin[$\sol$][0in]
                \part $\sqrt{2.25}=$  \SQUAREROOT{2.25}{\sol}    \fillin[$\sol$][0in] \\
                \part $\sqrt{3600}=$  \SQUAREROOT{3600}{\sol}    \fillin[$\sol$][0in]
            \end{parts}
        \end{multicols}
    }

    \newpage
    % \section*{\ifprintanswers{Sistema de unidades}\else{}\fi}
    % \subsection*{\ifprintanswers{Unidades de longitud y masa}\else{}\fi}

    \question[4]{Convierte las siguientes unidades de longitud y de masa como se te pide:

        \begin{multicols}{2}
            \begin{parts}\large
                \part  54 metros ($m$) a hectómetros ($Hm$).       \\ \fillin[$54\divisionsymbol 10 \divisionsymbol 10=0.54$][0in] \\[1em]
                \part  2.9 decagramos ($Dg$) a miligramos ($mg$).  \\ \fillin[$2.9\times 10 \times 10 \times 10\times 10=29000$][0in] \\[1em]
                \part  149 centímetros ($cm$) a decámetros ($Dm$). \\ \fillin[$149\divisionsymbol 10 \divisionsymbol 10 \divisionsymbol 10 =0.194$][0in] \\[1em]
                \part  90.4 miligramos ($mg$) a centigramos ($cg$). \\ \fillin[$90.4\divisionsymbol 10 =9.04$][0in] \\[1em]
            \end{parts}
        \end{multicols}
    }

    % \subsection*{\ifprintanswers{Unidades de capacidad}\else{}\fi}
    \question[4]{Convierte las siguientes unidades de capacidad como se te pide:

        \begin{multicols}{2}
            \begin{parts}\large
                \part 702  mililitros ($mL$) a decalitros ($DL$).         \\ \fillin[$702 \divisionsymbol 10 \divisionsymbol 10\divisionsymbol 10 \divisionsymbol 10=0.0702$][0in]  \\[1em]
                \part 1.9   litros ($L$) a mililitros ($mL$).             \\ \fillin[$1.9  \times 10 \times 10 \times 10=19000$][0in]  \\[1em]
                \part 8200 litros ($L$) a metros cúbicos ($m^3$).		  \\ \fillin[$8200\divisionsymbol 1000=8.2$][0in] \\[1em]
                \part 4.8  decímetros cúbicos ($dm^3$) a litros ($L$).	  \\ \fillin[$4.8=4.8$][0in] \\[1em]
            \end{parts}
        \end{multicols}

    }
    % \subsection*{\ifprintanswers{Unidades de área y volumen}\else{}\fi}
    \question[4]{Convierte las siguientes unidades de área y volumen como se te pide:

            \begin{parts}\large
                \part  8 kilómetros cuadrados ($Km^2$) a metros cuadrados ($m^2$) \\ \fillin[$8\times 100 \times 100=80000$][0in] \\[3em]
                \part  88 metros cuadrados ($m^2$) a kilómetros cuadrados ($Km^2$) \\ \fillin[$88\divisionsymbol 100 \divisionsymbol 100\divisionsymbol 100=0.00088$][0in] \\[3em]
            \end{parts}
    }
\end{questions}
\end{document}