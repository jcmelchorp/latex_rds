Pablo está ahorrando para comprarse una tablet cuyo precio es \$13 000.00. Ya
tiene \$2 500.00 y planea ahorrar \$420.00 cada semana.

\begin{parts}
    ¿Con cuáles reglas puede calcular el dinero que tendrá en cualquier semana?

    \begin{oneparcheckboxes}
        \choice $420\left(n+\frac{2}{3}\right)$
        \CorrectChoice $20\left(21n+125\right)$
        \CorrectChoice $420n+2500$
        \choice $420n-2500$
    \end{oneparcheckboxes}

    \begin{solutionbox}{1.5cm}

    \end{solutionbox}

    ¿Por qué puede expresar su plan de ahorro por medio de reglas generales para sucesiones?

    \begin{solutionbox}{1.5cm}
        Por la regularidad de cada semana.
    \end{solutionbox}
    ¿En cuántas semanas habrá llegado a su meta?

    \begin{solutionbox}{2cm}
        Se debe cumplir que $420n + 2,500 = 13,000$. Entonces $n = 25$. Pablo tardará 25 semanas en juntar \$13,000.00 para su tablet.
    \end{solutionbox}
\end{parts}