Completa la Tabla \ref{tab:precio_aguacate} y responde a las preguntas.

% \ifprintanswers
\begin{table}[H]
    \rowcolors{2}{colorrds!10}{lightgray!10}
    \centering
    \caption{Precio del aguacate}
    \label{tab:precio_aguacate}
    \begin{tabular}{|c|c|c|}
        \toprule
        \rowcolor{colorrds!80}
        \textbf{\color{white}Peso} & \textbf{\color{white}Precio} & \textbf{\color{white}Razón entre precio y peso} \\\midrule
        1                          & \ifprintanswers59.16\fi      & \ifprintanswers59.16\fi                         \\\hline
        2                          & 118.32                       & \ifprintanswers59.16\fi                         \\\hline
        3                          & \ifprintanswers177.148\fi    & \ifprintanswers59.16\fi                         \\\hline
        4                          & \ifprintanswers236.26 \fi    & \ifprintanswers59.16\fi                         \\\cline{2-3}
        \bottomrule
    \end{tabular}
\end{table}


\begin{parts}
    \part ¿El precio del aguacate es proporcional a su peso?

    \begin{solutionbox}{1.5cm}
        Sí, pues la razón entre precio y peso es constante.
    \end{solutionbox}

    \part ¿Cuál es la constante de proporcionalidad para esta situación?
    \begin{solutionbox}{1.5cm}
        59.065
    \end{solutionbox}
\end{parts}