\begin{minipage}[t][][b]{0.8\textwidth}
    Una compañía de pizzas vende la pizza pequeña en \$30 pesos. Cada ingrediente cuesta \$8 pesos.
    Sabemos que el costo de una pizza con 0 ingredientes es de \$30 pesos; una pizza con 1 ingrediente cuesta \$8 pesos más,
    es decir, \$38 pesos, y así sucesivamente. A continuación mostramos una tabla que exhibe este hecho:

    \begin{table}[H]
        \rowcolors{2}{colorrds!10}{lightgray!10}
        \centering
        \caption{Costo de una pizza pequeña según la cantidad de ingredientes}
        \label{tab:pizza_ingredientes}
        \begin{tabular}{|c|c|c|}
            \toprule
            \rowcolor{colorrds!80}
            \textbf{\color{white}Ingredientes} & \textbf{\color{white}Costo} & \textbf{\color{white}Coordenada} \\\midrule
            0                                  & \$30                        & $(0,30)$                         \\\hline
            1                                  & \$38                        & $(1,38)$                         \\\hline
            2                                  & \$46                        & $(2,46)$                         \\\hline
            3                                  & \$54                        & $(3,54)$                         \\\hline
            4                                  & \$62                        & $(4,62)$                         \\\hline
            \bottomrule
        \end{tabular}
    \end{table}
    Podemos utilizar estos pares ordenados para crear la gráfica de la Figura \ref{fig:20230320232504}.
\end{minipage}\hfill
\begin{minipage}[t][][b]{0.2\textwidth}
    \begin{figure}[H]
        \centering
        \includegraphics[width=0.5\linewidth]{../images/20230320232504}
        \caption{}%Gráfica de la relación entre la cantidad de ingredientes y el costo en una pizza pequeña.}
        \label{fig:20230320232504}
    \end{figure}
\end{minipage}