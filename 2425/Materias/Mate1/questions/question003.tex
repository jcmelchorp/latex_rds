Obten el resultado de las siguientes operaciones tomando en cuenta la \textbf{jerarquía de operaciones}.

\begin{multicols}{2}
    \begin{parts}
        % \part $9\times10+3=$\fillin[93][1.5cm]

        % \begin{solutionbox}{2.2cm}

        % \end{solutionbox}

        \part $8-2\left(5-3+1\right)=$%\fillin[0][1.5cm]

        \begin{solutionbox}{3cm}
            \[
                \begin{array}{rlr}
                     & =8-2\left(5-3+1\right) & \text{\scriptsize Suma dentro del paréntesis} \\
                     & =8-2(3)                & \text{\scriptsize Multiplica 2 y 3}           \\
                     & =8-6                   & \text{\scriptsize Resta 8 y 6}                \\
                     & =2                     &
                \end{array}
            \]
        \end{solutionbox}

        % \part $4-1\times2=$\fillin[2][1.5cm]

        % \begin{solutionbox}{2.2cm}

        % \end{solutionbox}

        % \part $2+12 \divisionsymbol 2\times 3=$\fillin[2][1.5cm]

        % \begin{solutionbox}{2.2cm}

        % \end{solutionbox}

        % \part $3\times9+10\times\dfrac{36}{6}=$\fillin[87][1.5cm]

        % \begin{solutionbox}{2.2cm}

        % \end{solutionbox}

        % \part $9-3\times2=$\fillin[3][1.5cm]

        % \begin{solutionbox}{2.2cm}

        % \end{solutionbox}

        \part $6(4+2)=$%\fillin[30][1.5cm]

        \begin{solutionbox}{3cm}
            \[
                \begin{array}{rlr}
                     & =-6 (4+2) & \text{\scriptsize Se realiza primero el paréntesis} \\
                     & =-6 (6)   & \text{\scriptsize Multiplica -6 por 6}              \\
                     & =-36      &
                \end{array}
            \]
        \end{solutionbox}


        % \part $8 \divisionsymbol 4-\left(-10+8\right)=$\fillin[2][1.5cm]

        % \begin{solutionbox}{2.2cm}

        % \end{solutionbox}
    \end{parts}
\end{multicols}