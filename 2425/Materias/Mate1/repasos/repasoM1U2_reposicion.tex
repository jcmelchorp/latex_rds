\documentclass[12pt,addpoints,answers]{repaso}
\grado{1}
\nivel{Secundaria}
\cicloescolar{2024-2025}
\materia{Matemáticas}
\unidad{2}
\title{Practica la reposición a la Unidad}
\aprendizajes{
	\item Determina y usa la jerarquía de operaciones y los paréntesis en operaciones con números naturales, enteros y decimales (para multiplicación y división, sólo números positivos).
	\item Resuelve problemas de cálculo de porcentajes, de tanto porciento y de la cantidad base.
	\item Resuelve problemas de suma y resta con números enteros, fracciones y decimales positivos y negativos.
	\item Resuelve problemas de multiplicación con fracciones y decimales y de división con decimales.
}
\author{Melchor Pinto, J.C.}
\begin{document}
\INFO
\begin{multicols}{2}
	\tableofcontents
\end{multicols}
\vfill
\afterpage{\blankpage}
\begin{questions}
	\section{Operaciones con decimales}
	\subsection{Suma de decimales}

	\questionboxed[5]{Realiza las siguientes \textbf{sumas de decimales}:

		\begin{multicols}{5}
			\begin{parts}
				\part
				\ifprintanswers{\opadd[hfactor=decimal,resultstyle=\color{red},carryadd=true]{3441.6}{634.79}\\[1em]}
				\else{\opadd[hfactor=decimal,resultstyle=\color{white},carryadd=false]{3441.6}{634.79}\\[1.5em]}\fi

				\part
				\ifprintanswers{\opadd[hfactor=decimal,resultstyle=\color{red},carryadd=true]{4.908}{3.037}\\[1em]}
				\else{\opadd[hfactor=decimal,resultstyle=\color{white},carryadd=false]{4.908}{3.037}\\[1.5em]}\fi

				\part
				\ifprintanswers{\opadd[hfactor=decimal,resultstyle=\color{red},carryadd=true]{241.81}{23.48}\\[1em]}
				\else{\opadd[hfactor=decimal,resultstyle=\color{white},carryadd=false]{241.81}{23.48}\\[1.5em]}\fi

				\part
				\ifprintanswers{\opadd[hfactor=decimal,resultstyle=\color{red},carryadd=true]{36.494}{19.214}\\[1em]}
				\else{\opadd[hfactor=decimal,resultstyle=\color{white},carryadd=false]{36.494}{19.214}\\[1.5em]}\fi

				\part
				\ifprintanswers{\opadd[hfactor=decimal,resultstyle=\color{red},carryadd=true]{2314.3}{1923.9}\\[1em]}
				\else{\opadd[hfactor=decimal,resultstyle=\color{white},carryadd=false]{2314.3}{1923.9}\\[1.5em]}\fi
			\end{parts}
		\end{multicols}
	}

	\subsection{Resta de decimales}

	\questionboxed[5]{Realiza las siguientes \textbf{restas de decimales}:

		\begin{multicols}{3}
			\begin{parts}
				\part
				\ifprintanswers{\opsub[hfactor=decimal,resultstyle=\color{red},carrysub=true]{45.291}{40.093}}
				\else{\opsub[hfactor=decimal,resultstyle=\color{white},carrysub=false]{45.291}{40.093}}\fi\\[1em]
				\part
				\ifprintanswers{\opsub[hfactor=decimal,resultstyle=\color{red},carrysub=true]{5.234}{2.347}}
				\else{\opsub[hfactor=decimal,resultstyle=\color{white},carrysub=false]{5.234}{2.347}}\fi\\[1em]
				\part
				\ifprintanswers{\opsub[hfactor=decimal,resultstyle=\color{red},carrysub=true]{908.31}{134.67}}
				\else{\opsub[hfactor=decimal,resultstyle=\color{white},carrysub=false]{908.31}{134.67}}\fi\\[1em]
			\end{parts}
		\end{multicols}

	}

	\subsection{Multiplicación de decimales}

	\questionboxed[5]{Realiza las siguientes \textbf{multiplicaciones de decimales}:

		\begin{multicols}{3}
			\begin{parts}
				\part
				\ifprintanswers{\opmul[resultstyle=\color{red},displayintermediary=all]{87.31}{9.01}\\[1em]}
				\else{\opmul[resultstyle=\color{white},displayintermediary=None]{87.31}{9.01}\\[3em]}\fi

				\part
				\ifprintanswers{\opmul[resultstyle=\color{red},displayintermediary=all]{12.34}{7.4}\\[1em]}
				\else{\opmul[resultstyle=\color{white},displayintermediary=None]{12.34}{7.4}\\[3em]}\fi

				\part
				\ifprintanswers{\opmul[resultstyle=\color{red},displayintermediary=all]{738.4}{12.2}\\[1em]}
				\else{\opmul[resultstyle=\color{white},displayintermediary=None]{738.4}{12.2}\\[3em]}\fi
			\end{parts}
		\end{multicols}
	}

	\subsection{División de decimales}

	\questionboxed[5]{ Realiza las siguientes \textbf{divisiones con decimales}:

		\begin{multicols}{3}
			\begin{parts}
				\part
				\ifprintanswers{\opdiv[style=text,resultstyle=\color{red}]{187.772}{3.14}}
				\else{${187.772}\divisionsymbol{3.14}=$}\fi\\[1em]

				\part
				\ifprintanswers{\opdiv[style=text,resultstyle=\color{red}]{11.655}{2.1}}
				\else{\part ${11.655}\divisionsymbol{2.1}=$}\fi\\[1em]

				\part
				\ifprintanswers{\opdiv[style=text,resultstyle=\color{red}]{35.91}{5.7}}
				\else{\part ${35.91}\divisionsymbol{5.7}=$}\fi\\[1em]
			\end{parts}
		\end{multicols}

	}

	\subsection{Resolución de problemas}

	\questionboxed[5]{Resuelve los siguientes problemas:

		\begin{multicols}{3}
			\begin{parts}
				\part Una pintura tiene un costo de 33.24 pesos el litro, una persona compra 53 litros. ¿Cuánto debe pagar?

				\begin{solutionbox}{2.5cm}
					\opmul[resultstyle=\color{red},displayintermediary=all,carryadd=false,carrysub=false]{33.24}{53}
				\end{solutionbox}

				\part La mamá de Susana compró 11 metros de franela y pagó 103.40 pesos. ¿Cuánto cuesta el metro de franela?

				\begin{solutionbox}{2.5cm}
					\opdiv[style=text]{103.40}{11}
				\end{solutionbox}

				\part El precio de 385 artículos comerciales es de 1,232 pesos. ¿Cuál es el precio unitario de cada artículo?

				\begin{solutionbox}{2.5cm}
					\opdiv[style=text]{1232}{385}
				\end{solutionbox}
			\end{parts}
		\end{multicols}
	}


	\section{Operaciones con fracciones}
	\subsection{Suma y resta con denominadores iguales}
	\questionboxed[3]{ Realiza las siguientes sumas y restas de fracciones con \textbf{denominadores iguales}:
		\begin{multicols}{3}
			\begin{parts}
				\part $\dfrac{3}{5}+\dfrac{2}{5}=$ \fillin[$\dfrac{5}{5}=1$][0in]\\[0.5em]
				\part $\dfrac{7}{8}-\dfrac{3}{8}=$ \fillin[$\dfrac{5}{8}$][0in]\\[0.5em]
				\part $\dfrac{37}{12}-\dfrac{11}{12}=$ \fillin[$\dfrac{13}{6}$][0in]\\[0.5em]
				% \part $\dfrac{7}{4}+\dfrac{3}{4}=$ \fillin[$\dfrac{}{}+\dfrac{}{}=\dfrac{}{}$][0in]\\[1em]
				% \part $\dfrac{1}{3}+\dfrac{2}{3}=$ \fillin[$\dfrac{}{}+\dfrac{}{}=\dfrac{}{}$][0in]\\[1em]
				% \part $\dfrac{5}{4}+\dfrac{8}{4}=$ \fillin[$\dfrac{}{}+\dfrac{}{}=\dfrac{}{}$][0in]\\[1em]
			\end{parts}
		\end{multicols}
	}

	\subsection{Suma y resta denominadores diferentes}
	\questionboxed[5]{ Realiza las siguientes sumas y restas de fracciones con \bf{denominadores diferentes}:
		\begin{multicols}{3}
			\begin{parts}
				\part $\dfrac{3}{5}+\dfrac{2}{3}=$ \fillin[$\dfrac{9}{15}+\dfrac{10}{15}=\dfrac{19}{15}$][0in]\\[0.5em]
				\part $\dfrac{7}{8}+\dfrac{3}{4}=$ \fillin[$\dfrac{7}{8}+\dfrac{6}{8}=\dfrac{13}{8}$][0in]\\[0.5em]
				\part $\dfrac{2}{3}-\dfrac{1}{6}=$ \fillin[$\dfrac{4}{6}-\dfrac{1}{6} =\dfrac{3}{6} =\dfrac{1}{2}$][0in]\\[0.5em]
				\part $\dfrac{5}{6}-\dfrac{3}{8}=$ \fillin[$\dfrac{20}{24}-\dfrac{9}{24} =\dfrac{11}{24}$][0in]\\[0.5em]
				\part $\dfrac{4}{5}-\dfrac{3}{10}=$ \fillin[$\dfrac{8}{10}-\dfrac{3}{10}  =\dfrac{5}{10}=\dfrac{1}{2}$][0in]\\[0.5em]
				\part $\dfrac{1}{3}-\dfrac{1}{5}=$ \fillin[$\dfrac{5}{15}-\dfrac{3}{15} = \dfrac{2}{15}$][0in]\\[0.5em]
			\end{parts}
		\end{multicols}
	}

	\subsection{Multiplicación de fracciones}
	\questionboxed[5]{ Realiza las siguientes \textbf{multiplicación de fracciones}:
		\begin{multicols}{3}
			\begin{parts}
				% \part $\dfrac{3}{5}\times\dfrac{2}{3}=$ \fillin[$\dfrac{6}{15}$][0in]
				\part $\dfrac{7}{8}\times\dfrac{3}{4}=$ \fillin[$\dfrac{21}{32}$][0in]\\[0.5em]
				\part $\dfrac{4}{9}\times 2=$ \fillin[$\dfrac{4}{9}\times \dfrac{2}{1}=\dfrac{8}{9}$][0in]\\[0.5em]
				\part $4\times \dfrac{1}{5}=$ \fillin[$\dfrac{4}{1}\times \dfrac{1}{5}=\dfrac{4}{5}$][0in]\\[0.5em]
				\part $\dfrac{4}{3}\times\dfrac{7}{8}=$ \fillin[$\dfrac{28}{24}=\dfrac{7}{6}$][0in]\\[0.5em]
				\part $1\dfrac{5}{8}\times 1\dfrac{8}{9}=$ \fillin[$\dfrac{13}{8}\times\dfrac{17}{9}=\dfrac{221}{72}$][0in]\\[0.5em]
				\part $\dfrac{9}{5}\times\dfrac{15}{4}=$ \fillin[$\dfrac{135}{20}=\dfrac{27}{4}$][0in]\\[0.5em]
			\end{parts}
		\end{multicols}
	}

	\subsection{División de fracciones}
	\questionboxed[2]{ Realiza las siguientes \textbf{división de fracciones}:
		\begin{multicols}{3}
			\begin{parts}
				\part $\dfrac{5}{3}\divisionsymbol\dfrac{6}{15}=$ \fillin[$\dfrac{75}{18}=\dfrac{25}{6}$][0in]
				\part $ \dfrac{4}{5} \divisionsymbol\dfrac{2}{3}=$ \fillin[$\dfrac{12}{10}=\dfrac{6}{5}$][0in]\\[1em]
				\part $ \dfrac{7}{12}\divisionsymbol\dfrac{2}{3}=$ \fillin[$\dfrac{21}{24}=\dfrac{7}{8}$][0in]\\[1em]
			\end{parts}
		\end{multicols}
	}

	\subsection{Resolución de problemas}
	\questionboxed[5]{ Resuelve los siguientes problemas:
		
	\begin{multicols}{2}
		\begin{parts}
			\part Un granjero siembra 2/5 de su granja con maíz y 3/10 con soya, ¿qué cantidad de su granja queda por sembrar?
			\begin{solutionbox}{3cm}
				Para conocer la cantidad de su granja que queda por sembrar, se debe restar 2/5 y 3/10 a 1 (que representa la totalidad del terreno); entonces:
				\[1-\dfrac{2}{5}-\dfrac{3}{10}=\dfrac{10}{10}-\dfrac{4}{10}-\dfrac{3}{10}=\dfrac{3}{10}\]

			\end{solutionbox}

			\part Un reloj se adelanta 3/7 de minuto cada hora. ¿Cuánto se adelantará en 5 horas?
			\begin{solutionbox}{3cm}
				Para conocer cuánto se adelantará en 5 horas, se debe multiplicar 3/7 por 5; entonces:
				\[\dfrac{3}{7}\times 5=\dfrac{15}{7}\]
			\end{solutionbox}
		\end{parts}
	\end{multicols}
	}

	\section{Porcentajes}
	\subsection{Porcentajes a decimal}
	\questionboxed[5]{ Escribe como \textbf{decimal} los siguientes porcentajes:
		\begin{multicols}{3}
			\begin{parts}
				\part 25\% = \fillin[$\dfrac{25\%}{100\%}=0.25$][0in]\\[0.25em]
				\part 75\% = \fillin[$\dfrac{75\%}{100\%}=0.75$][0in]\\[0.25em]
				\part 50\% = \fillin[$\dfrac{50\%}{100\%}=0.5$][0in]\\[0.25em]
				\part 10\% = \fillin[$\dfrac{10\%}{100\%}=0.1$][0in]\\[0.25em]
				\part 5\% = \fillin[$\dfrac{5\%}{100\%}=0.05$][0in]\\[0.25em]
				\part 0.5\% = \fillin[$\dfrac{0.5\%}{100\%}=0.005$][0in]\\[0.25em]
			\end{parts}
		\end{multicols}
	}

	\subsection{Decimal a porcentaje}
	\questionboxed[5]{ Escribe como \textbf{porcentaje} los siguientes decimales:
		\begin{multicols}{3}
			\begin{parts}
				\part $0.52=$  \fillin[$0.52 \times 100\% = 52\%$][0in]\\[0.25em]
				\part $0.09=$  \fillin[$0.09 \times 100\%= 9\%$][0in]
				\part $6.5=$   \fillin[$6.5 \times 100\% = 650\%$][0in]\\[0.25em]
				\part $0.704=$ \fillin[$0.704 \times 100\% = 70.4\%$][0in]
				\part $0.1=$   \fillin[$0.1 \times 100\% = 10\%$][0in]\\[0.25em]
				\part $1=$     \fillin[$1 \times 100\% = 100\%$][0in]
			\end{parts}
		\end{multicols}
	}

	\subsection{Porcentaje de cantidades}
	\questionboxed[5]{ Calcula el porcentaje de las siguientes cantidades:
		\begin{multicols}{3}
			\begin{parts}
				\part 60\% de 360 = \MULTIPLY{6}{36}{\sol} \fillin[$0.60\times 360=\sol$][1in]
				\part 16\% de 900 = \MULTIPLY{16}{9}{\sol} \fillin[$0.16\times 900=\sol$][1in]
				\part 30\% de 600 = \MULTIPLY{30}{6}{\sol} \fillin[$0.30\times 600=\sol$][1in]
				\part 3\% de 1200 = \MULTIPLY{3}{12}{\sol} \fillin[$0.03\times 1200=\sol$][1in]
				\part 5\% de 7100 = \MULTIPLY{5}{71}{\sol} \fillin[$0.05\times 7100=\sol$][1in]
				\part 45\% de 800 = \MULTIPLY{45}{8}{\sol} \fillin[$0.45\times 800=\sol$][1in]

				\columnbreak%

				\part Si se sabe que 210 es el 21\% de cierta cantidad, ¿cuál es esta cantidad?
				
				\begin{solutionbox}{3cm}
					Para conocer la cantidad, se debe dividir 210 entre 21\%; entonces:\[\dfrac{100\%\times 210}{21\%}=1000\]
				\end{solutionbox}

				% \part Si se sabe que 200 es el 250\% de cierta cantidad, ¿cuál es esta cantidad?
			
				% \begin{solutionbox}{2cm}
				% 	Para conocer la cantidad, se debe dividir 200 entre 250; entonces:
				% 	\[100\times\dfrac{200}{250}=80\]
				% \end{solutionbox}

				\part Si se sabe que 120 es el 96\% de cierta cantidad, ¿cuál es esta cantidad?
			
				\begin{solutionbox}{3cm}
					Para conocer la cantidad, se debe dividir 120 entre 96\%; entonces:\[\dfrac{100\%\times 120}{96\%}=125\]
				\end{solutionbox}
			\end{parts}
		\end{multicols}
	}

	% \questionboxed[5]{Resuelve los siguientes problemas con porcentajes:
	%       \begin{parts}

	%       \end{parts}
	% }

	\subsection{Resolución de problemas}
	\questionboxed[5]{ Resuelve los siguientes problemas:
		\begin{parts}
			\part El costo de una computadora es de \$12220 pesos, si la tasa de impuesto es del 16\%. ¿Cuánto será el total a pagar por la computadora?
			\begin{solutionbox}{2.5cm}
				Para conocer el total a pagar por la computadora, se debe multiplicar \$12220 por 16\%; entonces:
				\[\$12220\times 116\%= \$14175.20\]
				Por lo tanto, el total a pagar por la computadora es de \$14175.20 pesos.
			\end{solutionbox}

			\part El 24\% de los habitantes de un pueblo tienen menos de 30 años. ¿Cuántos habitantes tiene el pueblo si hay 120 jóvenes menores de 30 años?
			\begin{solutionbox}{2.5cm}
				Para conocer el total de habitantes del pueblo, se debe dividir 120 entre 24\%; entonces:
				\[\dfrac{100\%\times 120}{24\%}=500\]
				Por lo tanto, el pueblo tiene 500 habitantes.
			\end{solutionbox}
		\end{parts}
	}

	\section{Potencias y raíces}
	\subsection{Potenciación}
	\questionboxed[5]{ Realiza las siguientes potencias:
		\begin{multicols}{3}
			\begin{parts}
				\part $2^3=$ \fillin[8][0in]
				\part $3^2=$ \fillin[9][0in]
				\part $5^2=$ \fillin[25][0in]
				\part $10^4=$ \fillin[10000][0in]
				\part $3^5=$ \fillin[243][0in]
				\part $\left(\dfrac{1}{3}\right)^3=$ \fillin[$\dfrac{1}{27}$][0in]
				\part $\left(\dfrac{2}{3}\right)^4=$ \fillin[$\dfrac{16}{81}$][0in]
				\part $\left(\dfrac{1}{9}\right)^2=$ \fillin[$\dfrac{1}{81}$][0in]
				\part $\left(\dfrac{4}{3}\right)^2=$ \fillin[$\dfrac{1}{1000}$][0in]
				\part $\left(\dfrac{3}{2}\right)^5=$ \fillin[$\dfrac{1}{8}$][0in]
			\end{parts}
		\end{multicols}
	}

	\subsection{Notación científica}
	\questionboxed[5]{Escribe la forma desarrollada de los siguientes números:
		\begin{multicols}{3}
			\begin{parts}
				\part $1.0934\times10^{4}=$
				\part $3.39\times10^{3}=$
				\part $12\times10^{5}=$
				\part $4\times10^{2}=$
				\part $2.08\times10^{6}=$
				\part $0.5\times10^{3}=$
			\end{parts}
		\end{multicols}
	}

	\questionboxed[5]{Escribe con notación científica los siguientes números:
		\begin{multicols}{3}
			\begin{parts}
				\part $7600=$
				\part $0.04=$
				\part $5000000=$
				\part $0.1=$
				\part $25=$
				\part $1.01=$
			\end{parts}
		\end{multicols}
	}


	\subsection{Raíces}
	\questionboxed[5]{Calcula las siguientes raíces cuadradas:
		\begin{multicols}{3}
			\begin{parts}
				\part $\sqrt{169}=$
				\part $\sqrt{1.44}=$
				\part $\sqrt{0.09}=$
				\part $\sqrt{2.25}=$
				\part $\sqrt{196}=$
				\part $\sqrt{900}$
			\end{parts}
		\end{multicols}
	}

	\section{Sistema de unidades}
	\subsection{Unidades de longitud y masa}
	\questionboxed[5]{Convierte las siguientes unidades de longitud y de masa como se te pide:
		\begin{parts}
			\part Convierte 4.9 kilómetros a metros.
			\part Convierte 34 metros a hectómetros
			\part Convierte 98 milímetros a centímetros
			\part Convierte 134 kilómetros a metros
			\part Convierte 134 centímetros a decámetros
			\part Convierte 342 gramos a hectogramos.
			\part Convierte 8334 centigramos a gramos.
			\part Convierte 93.4 miligramos a centigramos.
			\part Convierte 29 decagramos a miligramos.
			\part Convierte 9 gramos a miligramos.
		\end{parts}
	}

	\subsection{Unidades de capacidad}
	\questionboxed[5]{Convierte las siguientes unidades de capacidad como se te pide:
		\begin{parts}
			\part Convierte 27 hectolitros a decilitros.
			\part Convierte 8 mililitros a centilitros.
			\part Convierte 1094 mililitros a decilitros.
			\part Convierte 702 mililitros a decilitros.
			\part Convierte 19 litros a mililitros.
			\part Convierte 8200 litros a metros cúbicos.
			\part Convierte 4.8 decímetros cúbicos a litros.
			\part Convierte 750 litros a metros cúbicos.
			\part Convierte 567 milímetros cúbicos a litros.
			\part Convierte 4100 litros a metros cúbicos.

		\end{parts}
	}
	\subsection{Unidades de área y volumen}
	\questionboxed[5]{Convierte las siguientes unidades de área y volumen como se te pide:
		\begin{parts}
			\part Convierte 8.03 metros cúbicos a milímetros cúbicos
			\part Convierte 8 kilómetros cuadrados a metros cuadrados
			\part Convierte 88 metros cuadrados a kilómetros cuadrados
			\part Convierte 18 decámetros cúbicos a milímetros cúbicos
			\part Convierte 801 milímetros cuadrados a decámetros cuadrados
		\end{parts}
	}
\end{questions}
\end{document}