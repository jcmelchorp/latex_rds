\documentclass[12pt,addpoints]{repaso}
\grado{1}
\nivel{Secundaria}
\cicloescolar{2024-2025}
\materia{Matemáticas}
\unidad{2}
\title{Practica la reposición a la Unidad}
\aprendizajes{
	\item Determina y usa la jerarquía de operaciones y los paréntesis en operaciones con números naturales, enteros y decimales (para multiplicación y división, sólo números positivos).
	\item Resuelve problemas de cálculo de porcentajes, de tanto porciento y de la cantidad base.
	\item Resuelve problemas de suma y resta con números enteros, fracciones y decimales positivos y negativos.
	\item Resuelve problemas de multiplicación con fracciones y decimales y de división con decimales.
}
\author{Melchor Pinto, J.C.}
\begin{document}
\INFO%
\begin{multicols}{2}
	\tableofcontents
\end{multicols}
\vfill
\afterpage{\blankpage}
\begin{questions}
	\section{Operaciones con decimales}
	\subsection{Suma de decimales}

	\questionboxed[5]{Realiza las siguientes \textbf{sumas de decimales}:

		\begin{multicols}{5}
			\begin{parts}
				\part
				\ifprintanswers{\opadd[hfactor=decimal,resultstyle=\color{red},carryadd=true]{3441.6}{634.79}\\[1em]}
				\else{\opadd[hfactor=decimal,resultstyle=\color{white},carryadd=false]{3441.6}{634.79}\\[1.5em]}\fi

				\part
				\ifprintanswers{\opadd[hfactor=decimal,resultstyle=\color{red},carryadd=true]{4.908}{3.037}\\[1em]}
				\else{\opadd[hfactor=decimal,resultstyle=\color{white},carryadd=false]{4.908}{3.037}\\[1.5em]}\fi

				\part
				\ifprintanswers{\opadd[hfactor=decimal,resultstyle=\color{red},carryadd=true]{241.81}{23.48}\\[1em]}
				\else{\opadd[hfactor=decimal,resultstyle=\color{white},carryadd=false]{241.81}{23.48}\\[1.5em]}\fi

				\part
				\ifprintanswers{\opadd[hfactor=decimal,resultstyle=\color{red},carryadd=true]{36.494}{19.214}\\[1em]}
				\else{\opadd[hfactor=decimal,resultstyle=\color{white},carryadd=false]{36.494}{19.214}\\[1.5em]}\fi

				\part
				\ifprintanswers{\opadd[hfactor=decimal,resultstyle=\color{red},carryadd=true]{2314.3}{1923.9}\\[1em]}
				\else{\opadd[hfactor=decimal,resultstyle=\color{white},carryadd=false]{2314.3}{1923.9}\\[1.5em]}\fi
			\end{parts}
		\end{multicols}
	}

	\subsection{Resta de decimales}

	\questionboxed[5]{Realiza las siguientes \textbf{restas de decimales}:

		\begin{multicols}{3}
			\begin{parts}
				\part
				\ifprintanswers{\opsub[hfactor=decimal,resultstyle=\color{red},carrysub=true]{45.291}{40.093}}
				\else{\opsub[hfactor=decimal,resultstyle=\color{white},carrysub=false]{45.291}{40.093}}\fi\\[1em]
				\part
				\ifprintanswers{\opsub[hfactor=decimal,resultstyle=\color{red},carrysub=true]{5.234}{2.347}}
				\else{\opsub[hfactor=decimal,resultstyle=\color{white},carrysub=false]{5.234}{2.347}}\fi\\[1em]
				\part
				\ifprintanswers{\opsub[hfactor=decimal,resultstyle=\color{red},carrysub=true]{908.31}{134.67}}
				\else{\opsub[hfactor=decimal,resultstyle=\color{white},carrysub=false]{908.31}{134.67}}\fi\\[1em]
			\end{parts}
		\end{multicols}

	}

	\subsection{Multiplicación de decimales}

	\questionboxed[5]{Realiza las siguientes \textbf{multiplicaciones de decimales}:

		\begin{multicols}{3}
			\begin{parts}
				\part
				\ifprintanswers{\opmul[resultstyle=\color{red},displayintermediary=all]{87.31}{9.01}\\[1em]}
				\else{\opmul[resultstyle=\color{white},displayintermediary=None]{87.31}{9.01}\\[3em]}\fi

				\part
				\ifprintanswers{\opmul[resultstyle=\color{red},displayintermediary=all]{12.34}{7.4}\\[1em]}
				\else{\opmul[resultstyle=\color{white},displayintermediary=None]{12.34}{7.4}\\[3em]}\fi

				\part
				\ifprintanswers{\opmul[resultstyle=\color{red},displayintermediary=all]{738.4}{12.2}\\[1em]}
				\else{\opmul[resultstyle=\color{white},displayintermediary=None]{738.4}{12.2}\\[3em]}\fi
			\end{parts}
		\end{multicols}
	}

	\subsection{División de decimales}

	\questionboxed[5]{ Realiza las siguientes \textbf{divisiones con decimales}:

		\begin{multicols}{3}
			\begin{parts}
				\part
				\ifprintanswers{\opdiv[style=text,resultstyle=\color{red}]{187.772}{3.14}}
				\else{${187.772}\divisionsymbol{3.14}=$}\fi\\[1em]

				\part
				\ifprintanswers{\opdiv[style=text,resultstyle=\color{red}]{11.655}{2.1}}
				\else{ ${11.655}\divisionsymbol{2.1}=$}\fi\\[1em]

				\part
				\ifprintanswers{\opdiv[style=text,resultstyle=\color{red}]{35.91}{5.7}}
				\else{ ${35.91}\divisionsymbol{5.7}=$}\fi\\[1em]
			\end{parts}
		\end{multicols}

	}

	\subsection{Resolución de problemas}

	\questionboxed[5]{Resuelve los siguientes problemas:

		\begin{multicols}{3}
			\begin{parts}
				\part Una pintura tiene un costo de 33.24 pesos el litro, una persona compra 53 litros. ¿Cuánto debe pagar?

				\begin{solutionbox}{2.5cm}
					\opmul[resultstyle=\color{red},displayintermediary=all,carryadd=false,carrysub=false]{33.24}{53}
				\end{solutionbox}

				\part La mamá de Susana compró 11 metros (m) de franela y pagó 103.40 pesos. ¿Cuánto cuesta el metro de franela?

				\begin{solutionbox}{2.5cm}
					\opdiv[style=text]{103.40}{11}
				\end{solutionbox}

				\part El precio de 385 artículos comerciales es de 1,232 pesos. ¿Cuál es el precio unitario de cada artículo?

				\begin{solutionbox}{2.5cm}
					\opdiv[style=text]{1232}{385}
				\end{solutionbox}
			\end{parts}
		\end{multicols}
	}


	\section{Operaciones con fracciones}
	\subsection{Suma y resta con denominadores iguales}
	\questionboxed[3]{ Realiza las siguientes sumas y restas de fracciones con \textbf{denominadores iguales}:
		\begin{multicols}{3}
			\begin{parts}
				\part $\dfrac{3}{5}+\dfrac{2}{5}=$ \fillin[$\dfrac{5}{5}=1$][0in]\\[0.5em]
				\part $\dfrac{7}{8}-\dfrac{3}{8}=$ \fillin[$\dfrac{5}{8}$][0in]\\[0.5em]
				\part $\dfrac{37}{12}-\dfrac{11}{12}=$ \fillin[$\dfrac{13}{6}$][0in]\\[0.5em]
				% \part $\dfrac{7}{4}+\dfrac{3}{4}=$ \fillin[$\dfrac{}{}+\dfrac{}{}=\dfrac{}{}$][0in]\\[1em]
				% \part $\dfrac{1}{3}+\dfrac{2}{3}=$ \fillin[$\dfrac{}{}+\dfrac{}{}=\dfrac{}{}$][0in]\\[1em]
				% \part $\dfrac{5}{4}+\dfrac{8}{4}=$ \fillin[$\dfrac{}{}+\dfrac{}{}=\dfrac{}{}$][0in]\\[1em]
			\end{parts}
		\end{multicols}
	}

	\subsection{Suma y resta denominadores diferentes}
	\questionboxed[5]{ Realiza las siguientes sumas y restas de fracciones con \bf{denominadores diferentes}:
		\begin{multicols}{3}
			\begin{parts}
				\part $\dfrac{3}{5}+\dfrac{2}{3}=$ \fillin[$\dfrac{9}{15}+\dfrac{10}{15}=\dfrac{19}{15}$][0in]\\[0.5em]
				\part $\dfrac{7}{8}+\dfrac{3}{4}=$ \fillin[$\dfrac{7}{8}+\dfrac{6}{8}=\dfrac{13}{8}$][0in]\\[0.5em]
				\part $\dfrac{2}{3}-\dfrac{1}{6}=$ \fillin[$\dfrac{4}{6}-\dfrac{1}{6} =\dfrac{3}{6} =\dfrac{1}{2}$][0in]\\[0.5em]
				\part $\dfrac{5}{6}-\dfrac{3}{8}=$ \fillin[$\dfrac{20}{24}-\dfrac{9}{24} =\dfrac{11}{24}$][0in]\\[0.5em]
				\part $\dfrac{4}{5}-\dfrac{3}{10}=$ \fillin[$\dfrac{8}{10}-\dfrac{3}{10}  =\dfrac{5}{10}=\dfrac{1}{2}$][0in]\\[0.5em]
				\part $\dfrac{1}{3}-\dfrac{1}{5}=$ \fillin[$\dfrac{5}{15}-\dfrac{3}{15} = \dfrac{2}{15}$][0in]\\[0.5em]
			\end{parts}
		\end{multicols}
	}

	\subsection{Multiplicación de fracciones}
	\questionboxed[5]{ Realiza las siguientes \textbf{multiplicación de fracciones}:
		\begin{multicols}{3}
			\begin{parts}
				% \part $\dfrac{3}{5}\times\dfrac{2}{3}=$ \fillin[$\dfrac{6}{15}$][0in]
				\part $\dfrac{7}{8}\times\dfrac{3}{4}=$ \fillin[$\dfrac{21}{32}$][0in]\\[0.5em]
				\part $\dfrac{4}{9}\times 2=$ \fillin[$\dfrac{4}{9}\times \dfrac{2}{1}=\dfrac{8}{9}$][0in]\\[0.5em]
				\part $4\times \dfrac{1}{5}=$ \fillin[$\dfrac{4}{1}\times \dfrac{1}{5}=\dfrac{4}{5}$][0in]\\[0.5em]
				\part $\dfrac{4}{3}\times\dfrac{7}{8}=$ \fillin[$\dfrac{28}{24}=\dfrac{7}{6}$][0in]\\[0.5em]
				\part $1\dfrac{5}{8}\times 1\dfrac{8}{9}=$ \fillin[$\dfrac{13}{8}\times\dfrac{17}{9}=\dfrac{221}{72}$][0in]\\[0.5em]
				\part $\dfrac{9}{5}\times\dfrac{15}{4}=$ \fillin[$\dfrac{135}{20}=\dfrac{27}{4}$][0in]\\[0.5em]
			\end{parts}
		\end{multicols}
	}

	\subsection{División de fracciones}
	\questionboxed[2]{ Realiza las siguientes \textbf{división de fracciones}:
		\begin{multicols}{3}
			\begin{parts}
				\part $\dfrac{5}{3}\divisionsymbol\dfrac{6}{15}=$ \fillin[$\dfrac{75}{18}=\dfrac{25}{6}$][0in]
				\part $ \dfrac{4}{5} \divisionsymbol\dfrac{2}{3}=$ \fillin[$\dfrac{12}{10}=\dfrac{6}{5}$][0in]\\[1em]
				\part $ \dfrac{7}{12}\divisionsymbol\dfrac{2}{3}=$ \fillin[$\dfrac{21}{24}=\dfrac{7}{8}$][0in]\\[1em]
			\end{parts}
		\end{multicols}
	}

	\subsection{Resolución de problemas}

	\questionboxed[5]{ Resuelve los siguientes problemas:
		
	\begin{multicols}{2}
		\begin{parts}
			\part Un granjero siembra 2/5 de su granja con maíz y 3/10 con soya, ¿qué cantidad de su granja queda por sembrar?
			\begin{solutionbox}{3cm}
				Para conocer la cantidad de su granja que queda por sembrar, se debe restar 2/5 y 3/10 a 1 (que representa la totalidad del terreno); entonces:
				\[1-\dfrac{2}{5}-\dfrac{3}{10}=\dfrac{10}{10}-\dfrac{4}{10}-\dfrac{3}{10}=\dfrac{3}{10}\]

			\end{solutionbox}

			\part Un reloj se adelanta 3/7 de minuto cada hora. ¿Cuánto se adelantará en 5 horas?
			\begin{solutionbox}{3cm}
				Para conocer cuánto se adelantará en 5 horas, se debe multiplicar 3/7 por 5; entonces:
				\[\dfrac{3}{7}\times 5=\dfrac{15}{7}\]
			\end{solutionbox}
		\end{parts}
	\end{multicols}
	}

	\section{Porcentajes}
	\subsection{Porcentajes a decimal}

	\questionboxed[5]{ Escribe como \textbf{decimal} los siguientes porcentajes:
	
	\begin{multicols}{3}
			\begin{parts}
				\part 25\% = \fillin[$\dfrac{25\%}{100\%}=0.25$][0in]\\[0.25em]
				\part 75\% = \fillin[$\dfrac{75\%}{100\%}=0.75$][0in]\\[0.25em]
				\part 50\% = \fillin[$\dfrac{50\%}{100\%}=0.5$][0in]\\[0.25em]
				\part 10\% = \fillin[$\dfrac{10\%}{100\%}=0.1$][0in]\\[0.25em]
				\part 5\% = \fillin[$\dfrac{5\%}{100\%}=0.05$][0in]\\[0.25em]
				\part 0.5\% = \fillin[$\dfrac{0.5\%}{100\%}=0.005$][0in]\\[0.25em]
			\end{parts}
		\end{multicols}
	}

	\subsection{Decimal a porcentaje}

	\questionboxed[5]{Escribe como \textbf{porcentaje} los siguientes decimales:
		
	\begin{multicols}{3}
			\begin{parts}
				\part $0.52=$  \fillin[$0.52 \times 100\% = 52\%$][0in]
				\part $0.09=$  \fillin[$0.09 \times 100\%= 9\%$][0in]
				\part $1.5=$   \fillin[$1.5 \times 100\% = 150\%$][0in]
				\part $0.404=$ \fillin[$0.404 \times 100\% = 40.4\%$][0in]
				\part $0.1=$   \fillin[$0.1 \times 100\% = 10\%$][0in]
				\part $1=$     \fillin[$1 \times 100\% = 100\%$][0in]
				\part $0.12=$  \fillin[$0.12 \times 100\% = 12\%$][0in]
				\part $0.103=$  \fillin[$0.09 \times 100\%= 9\%$][0in]
				\part $0.001$   \fillin[$0.001 \times 100\% = 0.1\%$][0in]
			\end{parts}
		\end{multicols}
	}

	\subsection{Porcentaje de cantidades}
	
	\questionboxed[5]{Calcula el porcentaje de las siguientes cantidades:
	
	\begin{multicols}{3}
			\begin{parts}
				\part 60\% de 360 = \MULTIPLY{6}{36}{\sol} \fillin[$0.60\times 360=\sol$][1in]
				\part 16\% de 900 = \MULTIPLY{16}{9}{\sol} \fillin[$0.16\times 900=\sol$][1in]
				\part 30\% de 600 = \MULTIPLY{30}{6}{\sol} \fillin[$0.30\times 600=\sol$][1in]
				\part 3\% de 1200 = \MULTIPLY{3}{12}{\sol} \fillin[$0.03\times 1200=\sol$][1in]
				\part 5\% de 7100 = \MULTIPLY{5}{71}{\sol} \fillin[$0.05\times 7100=\sol$][1in]
				\part 45\% de 800 = \MULTIPLY{45}{8}{\sol} \fillin[$0.45\times 800=\sol$][1in]

				\columnbreak%

				\part Si se sabe que 210 es el 21\% de cierta cantidad, ¿cuál es esta cantidad?
				
				\begin{solutionbox}{3cm}
					Para conocer la cantidad, se debe dividir 210 entre 21\%; entonces:\[\dfrac{100\%\times 210}{21\%}=1000\]
				\end{solutionbox}

				% \part Si se sabe que 200 es el 250\% de cierta cantidad, ¿cuál es esta cantidad?
			
				% \begin{solutionbox}{2cm}
				% 	Para conocer la cantidad, se debe dividir 200 entre 250; entonces:
				% 	\[100\times\dfrac{200}{250}=80\]
				% \end{solutionbox}

				\part Si se sabe que 120 es el 96\% de cierta cantidad, ¿cuál es esta cantidad?
			
				\begin{solutionbox}{3cm}
					Para conocer la cantidad, se debe dividir 120 entre 96\%; entonces:\[\dfrac{100\%\times 120}{96\%}=125\]
				\end{solutionbox}
			\end{parts}
		\end{multicols}
	}

	% \questionboxed[5]{Resuelve los siguientes problemas con porcentajes:
	%       \begin{parts}

	%       \end{parts}
	% }

	\subsection{Resolución de problemas}
	
	\questionboxed[5]{ Resuelve los siguientes problemas:
	
	\begin{parts}
			\part El costo de una computadora es de \$12220 pesos, si la tasa de impuesto es del 16\%. ¿Cuánto será el total a pagar por la computadora?
			\begin{solutionbox}{2.5cm}
				Para conocer el total a pagar por la computadora, se debe multiplicar \$12220 por 16\%; entonces:
				\[\$12220\times 116\%= \$14175.20\]
				Por lo tanto, el total a pagar por la computadora es de \$14175.20 pesos.
			\end{solutionbox}

			\part El 24\% de los habitantes de un pueblo tienen menos de 30 años. ¿Cuántos habitantes tiene el pueblo si hay 120 jóvenes menores de 30 años?
			\begin{solutionbox}{2.5cm}
				Para conocer el total de habitantes del pueblo, se debe dividir 120 entre 24\%; entonces:
				\[\dfrac{100\%\times 120}{24\%}=500\]
				Por lo tanto, el pueblo tiene 500 habitantes.
			\end{solutionbox}
		\end{parts}
	}

	\newpage
	\section{Potencias y raíces}
	\subsection{Potenciación}

	\questionboxed[5]{Realiza las siguientes potencias:
	
	\begin{multicols}{3}
			\begin{parts}
				\part $3^4=$ \POWER{3}{4}{\sol} \fillin[$3\times 3 \times 3 \times 3=\sol$][0in]
				\part $10^3=$ \POWER{10}{3}{\sol} \fillin[$10\times 10\times 10=\sol$][0in]
				\part $25^2=$ \POWER{25}{2}{\sol} \fillin[$25\times 25=\sol$][0in]
				\part $2^6=$ \POWER{2}{6}{\sol} \fillin[$2\times 2\times 2\times 2\times 2\times 2=\sol$][0in]
				\part $4^3=$ \POWER{4}{3}{\sol} \fillin[$4\times 4 \times 4=\sol$][0in]
				\part $\left(\dfrac{1}{3}\right)^3=$ \fillin[$\dfrac{1}{27}$][0in]
				\part $\left(\dfrac{2}{3}\right)^4=$ \fillin[$\dfrac{16}{81}$][0in]
				\part $\left(\dfrac{10}{5}\right)^4=$ \fillin[$\dfrac{10000}{625}=16$][0in]
				\part $\left(\dfrac{5}{9}\right)^2=$ \fillin[$\dfrac{25}{81}$][0in]
				\part $\left(\dfrac{6}{2}\right)^3=$ \fillin[$\dfrac{216}{8}=27$][0in]
				\part $\left(\dfrac{3}{6}\right)^2=$ \fillin[$\dfrac{9}{36}=\dfrac{1}{4}$][0in]
			\end{parts}
		\end{multicols}
	}

	\subsection{Notación científica}

	\questionboxed[5]{Escribe la forma desarrollada de los siguientes números:
	
	\begin{multicols}{3}
			\begin{parts}
				\part $1.025\times10^{2}=$  \fillin[$102.5$][0in]
				\part $3.94\times10^{5}=$   \fillin[$394000$][0in]
				\part $12\times10^{8}=$     \fillin[$1200000000$][0in]
				\part $4\times10^{-2}=$      \fillin[$0.04$][0in]
				\part $2.08\times10^{-6}=$   \fillin[$0.00000208$][0in]
				\part $0.5\times10^{-3}=$    \fillin[$0.0005$][0in]
			\end{parts}
		\end{multicols}
	}

	\questionboxed[5]{Escribe con notación científica los siguientes números:
	
	\begin{multicols}{3}
			\begin{parts}
				\part $76000=$			\fillin[$7.6\times 10^{4}$][0in]
				\part $0.0104=$			\fillin[$1.04\times 10^{-2}$][0in]
				\part $83000000=$		 \fillin[$8.3\times 10^{7}$][0in]
				\part $0.00009=$		\fillin[$9\times 10^{-5}$][0in]
				\part $5000000000000=$	 \fillin[$5\times 10^{12}$][0in]
				\part $0.0000000002=$	\fillin[$2\times 10^{-10}$][0in]
			\end{parts}
		\end{multicols}
	}


	\subsection{Raíces}
	
	\questionboxed[5]{Calcula las siguientes raíces cuadradas:
	
	\begin{multicols}{3}
			\begin{parts}
				\part $\sqrt{169}=$   \SQUAREROOT{169}{\sol} \fillin[\sol][0in]
				\part $\sqrt{400}=$   \SQUAREROOT{400}{\sol} \fillin[\sol][0in]
				\part $\sqrt{6.25}=$  \SQUAREROOT{6.25}{\sol} \fillin[\sol][0in]
				\part $\sqrt{1.44}=$  \SQUAREROOT{1.44}{\sol} \fillin[\sol][0in]
				\part $\sqrt{0.36}=$   \fillin[0.6][0in]
				\part $\sqrt{2.25}=$  \SQUAREROOT{2.25}{\sol} \fillin[\sol][0in]
				\part $\sqrt{196}=$	  \SQUAREROOT{196}{\sol} \fillin[\sol][0in]
				\part $\sqrt{3600}=$  \SQUAREROOT{3600}{\sol} \fillin[\sol][0in]
				\part $\sqrt{900}$ 	  \SQUAREROOT{900}{\sol} \fillin[\sol][0in]
			\end{parts}
		\end{multicols}
	}

	\newpage
	\section{Sistema de unidades}
	\subsection{Unidades de longitud y masa}
	
	\questionboxed[5]{Convierte las siguientes unidades de longitud y de masa como se te pide:
		
	\begin{multicols}{2}
	\begin{parts}
			\part  3.8 kilómetros ($Km$) a metros ($m$).       \\ \fillin[$3.8\times 10 \times 10 \times 10=3800$][0in]
			\part  54 metros ($m$) a hectómetros ($Hm$).       \\ \fillin[$54\divisionsymbol 10 \divisionsymbol 10=0.54$][0in]
			\part  88 milímetros ($mm$) a centímetros ($cm$)   \\ \fillin[$88\divisionsymbol 10 =8.8$][0in]
			\part  123 kilómetros ($Km$) a metros ($m$)		   \\ \fillin[$123\times 10 \times 10 \times 10=123000$][0in]
			\part  149 centímetros ($cm$) a decámetros ($Dm$). \\ \fillin[$149\divisionsymbol 10 \divisionsymbol 10 \divisionsymbol 10 =0.194$][0in]
			\part  6.5 gramos ($g$) a hectogramos ($Hg$).	   \\ \fillin[$6.5\divisionsymbol 10 \divisionsymbol 10=0.065$][0in]
			\part  8674 centigramos ($cg$) a gramos ($g$).     \\ \fillin[$8674\divisionsymbol 10 \divisionsymbol 10=86.74$][0in]
			\part  90.4 miligramos ($mg$) a centigramos ($cg$). \\ \fillin[$90.4\divisionsymbol 10 =9.04$][0in]
			\part  2.9 decagramos ($Dg$) a miligramos ($mg$).  \\ \fillin[$2.9\times 10 \times 10 \times 10\times 10=29000$][0in]
			\part  9.01 gramos ($g$) a miligramos ($mg$).      \\ \fillin[$9.01\times 10 \times 10 \times 10=9010$][0in]
		\end{parts}
	\end{multicols}
	}

	\subsection{Unidades de capacidad}
	
	\questionboxed[5]{Convierte las siguientes unidades de capacidad como se te pide:
	
	\begin{multicols}{2}
	\begin{parts}
			\part 27   hectolitros ($HL$) a centilitros ($cL$).       \\ \fillin[$27  \times 10 \times 10 \times 10\times 10=270000$][0in] 
			\part 8    mililitros ($mL$) a centilitros ($cL$).        \\ \fillin[$8   \divisionsymbol 10 \divisionsymbol 10=0.08$][0in] 
			\part 1094 mililitros ($mL$) a decilitros ($dL$).         \\ \fillin[$1094\divisionsymbol 10 \divisionsymbol 10=10.94$][0in] 
			\part 702  mililitros ($mL$) a decalitros ($DL$).         \\ \fillin[$702 \divisionsymbol 10 \divisionsymbol 10\divisionsymbol 10 \divisionsymbol 10=0.0702$][0in] 
			\part 1.9   litros ($L$) a mililitros ($mL$).             \\ \fillin[$1.9  \times 10 \times 10 \times 10=19000$][0in] 
			\part 8200 litros ($L$) a metros cúbicos ($m^3$).		  \\ \fillin[$8200\divisionsymbol 1000=8.2$][0in]
			\part 4.8  decímetros cúbicos ($dm^3$) a litros ($L$).	  \\ \fillin[$4.8=4.8$][0in]
			\part 750  litros ($L$) a metros cúbicos ($m^3$).	 	  \\ \fillin[$750\divisionsymbol 1000=0.75$][0in]
			\part 567  milímetros cúbicos ($mm^3$) a litros ($L$).	  \\ \fillin[$567 \divisionsymbol 1000 \divisionsymbol 1000=0.000567$][0in]
			\part 4100 litros ($L$) a metros cúbicos ($m^3$).		  \\ \fillin[$4100\divisionsymbol 1000=4.1$][0in]
		\end{parts}
	\end{multicols}
	}

	\subsection{Unidades de área y volumen}
	
	\questionboxed[5]{Convierte las siguientes unidades de área y volumen como se te pide:

	\begin{parts}
			\part  8.8 metros cúbicos ($m^3$) a milímetros cúbicos ($mm^3$) \hfill \fillin[$8.8\times 1000 \times 1000 \times 1000=8800000000$][0in]
			\part  8 kilómetros cuadrados ($Km^2$) a metros cuadrados ($m^2$) \hfill \fillin[$8\times 100 \times 100=80000$][0in]
			\part  88 metros cuadrados ($m^2$) a kilómetros cuadrados ($Km^2$) \hfill \fillin[$88\divisionsymbol 100 \divisionsymbol 100\divisionsymbol 100=0.00088$][0in]
			\part  18 decámetros cúbicos ($Dm^3$) a centímetros cúbicos ($cm^3$)	\hfill \fillin[$18\times 1000 \times 1000 \times 1000=18000000000$][0in]
			\part  801 milímetros cuadrados ($mm^2$) a decámetros cuadrados ($Dm^2$)	\hfill \fillin[$801\divisionsymbol 100 \divisionsymbol 100\divisionsymbol 100 \divisionsymbol 100=0.000801$][0in]
		\end{parts}
	}
\end{questions}
\end{document}