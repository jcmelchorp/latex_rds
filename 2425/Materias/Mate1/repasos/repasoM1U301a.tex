\documentclass[12pt,addpoints]{repaso}
\grado{1}
\nivel{Secundaria}
\cicloescolar{2022-2023}
\materia{Matemáticas}
\unidad{3}
\title{Preparación para el examen de la Unidad}
\aprendizajes{
    \item Resuelve problemas mediante la formulación y solución algebraica de ecuaciones lineales.
    \item Analiza y compara situaciones de variación lineal a partir de sus representaciones tabular, gráfica y algebraica. Interpreta y resuelve problemas que se modelan con estos tipos de variación.
    \item Calcula valores faltantes en problemas de proporcionalidad directa, con constante natural, fracción o decimal (incluyendo tablas de variación).
    }
\author{Melchor Pinto, J.C.}
\begin{document}
\INFO%
\begin{questions}
    \questionboxed[10]{\include*{../questions/question077c}}
    \ejemplosboxed[\include*{../questions/question050e}]
    \questionboxed[10]{\include*{../questions/question050a}}
    \questionboxed[4]{\include*{../../Mate 3/questions/question028d}}
    \questionboxed[4]{\include*{../../Mate 3/questions/question028b}}
    \questionboxed[4]{\include*{../../Mate 3/questions/question028j}}
    \ejemplosboxed[\include*{../questions/question031a}]
    \questionboxed[20]{\include*{../questions/question031}}
    \ejemplosboxed[\include*{../questions/question075b}]
    \questionboxed[5]{\include*{../questions/question075c}}
    \ejemplosboxed[\include*{../questions/question076a}]
    \questionboxed[5]{\include*{../questions/question076b}}
    \questionboxed[4]{\include*{../questions/question076c}}
    \questionboxed[4]{\include*{../questions/question071}}
    \questionboxed[30]{\include*{../questions/question050f}}
\end{questions}
\end{document}