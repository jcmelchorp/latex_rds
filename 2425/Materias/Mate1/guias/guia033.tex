\documentclass[12pt,addpoints,answers]{guia}
\grado{1$^\circ$ de Secundaria}
\cicloescolar{2022-2023}
\materia{Matemáticas 1}
\guia{33}
\unidad{3}
\title{Análisis de variación lineal}
\aprendizajes{\item Analiza y compara situaciones de variación lineal a partir de sus representaciones tabular, gráfica y algebraica.}
\author{JC Melchor Pinto}
\begin{document}
\INFO%
\ejemplosboxed[\include*{../questions/question075a}]
\begin{questions}
    \questionboxed[10]{\include*{../questions/question075b}}
    \questionboxed[15]{\include*{../questions/question075c}}
    \questionboxed[15]{\include*{../questions/question074a}}
    \questionboxed[15]{\include*{../questions/question073a}}
    \questionboxed[15]{\include*{../questions/question076a}}
    \questionboxed[15]{\include*{../questions/question076b}}
    \questionboxed[15]{\include*{../questions/question076c}}
\end{questions}
\end{document}

