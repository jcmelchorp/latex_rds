\documentclass[12pt,addpoints]{evalua}
\grado{6$^\circ$ de Primaria}
\cicloescolar{2024-2025}
\materia{Matemáticas}
\unidad{}
\title{Examen General}
\aprendizajes{\tiny%
% \item Estudio de los números.
\item Expresa oralmente la sucesión numérica hasta billones, en español y hasta donde sea posible, en su lengua materna, de manera ascendente y descendente a partir de un número natural dado. Ordena, lee y escribe números naturales de más de nueve cifras e interpreta números decimales en diferentes contextos. Identifica semejanzas y diferencias entre el sistema de numeración decimal y otros sistemas como el maya y el romano
	% \item Suma y resta, su relación como operaciones inversas.
	\item A partir de situaciones problemáticas vinculadas a diferentes contextos, suma y resta números decimales y fracciones con diferentes denominadores.
	% \item Multiplicación y división, su relación como operaciones inversas.
	\item Resuelve situaciones problemáticas vinculadas a diferentes contextos que implican dividir números decimales entre naturales. También, dividir números fraccionarios entre números naturales.
	% \item Relaciones de proporcionalidad.
	\item A partir de situaciones problemáticas de proporcionalidad vinculadas a diferentes contextos, determina valores faltantes en las que en ocasiones se conoce el valor unitario y en otras no.
	% \item Ubicación espacial.
	\item Lee, interpreta y elabora planos para comunicar la ubicación de seres vivos y objetos.
	% \item Figuras y cuerpos geométricos y sus características.
	\item Explora y reconoce las características del cilindro y cono; anticipa y comprueba desarrollos planos que permiten construirlos.
	% \item Perímetro, área y noción de volumen.
	\item Resuelve situaciones problemáticas que implican calcular el perímetro y área de figuras compuestas por triángulos y cuadriláteros. Resuelve problemas que implican construir, estimar y comparar el volumen de cuerpos y prismas rectos rectangulares mediante el conteo de cubos, y reconoce que existen diferentes cuerpos con el mismo volumen.
	% \item Organización e interpretación de datos.
	\item Interpreta información cuantitativa y cualitativa contenida en tablas, gráficas de barras y circulares para responder preguntas vinculadas a diferentes contextos; construye gráficas de barras. Genera y organiza datos, determina la moda, la media aritmética y el rango para responder preguntas vinculadas a diferentes contextos.
	% \item Nociones de probabilidad.
	\item Clasifica eventos de diversos contextos utilizando términos como seguro, imposible, probable, muy probable o poco probable que sucedan.
}
\author{Prof.: Julio César Melchor Pinto}
\begin{document}
\tableofcontents
\newpage%
\begin{questions}
   \addcontentsline{toc}{section}{Unidad 1}
   \section*{Unidad 1}
   \question[10]
       % UNIDAD 1                  
   % \section*{\ifprintanswers{Introducción a las fracciones           }\else{}\fi}
% \subsection*{\ifprintanswers{Clasificación de fracciones             }\else{}\fi}
% \subsection*{\ifprintanswers{Representación de fracciones            }\else{}\fi}
% \subsection*{\ifprintanswers{Nombre de fracciones                    }\else{}\fi}
% \subsection*{\ifprintanswers{Fracciones en la recta numérica         }\else{}\fi}
% \subsection*{\ifprintanswers{Conversión de fracciones                }\else{}\fi}
   % \section*{\ifprintanswers{Multiplicaciones y Divisiones           }\else{}\fi}
% \subsection*{\ifprintanswers{Multiplicaciones 1                      }\else{}\fi}
% \subsection*{\ifprintanswers{Multiplicaciones 2                      }\else{}\fi}
% \subsection*{\ifprintanswers{Divisiones 1                            }\else{}\fi}
% \subsection*{\ifprintanswers{Divisiones 2                            }\else{}\fi}
% \subsection*{\ifprintanswers{Resolucion de problemas                 }\else{}\fi}
   % \section*{\ifprintanswers{Sumas y Restas                          }\else{}\fi}
% \subsection*{\ifprintanswers{Sumas 1                                 }\else{}\fi}
% \subsection*{\ifprintanswers{Sumas 2                                 }\else{}\fi}
% \subsection*{\ifprintanswers{Restas 1                                }\else{}\fi}
% \subsection*{\ifprintanswers{Restas 2                                }\else{}\fi}
% \subsection*{\ifprintanswers{Resolucion de problemas                 }\else{}\fi}
   % \section*{\ifprintanswers{Números decimales                       }\else{}\fi}
% \subsection*{\ifprintanswers{Posición decimal y notación desarrollada}\else{}\fi}
% \subsection*{\ifprintanswers{Nombre de decimales                     }\else{}\fi}
% \subsection*{\ifprintanswers{Decimales en la recta numérica          }\else{}\fi}
% \subsection*{\ifprintanswers{Comparación de decimales                }\else{}\fi}
% \subsection*{\ifprintanswers{Redondeo de decimales                   }\else{}\fi}

   % UNIDAD 2                  
   % \section*{\ifprintanswers{Operaciones con decimales               }\else{}\fi}
% \subsection*{\ifprintanswers{Suma de decimales                       }\else{}\fi}
% \subsection*{\ifprintanswers{Resta de decimales                      }\else{}\fi}
% \subsection*{\ifprintanswers{Multiplicación de decimales             }\else{}\fi}
% \subsection*{\ifprintanswers{División de decimales                   }\else{}\fi}
% \subsection*{\ifprintanswers{Resolución de problemas                 }\else{}\fi}
   % \section*{\ifprintanswers{Simplificación de fracciones            }\else{}\fi}
% \subsection*{\ifprintanswers{Comparación de fracciones               }\else{}\fi}
% \subsection*{\ifprintanswers{Fracciones equivalentes                 }\else{}\fi}
% \subsection*{\ifprintanswers{Mínimo Común Múltiplo                   }\else{}\fi}
% \subsection*{\ifprintanswers{Máxico Común Divisor                    }\else{}\fi}
% \subsection*{\ifprintanswers{Simplificación de fracciones            }\else{}\fi}
   % \section*{\ifprintanswers{Suma y resta de fracciones              }\else{}\fi}
% \subsection*{\ifprintanswers{Suma de decimales                       }\else{}\fi}
% \subsection*{\ifprintanswers{Resta de decimales                      }\else{}\fi}
% \subsection*{\ifprintanswers{Multiplicación de decimales             }\else{}\fi}
% \subsection*{\ifprintanswers{División de decimales                   }\else{}\fi}
% \subsection*{\ifprintanswers{Resolución de problemas                 }\else{}\fi}
   % \section*{\ifprintanswers{Multiplicación y divisioón de fracciones}\else{}\fi}
% \subsection*{\ifprintanswers{Suma y resta con denominadores iguales  }\else{}\fi}
% \subsection*{\ifprintanswers{Suma y resta denominadores diferentes   }\else{}\fi}
% \subsection*{\ifprintanswers{Multiplicación de fracciones            }\else{}\fi}
% \subsection*{\ifprintanswers{División de fracciones                  }\else{}\fi}
% \subsection*{\ifprintanswers{Resolución de problemas                 }\else{}\fi}
   % \section*{\ifprintanswers{Decimales y porcentajes                 }\else{}\fi}
% \subsection*{\ifprintanswers{Ubicación en la recta numérica          }\else{}\fi}
% \subsection*{\ifprintanswers{Porcentajes a decimal                   }\else{}\fi}
% \subsection*{\ifprintanswers{Operaciones con múltiplos de 10         }\else{}\fi}
% \subsection*{\ifprintanswers{Conversión de fracciones a decimales    }\else{}\fi}
% \subsection*{\ifprintanswers{Conversión de decimales a fracciones    }\else{}\fi}
                                                                                                                                                                                                                                                                                                                                                                                                                                                                                                          
   % UNIDAD 3                  
   % \section*{\ifprintanswers{Círculo                         }\else{}\fi}
% \subsection*{\ifprintanswers{Diámetro de un círculo          }\else{}\fi}
% \subsection*{\ifprintanswers{Radio de un círculo             }\else{}\fi}
% \subsection*{\ifprintanswers{Perímetro                       }\else{}\fi}
% \subsection*{\ifprintanswers{Área                            }\else{}\fi}
% \subsection*{\ifprintanswers{Resolución de problemas         }\else{}\fi}
   % \section*{\ifprintanswers{Cuerpos Geométricos             }\else{}\fi}
% \subsection*{\ifprintanswers{Nombre de cuerpos geométricos   }\else{}\fi}
% \subsection*{\ifprintanswers{Elementos de cuerpos geométricos}\else{}\fi}
% \subsection*{\ifprintanswers{Área lateral                    }\else{}\fi}
% \subsection*{\ifprintanswers{Área total                      }\else{}\fi}
% \subsection*{\ifprintanswers{Volumen                         }\else{}\fi}
   % \section*{\ifprintanswers{Figuras Geométricas             }\else{}\fi}
% \subsection*{\ifprintanswers{Nombre de figuras               }\else{}\fi}
% \subsection*{\ifprintanswers{Elementos de figuras            }\else{}\fi}
% \subsection*{\ifprintanswers{Perímetro                       }\else{}\fi}
% \subsection*{\ifprintanswers{Área                            }\else{}\fi}
% \subsection*{\ifprintanswers{Resolución de problemas         }\else{}\fi}
   % \section*{\ifprintanswers{Sistema de Unidades             }\else{}\fi}
% \subsection*{\ifprintanswers{Operaciones con múltiplos de 10 }\else{}\fi}
% \subsection*{\ifprintanswers{Unidades de longitud            }\else{}\fi}
% \subsection*{\ifprintanswers{Unidades de masa                }\else{}\fi}
% \subsection*{\ifprintanswers{Unidades de capacidad           }\else{}\fi}
% \subsection*{\ifprintanswers{Unidades de área y volumen      }\else{}\fi}
\end{questions}
\end{document}