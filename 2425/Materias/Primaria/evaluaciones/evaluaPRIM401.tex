\documentclass[12pt,addpoints]{evalua}
\grado{1$^\circ$ de Secundaria}
\cicloescolar{2024-2025}
\materia{Matemáticas 1 \normalfont \color{darkgray} \small con adecuación curricular a \\Matemáticas 4$^\circ$ de Primaria.}
\unidad{1, 2 y 3}
\title{Examen de la Unidad}
\aprendizajes{\tiny%
   \item Expresa oralmente la sucesión numérica hasta cuatro cifras, en español y hasta donde sea posible, en su lengua materna, de manera ascendente y descendente a partir de un número natural dado.
   \item Representa, con apoyo de material concreto y modelos gráficos, fracciones: medios, cuartos, octavos, dieciseisavos, para expresar el resultado de mediciones y repartos en situaciones vinculadas a su contexto.
   \item Resuelve situaciones problemáticas vinculadas a su contexto que implican sumas o restas de números naturales de hasta cuatro cifras utilizando los algoritmos convencionales y números decimales hasta centésimos, con apoyo de material concreto y representaciones gráficas.
   \item Resuelve situaciones problemáticas que implican sumas o restas de fracciones con diferente denominador (tercios, quintos, sextos, novenos y décimos) vinculados a su contexto, mediante diversos procedimientos, en particular, la equivalencia.
   \item Resuelve situaciones problemáticas vinculadas a su contexto que implican multiplicaciones de números naturales de hasta tres por dos cifras, a partir de diversas descomposiciones aditivas y el algoritmo convencional y el uso de un algoritmo para dividir números naturales de hasta tres cifras entre un número de una o dos cifras; reconoce al cociente y al residuo como resultado de una división.
}
\author{Prof.: Julio César Melchor Pinto}
\begin{document}
\begin{questions}
	\question[10]
	% UNIDAD 1                  
	% \section*{\ifprintanswers{Escritura de cantidades                        }\else{}\fi}
	% \subsection*{\ifprintanswers{Escritura de cantidades 1                      }\else{}\fi}
	% \subsection*{\ifprintanswers{Escritura de cantidades 2                      }\else{}\fi}
	% \subsection*{\ifprintanswers{Escritura de cantidades 3                      }\else{}\fi}
	% \subsection*{\ifprintanswers{Escritura de cantidades 4                      }\else{}\fi}
	% \subsection*{\ifprintanswers{Escritura de cantidades 5                      }\else{}\fi}
	% \section*{\ifprintanswers{Números romanos                                }\else{}\fi}
	% \subsection*{\ifprintanswers{Números romanos 1                              }\else{}\fi}
	% \subsection*{\ifprintanswers{Números romanos 2                              }\else{}\fi}
	% \subsection*{\ifprintanswers{Números romanos 3                              }\else{}\fi}
	% \subsection*{\ifprintanswers{Números romanos 4                              }\else{}\fi}
	% \subsection*{\ifprintanswers{Números romanos 5                              }\else{}\fi}
	% \section*{\ifprintanswers{Sistema decimal                                }\else{}\fi}
	% \subsection*{\ifprintanswers{Posicionamiento decimal 1                      }\else{}\fi}
	% \subsection*{\ifprintanswers{Posicionamiento decimal 2                      }\else{}\fi}
	% \subsection*{\ifprintanswers{Notación desarrollada 1                        }\else{}\fi}
	% \subsection*{\ifprintanswers{Notación desarrollada 2                        }\else{}\fi}
	% \subsection*{\ifprintanswers{Posicionamiento decimal y Notación desarrollada}\else{}\fi}
	% \section*{\ifprintanswers{Tablas de multiplicar                          }\else{}\fi}
	% \subsection*{\ifprintanswers{Tabla del 1 y 2                                }\else{}\fi}
	% \subsection*{\ifprintanswers{Tabla del 3 y 4                                }\else{}\fi}
	% \subsection*{\ifprintanswers{Tabla del 5 y 6                                }\else{}\fi}
	% \subsection*{\ifprintanswers{Tabla del 7 y 8                                }\else{}\fi}
	% \subsection*{\ifprintanswers{Tabla del 9 y 10                               }\else{}\fi}

	% UNIDAD 2                  
	% \section*{\ifprintanswers{Números decimales            }\else{}\fi}
	% \subsection*{\ifprintanswers{Posicionamiento decimal      }\else{}\fi}
	% \subsection*{\ifprintanswers{Notación desarrollada        }\else{}\fi}
	% \subsection*{\ifprintanswers{Nombre de decimales          }\else{}\fi}
	% \subsection*{\ifprintanswers{Suma de decimales            }\else{}\fi}
	% \subsection*{\ifprintanswers{Resta de decimales           }\else{}\fi}
	% \section*{\ifprintanswers{Sumas                        }\else{}\fi}
	% \subsection*{\ifprintanswers{Sumas sin acarreos 1         }\else{}\fi}
	% \subsection*{\ifprintanswers{Sumas sin acarreos 2         }\else{}\fi}
	% \subsection*{\ifprintanswers{Sumas con acarreos 1         }\else{}\fi}
	% \subsection*{\ifprintanswers{Sumas con acarreos 2         }\else{}\fi}
	% \subsection*{\ifprintanswers{Repaso de sumas              }\else{}\fi}
	% \section*{\ifprintanswers{Restas                       }\else{}\fi}
	% \subsection*{\ifprintanswers{Restas con múltiplos de 100  }\else{}\fi}
	% \subsection*{\ifprintanswers{Restas con transformación 1  }\else{}\fi}
	% \subsection*{\ifprintanswers{Restas con transformación 2  }\else{}\fi}
	% \subsection*{\ifprintanswers{Restas con ceros intermedios }\else{}\fi}
	% \subsection*{\ifprintanswers{Restas con transformación 3  }\else{}\fi}
	% \section*{\ifprintanswers{Multiplicaciones             }\else{}\fi}
	% \subsection*{\ifprintanswers{Multiplicaciones 1           }\else{}\fi}
	% \subsection*{\ifprintanswers{Multiplicaciones 2           }\else{}\fi}
	% \subsection*{\ifprintanswers{Multiplicaciones con 2 cifras}\else{}\fi}
	% \subsection*{\ifprintanswers{Multiplicaciones con 3 cifras}\else{}\fi}
	% \subsection*{\ifprintanswers{Multiplicaciones con 4 cifras}\else{}\fi}
	% \section*{\ifprintanswers{Divisiones                   }\else{}\fi}
	% \subsection*{\ifprintanswers{Divisiones sin residuo 1     }\else{}\fi}
	% \subsection*{\ifprintanswers{Divisiones sin residuo 2     }\else{}\fi}
	% \subsection*{\ifprintanswers{Divisiones sin residuo 3     }\else{}\fi}
	% \subsection*{\ifprintanswers{Divisiones con residuo 1     }\else{}\fi}
	% \subsection*{\ifprintanswers{Divisiones con residuo 2     }\else{}\fi}           

	% UNIDAD 3                  
	% \section*{\ifprintanswers{Introducción a fracciones                  }\else{}\fi}
	% \subsection*{\ifprintanswers{Clasificación de fracciones                }\else{}\fi}
	% \subsection*{\ifprintanswers{Representación de fracciones               }\else{}\fi}
	% \subsection*{\ifprintanswers{Nombre de fracciones                       }\else{}\fi}
	% \subsection*{\ifprintanswers{Conversión de fracciones mixtas a impropias}\else{}\fi}
	% \subsection*{\ifprintanswers{Conversión de fracciones impropias a mixtas}\else{}\fi}
	% \section*{\ifprintanswers{Operaciones con fracciones                 }\else{}\fi}
	% \subsection*{\ifprintanswers{Suma de fracciones                         }\else{}\fi}
	% \subsection*{\ifprintanswers{Resta de fracciones                        }\else{}\fi}
	% \subsection*{\ifprintanswers{Multiplicación de fracciones               }\else{}\fi}
	% \subsection*{\ifprintanswers{División de fracciones                     }\else{}\fi}
	% \subsection*{\ifprintanswers{Operaciones de fracciones mixtas           }\else{}\fi}
	% \section*{\ifprintanswers{Figuras geométricas                        }\else{}\fi}
	% \subsection*{\ifprintanswers{Nombre de figuras                          }\else{}\fi}
	% \subsection*{\ifprintanswers{Elementos de figuras                       }\else{}\fi}
	% \subsection*{\ifprintanswers{Perímetros 1                               }\else{}\fi}
	% \subsection*{\ifprintanswers{Perímetros 2                               }\else{}\fi}
	% \subsection*{\ifprintanswers{Área de figuras                            }\else{}\fi}
	% \section*{\ifprintanswers{Sistema de unidades                        }\else{}\fi}
	% \subsection*{\ifprintanswers{Reloj                                      }\else{}\fi}
	% \subsection*{\ifprintanswers{Multiplicaciones por múltiplos de 10       }\else{}\fi}
	% \subsection*{\ifprintanswers{Unidades de tiempo                         }\else{}\fi}
	% \subsection*{\ifprintanswers{Unidades de longitud                       }\else{}\fi}
	% \subsection*{\ifprintanswers{Unidades de masa                           }\else{}\fi}
\end{questions}
\end{document}