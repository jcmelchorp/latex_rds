\documentclass[12pt,addpoints]{repaso}
\grado{3}
\nivel{Primaria}
\cicloescolar{2024-2025}
\materia{Matemáticas}
\unidad{2}
\title{Practica la Unidad}
\aprendizajes{\scriptsize%
\item Expresa oralmente la sucesión numérica hasta cuatro cifras, en español y hasta donde sea posible, en su lengua materna, de manera ascendente y descendente a partir de un número natural dado.\\[-1.8em]
\item Representa, con apoyo de material concreto y modelos gráficos, fracciones: medios, cuartos, octavos, dieciseisavos, para expresar el resultado de mediciones y repartos en situaciones vinculadas a su contexto.\\[-1.8em]
\item Resuelve situaciones problemáticas vinculadas a su contexto que implican sumas, restas, multiplicación y división de números naturales de hasta tres cifras utilizando el algoritmo convencional y que impliquen, medición, estimación y comparación, de longitudes, masas y capacidades, con el uso del metro, kilogramo, litro y medios y cuartos de estas unidades; en el caso de la longitud, el decímetro y centímetro.\\[-1.8em]
\item Resuelve problemas de suma, resta, multiplicación y división vinculados a su contexto, que impliquen el uso de fracciones (medios, cuartos, octavos, dieciseisavos), con el apoyo de material concreto o representaciones gráficas.
   }
\author{Melchor Pinto, JC}
\begin{document}
\INFO
\begin{questions}
	% UNIDAD 2

	% \section*{\ifprintanswers{Tablas de multiplicar 2        }\else{}\fi}
	% \subsection*{\ifprintanswers{Tabla del 6                    }\else{}\fi}
	% \subsection*{\ifprintanswers{Tabla del 7                    }\else{}\fi}
	% \subsection*{\ifprintanswers{Tabla del 8                    }\else{}\fi}
	% \subsection*{\ifprintanswers{Tabla del 9                    }\else{}\fi}
	% \subsection*{\ifprintanswers{Tabla del 10                   }\else{}\fi}

	\questionboxed[8]{Reponde las siguientes tablas de multiplicar:

		\begin{multicols}{4}
			\begin{parts}\Large
				\part $5 \times 9=$ \fillin[$45$][0cm]
				\part $5 \times 6=$ \fillin[$30$][0cm]
				\part $6 \times 8=$ \fillin[$48$][0cm]
				\part $6 \times 9=$ \fillin[$54$][0cm]
				\part $3 \times 6=$ \fillin[$18$][0cm]
				\part $2 \times 7=$ \fillin[$14$][0cm]
				\part $4 \times 7=$ \fillin[$28$][0cm]
				\part $3 \times 8=$ \fillin[$24$][0cm]
				\part $2 \times 9=$ \fillin[$18$][0cm]
				\part $4 \times 4=$ \fillin[$16$][0cm]
				\part $7 \times 7=$ \fillin[$49$][0cm]
				\part $7 \times 5=$ \fillin[$35$][0cm]
				\part $5 \times 4=$ \fillin[$20$][0cm]
				\part $8 \times 7=$ \fillin[$56$][0cm]
				\part $7 \times 6=$ \fillin[$42$][0cm]
				\part $9 \times 7=$ \fillin[$63$][0cm]
			\end{parts}
		\end{multicols}
	}

	\questionboxed[8]{Completa las siguientes tablas de multiplicar:

		\begin{multicols}{4}
			\begin{parts}\Large
				\part $\fillin[6][0.5cm] \times 6= 36$
				\part $\fillin[8][0.5cm] \times 8= 64$
				\part $\fillin[7][0.5cm] \times 8= 56$
				\part $5 \times \fillin[10][0.5cm]=50$
				\part $4 \times \fillin[8][0.5cm]=32$
				\part $8 \times \fillin[5][0.5cm]= 40$
				\part $\fillin[6][0.5cm] \times 4= 24$
				\part $7 \times \fillin[7][0.5cm]= 49$
				\part $\fillin[8][0.5cm] \times 3= 24$
				\part $9 \times \fillin[8][0.5cm]= 72$
				\part $\fillin[9][0.5cm] \times 5= 45$
				\part $6 \times \fillin[7][0.5cm]= 42$
				\part $\fillin[9][0.5cm] \times 9= 81$
				\part $4 \times \fillin[9][0.5cm]= 36$
				\part $\fillin[7][0.5cm] \times 4= 28$
				\part $\fillin[9][0.5cm] \times 3= 21$
			\end{parts}
		\end{multicols}
	}

	% \section*{\ifprintanswers{Sumas 1                        }\else{}\fi}
	% \subsection*{\ifprintanswers{Sumando con 1, 2 y 3           }\else{}\fi}
	% \subsection*{\ifprintanswers{Sumando con 4, 5 y 6           }\else{}\fi}
	% \subsection*{\ifprintanswers{Sumando con 7, 8 y 9           }\else{}\fi}
	% \subsection*{\ifprintanswers{Sumando números entre 10 y 20  }\else{}\fi}
	% \subsection*{\ifprintanswers{Sumando números entre 20 y 50  }\else{}\fi}

	% \section*{\ifprintanswers{Sumas 2                        }\else{}\fi}
	% \subsection*{\ifprintanswers{Sumas hasta el 100             }\else{}\fi}
	% \subsection*{\ifprintanswers{Sumas hasta el 500             }\else{}\fi}
	% \subsection*{\ifprintanswers{Sumas con acarreos 1           }\else{}\fi}
	% \subsection*{\ifprintanswers{Sumas con acarreos 2           }\else{}\fi}
	% \subsection*{\ifprintanswers{Sumas con acarreos 3           }\else{}\fi}

	\questionboxed[8]{Realiza las siguientes sumas:

		\begin{multicols}{4}
			\begin{parts}
				\part \ifprintanswers{\large  \quad   \opadd[hfactor=decimal,resultstyle=\color{red},carryadd=true,carrysub=false]{37854}{18581} }
				\else{          \large  \quad  \opadd[hfactor=decimal,resultstyle=\color{white},carryadd=false,carrysub=false]{37854}{18581} }
				\fi
				\part \ifprintanswers{\large  \quad   \opadd[hfactor=decimal,resultstyle=\color{red},carryadd=true,carrysub=false]{3234}{24156} }
				\else{          \large  \quad  \opadd[hfactor=decimal,resultstyle=\color{white},carryadd=false,carrysub=false]{3234}{24156} }
				\fi
				\part \ifprintanswers{\large  \quad   \opadd[hfactor=decimal,resultstyle=\color{red},carryadd=true,carrysub=false]{30985}{19562} }
				\else{          \large  \quad  \opadd[hfactor=decimal,resultstyle=\color{white},carryadd=false,carrysub=false]{30985}{19562} }
				\fi
				\part \ifprintanswers{\large  \quad   \opadd[hfactor=decimal,resultstyle=\color{red},carryadd=true,carrysub=false]{2849}{2415} }
				\else{          \large  \quad  \opadd[hfactor=decimal,resultstyle=\color{white},carryadd=false,carrysub=false]{2849}{2415} }
				\fi
				\part \ifprintanswers{\large  \quad   \opadd[hfactor=decimal,resultstyle=\color{red},carryadd=true,carrysub=false]{31085}{19001} }
				\else{          \large  \quad  \opadd[hfactor=decimal,resultstyle=\color{white},carryadd=false,carrysub=false]{31085}{19001} }
				\fi
				\part \ifprintanswers{\large  \quad   \opadd[hfactor=decimal,resultstyle=\color{red},carryadd=true,carrysub=false]{35701}{25484} }
				\else{          \large  \quad  \opadd[hfactor=decimal,resultstyle=\color{white},carryadd=false,carrysub=false]{35701}{25484} }
				\fi
				\part \ifprintanswers{\large  \quad   \opadd[hfactor=decimal,resultstyle=\color{red},carryadd=true,carrysub=false]{45668}{19624} }
				\else{          \large  \quad  \opadd[hfactor=decimal,resultstyle=\color{white},carryadd=false,carrysub=false]{45668}{19624} }
				\fi
				\part \ifprintanswers{\large  \quad   \opadd[hfactor=decimal,resultstyle=\color{red},carryadd=true,carrysub=false]{58718}{3652} }
				\else{          \large  \quad  \opadd[hfactor=decimal,resultstyle=\color{white},carryadd=false,carrysub=false]{58718}{3652} }
				\fi
			\end{parts}
		\end{multicols}
	}

	% \section*{\ifprintanswers{Restas 1                       }\else{}\fi}
	% \subsection*{\ifprintanswers{Restando con 1, 2 y 3          }\else{}\fi}
	% \subsection*{\ifprintanswers{Restando con 4, 5, 6, 7, 8 y 9 }\else{}\fi}
	% \subsection*{\ifprintanswers{Restas como sumas 1            }\else{}\fi}
	% \subsection*{\ifprintanswers{Restas como suma 2             }\else{}\fi}
	% \subsection*{\ifprintanswers{Restas como suma 3             }\else{}\fi}


	% \section*{\ifprintanswers{Restas 2                       }\else{}\fi}
	% \subsection*{\ifprintanswers{Restas sin transformación 1    }\else{}\fi}
	% \subsection*{\ifprintanswers{Restas sin transformación 2    }\else{}\fi}
	% \subsection*{\ifprintanswers{Restas con transformación 1    }\else{}\fi}
	% \subsection*{\ifprintanswers{Restas con transformación 2    }\else{}\fi}
	% \subsection*{\ifprintanswers{Restas con transformación 3    }\else{}\fi}

	\questionboxed[8]{Realiza las siguientes restas:

		\begin{multicols}{4}
			\begin{parts}
				\part \ifprintanswers{\large  \quad   \opsub[hfactor=decimal,resultstyle=\color{red},carryadd=true,carrysub=true]{4000}{2267} }
				\else{          \large  \quad  \opsub[hfactor=decimal,resultstyle=\color{white},carryadd=false,carrysub=false]{4000}{2267} }
				\fi
				\part \ifprintanswers{\large  \quad   \opsub[hfactor=decimal,resultstyle=\color{red},carryadd=true,carrysub=true]{800}{744} }
				\else{          \large  \quad  \opsub[hfactor=decimal,resultstyle=\color{white},carryadd=false,carrysub=false]{800}{744} }
				\fi
				\part \ifprintanswers{\large  \quad   \opsub[hfactor=decimal,resultstyle=\color{red},carryadd=true,carrysub=true]{3500}{308} }
				\else{          \large  \quad  \opsub[hfactor=decimal,resultstyle=\color{white},carryadd=false,carrysub=false]{3500}{308} }
				\fi
				\part \ifprintanswers{\large  \quad   \opsub[hfactor=decimal,resultstyle=\color{red},carryadd=true,carrysub=true]{3000}{189} }
				\else{          \large  \quad  \opsub[hfactor=decimal,resultstyle=\color{white},carryadd=false,carrysub=false]{3000}{189} }
				\fi
				\part \ifprintanswers{\large  \quad   \opsub[hfactor=decimal,resultstyle=\color{red},carryadd=true,carrysub=true]{1200}{966} }
				\else{          \large  \quad  \opsub[hfactor=decimal,resultstyle=\color{white},carryadd=false,carrysub=false]{1200}{966} }
				\fi
				\part \ifprintanswers{\large  \quad   \opsub[hfactor=decimal,resultstyle=\color{red},carryadd=true,carrysub=true]{3300}{2117} }
				\else{          \large  \quad  \opsub[hfactor=decimal,resultstyle=\color{white},carryadd=false,carrysub=false]{3300}{2117} }
				\fi
				\part \ifprintanswers{\large  \quad   \opsub[hfactor=decimal,resultstyle=\color{red},carryadd=true,carrysub=true]{2000}{1251} }
				\else{          \large  \quad  \opsub[hfactor=decimal,resultstyle=\color{white},carryadd=false,carrysub=false]{2000}{1251} }
				\fi
				\part \ifprintanswers{\large  \quad   \opsub[hfactor=decimal,resultstyle=\color{red},carryadd=true,carrysub=true]{2400}{2023} }
				\else{          \large  \quad  \opsub[hfactor=decimal,resultstyle=\color{white},carryadd=false,carrysub=false]{2400}{2023} }
				\fi
				% \part \ifprintanswers{\large  \quad   \opsub[hfactor=decimal,resultstyle=\color{red},carryadd=true,carrysub=false]{2400}{211} }
				% \else{          \large  \quad  \opsub[hfactor=decimal,resultstyle=\color{white},carryadd=false,carrysub=false]{2400}{211} }
				% \fi
				% \part \ifprintanswers{\large  \quad   \opsub[hfactor=decimal,resultstyle=\color{red},carryadd=true,carrysub=false]{1500}{1044} }
				% \else{          \large  \quad  \opsub[hfactor=decimal,resultstyle=\color{white},carryadd=false,carrysub=false]{1500}{1044} }
				% \fi
				% \part \ifprintanswers{\large  \quad   \opsub[hfactor=decimal,resultstyle=\color{red},carryadd=true,carrysub=false]{2000}{1105} }
				% \else{          \large  \quad  \opsub[hfactor=decimal,resultstyle=\color{white},carryadd=false,carrysub=false]{2000}{1105} }
				% \fi
				% \part \ifprintanswers{\large  \quad   \opsub[hfactor=decimal,resultstyle=\color{red},carryadd=true,carrysub=false]{1600}{669} }
				% \else{          \large  \quad  \opsub[hfactor=decimal,resultstyle=\color{white},carryadd=false,carrysub=false]{1600}{669} }
				% \fi
			\end{parts}
		\end{multicols}
	}

	

	% \section*{\ifprintanswers{Restas 3                                   }\else{}\fi}
	% \subsection*{\ifprintanswers{Restas con transformación 1                }\else{}\fi}
	% \subsection*{\ifprintanswers{Restas con transformación 2                }\else{}\fi}
	% \subsection*{\ifprintanswers{Minuendos múltiplos de 10                  }\else{}\fi}
	% \subsection*{\ifprintanswers{Minuendos con ceros intermedios            }\else{}\fi}
	% \subsection*{\ifprintanswers{Repaso de restas                           }\else{}\fi}
\end{questions}
\end{document}