\documentclass[12pt,addpoints]{repaso}
\grado{3}
\nivel{Primaria}
\cicloescolar{2024-2025}
\materia{Matemáticas}
\unidad{1}
\title{Practica la Unidad}
\aprendizajes{\scriptsize%
\item Expresa oralmente la sucesión numérica hasta cuatro cifras, en español y hasta donde sea posible, en su lengua materna, de manera ascendente y descendente a partir de un número natural dado.\\[-1.8em]
\item Representa, con apoyo de material concreto y modelos gráficos, fracciones: medios, cuartos, octavos, dieciseisavos, para expresar el resultado de mediciones y repartos en situaciones vinculadas a su contexto.\\[-1.8em]
\item Resuelve situaciones problemáticas vinculadas a su contexto que implican sumas, restas, multiplicación y división de números naturales de hasta tres cifras utilizando el algoritmo convencional y que impliquen, medición, estimación y comparación, de longitudes, masas y capacidades, con el uso del metro, kilogramo, litro y medios y cuartos de estas unidades; en el caso de la longitud, el decímetro y centímetro.\\[-1.8em]
\item Resuelve problemas de suma, resta, multiplicación y división vinculados a su contexto, que impliquen el uso de fracciones (medios, cuartos, octavos, dieciseisavos), con el apoyo de material concreto o representaciones gráficas.
   }
\author{Melchor Pinto, JC}
\begin{document}
\INFO
\begin{questions}
	% UNIDAD 1                  
	% \section*{\ifprintanswers{Escritura de cantidades}\else{}\fi}
	% \subsection*{\ifprintanswers{Escritura de cantidades 1 }\else{}\fi}
	% \subsection*{\ifprintanswers{Escritura de cantidades 2 }\else{}\fi}
	% \subsection*{\ifprintanswers{Escritura de cantidades 3 }\else{}\fi}
	% \subsection*{\ifprintanswers{Escritura de cantidades 4 }\else{}\fi}
	% \subsection*{\ifprintanswers{Escritura de cantidades 5 }\else{}\fi}

	\questionboxed[4]{Escribe sore la línea los siguientes números

		\begin{multicols}{2}
			\begin{parts}\large
				\part \fillin[  65][1.5cm] Sesenta y cinco.
				\part \fillin[ 109][1.5cm] Ciento nueve.
				\part \fillin[ 254][1.5cm] Doscientos cincuenta y cuatro.
				\part \fillin[ 314][1.5cm] Trescientos catorce.
				\part \fillin[ 431][1.5cm] Cuatrocientos treinta y uno.
				\part \fillin[1024][1.5cm] Mil veinticuatro.s
				\part \fillin[1849][1.5cm] Mil ochocientos cuarenta y nueve.
				% \part \fillin[1310][1.1cm] Mil trescientos diez.                      
				\part \fillin[ 703][1.5cm] Setecientos tres.
			\end{parts}
		\end{multicols}
	}
	% \section*{\ifprintanswers{Sistema decimal 1}\else{}\fi}

	% \subsection*{\ifprintanswers{Notación desarrollada 1 }\else{}\fi}
	% \subsection*{\ifprintanswers{Notación desarrollada 2 }\else{}\fi}
	% \subsection*{\ifprintanswers{Notación desarrollada 3 }\else{}\fi}


	\questionboxed[4]{Escribe la notación desarrollada de cada uno de los siguientes números:

		\begin{multicols}{2}
			\begin{parts}\Large
				\part $15984=$ \fillin[$10000+5000+900+80+4$][2.4in]
				\part $4936 =$ \fillin[$4000+900+30+6$][2.4in]
				\part $27545=$ \fillin[$20000+7000+500+40+5$][2.4in]
				\part $6215 =$ \fillin[$6000+200+10+5$][2.4in]
				\part $5454 =$ \fillin[$5000+400+50+4$][2.4in]
				\part $6451 =$ \fillin[$6000+400+50+1$][2.4in]
				\part $19679=$ \fillin[$10000+9000+600+70+9$][2.4in]
				\part $26324=$ \fillin[$20000+6000+300+20+4$][2.4in]
				\part $5717 =$ \fillin[$5000+700+10+7$][2.4in]
				\part $31126=$ \fillin[$30000+1000+100+20+6$][2.4in]
				\part $4818 =$ \fillin[$4000+800+10+8$][2.4in]
				\part $7145 =$ \fillin[$7000+100+40+5$][2.4in]
			\end{parts}
		\end{multicols}
	}
	% \subsection*{\ifprintanswers{Posicionamiento decimal 1 }\else{}\fi}
	% \subsection*{\ifprintanswers{Posicionamiento decimal 2 }\else{}\fi}

	\questionboxed[4]{Señala la opción que responda correctamente a cada una de las siguientes preguntas:

		\begin{multicols}{2}
			\begin{parts}\large
				\part ¿Qué lugar ocupa el 6 en 6418?     \fillin[C][0.5cm]
				\part ¿Qué lugar ocupa el 2 en 206418?   \fillin[A][0.5cm]
				\part ¿Qué lugar ocupa el 2 en 87264?    \fillin[D][0.5cm]
				\part ¿Qué lugar ocupa el 1 en 1681?     \fillin[F][0.5cm]
				\part ¿Qué lugar ocupa el 1 en 6138?     \fillin[D][0.5cm]
				\part ¿Qué lugar ocupa el 8 en 198114?   \fillin[C][0.5cm]
				\part ¿Qué lugar ocupa el 7 en 46878?    \fillin[E][0.5cm]
				\part ¿Qué lugar ocupa el 4 en 149778?   \fillin[B][0.5cm]
			\end{parts}

			\columnbreak%

			\begin{choices}\Large
				\choice {\color{red}centenas de millar.}
				\choice {\color{blue}decenas de millar.}
				\choice {\color{Goldenrod}unidades de millar.}
				\choice {\color{red}centenas.}
				\choice {\color{blue}decenas.}
				\choice {\color{Goldenrod}unidades.}
			\end{choices}
		\end{multicols}
	}



	% \section*{\ifprintanswers{Sistema decimal 2}\else{}\fi}
	% \subsection*{\ifprintanswers{Posicionamiento decimal 1 }\else{}\fi}
	% \subsection*{\ifprintanswers{Posicionamiento decimal 2 }\else{}\fi}

	\questionboxed[3]{Señala la opción que responda correctamente a cada una de las siguientes preguntas:

		\begin{multicols}{2}
			\begin{parts}\large
				\part En el número 1.829, ¿qué número ocupa la posición de las centésimas?

				\begin{oneparcheckboxes}
					\choice 1 \CorrectChoice 2 \choice 6 \choice 8 \choice 9
				\end{oneparcheckboxes}

				\part En el número 2.087, ¿qué número ocupa la posición de las décimas?

				\begin{oneparcheckboxes}
					\CorrectChoice 0 \choice 2 \choice 7 \choice 8 \choice 9
				\end{oneparcheckboxes}

				\part En el número 5.928, ¿qué número ocupa la posición de las décimas?

				\begin{oneparcheckboxes}
					\choice 5 \choice 2 \choice 6 \choice 8 \CorrectChoice 9
				\end{oneparcheckboxes}

				\part En el número 3.284, ¿qué número ocupa la posición de las milésimas?

				\begin{oneparcheckboxes}
					\choice 2 \choice 3 \CorrectChoice 4  \choice 8 \choice 9
				\end{oneparcheckboxes}

				\part En el número 1.285, ¿qué número ocupa la posición de las décimas?

				\begin{oneparcheckboxes}
					\choice 1 \CorrectChoice 2 \choice 5 \choice 8 \choice 9
				\end{oneparcheckboxes}

				\part En el número 1.823, ¿qué número ocupa la posición de las milésimas?

				\begin{oneparcheckboxes}
					\choice 1 \choice 2 \CorrectChoice 3 \choice 6 \choice 8
				\end{oneparcheckboxes}
			\end{parts}
		\end{multicols}
	}

	\questionboxed[6]{Escribe los siguientes números

		\begin{multicols}{2}
			\begin{parts}\large
				\part Veinticinco enteros ocho décimas                  \\ \hfill \fillin[$25.8$][1cm]
				\part Seis enteros ciento veintiocho milésimas          \\ \hfill \fillin[$6.128$][1cm]
				\part Catorce enteros veintinueve centésimas            \\ \hfill \fillin[$14.29$][1cm]
				\part Cuarenta enteros dos décimas                      \\ \hfill \fillin[$40.2$][1cm]
				\part Tres enteros cincuenta y ocho centésimas          \\ \hfill \fillin[$3.58$][1cm]
				\part Cuatro enteros sesenta y nueve milésimas          \\ \hfill \fillin[$4.069$][1cm]
				\part Siete enteros cuatro décimas                      \\ \hfill \fillin[$ 7.4$][1cm]
				% \part Dos enteros siete décimas                         \\ \hfill \fillin[$2.7$][1cm]
				% \part Cuatro enteros ocho milésimas                     \\ \hfill \fillin[$4.008$][1cm]
				% \part Siete enteros setenta y siete centésimas          \\ \hfill \fillin[$7.77$][1cm]
				% \part Once enteros ochenta y nueve centésimas           \\ \hfill \fillin[$11.89$][1cm]
				\part Treinta y ocho enteros nueve décimas              \\ \hfill \fillin[$38.9$][1cm]
			\end{parts}
		\end{multicols}
	}

	% \subsection*{\ifprintanswers{Notación desarrollada 1 }\else{}\fi}
	% \subsection*{\ifprintanswers{Notación desarrollada 2  }\else{}\fi}
	% \subsection*{\ifprintanswers{Posicionamiento decimal y Notación desarrollada }\else{}\fi}

	\questionboxed[4]{Señala la opción que responda correctamente a cada una de las siguientes preguntas:

		\begin{multicols}{2}
			\begin{parts}\large
				\part En el número 3658, ¿qué número ocupa la posición de las decenas?

				\begin{oneparcheckboxes}
					\choice 3 \CorrectChoice 5 \choice 6 \choice 8 \choice 9
				\end{oneparcheckboxes}

				\part En el número 17542, ¿qué número ocupa la posición de las unidades de millar?

				\begin{oneparcheckboxes}
					\choice 1 \CorrectChoice 7 \choice 5 \choice 4 \choice 2
				\end{oneparcheckboxes}

				\part En el número 5984, ¿qué número ocupa la posición de las centenas?

				\begin{oneparcheckboxes}
					\choice 4 \choice 2 \choice 5 \choice 8 \CorrectChoice 9
				\end{oneparcheckboxes}

				\part En el número 7841, ¿qué número ocupa la posición de las decenas?

				\begin{oneparcheckboxes}
					\choice 1 \choice 7 \choice 8 \CorrectChoice 4 \choice 2
				\end{oneparcheckboxes}

				\part En el número 3918, ¿qué número ocupa la posición de las centenas?

				\begin{oneparcheckboxes}
					\choice 3 \choice 1 \choice 6 \choice 8 \CorrectChoice 9
				\end{oneparcheckboxes}

				\part En el número 3621, ¿qué número ocupa la posición de las decenas?

				\begin{oneparcheckboxes}
					\CorrectChoice 2 \choice 3 \choice 6 \choice 8 \choice 1
				\end{oneparcheckboxes}

				\part En el número 51362, ¿qué número ocupa la posición de las decenas de millar?

				\begin{oneparcheckboxes}
					\choice 3 \CorrectChoice 5 \choice 6 \choice 1 \choice 2
				\end{oneparcheckboxes}

				\part En el número 7584, ¿qué número ocupa la posición de las decenas?

				\begin{oneparcheckboxes}
					\choice 3 \choice 5 \choice 7 \CorrectChoice 8 \choice 4
				\end{oneparcheckboxes}

				% \part En el número 9654, ¿qué número ocupa la posición de las centenas?

				% \begin{oneparcheckboxes}
				%    \choice 3 \choice 5 \CorrectChoice 6 \choice 4 \choice 9
				% \end{oneparcheckboxes}

				% \part En el número 240679, ¿qué número ocupa la posición de las centenas de millar?

				% \begin{oneparcheckboxes}
				%    \choice 0 \choice 6 \CorrectChoice 2 \choice 7 \choice 9 \choice 4
				% \end{oneparcheckboxes}
				% \part En el número 41589, ¿qué número ocupa la posición de las decenas de millar?
				% \part En el número 8459, ¿qué número ocupa la posición de las centenas?
				% \part En el número 10562, ¿qué número ocupa la posición de las centenas?
				% \part En el número 24781, ¿qué número ocupa la posición de las decenas de millar?
				% \part En el número 7856, ¿qué número ocupa la posición de las decenas?
			\end{parts}
		\end{multicols}
	}

	% \section*{\ifprintanswers{Tablas de multiplicar 1}\else{}\fi}
	% \subsection*{\ifprintanswers{Tabla del 1  }\else{}\fi}
	% \subsection*{\ifprintanswers{Tabla del 2 }\else{}\fi}
	% \subsection*{\ifprintanswers{Tabla del 3 }\else{}\fi}
	% \subsection*{\ifprintanswers{Tabla del 4 }\else{}\fi}
	% \subsection*{\ifprintanswers{Tabla del 5 }\else{}\fi}

	
\end{questions}
\end{document}