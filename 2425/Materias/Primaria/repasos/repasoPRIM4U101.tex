\documentclass[12pt,addpoints]{repaso}
\grado{4}
\nivel{Primaria}
\cicloescolar{2024-2025}
\materia{Matemáticas}
\unidad{1}
\title{Practica la Unidad}
\aprendizajes{\tiny
      \item Expresa oralmente la sucesión numérica hasta cuatro cifras, en español y hasta donde sea posible, en su lengua materna, de manera ascendente y descendente a partir de un número natural dado.
      \item Representa, con apoyo de material concreto y modelos gráficos, fracciones: medios, cuartos, octavos, dieciseisavos, para expresar el resultado de mediciones y repartos en situaciones vinculadas a su contexto.
      \item Resuelve situaciones problemáticas vinculadas a su contexto que implican sumas o restas de números naturales de hasta cuatro cifras utilizando los algoritmos convencionales y números decimales hasta centésimos, con apoyo de material concreto y representaciones gráficas.
	  \item Resuelve situaciones problemáticas que implican sumas o restas de fracciones con diferente denominador (tercios, quintos, sextos, novenos y décimos) vinculados a su contexto, mediante diversos procedimientos, en particular, la equivalencia.
	  \item Resuelve situaciones problemáticas vinculadas a su contexto que implican multiplicaciones de números naturales de hasta tres por dos cifras, a partir de diversas descomposiciones aditivas y el algoritmo convencional y el uso de un algoritmo para dividir números naturales de hasta tres cifras entre un número de una o dos cifras; reconoce al cociente y al residuo como resultado de una división.
	  }
\author{Melchor Pinto, JC}
\begin{document}
\INFO
\begin{questions}
	\questionboxed[2]{Escribe sore la línea los siguientes números

		\begin{multicols}{2}
			\begin{parts}\large
				\part  \fillin[$14005$][1.1cm] Catorce mil cinco.
				\part  \fillin[$11524$][1.1cm] Once mil quinientos veinticuatro.
				\part  \fillin[$13642$][1.1cm] Trece mil seiscientos cuarenta y dos.
				\part  \fillin[$10189$][1.1cm] Diez mil ciento ochenta y nueve.
				\part  \fillin[$13990$][1.1cm] Trece mil novecientos noventa.
				\part  \fillin[$11300$][1.1cm] Once mil trescientos.
				\part  \fillin[$14400$][1.1cm] Catorce mil cuatrocientos.
				\part  \fillin[$12881$][1.1cm] Doce mil ochocientos ochenta y uno.
				\part  \fillin[$10711$][1.1cm] Diez mil setecientos once.
				\part  \fillin[$11740$][1.1cm] Once mil setecientos cuarenta.
				\part  \fillin[$10298$][1.1cm] Diez mil doscientos noventa y ocho.
				\part  \fillin[$13422$][1.1cm] Trece mil cuatrocientos veintidos.
			\end{parts}
		\end{multicols}
	}
	% \section*{\ifprintanswers{Números romanos                                }\else{}\fi}
	% \subsection*{\ifprintanswers{Números romanos 1                              }\else{}\fi}
	% \subsection*{\ifprintanswers{Números romanos 2                              }\else{}\fi}
	% \subsection*{\ifprintanswers{Números romanos 3                              }\else{}\fi}
	% \subsection*{\ifprintanswers{Números romanos 4                              }\else{}\fi}
	% \subsection*{\ifprintanswers{Números romanos 5                              }\else{}\fi}
	\questionboxed[2]{Escribe el valor de los siguientes números romanos

		\begin{multicols}{4}
			\begin{parts}\large
				\part \fillin[$16$][1cm] XVI
				\part \fillin[$482$][1cm] CDLXXXII
				\part \fillin[$18$][1cm] XVIII
				\part \fillin[$98$][1cm] XCVIII
				\part \fillin[$64$][1cm] LXIV
				\part \fillin[$199$][1cm] CXCIX
				\part \fillin[$36$][1cm] XXXVI
				\part \fillin[$42$][1cm] XLII
				\part \fillin[$37$][1cm] XXXVII
				\part \fillin[$63$][1cm] LXIII
				\part \fillin[$29$][1cm] XXIX
				\part \fillin[$34$][1cm] XXXIV
			\end{parts}
		\end{multicols}
	}

	\questionboxed[2]{Escribe en números romanos los siguientes números

		\begin{multicols}{4}
			\begin{parts}\large
				\part 38  \fillin[XXXVIII][2.5cm]
				\part 150 \fillin[CL][2.5cm]
				\part 82  \fillin[LXXXII][2.5cm]
				\part 199 \fillin[CXCIX][2.5cm]
				\part 46  \fillin[XLVI][2.5cm]
				\part 98  \fillin[XCVIII][2.5cm]
				\part 482 \fillin[CDLXXXII][2.5cm]
				\part 28  \fillin[XXVIII][2.5cm]
				\part 45  \fillin[XLV][2.5cm]
				\part 94  \fillin[XCIV][2.5cm]
				\part 308 \fillin[CCCVIII][2.5cm]
				\part 40  \fillin[XL][2.5cm]
			\end{parts}
		\end{multicols}
	}

	% \section*{\ifprintanswers{Sistema decimal                                }\else{}\fi}
	% \subsection*{\ifprintanswers{Posicionamiento decimal 1                      }\else{}\fi}
	% \subsection*{\ifprintanswers{Posicionamiento decimal 2                      }\else{}\fi}

	\questionboxed[2]{Señala la opción que responda correctamente a cada una de las siguientes preguntas:

		\begin{multicols}{2}
			\begin{parts}\large
				\part ¿Qué lugar ocupa el 6 en 6418?     \fillin[C][0.5cm]
				\part ¿Qué lugar ocupa el 2 en 206418?   \fillin[A][0.5cm]
				\part ¿Qué lugar ocupa el 2 en 87264?    \fillin[D][0.5cm]
				\part ¿Qué lugar ocupa el 1 en 1681?     \fillin[F][0.5cm]
				\part ¿Qué lugar ocupa el 1 en 6138?     \fillin[D][0.5cm]
				\part ¿Qué lugar ocupa el 8 en 198114?   \fillin[C][0.5cm]
				\part ¿Qué lugar ocupa el 7 en 46878?    \fillin[E][0.5cm]
				\part ¿Qué lugar ocupa el 4 en 149778?   \fillin[B][0.5cm]
			\end{parts}

			\columnbreak%

			\begin{choices}\Large
				\choice {\color{red}centenas de millar.}
				\choice {\color{blue}decenas de millar.}
				\choice {\color{Goldenrod}unidades de millar.}
				\choice {\color{red}centenas.}
				\choice {\color{blue}decenas.}
				\choice {\color{Goldenrod}unidades.}
			\end{choices}
		\end{multicols}
	}

	% \subsection*{\ifprintanswers{Notación desarrollada 1                        }\else{}\fi}
	% \subsection*{\ifprintanswers{Notación desarrollada 2                        }\else{}\fi}


	\questionboxed[4]{Escribe la notación desarrollada de cada uno de los siguientes números:

		\begin{multicols}{2}
			\begin{parts}\Large
				\part $15984=$ \fillin[$10000+5000+900+80+4$][2.4in]
				\part $4936 =$ \fillin[$4000+900+30+6$][2.4in]
				\part $27545=$ \fillin[$20000+7000+500+40+5$][2.4in]
				\part $6215 =$ \fillin[$6000+200+10+5$][2.4in]
				\part $5454 =$ \fillin[$5000+400+50+4$][2.4in]
				\part $6451 =$ \fillin[$6000+400+50+1$][2.4in]
				\part $19679=$ \fillin[$10000+9000+600+70+9$][2.4in]
				\part $26324=$ \fillin[$20000+6000+300+20+4$][2.4in]
				\part $5717 =$ \fillin[$5000+700+10+7$][2.4in]
				\part $31126=$ \fillin[$30000+1000+100+20+6$][2.4in]
				\part $4818 =$ \fillin[$4000+800+10+8$][2.4in]
				\part $7145 =$ \fillin[$7000+100+40+5$][2.4in]
			\end{parts}
		\end{multicols}
	}

	% \subsection*{\ifprintanswers{Posicionamiento decimal y Notación desarrollada}\else{}\fi}

	\questionboxed[2]{Señala la opción que responda correctamente a cada una de las siguientes preguntas:

		\begin{multicols}{2}
			\begin{parts}\large
				\part En el número 3658, ¿qué número ocupa la posición de las decenas?

				\begin{oneparcheckboxes}
					\choice 3 \CorrectChoice 5 \choice 6 \choice 8 \choice 9
				\end{oneparcheckboxes}

				\part En el número 17542, ¿qué número ocupa la posición de las unidades de millar?

				\begin{oneparcheckboxes}
					\choice 1 \CorrectChoice 7 \choice 5 \choice 4 \choice 2
				\end{oneparcheckboxes}

				\part En el número 5984, ¿qué número ocupa la posición de las centenas?

				\begin{oneparcheckboxes}
					\choice 4 \choice 2 \choice 5 \choice 8 \CorrectChoice 9
				\end{oneparcheckboxes}

				\part En el número 7841, ¿qué número ocupa la posición de las decenas?

				\begin{oneparcheckboxes}
					\choice 1 \choice 7 \choice 8 \CorrectChoice 4 \choice 2
				\end{oneparcheckboxes}

				\part En el número 3918, ¿qué número ocupa la posición de las centenas?

				\begin{oneparcheckboxes}
					\choice 3 \choice 1 \choice 6 \choice 8 \CorrectChoice 9
				\end{oneparcheckboxes}

				\part En el número 3621, ¿qué número ocupa la posición de las decenas?

				\begin{oneparcheckboxes}
					\CorrectChoice 2 \choice 3 \choice 6 \choice 8 \choice 1
				\end{oneparcheckboxes}

				\part En el número 51362, ¿qué número ocupa la posición de las decenas de millar?

				\begin{oneparcheckboxes}
					\choice 3 \CorrectChoice 5 \choice 6 \choice 1 \choice 2
				\end{oneparcheckboxes}

				\part En el número 7584, ¿qué número ocupa la posición de las decenas?

				\begin{oneparcheckboxes}
					\choice 3 \choice 5 \choice 7 \CorrectChoice 8 \choice 4
				\end{oneparcheckboxes}

				\part En el número 9654, ¿qué número ocupa la posición de las centenas?

				\begin{oneparcheckboxes}
					\choice 3 \choice 5 \CorrectChoice 6 \choice 4 \choice 9
				\end{oneparcheckboxes}

				\part En el número 240679, ¿qué número ocupa la posición de las centenas de millar?

				\begin{oneparcheckboxes}
					\choice 0 \choice 6 \CorrectChoice 2 \choice 7 \choice 9 \choice 4
				\end{oneparcheckboxes}
				% \part En el número 41589, ¿qué número ocupa la posición de las decenas de millar?
				% \part En el número 8459, ¿qué número ocupa la posición de las centenas?
				% \part En el número 10562, ¿qué número ocupa la posición de las centenas?
				% \part En el número 24781, ¿qué número ocupa la posición de las decenas de millar?
				% \part En el número 7856, ¿qué número ocupa la posición de las decenas?
			\end{parts}
		\end{multicols}
	}


	% \section*{\ifprintanswers{Tablas de multiplicar                          }\else{}\fi}
	% \subsection*{\ifprintanswers{Tabla del 1 y 2                                }\else{}\fi}
	% \subsection*{\ifprintanswers{Tabla del 3 y 4                                }\else{}\fi}
	% \subsection*{\ifprintanswers{Tabla del 5 y 6                                }\else{}\fi}
	% \subsection*{\ifprintanswers{Tabla del 7 y 8                                }\else{}\fi}
	% \subsection*{\ifprintanswers{Tabla del 9 y 10                               }\else{}\fi}
	\questionboxed[3]{Reponde las siguientes tablas de multiplicar:

		\begin{multicols}{4}
			\begin{parts}\Large
				\part $5 \times 9=$ \fillin[$45$][0cm]
				\part $5 \times 6=$ \fillin[$30$][0cm]
				\part $6 \times 8=$ \fillin[$48$][0cm]
				\part $6 \times 9=$ \fillin[$54$][0cm]
				\part $3 \times 6=$ \fillin[$18$][0cm]
				\part $2 \times 7=$ \fillin[$14$][0cm]
				\part $4 \times 7=$ \fillin[$28$][0cm]
				\part $3 \times 8=$ \fillin[$24$][0cm]
				\part $2 \times 9=$ \fillin[$18$][0cm]
				\part $4 \times 4=$ \fillin[$16$][0cm]
				\part $7 \times 7=$ \fillin[$49$][0cm]
				\part $7 \times 5=$ \fillin[$35$][0cm]
				\part $5 \times 4=$ \fillin[$20$][0cm]
				\part $8 \times 7=$ \fillin[$56$][0cm]
				\part $7 \times 6=$ \fillin[$42$][0cm]
				\part $9 \times 7=$ \fillin[$63$][0cm]
			\end{parts}
		\end{multicols}
	}

	\questionboxed[3]{Completa las siguientes tablas de multiplicar:

		\begin{multicols}{4}
			\begin{parts}\Large
				\part $\fillin[6][0.5cm] \times 6= 36$
				\part $\fillin[8][0.5cm] \times 8= 64$
				\part $\fillin[7][0.5cm] \times 8= 56$
				\part $5 \times \fillin[10][0.5cm]=50$
				\part $4 \times \fillin[8][0.5cm]=32$
				\part $8 \times \fillin[5][0.5cm]= 40$
				\part $\fillin[6][0.5cm] \times 4= 24$
				\part $7 \times \fillin[7][0.5cm]= 49$
				\part $\fillin[8][0.5cm] \times 3= 24$
				\part $9 \times \fillin[8][0.5cm]= 72$
				\part $\fillin[9][0.5cm] \times 5= 45$
				\part $6 \times \fillin[7][0.5cm]= 42$
				\part $\fillin[9][0.5cm] \times 9= 81$
				\part $4 \times \fillin[9][0.5cm]= 36$
				\part $\fillin[7][0.5cm] \times 4= 28$
				\part $\fillin[9][0.5cm] \times 3= 21$
			\end{parts}
		\end{multicols}
	}
\end{questions}
\end{document}