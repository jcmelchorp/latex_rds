\documentclass[12pt,addpoints]{repaso}
\grado{5}
\nivel{Primaria}
\cicloescolar{2024-2025}
\materia{Matemáticas}
\unidad{1}
\title{Practica la Unidad}
\aprendizajes{\scriptsize%
	% \item Estudio de los números.
\item Ordena, lee, escribe e identifica regularidades en números naturales de hasta nueve cifras. Lee, escribe y ordena números decimales hasta diezmilésimos en notación decimal y letra, y los interpreta en diferentes contextos.
	% \item Suma y resta, su relación como operaciones inversas.
	\item Propone y resuelve situaciones problemáticas que implican sumas y restas con números decimales utilizando el algoritmo convencional y fracciones con diferentes denominadores. 
	% \item Multiplicación y división, su relación como operaciones inversas.
	\item Resuelve situaciones problemáticas vinculadas a diferentes contextos que implican multiplicar números fraccionarios y números decimales, con un número natural como multiplicador.También, dividir números naturales y el cociente resulte un número decimal.
	% \item Relaciones de proporcionalidad.
	\item Resuelve situaciones problemáticas de proporcionalidad en las que determina valores faltantes de números naturales, a partir de diferentes estrategias (cálculo del valor unitario, de dobles, triples o mitades).
	% \item Ubicación espacial.
	\item Elabora e interpreta croquis para comunicar la ubicación de seres vivos, objetos, trayectos o lugares.
	% \item Medición de la longitud, masa y capacidad.
	% \item Figuras y cuerpos geométricos y sus características.
	\item Reconoce y describe semejanzas y diferencias entre un prisma y una pirámide; propone desarrollos planos para construir prismas rectos cuadrangulares o rectangulares.
	% \item Perímetro, área y noción de volumen.
	\item Calcula el perímetro y área de diferentes polígonos. Construye y usa fórmulas para calcular el perímetro de cualquier polígono, a partir de sumar la longitud de todos sus lados o multiplicar el número de lados por la medida de uno de ellos.
	% \item Organización e interpretación de datos.
	\item Construye tablas y gráficas de barras, e interpreta información cuantitativa y cualitativa contenida en ellas.
	% \item Nociones de probabilidad. 
	\item Identifica situaciones de distintos contextos en las que interviene o no el azar; registra resultados de experiencias aleatorias en tablas de frecuencias y expresa la frecuencia absoluta y la relativa.
	  }
\author{Melchor Pinto, JC}
\begin{document}
\INFO
\begin{multicols}{2}
	\tableofcontents
\end{multicols}
\begin{questions}\large
	\addcontentsline{toc}{section}{Unidad 1}
	\section*{Unidad 1}
	\addcontentsline{toc}{subsection}{Números romanos}
	\subsection*{Números romanos}

	\questionboxed[1]{Escribe el valor de los siguientes números romanos

		\begin{multicols}{3}
			\begin{parts}
				\part \fillin[$36$][1cm] XXXVI
				\part \fillin[$42$][1cm] XLII
				\part \fillin[$63$][1cm] LXIII
				\part \fillin[$29$][1cm] XXIX
				\part \fillin[$482$][1cm] CDLXXXII
				\part \fillin[$544$][1cm] DXLIV
				\part \fillin[$671$][1cm] DCLXXI
				\part \fillin[$199$][1cm] CXCIX
				\part \fillin[$2916$][1cm] MMCMXVI
				\part \fillin[$1085$][1cm] MLXXXV
				\part \fillin[$1144$][1cm] MCXLIV
				\part \fillin[$2127$][1cm] MMCXXVII
			\end{parts}
		\end{multicols}
	}

	\questionboxed[1]{Escribe en números romanos los siguientes números

		\begin{multicols}{4}
			\begin{parts}
				\part 38   \hfill \fillin[XXXVIII][2cm]
				\part 150  \hfill \fillin[CL][2cm]
				\part 795  \hfill  \fillin[DCCXCV][2cm]
				\part 199  \hfill \fillin[CXCIX][2cm]
				\part 46   \hfill \fillin[XLVI][2cm]
				\part 98   \hfill \fillin[XCVIII][2cm]
				\part 482  \hfill \fillin[CDLXXXII][2cm]
				\part 2091 \hfill   \fillin[MMXCI][2cm]
				\part 897  \hfill  \fillin[DCCCXCVII][2cm]
				\part 94   \hfill \fillin[XCIV][2cm]
				\part 308  \hfill \fillin[CCCVIII][2cm]
				\part 649  \hfill  \fillin[DCXLIX][2cm]
			\end{parts}
		\end{multicols}
	}

	\addcontentsline{toc}{subsection}{Sumas y restas}
	\subsection*{Sumas y restas}

	\questionboxed[1]{Realiza las siguientes sumas y restas:

		\begin{multicols}{4}
			\begin{parts}
				\part
				\ifprintanswers{\opadd[hfactor=decimal,resultstyle=\color{red},carryadd=true]{17}{18}}
				\else{\opadd[hfactor=decimal,resultstyle=\color{white},carryadd=false]{17}{18}\\[0.5cm]}\fi

				\part
				\ifprintanswers{\opadd[hfactor=decimal,resultstyle=\color{red},carryadd=true]{1155}{893}}
				\else{\opadd[hfactor=decimal,resultstyle=\color{white},carryadd=false]{1155}{893}\\[0.5cm]}\fi

				\part
				\ifprintanswers{\opadd[hfactor=decimal,resultstyle=\color{red},carryadd=true]{26}{19}}
				\else{\opadd[hfactor=decimal,resultstyle=\color{white},carryadd=false]{26}{19}\\[0.5cm]}\fi

				\part
				\ifprintanswers{\opadd[hfactor=decimal,resultstyle=\color{red},carryadd=true]{2271}{1028}}
				\else{\opadd[hfactor=decimal,resultstyle=\color{white},carryadd=false]{2271}{1028}\\[0.5cm]}\fi

				\part
				\ifprintanswers{\opadd[hfactor=decimal,resultstyle=\color{red},carryadd=true]{182}{149}}
				\else{\opadd[hfactor=decimal,resultstyle=\color{white},carryadd=false]{182}{149}\\[0.5cm]}\fi

				\part
				\ifprintanswers{\opadd[hfactor=decimal,resultstyle=\color{red},carryadd=true]{7449}{4358}}
				\else{\opadd[hfactor=decimal,resultstyle=\color{white},carryadd=false]{7449}{4358}\\[0.5cm]}\fi

				\part \ifprintanswers{   \opsub[hfactor=decimal,resultstyle=\color{red},carryadd=true,carrysub=true]{706}{589} }
				\else{            \opsub[hfactor=decimal,resultstyle=\color{white},carryadd=false,carrysub=false]{706}{589}\\[0.5cm] }
				\fi

				\part \ifprintanswers{   \opsub[hfactor=decimal,resultstyle=\color{red},carryadd=true,carrysub=true]{3004}{1242} }
				\else{            \opsub[hfactor=decimal,resultstyle=\color{white},carryadd=false,carrysub=false]{3004}{1242}\\[0.5cm] }
				\fi

				\part \ifprintanswers{   \opsub[hfactor=decimal,resultstyle=\color{red},carryadd=true,carrysub=true]{1600}{669} }
				\else{            \opsub[hfactor=decimal,resultstyle=\color{white},carryadd=false,carrysub=false]{1600}{669} \\[0.5cm]}
				\fi

				\part \ifprintanswers{   \opsub[hfactor=decimal,resultstyle=\color{red},carryadd=true,carrysub=true]{4005}{2831} }
				\else{            \opsub[hfactor=decimal,resultstyle=\color{white},carryadd=false,carrysub=false]{4005}{2831}\\[0.5cm] }
				\fi

				\part \ifprintanswers{   \opsub[hfactor=decimal,resultstyle=\color{red},carryadd=true,carrysub=true]{1200}{966} }
				\else{            \opsub[hfactor=decimal,resultstyle=\color{white},carryadd=false,carrysub=false]{1200}{966} \\[0.5cm]}
				\fi

				% \part \ifprintanswers{   \opsub[hfactor=decimal,resultstyle=\color{red},carryadd=true,carrysub=true]{42784}{34180} }
				% \else{            \opsub[hfactor=decimal,resultstyle=\color{white},carryadd=false,carrysub=false]{42784}{34180} \\[0.5cm]}
				% \fi

				\part \ifprintanswers{   \opsub[hfactor=decimal,resultstyle=\color{red},carryadd=true,carrysub=true]{800}{744} }
				\else{            \opsub[hfactor=decimal,resultstyle=\color{white},carryadd=false,carrysub=false]{800}{744} \\[0.5cm]}
				\fi

				% \part \ifprintanswers{   \opsub[hfactor=decimal,resultstyle=\color{red},carryadd=true,carrysub=true]{37881}{24049} }
				% \else{            \opsub[hfactor=decimal,resultstyle=\color{white},carryadd=false,carrysub=false]{37881}{24049}\\[0.5cm] }
				% \fi
			\end{parts}
		\end{multicols}
	}

	% \subsection*{\ifprintanswers{Resolución de problemas 
	\questionboxed[2]{Resuelve los siguientes problemas sobre sumas y restas:

		\begin{multicols}{2}
			\begin{parts}
				\part El total de mis compras es de 315 pesos, ¿cuánto dinero recibiré de cambio si pago con un billete de 500 pesos?

				\begin{solutionbox}{1cm}
					\opsub[style=text]{500}{315}
				\end{solutionbox}

				\part Luis tiene ahorrado 257 pesos, si su abuelo le regala 360 pesos más, ¿cuánto dinero tiene en total Luis?

				\begin{solutionbox}{1cm}
					\opadd[style=text]{257}{360}
				\end{solutionbox}

				\part Jorge está armando un rompecabezas de 500 piezas, si ha puesto 233 piezas, ¿cuántas piezas le faltan por poner a Jorge?

				\begin{solutionbox}{1cm}
					\opsub[style=text]{500}{233}
				\end{solutionbox}

				\part Carlos mide 183 centímetros y es 8 centímetros más alto que Julio, ¿cuántos centímetros mide Julio?

				\begin{solutionbox}{1cm}
					\opsub[style=text]{183}{8}
				\end{solutionbox}
			\end{parts}
		\end{multicols}
	}

	\addcontentsline{toc}{subsection}{Multiplicación}
	\subsection*{Multiplicación}
	% \subsection*{\ifprintanswers{Tablas de multiplicar                  }
	% \subsection*{\ifprintanswers{Multiplicaciones 1                     }
	% \subsection*{\ifprintanswers{Multiplicaciones 2                     }
	% \subsection*{\ifprintanswers{Multiplicaciones 3                     }
	% \subsection*{\ifprintanswers{Resolución de problemas 
	\questionboxed[2]{Reponde las siguientes tablas de multiplicar:

		\begin{multicols}{4}
			\begin{parts}
				\part $ 5 \times 9=$ \fillin[45][0.5cm]
				\part $4 \times \fillin[$8$][0.5cm]=32$
				\part $6 \times 8=$ \fillin[48][0.5cm]
				\part $8 \times \fillin[$5$][0.5cm]= 40$
				\part $7 \times 6=$ \fillin[42][0.5cm]
				\part $\fillin[$6$][0.5cm] \times 4= 24$
				\part $9 \times 7=$ \fillin[63][0.5cm]
				\part $7 \times \fillin[$7$][0.5cm]= 49$
				\part $6 \times 9=$ \fillin[54][0.5cm]
				\part $\fillin[$0$][0.5cm] \times 8= 0$
				\part $5 \times 6=$ \fillin[30][0.5cm]
				\part $9 \times \fillin[$8$][0.5cm]= 72$
				\part $4 \times 7=$ \fillin[28][0.5cm]
				\part $\fillin[$9$][0.5cm] \times 1= 9$
				\part $3 \times 8=$ \fillin[24][0.5cm]
				\part $6 \times \fillin[$7$][0.5cm]= 42$
			\end{parts}
		\end{multicols}
	}



	\questionboxed[2]{Realiza las siguientes multiplicaciones:

		\begin{multicols}{3}
			\begin{parts}
				\part \ifprintanswers{\normalsize\opmul[hfactor=decimal,resultstyle=\color{red},displayintermediary=None]{314}{2} }
				\else{\opmul[hfactor=decimal,resultstyle=\color{white},displayintermediary=None]{314}{2}}\\[2em]\fi

				\part \ifprintanswers{\normalsize\opmul[hfactor=decimal,resultstyle=\color{red},displayintermediary=all]{283}{44} }
				\else{\opmul[hfactor=decimal,resultstyle=\color{white},displayintermediary=None]{283}{44}}\\[2em]\fi

				\part \ifprintanswers{\normalsize\opmul[hfactor=decimal,resultstyle=\color{red},displayintermediary=None]{2781}{5} }
				\else{\opmul[hfactor=decimal,resultstyle=\color{white},displayintermediary=None]{2781}{5}}\\[2em]\fi

				\part \ifprintanswers{\normalsize\opmul[hfactor=decimal,resultstyle=\color{red},displayintermediary=all]{3914}{106} }
				\else{\opmul[hfactor=decimal,resultstyle=\color{white},displayintermediary=None]{3914}{106}}\\[2em]\fi

				\part \ifprintanswers{\normalsize\opmul[hfactor=decimal,resultstyle=\color{red},displayintermediary=all]{255}{24} }
				\else{\opmul[hfactor=decimal,resultstyle=\color{white},displayintermediary=None]{255}{24}}\\[2em]\fi

				\part \ifprintanswers{\normalsize\opmul[hfactor=decimal,resultstyle=\color{red},displayintermediary=all]{3533}{29} }
				\else{\opmul[hfactor=decimal,resultstyle=\color{white},displayintermediary=None]{3533}{29}}\\[2em]\fi
			\end{parts}
		\end{multicols}
	}

	\questionboxed[2]{Resuelve los siguientes problemas sobre multiplicaciones:

		\begin{multicols}{2}
			\begin{parts}
				\part Una escuela tiene 6 salones, si cada salón tiene 25 alumnos. ¿Cuántos alumnos tiene en total la escuela?

				\begin{solutionbox}{1cm}
					\opmul[style=text]{6}{25}
				\end{solutionbox}

				\part Una cubeta de pintura cuesta 2345 pesos, ¿cuánto se pagará por 3 cubetas de pintura?

				\begin{solutionbox}{1cm}
					\opmul[style=text]{3}{2345}
				\end{solutionbox}

				\part Una secretaria puede escribir 36 palabras por minuto si continua con este ritmo, ¿cuántas palabras puede escribir en 12 minutos?

				\begin{solutionbox}{1cm}
					\opmul[style=text]{36}{12}
				\end{solutionbox}

				\part Cristina compró 5 cajas de leche de soya, si cada caja tiene 12 envases de leche, ¿cuántos envases de leche compró Cristina?

				\begin{solutionbox}{1cm}
					\opmul[style=text]{5}{12}
				\end{solutionbox}

				\part Mariana fue a la frutería y compró 3 kilogramos de uvas, si el kilogramo cuesta 84 pesos. ¿Cuánto pagó en total Mariana?

				\begin{solutionbox}{1cm}
					\opmul[style=text]{3}{84}
				\end{solutionbox}

				\part Laura compró 28 paquetes de galletas, si cada paquete tiene 18 galletas. ¿Cuántas galletas tiene en total Laura?

				\begin{solutionbox}{1cm}
					\opmul[style=text]{28}{18}
				\end{solutionbox}

			\end{parts}
		\end{multicols}
	}

	\addcontentsline{toc}{subsection}{División}
	\subsection*{División}

	\questionboxed[2]{Calcula el {\color{red}cociente} y {\color{blue} residuo} de las siguientes divisiones de números enteros:

		\begin{multicols}{4}
			\begin{parts}
				\part \ifprintanswers{\opidiv[resultstyle=\color{red},remainderstyle.1=\color{blue!100!white}]{23}{6}} \\[2em]
				\else{           $6 \overline{) \ 23\ }$} \\[4em]
				\fi

				\part \ifprintanswers{\opidiv[resultstyle=\color{red},remainderstyle.2=\color{blue!100!white}]{200}{3}} \\[2em]
				\else{           $3 \overline{) \ 200\ }$} \\[4em]
				\fi

				\part \ifprintanswers{\opidiv[resultstyle=\color{red},remainderstyle.2=\color{blue!100!white}]{99}{8}} \\[2em]
				\else{           $8 \overline{) \ 99\ }$} \\[4em]
				\fi

				\part \ifprintanswers{\opidiv[resultstyle=\color{red},remainderstyle.2=\color{blue!100!white}]{283}{6}} \\[2em]
				\else{           $6 \overline{) \ 283\ }$} \\[4em]
				\fi

				\part \ifprintanswers{\opidiv[resultstyle=\color{red},remainderstyle.3=\color{blue!100!white}]{4032}{8}} \\[2em]
				\else{           $8 \overline{) \ 4032\ }$} \\[4em]
				\fi

				\part \ifprintanswers{\opidiv[resultstyle=\color{red},remainderstyle.2=\color{blue!100!white}]{644}{8}} \\[2em]
				\else{           $8 \overline{) \ 644\ }$} \\[4em]
				\fi

				\part \ifprintanswers{\opidiv[resultstyle=\color{red},remainderstyle.2=\color{blue!100!white}]{656}{7}} \\[2em]
				\else{           $7 \overline{) \ 656\ }$} \\[4em]
				\fi

				\part \ifprintanswers{\opidiv[resultstyle=\color{red},remainderstyle.3=\color{blue!100!white}]{2303}{7}} \\[2em]
				\else{           $7 \overline{) \ 2303\ }$} \\[4em]
				\fi
			\end{parts}
		\end{multicols}
	}


	% \subsection*{\ifprintanswers{Tablas de multiplicar                  }
	% \subsection*{\ifprintanswers{Multiplicaciones 1                     }
	% \subsection*{\ifprintanswers{Multiplicaciones 2                     }
	% \subsection*{\ifprintanswers{Multiplicaciones 3                     }
	% \subsection*{\ifprintanswers{Resolución de problemas                }

	\addcontentsline{toc}{subsection}{Sistema decimal}
	\subsection*{Sistema decimal}
	% \subsection*{\ifprintanswers{Posicionamiento decimal                }


	\questionboxed[2]{Señala la opción que responda correctamente a cada una de las siguientes preguntas:

		\begin{multicols}{2}
			\begin{parts}
				\part En el número 3658, ¿qué número ocupa la posición de las decenas?

				\begin{oneparcheckboxes}
					\choice 3 \CorrectChoice 5 \choice 6 \choice 8 \choice 9
				\end{oneparcheckboxes}

				\part En el número 17542, ¿qué número ocupa la posición de las unidades de millar?

				\begin{oneparcheckboxes}
					\choice 1 \CorrectChoice 7 \choice 5 \choice 4 \choice 2
				\end{oneparcheckboxes}

				\part En el número 5984, ¿qué número ocupa la posición de las centenas?

				\begin{oneparcheckboxes}
					\choice 4 \choice 2 \choice 5 \choice 8 \CorrectChoice 9
				\end{oneparcheckboxes}

				\part En el número 7841, ¿qué número ocupa la posición de las decenas?

				\begin{oneparcheckboxes}
					\choice 1 \choice 7 \choice 8 \CorrectChoice 4 \choice 2
				\end{oneparcheckboxes}

				\part En el número 3918, ¿qué número ocupa la posición de las centenas?

				\begin{oneparcheckboxes}
					\choice 3 \choice 1 \choice 6 \choice 8 \CorrectChoice 9
				\end{oneparcheckboxes}

				\part En el número 3621, ¿qué número ocupa la posición de las decenas?

				\begin{oneparcheckboxes}
					\CorrectChoice 2 \choice 3 \choice 6 \choice 8 \choice 1
				\end{oneparcheckboxes}

				\part En el número 51362, ¿qué número ocupa la posición de las decenas de millar?

				\begin{oneparcheckboxes}
					\choice 3 \CorrectChoice 5 \choice 6 \choice 1 \choice 2
				\end{oneparcheckboxes}

				\part En el número 7584, ¿qué número ocupa la posición de las decenas?

				\begin{oneparcheckboxes}
					\choice 3 \choice 5 \choice 7 \CorrectChoice 8 \choice 4
				\end{oneparcheckboxes}

				\part En el número 9654, ¿qué número ocupa la posición de las centenas?

				\begin{oneparcheckboxes}
					\choice 3 \choice 5 \CorrectChoice 6 \choice 4 \choice 9
				\end{oneparcheckboxes}

				\part En el número 240679, ¿qué número ocupa la posición de las centenas de millar?

				\begin{oneparcheckboxes}
					\choice 6 \CorrectChoice 2 \choice 7 \choice 9 \choice 4
				\end{oneparcheckboxes}
				% \part En el número 41589, ¿qué número ocupa la posición de las decenas de millar?
				% \part En el número 8459, ¿qué número ocupa la posición de las centenas?
				% \part En el número 10562, ¿qué número ocupa la posición de las centenas?
				% \part En el número 24781, ¿qué número ocupa la posición de las decenas de millar?
				% \part En el número 7856, ¿qué número ocupa la posición de las decenas?
			\end{parts}
		\end{multicols}
	}

	\questionboxed[2]{Señala la opción que responda correctamente a cada una de las siguientes preguntas:

		\begin{multicols}{2}
			\begin{parts}
				\part ¿Qué lugar ocupa el 2 en 87264?    \fillin[D][0.5cm]
				\part ¿Qué lugar ocupa el 1 en 1684?     \fillin[F][0.5cm]
				\part ¿Qué lugar ocupa el 1 en 6138?     \fillin[D][0.5cm]
				\part ¿Qué lugar ocupa el 8 en 198114?   \fillin[C][0.5cm]
				\part ¿Qué lugar ocupa el 2 en 206418?   \fillin[A][0.5cm]
				\part ¿Qué lugar ocupa el 6 en 6418?     \fillin[C][0.5cm]
				\part ¿Qué lugar ocupa el 7 en 46878?    \fillin[E][0.5cm]
				\part ¿Qué lugar ocupa el 4 en 149778?   \fillin[B][0.5cm]
			\end{parts}

			\columnbreak%

			\begin{choices}
				\choice {\color{red!80}centenas de millar.}
				\choice {\color{blue}decenas de millar.}
				\choice {\color{Goldenrod!70!Brown}unidades de millar.}
				\choice {\color{red!80}centenas.}
				\choice {\color{blue}decenas.}
				\choice {\color{Goldenrod!70!Brown}unidades.}
			\end{choices}
		\end{multicols}
	}




	% \subsection*{\ifprintanswers{Notación desarrollada 1                }
	% \subsection*{\ifprintanswers{Notación desarrollada 2                }
	\questionboxed[2]{Escribe la notación desarrollada de cada uno de los siguientes números:

		\begin{multicols}{2}
			\begin{parts}
				\part $15984=$ \fillin[$10000+5000+900+80+4$][2.4in] \\
				\part $4936 =$ \fillin[$4000+900+30+6$][2.4in] \\
				\part $27545=$ \fillin[$20000+7000+500+40+5$][2.4in] \\
				\part $6215 =$ \fillin[$6000+200+10+5$][2.4in] \\
				\part $5454 =$ \fillin[$5000+400+50+4$][2.4in] \\
				\part $6451 =$ \fillin[$6000+400+50+1$][2.4in] \\
				\part $19679=$ \fillin[$10000+9000+600+70+9$][2.4in] \\
				\part $26324=$ \fillin[$20000+6000+300+20+4$][2.4in] \\
				\part $5717 =$ \fillin[$5000+700+10+7$][2.4in] \\
				\part $31126=$ \fillin[$30000+1000+100+20+6$][2.4in] \\
				\part $4818 =$ \fillin[$4000+800+10+8$][2.4in] \\
				\part $7145 =$ \fillin[$7000+100+40+5$][2.4in] \\
			\end{parts}
		\end{multicols}
	}

	% \subsection*{\ifprintanswers{Escritura de cantidades 1              }
	% \subsection*{\ifprintanswers{Escritura de cantidades 2              }

	\questionboxed[2]{Escribe sore la línea los siguientes números:

		\begin{multicols}{2}
			\begin{parts}
				\part \fillin[  254][1.5cm] Doscientos cincuenta y cuatro.  		\\
				\part \fillin[  314][1.5cm] Trescientos catorce.  		\\
				\part \fillin[  431][1.5cm] Cuatrocientos treinta y uno.  		\\
				\part \fillin[ 1024][1.5cm] Mil veinticuatro.s  		\\
				\part \fillin[ 1849][1.5cm] Mil ochocientos cuarenta y nueve.  		\\
				\part \fillin[14005][1.5cm] Catorce mil cinco.  		\\
				\part \fillin[113013][1.5cm] Ciento trece mil trece.  		\\
				\part \fillin[4400][1.5cm] Cuatro mil cuatrocientos.  		\\
				\part \fillin[15081][1.5cm] Quince mil ochenta y uno.  		\\
				\part \fillin[19111][1.5cm] Diescinueve mil ciento once.  		\\
				\part \fillin[304300][1.5cm] Trescientos cuatro mil trescientos.  		\\
				\part \fillin[120022][1.5cm] Ciento Veinte mil veintidos.  		\\
			\end{parts}
		\end{multicols}
	}
\end{questions}
\end{document}