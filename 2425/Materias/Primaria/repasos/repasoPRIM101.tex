\documentclass[12pt,addpoints]{repaso}
\grado{1}
\nivel{Primaria}
\cicloescolar{2024-2025}
\materia{Matemáticas}
\unidad{1, 2 y 3}
\title{Practica la Unidad}
\aprendizajes{\scriptsize%
\item Expresa oralmente la sucesión numérica hasta cuatro cifras, en español y hasta donde sea posible, en su lengua materna, de manera ascendente y descendente a partir de un número natural dado.\\[-1.8em]
\item Representa, con apoyo de material concreto y modelos gráficos, fracciones: medios, cuartos, octavos, dieciseisavos, para expresar el resultado de mediciones y repartos en situaciones vinculadas a su contexto.\\[-1.8em]
\item Resuelve situaciones problemáticas vinculadas a su contexto que implican sumas, restas, multiplicación y división de números naturales de hasta tres cifras utilizando el algoritmo convencional y que impliquen, medición, estimación y comparación, de longitudes, masas y capacidades, con el uso del metro, kilogramo, litro y medios y cuartos de estas unidades; en el caso de la longitud, el decímetro y centímetro.\\[-1.8em]
   }
\author{Melchor Pinto, JC}
\begin{document}
\INFO%
% \begin{multicols}{2}
\tableofcontents
% \end{multicols}
\newpage
\begin{questions}\large
	\section{Unidad}

	\subsection{Conteo de números}
	\subsection{Escritura de cantidades}
	\subsection{Recta numérica}
	\subsection{Sistema decimal}

	\section{Unidad}

	\subsection{Sumas}
	\subsection{Restas}

	\section{Unidad}

	\subsection{Tabla del 1}
	\subsection{Tabla del 2}
	\subsection{Tabla del 3}
	\subsection{Miselánea}
	\questionboxed[10]{}
\end{questions}
\end{document}