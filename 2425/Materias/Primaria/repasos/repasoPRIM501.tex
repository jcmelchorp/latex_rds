\documentclass[12pt,addpoints]{repaso}
\grado{5}
\nivel{Primaria}
\cicloescolar{2024-2025}
\materia{Matemáticas}
\unidad{1, 2 y 3}
\title{Practica la Unidad}
\aprendizajes{\scriptsize%
	% \item Estudio de los números.
\item Ordena, lee, escribe e identifica regularidades en números naturales de hasta nueve cifras. Lee, escribe y ordena números decimales hasta diezmilésimos en notación decimal y letra, y los interpreta en diferentes contextos.
	% \item Suma y resta, su relación como operaciones inversas.
	\item Propone y resuelve situaciones problemáticas que implican sumas y restas con números decimales utilizando el algoritmo convencional y fracciones con diferentes denominadores. 
	% \item Multiplicación y división, su relación como operaciones inversas.
	\item Resuelve situaciones problemáticas vinculadas a diferentes contextos que implican multiplicar números fraccionarios y números decimales, con un número natural como multiplicador.También, dividir números naturales y el cociente resulte un número decimal.
	% \item Relaciones de proporcionalidad.
	\item Resuelve situaciones problemáticas de proporcionalidad en las que determina valores faltantes de números naturales, a partir de diferentes estrategias (cálculo del valor unitario, de dobles, triples o mitades).
	% \item Ubicación espacial.
	\item Elabora e interpreta croquis para comunicar la ubicación de seres vivos, objetos, trayectos o lugares.
	% \item Medición de la longitud, masa y capacidad.
	% \item Figuras y cuerpos geométricos y sus características.
	\item Reconoce y describe semejanzas y diferencias entre un prisma y una pirámide; propone desarrollos planos para construir prismas rectos cuadrangulares o rectangulares.
	% \item Perímetro, área y noción de volumen.
	\item Calcula el perímetro y área de diferentes polígonos. Construye y usa fórmulas para calcular el perímetro de cualquier polígono, a partir de sumar la longitud de todos sus lados o multiplicar el número de lados por la medida de uno de ellos.
	% \item Organización e interpretación de datos.
	\item Construye tablas y gráficas de barras, e interpreta información cuantitativa y cualitativa contenida en ellas.
	% \item Nociones de probabilidad. 
	\item Identifica situaciones de distintos contextos en las que interviene o no el azar; registra resultados de experiencias aleatorias en tablas de frecuencias y expresa la frecuencia absoluta y la relativa.
	  }
\author{Melchor Pinto, JC}
\begin{document}
\INFO%
\begin{questions}
      \questionboxed[10]{}                                                                                                                                                                                                                                                                                                                                                                                                                                                                                                                                                                                                                                                                                                                                                                                                                                                                                                                                                                                                                                                                                                                                                
       % UNIDAD 1                  
   % \section*{\ifprintanswers{Números romanos                         }\else{}\fi}
% \subsection*{\ifprintanswers{Números romanos 1                       }\else{}\fi}
% \subsection*{\ifprintanswers{Números romanos 2                       }\else{}\fi}
% \subsection*{\ifprintanswers{Números romanos 3                       }\else{}\fi}
% \subsection*{\ifprintanswers{Números romanos 4                       }\else{}\fi}
% \subsection*{\ifprintanswers{Números romanos 5                       }\else{}\fi}
   % \section*{\ifprintanswers{Sumas y restas                          }\else{}\fi}
% \subsection*{\ifprintanswers{Sumas 1                                 }\else{}\fi}
% \subsection*{\ifprintanswers{Sumas 2                                 }\else{}\fi}
% \subsection*{\ifprintanswers{Restas 1                                }\else{}\fi}
% \subsection*{\ifprintanswers{Restas 2                                }\else{}\fi}
% \subsection*{\ifprintanswers{Resolución de problemas                 }\else{}\fi}
   % \section*{\ifprintanswers{Multiplicación                          }\else{}\fi}
% \subsection*{\ifprintanswers{Tablas de multiplicar                   }\else{}\fi}
% \subsection*{\ifprintanswers{Multiplicaciones 1                      }\else{}\fi}
% \subsection*{\ifprintanswers{Multiplicaciones 2                      }\else{}\fi}
% \subsection*{\ifprintanswers{Multiplicaciones 3                      }\else{}\fi}
% \subsection*{\ifprintanswers{Resolución de problemas                 }\else{}\fi}
   % \section*{\ifprintanswers{División                                }\else{}\fi}
% \subsection*{\ifprintanswers{Tablas de multiplicar                   }\else{}\fi}
% \subsection*{\ifprintanswers{Multiplicaciones 1                      }\else{}\fi}
% \subsection*{\ifprintanswers{Multiplicaciones 2                      }\else{}\fi}
% \subsection*{\ifprintanswers{Multiplicaciones 3                      }\else{}\fi}
% \subsection*{\ifprintanswers{Resolución de problemas                 }\else{}\fi}
   % \section*{\ifprintanswers{Sistema decimal                         }\else{}\fi}
% \subsection*{\ifprintanswers{Posicionamiento decimal                 }\else{}\fi}
% \subsection*{\ifprintanswers{Notación desarrollada 1                 }\else{}\fi}
% \subsection*{\ifprintanswers{Notación desarrollada 2                 }\else{}\fi}
% \subsection*{\ifprintanswers{Escritura de cantidades 1               }\else{}\fi}
% \subsection*{\ifprintanswers{Escritura de cantidades 2               }\else{}\fi}
   % \section*{\ifprintanswers{Números decimales                       }\else{}\fi}
% \subsection*{\ifprintanswers{Posición decimal y notación desarrollada}\else{}\fi}
% \subsection*{\ifprintanswers{Nombre de decimales                     }\else{}\fi}
% \subsection*{\ifprintanswers{Suma de decimales                       }\else{}\fi}
% \subsection*{\ifprintanswers{Resta de decimales                      }\else{}\fi}
% \subsection*{\ifprintanswers{Multiplicación de decimales             }\else{}\fi}                                 
                                                             
   % UNIDAD 2                                             
   % \section*{\ifprintanswers{Introducción a las fracciones          }\else{}\fi}
% \subsection*{\ifprintanswers{Clasificación de fracciones            }\else{}\fi}
% \subsection*{\ifprintanswers{Representación de fracciones           }\else{}\fi}
% \subsection*{\ifprintanswers{Nombre de fracciones                   }\else{}\fi}
% \subsection*{\ifprintanswers{Fracciones en la recta numérica        }\else{}\fi}
% \subsection*{\ifprintanswers{Conversión de fracciones               }\else{}\fi}
   % \section*{\ifprintanswers{Suma y resta de fracciones             }\else{}\fi}
% \subsection*{\ifprintanswers{Simplificación de fracciones           }\else{}\fi}
% \subsection*{\ifprintanswers{Suma y resta con denominadores iguales }\else{}\fi}
% \subsection*{\ifprintanswers{Suma con denominadores diferentes      }\else{}\fi}
% \subsection*{\ifprintanswers{Resta con denominadores diferentes     }\else{}\fi}
% \subsection*{\ifprintanswers{Sumas y restas con fracciones mixtas   }\else{}\fi}
   % \section*{\ifprintanswers{Multiplicación y división de fracciones}\else{}\fi}
% \subsection*{\ifprintanswers{Multiplicación de fracciones           }\else{}\fi}
% \subsection*{\ifprintanswers{División de fracciones                 }\else{}\fi}
% \subsection*{\ifprintanswers{Multiplicación y división con enteros  }\else{}\fi}
% \subsection*{\ifprintanswers{Multiplicación con fracciones mixtas   }\else{}\fi}
% \subsection*{\ifprintanswers{División con fracciones mixtas         }\else{}\fi}
   % \section*{\ifprintanswers{MCD y MCM                              }\else{}\fi}
% \subsection*{\ifprintanswers{Factores primos                        }\else{}\fi}
% \subsection*{\ifprintanswers{Fracciones equivalentes                }\else{}\fi}
% \subsection*{\ifprintanswers{Mínimo común múltiplo                  }\else{}\fi}
% \subsection*{\ifprintanswers{Máximo común divisor                   }\else{}\fi}
% \subsection*{\ifprintanswers{Simplificación de fracciones           }\else{}\fi}
   % \section*{\ifprintanswers{Decimales y porcentajes                }\else{}\fi}
% \subsection*{\ifprintanswers{Decimales en la recta númerica         }\else{}\fi}
% \subsection*{\ifprintanswers{Porcentaje como decimales              }\else{}\fi}
% \subsection*{\ifprintanswers{Porcentaje de números                  }\else{}\fi}
% \subsection*{\ifprintanswers{Conversión de decimales a fracciones   }\else{}\fi}
% \subsection*{\ifprintanswers{Conversión de fracciones a decimales   }\else{}\fi}  

   % UNIDAD 3                                                              
   % \section*{\ifprintanswers{Estadística y gráficas              }\else{}\fi}                                            
% \subsection*{\ifprintanswers{Mediana                             }\else{}\fi}                                            
% \subsection*{\ifprintanswers{Moda                                }\else{}\fi}                                            
% \subsection*{\ifprintanswers{Media                               }\else{}\fi}
% \subsection*{\ifprintanswers{Interpretación de gráficas          }\else{}\fi}
% \subsection*{\ifprintanswers{Tablas de variación                 }\else{}\fi}
   % \section*{\ifprintanswers{Círculo                             }\else{}\fi}
% \subsection*{\ifprintanswers{Diámetro de un círculo              }\else{}\fi}
% \subsection*{\ifprintanswers{Radio de un círculo                 }\else{}\fi}
% \subsection*{\ifprintanswers{Perímetro de un círculo             }\else{}\fi}
% \subsection*{\ifprintanswers{Área de un círculo                  }\else{}\fi}
% \subsection*{\ifprintanswers{Líneas del círculo                  }\else{}\fi}
   % \section*{\ifprintanswers{Figuras geométricas                 }\else{}\fi}
% \subsection*{\ifprintanswers{Nombre de figuras                   }\else{}\fi}
% \subsection*{\ifprintanswers{Elementos de figuras                }\else{}\fi}
% \subsection*{\ifprintanswers{Perímetro                           }\else{}\fi}
% \subsection*{\ifprintanswers{Área                                }\else{}\fi}
% \subsection*{\ifprintanswers{Resolución de problemas             }\else{}\fi}
   % \section*{\ifprintanswers{Resolución de problemas             }\else{}\fi}
% \subsection*{\ifprintanswers{Unidades de tiempo y reloj          }\else{}\fi}
% \subsection*{\ifprintanswers{Cuerpos geométricos 1               }\else{}\fi}
% \subsection*{\ifprintanswers{Cuerpos geométricos 2               }\else{}\fi}
% \subsection*{\ifprintanswers{Resolución de problemas 1           }\else{}\fi}
% \subsection*{\ifprintanswers{Resolución de problemas 2           }\else{}\fi}
   % \section*{\ifprintanswers{Sistema de unidades                 }\else{}\fi}
% \subsection*{\ifprintanswers{Multiplicaciones por múltiplos de 10}\else{}\fi}
% \subsection*{\ifprintanswers{Divisiones por múltiplos de 10      }\else{}\fi}
% \subsection*{\ifprintanswers{Unidades de longitud                }\else{}\fi}
% \subsection*{\ifprintanswers{Unidades de masa                    }\else{}\fi}
% \subsection*{\ifprintanswers{Unidades de capacidad               }\else{}\fi}

\end{questions}
\end{document}