\documentclass[12pt,addpoints,answers]{repaso}
\grado{1}
\nivel{Secundaria}
\cicloescolar{2024-2025}
\materia{Matemáticas 1 \normalfont \color{darkgray} \\[-0.2em] \small con adecuación curricular a Matemáticas 6$^\circ$ de Primaria}
\unidad{1, 2 y 3}
\title{Practica la Unidad}
\aprendizajes{\scriptsize%
% \item Estudio de los números.
\item Expresa oralmente la sucesión numérica hasta billones, en español y hasta donde sea posible, en su lengua materna, de manera ascendente y descendente a partir de un número natural dado. Ordena, lee y escribe números naturales de más de nueve cifras e interpreta números decimales en diferentes contextos. Identifica semejanzas y diferencias entre el sistema de numeración decimal y otros sistemas como el maya y el romano
	% \item Suma y resta, su relación como operaciones inversas.
	\item A partir de situaciones problemáticas vinculadas a diferentes contextos, suma y resta números decimales y fracciones con diferentes denominadores.
	% \item Multiplicación y división, su relación como operaciones inversas.
	\item Resuelve situaciones problemáticas vinculadas a diferentes contextos que implican dividir números decimales entre naturales. También, dividir números fraccionarios entre números naturales.
	% \item Relaciones de proporcionalidad.
	\item A partir de situaciones problemáticas de proporcionalidad vinculadas a diferentes contextos, determina valores faltantes en las que en ocasiones se conoce el valor unitario y en otras no.
	% \item Ubicación espacial.
	\item Lee, interpreta y elabora planos para comunicar la ubicación de seres vivos y objetos.
	% \item Figuras y cuerpos geométricos y sus características.
	\item Explora y reconoce las características del cilindro y cono; anticipa y comprueba desarrollos planos que permiten construirlos.
	% \item Perímetro, área y noción de volumen.
	\item Resuelve situaciones problemáticas que implican calcular el perímetro y área de figuras compuestas por triángulos y cuadriláteros. Resuelve problemas que implican construir, estimar y comparar el volumen de cuerpos y prismas rectos rectangulares mediante el conteo de cubos, y reconoce que existen diferentes cuerpos con el mismo volumen.
	% \item Organización e interpretación de datos.
	\item Interpreta información cuantitativa y cualitativa contenida en tablas, gráficas de barras y circulares para responder preguntas vinculadas a diferentes contextos; construye gráficas de barras. Genera y organiza datos, determina la moda, la media aritmética y el rango para responder preguntas vinculadas a diferentes contextos.
	% \item Nociones de probabilidad.
	\item Clasifica eventos de diversos contextos utilizando términos como seguro, imposible, probable, muy probable o poco probable que sucedan.
}
\author{Melchor Pinto, JC}
\begin{document}
\INFO
\begin{multicols}{2}
	\tableofcontents
\end{multicols}\newpage
\begin{questions}\large
	\addcontentsline{toc}{section}{Unidad 1}
	\section*{Unidad 1}
	\addcontentsline{toc}{subsection}{Introducción a las fracciones}
	\subsection*{Introducción a las fracciones}
	
	\questionboxed[10]{

	}

	% \subsection*{Clasificación de fracciones   }
	% \subsection*{Representación de fracciones  }
	% \subsection*{Nombre de fracciones          }
	% \subsection*{Fracciones en la recta numérica         }
	% \subsection*{Conversión de fracciones      }

	\addcontentsline{toc}{subsection}{Multiplicaciones y Divisiones}
	\subsection*{Multiplicaciones y Divisiones}
	% \subsection*{Multiplicaciones 1            }
	% \subsection*{Multiplicaciones 2            }
	% \subsection*{Divisiones 1                  }
	% \subsection*{Divisiones 2                  }
	% \subsection*{Resolucion de problemas       }

	\addcontentsline{toc}{subsection}{Sumas y Restas}
	\subsection*{Sumas y Restas}
	% \subsection*{Sumas 1                       }
	% \subsection*{Sumas 2                       }
	% \subsection*{Restas 1                      }
	% \subsection*{Restas 2                      }
	% \subsection*{Resolucion de problemas       }

	\addcontentsline{toc}{subsection}{Números decimales}
	\subsection*{Números decimales}
	% \subsection*{Posición decimal y notación desarrollada}
	% \subsection*{Nombre de decimales           }
	% \subsection*{Decimales en la recta numérica}
	% \subsection*{Comparación de decimales      }
	% \subsection*{Redondeo de decimales         }

	\addcontentsline{toc}{section}{Unidad 2}
	\section*{Unidad 2}

	\addcontentsline{toc}{subsection}{Operaciones con decimales}
	\subsection*{Operaciones con decimales}
	% \subsection*{Suma de decimales             }
	% \subsection*{Resta de decimales            }
	% \subsection*{Multiplicación de decimales   }
	% \subsection*{División de decimales         }
	% \subsection*{Resolución de problemas       }

	\addcontentsline{toc}{subsection}{Simplificación de fracciones}
	\subsection*{Simplificación de fracciones}
	% \subsection*{Comparación de fracciones     }
	% \subsection*{Fracciones equivalentes       }
	% \subsection*{Mínimo Común Múltiplo         }
	% \subsection*{Máxico Común Divisor          }
	% \subsection*{Simplificación de fracciones  }

	\addcontentsline{toc}{subsection}{Suma y resta de fracciones}
	\subsection*{Suma y resta de fracciones}
	% \subsection*{Suma de decimales             }
	% \subsection*{Resta de decimales            }
	% \subsection*{Multiplicación de decimales   }
	% \subsection*{División de decimales         }
	% \subsection*{Resolución de problemas       }

	\addcontentsline{toc}{subsection}{Multiplicación y división de fracciones}
	\subsection*{Multiplicación y división de fracciones}
	% \subsection*{Suma y resta con denominadores iguales  }
	% \subsection*{Suma y resta denominadores diferentes   }
	% \subsection*{Multiplicación de fracciones  }
	% \subsection*{División de fracciones        }
	% \subsection*{Resolución de problemas       }

	\addcontentsline{toc}{subsection}{Decimales y porcentajes}
	\subsection*{Decimales y porcentajes}
	% \subsection*{Ubicación en la recta numérica}
	% \subsection*{Porcentajes a decimal         }
	% \subsection*{Operaciones con múltiplos de 10         }
	% \subsection*{Conversión de fracciones a decimales    }
	% \subsection*{Conversión de decimales a fracciones    }

	\addcontentsline{toc}{section}{Unidad 3}
	\section*{Unidad 3}

	\addcontentsline{toc}{subsection}{Círculo}
	\subsection*{Círculo}
	% \subsection*{Diámetro de un círculo}
	% \subsection*{Radio de un círculo   }
	% \subsection*{Perímetro             }
	% \subsection*{Área                  }
	% \subsection*{Resolución de problemas         }

	\addcontentsline{toc}{subsection}{Cuerpos Geométricos}
	\subsection*{Cuerpos Geométricos}
	% \subsection*{Nombre de cuerpos geométricos   }
	% \subsection*{Elementos de cuerpos geométricos}
	% \subsection*{Área lateral          }
	% \subsection*{Área total            }
	% \subsection*{Volumen               }

	\addcontentsline{toc}{subsection}{Figuras Geométricas}
	\subsection*{Figuras Geométricas}
	% \subsection*{Nombre de figuras     }
	% \subsection*{Elementos de figuras  }
	% \subsection*{Perímetro             }
	% \subsection*{Área                  }
	% \subsection*{Resolución de problemas         }

	\addcontentsline{toc}{subsection}{Sistema de Unidades}
	\subsection*{Sistema de Unidades}
	% \subsection*{Operaciones con múltiplos de 10 }
	% \subsection*{Unidades de longitud  }
	% \subsection*{Unidades de masa      }
	% \subsection*{Unidades de capacidad }
	% \subsection*{Unidades de área y volumen      }


\end{questions}
\end{document}