\documentclass[12pt,addpoints,answers]{repaso}
\grado{1}
\nivel{Secundaria}
\cicloescolar{2024-2025}
\materia{Matemáticas 1 \normalfont \color{darkgray} \\[-0.2em] \small con adecuación curricular a Matemáticas 6$^\circ$ de Primaria}
\unidad{1, 2 y 3}
\title{Practica la Unidad}
\aprendizajes{\scriptsize%
% \item Estudio de los números.
\item Expresa oralmente la sucesión numérica hasta billones, en español y hasta donde sea posible, en su lengua materna, de manera ascendente y descendente a partir de un número natural dado. Ordena, lee y escribe números naturales de más de nueve cifras e interpreta números decimales en diferentes contextos. Identifica semejanzas y diferencias entre el sistema de numeración decimal y otros sistemas como el maya y el romano
	% \item Suma y resta, su relación como operaciones inversas.
	\item A partir de situaciones problemáticas vinculadas a diferentes contextos, suma y resta números decimales y fracciones con diferentes denominadores.
	% \item Multiplicación y división, su relación como operaciones inversas.
	\item Resuelve situaciones problemáticas vinculadas a diferentes contextos que implican dividir números decimales entre naturales. También, dividir números fraccionarios entre números naturales.
	% \item Relaciones de proporcionalidad.
	\item A partir de situaciones problemáticas de proporcionalidad vinculadas a diferentes contextos, determina valores faltantes en las que en ocasiones se conoce el valor unitario y en otras no.
	% \item Ubicación espacial.
	\item Lee, interpreta y elabora planos para comunicar la ubicación de seres vivos y objetos.
	% \item Figuras y cuerpos geométricos y sus características.
	\item Explora y reconoce las características del cilindro y cono; anticipa y comprueba desarrollos planos que permiten construirlos.
	% \item Perímetro, área y noción de volumen.
	\item Resuelve situaciones problemáticas que implican calcular el perímetro y área de figuras compuestas por triángulos y cuadriláteros. Resuelve problemas que implican construir, estimar y comparar el volumen de cuerpos y prismas rectos rectangulares mediante el conteo de cubos, y reconoce que existen diferentes cuerpos con el mismo volumen.
	% \item Organización e interpretación de datos.
	\item Interpreta información cuantitativa y cualitativa contenida en tablas, gráficas de barras y circulares para responder preguntas vinculadas a diferentes contextos; construye gráficas de barras. Genera y organiza datos, determina la moda, la media aritmética y el rango para responder preguntas vinculadas a diferentes contextos.
	% \item Nociones de probabilidad.
	\item Clasifica eventos de diversos contextos utilizando términos como seguro, imposible, probable, muy probable o poco probable que sucedan.
}
\author{Melchor Pinto, JC}
\begin{document}
\INFO
\begin{multicols}{2}
	\tableofcontents
\end{multicols}\newpage
\begin{questions}\large
	\addcontentsline{toc}{section}{Unidad 1}
	\section*{Unidad 1}
	\addcontentsline{toc}{subsection}{Introducción a las fracciones}
	\subsection*{Introducción a las fracciones}
	
	\questionboxed[10]{}

	% \addcontentsline{toc}{subsubsection}{Clasificación de fracciones}
	% \subsubsection*{Clasificación de fracciones}
	% \addcontentsline{toc}{subsubsection}{Representación de fracciones}
	% \subsubsection*{Representación de fracciones}
	% \addcontentsline{toc}{subsubsection}{Nombre de fracciones}
	% \subsubsection*{Nombre de fracciones}
	% \addcontentsline{toc}{subsubsection}{Fracciones en la recta numérica}
	% \subsubsection*{Fracciones en la recta numérica}
	% \addcontentsline{toc}{subsubsection}{Conversión de fracciones}
	% \subsubsection*{Conversión de fracciones}

	\addcontentsline{toc}{subsection}{Multiplicaciones y Divisiones}
	\subsection*{Multiplicaciones y Divisiones}
	% \addcontentsline{toc}{subsubsection}{Multiplicaciones 1}
	% \subsubsection*{Multiplicaciones 1}
	% \addcontentsline{toc}{subsubsection}{Multiplicaciones 2}
	% \subsubsection*{Multiplicaciones 2}
	% \addcontentsline{toc}{subsubsection}{Divisiones 1}
	% \subsubsection*{Divisiones 1}
	% \addcontentsline{toc}{subsubsection}{Divisiones 2}
	% \subsubsection*{Divisiones 2}
	% \addcontentsline{toc}{subsubsection}{Resolucion de problemas}
	% \subsubsection*{Resolucion de problemas}

	\addcontentsline{toc}{subsection}{Sumas y Restas}
	\subsection*{Sumas y Restas}
	% \addcontentsline{toc}{subsubsection}{Sumas 1}
	% \subsubsection*{Sumas 1}
	% \addcontentsline{toc}{subsubsection}{Sumas 2}
	% \subsubsection*{Sumas 2}
	% \addcontentsline{toc}{subsubsection}{Restas 1}
	% \subsubsection*{Restas 1}
	% \addcontentsline{toc}{subsubsection}{Restas 2}
	% \subsubsection*{Restas 2}
	% \addcontentsline{toc}{subsubsection}{Resolucion de problemas}
	% \subsubsection*{Resolucion de problemas}

	\addcontentsline{toc}{subsection}{Números decimales}
	\subsection*{Números decimales}
	% \addcontentsline{toc}{subsubsection}{Posición decimal y notación desarrollada}
	% \subsubsection*{Posición decimal y notación desarrollada}
	% \addcontentsline{toc}{subsubsection}{Nombre de decimales}
	% \subsubsection*{Nombre de decimales}
	% \addcontentsline{toc}{subsubsection}{Decimales en la recta numérica}
	% \subsubsection*{Decimales en la recta numérica}
	% \addcontentsline{toc}{subsubsection}{Comparación de decimales}
	% \subsubsection*{Comparación de decimales}
	% \addcontentsline{toc}{subsubsection}{Redondeo de decimales}
	% \subsubsection*{Redondeo de decimales}

	\addcontentsline{toc}{section}{Unidad 2}
	\section*{Unidad 2}

	\addcontentsline{toc}{subsection}{Operaciones con decimales}
	\subsection*{Operaciones con decimales}
	% \addcontentsline{toc}{subsubsection}{Suma de decimales}
	% \subsubsection*{Suma de decimales}
	% \addcontentsline{toc}{subsubsection}{Resta de decimales}
	% \subsubsection*{Resta de decimales}
	% \addcontentsline{toc}{subsubsection}{Multiplicación de decimales}
	% \subsubsection*{Multiplicación de decimales}
	% \addcontentsline{toc}{subsubsection}{División de decimales}
	% \subsubsection*{División de decimales}
	% \addcontentsline{toc}{subsubsection}{Resolución de problemas}
	% \subsubsection*{Resolución de problemas}

	\addcontentsline{toc}{subsection}{Simplificación de fracciones}
	\subsection*{Simplificación de fracciones}
	% \addcontentsline{toc}{subsubsection}{Comparación de fracciones}
	% \subsubsection*{Comparación de fracciones}
	% \addcontentsline{toc}{subsubsection}{Fracciones equivalentes}
	% \subsubsection*{Fracciones equivalentes}
	% \addcontentsline{toc}{subsubsection}{Mínimo Común Múltiplo}
	% \subsubsection*{Mínimo Común Múltiplo}
	% \addcontentsline{toc}{subsubsection}{Máxico Común Divisor}
	% \subsubsection*{Máxico Común Divisor}
	% \addcontentsline{toc}{subsubsection}{Simplificación de fracciones}
	% \subsubsection*{Simplificación de fracciones}

	\addcontentsline{toc}{subsection}{Suma y resta de fracciones}
	\subsection*{Suma y resta de fracciones}
	% \addcontentsline{toc}{subsubsection}{Suma de decimales}
	% \subsubsection*{Suma de decimales}
	% \addcontentsline{toc}{subsubsection}{Resta de decimales}
	% \subsubsection*{Resta de decimales}
	% \addcontentsline{toc}{subsubsection}{Multiplicación de decimales}
	% \subsubsection*{Multiplicación de decimales}
	% \addcontentsline{toc}{subsubsection}{División de decimales}
	% \subsubsection*{División de decimales}
	% \addcontentsline{toc}{subsubsection}{Resolución de problemas}
	% \subsubsection*{Resolución de problemas}

	\addcontentsline{toc}{subsection}{Multiplicación y división de fracciones}
	\subsection*{Multiplicación y división de fracciones}
	% \addcontentsline{toc}{subsubsection}{Suma y resta con denominadores iguales}
	% \subsubsection*{Suma y resta con denominadores iguales}
	% \addcontentsline{toc}{subsubsection}{Suma y resta denominadores diferentes}
	% \subsubsection*{Suma y resta denominadores diferentes}
	% \addcontentsline{toc}{subsubsection}{Multiplicación de fracciones}
	% \subsubsection*{Multiplicación de fracciones}
	% \addcontentsline{toc}{subsubsection}{División de fracciones}
	% \subsubsection*{División de fracciones}
	% \addcontentsline{toc}{subsubsection}{Resolución de problemas}
	% \subsubsection*{Resolución de problemas}

	\addcontentsline{toc}{subsection}{Decimales y porcentajes}
	\subsection*{Decimales y porcentajes}
	% \addcontentsline{toc}{subsubsection}{Ubicación en la recta numérica}
	% \subsubsection*{Ubicación en la recta numérica}
	% \addcontentsline{toc}{subsubsection}{Porcentajes a decimal}
	% \subsubsection*{Porcentajes a decimal}
	% \addcontentsline{toc}{subsubsection}{Operaciones con múltiplos de 10}
	% \subsubsection*{Operaciones con múltiplos de 10}
	% \addcontentsline{toc}{subsubsection}{Conversión de fracciones a decimales}
	% \subsubsection*{Conversión de fracciones a decimales}
	% \addcontentsline{toc}{subsubsection}{Conversión de decimales a fracciones}
	% \subsubsection*{Conversión de decimales a fracciones}

	\addcontentsline{toc}{section}{Unidad 3}
	\section*{Unidad 3}

	\addcontentsline{toc}{subsection}{Círculo}
	\subsection*{Círculo}
	% \addcontentsline{toc}{subsubsection}{Diámetro de un círculo}
	% \subsubsection*{Diámetro de un círculo}
	% \addcontentsline{toc}{subsubsection}{Radio de un círculo}
	% \subsubsection*{Radio de un círculo}
	% \addcontentsline{toc}{subsubsection}{Perímetro}
	% \subsubsection*{Perímetro}
	% \addcontentsline{toc}{subsubsection}{Área}
	% \subsubsection*{Área}
	% \addcontentsline{toc}{subsubsection}{Resolución de problemas}
	% \subsubsection*{Resolución de problemas}

	\addcontentsline{toc}{subsection}{Cuerpos Geométricos}
	\subsection*{Cuerpos Geométricos}
	% \addcontentsline{toc}{subsubsection}{Nombre de cuerpos geométricos}
	% \subsubsection*{Nombre de cuerpos geométricos}
	% \addcontentsline{toc}{subsubsection}{Elementos de cuerpos geométricos}
	% \subsubsection*{Elementos de cuerpos geométricos}
	% \addcontentsline{toc}{subsubsection}{Área lateral}
	% \subsubsection*{Área lateral}
	% \addcontentsline{toc}{subsubsection}{Área total}
	% \subsubsection*{Área total}
	% \addcontentsline{toc}{subsubsection}{Volumen}
	% \subsubsection*{Volumen}

	\addcontentsline{toc}{subsection}{Figuras Geométricas}
	\subsection*{Figuras Geométricas}
	% \addcontentsline{toc}{subsubsection}{Nombre de figuras}
	% \subsubsection*{Nombre de figuras}
	% \addcontentsline{toc}{subsubsection}{Elementos de figuras}
	% \subsubsection*{Elementos de figuras}
	% \addcontentsline{toc}{subsubsection}{Perímetro}
	% \subsubsection*{Perímetro}
	% \addcontentsline{toc}{subsubsection}{Área}
	% \subsubsection*{Área}
	% \addcontentsline{toc}{subsubsection}{}
	% \subsubsection*{Resolución de problemas}

	\addcontentsline{toc}{subsection}{Sistema de Unidades}
	\subsection*{Sistema de Unidades}
	% \addcontentsline{toc}{subsubsection}{Operaciones con múltiplos de 10}
	% \subsubsection*{Operaciones con múltiplos de 10}
	% \addcontentsline{toc}{subsubsection}{Unidades de longitud}
	% \subsubsection*{Unidades de longitud}
	% \addcontentsline{toc}{subsubsection}{Unidades de masa}
	% \subsubsection*{Unidades de masa}
	% \addcontentsline{toc}{subsubsection}{Unidades de capacidad}
	% \subsubsection*{Unidades de capacidad}
	% \addcontentsline{toc}{subsubsection}{Unidades de área y volumen}
	% \subsubsection*{Unidades de área y volumen}


\end{questions}
\end{document}