\documentclass[12pt,addpoints]{repaso}
\grado{1}
\nivel{Secundaria}
\cicloescolar{2024-2025}
\materia{Matemáticas 1 \normalfont \color{darkgray} \\[-0.2em] \small con adecuación curricular a Matemáticas 6$^\circ$ de Primaria}

\unidad{1, 2 y 3}
\title{Practica la Unidad}
\aprendizajes{%
      \item hola
}
\author{Melchor Pinto, JC}
\begin{document}
\INFO%
\begin{questions}\large
	\addcontentsline{toc}{section}{Unidad 1}
	\section*{Unidad 1}
	\questionboxed[10]{}


	\addcontentsline{toc}{subsection}{Introducción a las fracciones}
	\subsection*{Introducción a las fracciones}
	% \subsection*{Clasificación de fracciones   }
	% \subsection*{Representación de fracciones  }
	% \subsection*{Nombre de fracciones          }
	% \subsection*{Fracciones en la recta numérica         }
	% \subsection*{Conversión de fracciones      }

	\addcontentsline{toc}{subsection}{Multiplicaciones y Divisiones}
	\subsection*{Multiplicaciones y Divisiones}
	% \subsection*{Multiplicaciones 1            }
	% \subsection*{Multiplicaciones 2            }
	% \subsection*{Divisiones 1                  }
	% \subsection*{Divisiones 2                  }
	% \subsection*{Resolucion de problemas       }

	\addcontentsline{toc}{subsection}{Sumas y Restas}
	\subsection*{Sumas y Restas}
	% \subsection*{Sumas 1                       }
	% \subsection*{Sumas 2                       }
	% \subsection*{Restas 1                      }
	% \subsection*{Restas 2                      }
	% \subsection*{Resolucion de problemas       }

	\addcontentsline{toc}{subsection}{Números decimales}
	\subsection*{Números decimales}
	% \subsection*{Posición decimal y notación desarrollada}
	% \subsection*{Nombre de decimales           }
	% \subsection*{Decimales en la recta numérica}
	% \subsection*{Comparación de decimales      }
	% \subsection*{Redondeo de decimales         }

	\addcontentsline{toc}{section}{Unidad 2}
	\section*{Unidad 2}

	\addcontentsline{toc}{subsection}{Operaciones con decimales}
	\subsection*{Operaciones con decimales}
	% \subsection*{Suma de decimales             }
	% \subsection*{Resta de decimales            }
	% \subsection*{Multiplicación de decimales   }
	% \subsection*{División de decimales         }
	% \subsection*{Resolución de problemas       }

	\addcontentsline{toc}{subsection}{Simplificación de fracciones}
	\subsection*{Simplificación de fracciones}
	% \subsection*{Comparación de fracciones     }
	% \subsection*{Fracciones equivalentes       }
	% \subsection*{Mínimo Común Múltiplo         }
	% \subsection*{Máxico Común Divisor          }
	% \subsection*{Simplificación de fracciones  }

	\addcontentsline{toc}{subsection}{Suma y resta de fracciones}
	\subsection*{Suma y resta de fracciones}
	% \subsection*{Suma de decimales             }
	% \subsection*{Resta de decimales            }
	% \subsection*{Multiplicación de decimales   }
	% \subsection*{División de decimales         }
	% \subsection*{Resolución de problemas       }

	\addcontentsline{toc}{subsection}{Multiplicación y división de fracciones}
	\subsection*{Multiplicación y división de fracciones}
	% \subsection*{Suma y resta con denominadores iguales  }
	% \subsection*{Suma y resta denominadores diferentes   }
	% \subsection*{Multiplicación de fracciones  }
	% \subsection*{División de fracciones        }
	% \subsection*{Resolución de problemas       }

	\addcontentsline{toc}{subsection}{Decimales y porcentajes}
	\subsection*{Decimales y porcentajes}
	% \subsection*{Ubicación en la recta numérica}
	% \subsection*{Porcentajes a decimal         }
	% \subsection*{Operaciones con múltiplos de 10         }
	% \subsection*{Conversión de fracciones a decimales    }
	% \subsection*{Conversión de decimales a fracciones    }

	\addcontentsline{toc}{section}{Unidad 3}
	\section*{Unidad 3}

	\addcontentsline{toc}{subsection}{Círculo}
	\subsection*{Círculo}
	% \subsection*{Diámetro de un círculo}
	% \subsection*{Radio de un círculo   }
	% \subsection*{Perímetro             }
	% \subsection*{Área                  }
	% \subsection*{Resolución de problemas         }

	\addcontentsline{toc}{subsection}{Cuerpos Geométricos}
	\subsection*{Cuerpos Geométricos}
	% \subsection*{Nombre de cuerpos geométricos   }
	% \subsection*{Elementos de cuerpos geométricos}
	% \subsection*{Área lateral          }
	% \subsection*{Área total            }
	% \subsection*{Volumen               }

	\addcontentsline{toc}{subsection}{Figuras Geométricas}
	\subsection*{Figuras Geométricas}
	% \subsection*{Nombre de figuras     }
	% \subsection*{Elementos de figuras  }
	% \subsection*{Perímetro             }
	% \subsection*{Área                  }
	% \subsection*{Resolución de problemas         }

	\addcontentsline{toc}{subsection}{Sistema de Unidades}
	\subsection*{Sistema de Unidades}
	% \subsection*{Operaciones con múltiplos de 10 }
	% \subsection*{Unidades de longitud  }
	% \subsection*{Unidades de masa      }
	% \subsection*{Unidades de capacidad }
	% \subsection*{Unidades de área y volumen      }


\end{questions}
\end{document}