\documentclass[12pt,addpoints,answers]{repaso}
\grado{1}
\nivel{Secundaria}
\cicloescolar{2024-2025}
\materia{Matemáticas 1 \normalfont \color{darkgray} \\[-0.2em] \small con adecuación curricular a Matemáticas 3$^\circ$ de Primaria.}
\unidad{1, 2 y 3}
\title{Practica la reposición a la Unidad}
\aprendizajes{\scriptsize%
\item Expresa oralmente la sucesión numérica hasta cuatro cifras, en español y hasta donde sea posible, en su lengua materna, de manera ascendente y descendente a partir de un número natural dado.\\[-1.8em]
\item Representa, con apoyo de material concreto y modelos gráficos, fracciones: medios, cuartos, octavos, dieciseisavos, para expresar el resultado de mediciones y repartos en situaciones vinculadas a su contexto.\\[-1.8em]
\item Resuelve situaciones problemáticas vinculadas a su contexto que implican sumas, restas, multiplicación y división de números naturales de hasta tres cifras utilizando el algoritmo convencional y que impliquen, medición, estimación y comparación, de longitudes, masas y capacidades, con el uso del metro, kilogramo, litro y medios y cuartos de estas unidades; en el caso de la longitud, el decímetro y centímetro.\\[-1.8em]
\item Resuelve problemas de suma, resta, multiplicación y división vinculados a su contexto, que impliquen el uso de fracciones (medios, cuartos, octavos, dieciseisavos), con el apoyo de material concreto o representaciones gráficas.
   }
\author{Melchor Pinto, JC}
\begin{document}
\INFO%
% \begin{multicols}{2}
	\tableofcontents
% \end{multicols}
\newpage
\begin{questions}\large
	\addcontentsline{toc}{section}{Unidad 1}
	\section*{Unidad 1}
	\addcontentsline{toc}{subsection}{Escritura de cantidades}
	\subsection*{Escritura de cantidades}
	% \subsection*{\ifprintanswers{Escritura de cantidades 1 }\else{}\fi}
	% \subsection*{\ifprintanswers{Escritura de cantidades 2 }\else{}\fi}
	% \subsection*{\ifprintanswers{Escritura de cantidades 3 }\else{}\fi}
	% \subsection*{\ifprintanswers{Escritura de cantidades 4 }\else{}\fi}
	% \subsection*{\ifprintanswers{Escritura de cantidades 5 }\else{}\fi}

	\questionboxed[4]{Escribe sobre la línea los siguientes números:

		\begin{multicols}{2}
			\begin{parts}
				\part \fillin[  65][1.5cm] Sesenta y cinco.
				\part \fillin[ 109][1.5cm] Ciento nueve.
				\part \fillin[ 254][1.5cm] Doscientos cincuenta y cuatro.
				\part \fillin[ 314][1.5cm] Trescientos catorce.
				\part \fillin[ 431][1.5cm] Cuatrocientos treinta y uno.
				\part \fillin[1024][1.5cm] Mil veinticuatro.
				\part \fillin[1849][1.5cm] Mil ochocientos cuarenta y nueve.
				% \part \fillin[1310][1.1cm] Mil trescientos diez.                      
				\part \fillin[ 703][1.5cm] Setecientos tres.
			\end{parts}
		\end{multicols}
	}
	\addcontentsline{toc}{subsection}{Sistema decimal}
	\subsection*{Sistema decimal}
	% \subsection*{\ifprintanswers{Notación desarrollada 1 }\else{}\fi}
	% \subsection*{\ifprintanswers{Notación desarrollada 2 }\else{}\fi}
	% \subsection*{\ifprintanswers{Notación desarrollada 3 }\else{}\fi}


	\questionboxed[8]{Escribe la notación desarrollada de cada uno de los siguientes números:

		\begin{multicols}{2}
			\begin{parts}
				\part $84=$ \fillin[$80+4$][2.4in]
				\part $936 =$ \fillin[$900+30+6$][2.4in]
				\part $2096=$ \fillin[$2000+90+6$][2.4in]
				\part $6215 =$ \fillin[$6000+200+10+5$][2.4in]
				\part $4818 =$ \fillin[$4000+800+10+8$][2.4in]
				\part $7145 =$ \fillin[$7000+100+40+5$][2.4in]
				\part $19679=$ \fillin[$10000+9000+600+70+9$][2.4in]
				\part $26324=$ \fillin[$20000+6000+300+20+4$][2.4in]
				\part $5717 =$ \fillin[$5000+700+10+7$][2.4in]
				\part $31126=$ \fillin[$30000+1000+100+20+6$][2.4in]
				\part $4818 =$ \fillin[$4000+800+10+8$][2.4in]
				\part $7145 =$ \fillin[$7000+100+40+5$][2.4in]
			\end{parts}
		\end{multicols}
	}
	% \subsection*{\ifprintanswers{Posicionamiento decimal 1 }\else{}\fi}
	% \subsection*{\ifprintanswers{Posicionamiento decimal 2 }\else{}\fi}

	\questionboxed[6]{Señala la opción que responda correctamente a cada una de las siguientes preguntas:

		\begin{multicols}{2}
			\begin{parts}
				\part ¿Qué lugar ocupa el 6 en 6418?     \fillin[C][0.5cm]
				\part ¿Qué lugar ocupa el 2 en 206418?   \fillin[A][0.5cm]
				\part ¿Qué lugar ocupa el 2 en 87264?    \fillin[D][0.5cm]
				\part ¿Qué lugar ocupa el 1 en 1684?     \fillin[F][0.5cm]
				% \part ¿Qué lugar ocupa el 1 en 6138?     \fillin[D][0.5cm]
				% \part ¿Qué lugar ocupa el 8 en 198114?   \fillin[C][0.5cm]
				\part ¿Qué lugar ocupa el 7 en 46878?    \fillin[E][0.5cm]
				\part ¿Qué lugar ocupa el 4 en 149778?   \fillin[B][0.5cm]
			\end{parts}

			\columnbreak%

			\begin{choices}
				\choice {\color{red}centenas de millar.}
				\choice {\color{blue}decenas de millar.}
				\choice {\color{Goldenrod}unidades de millar.}
				\choice {\color{red}centenas.}
				\choice {\color{blue}decenas.}
				\choice {\color{Goldenrod}unidades.}
			\end{choices}
		\end{multicols}
	}



	% \section*{\ifprintanswers{Sistema decimal 2}\else{}\fi}
	% \subsection*{\ifprintanswers{Posicionamiento decimal 1 }\else{}\fi}
	% \subsection*{\ifprintanswers{Posicionamiento decimal 2 }\else{}\fi}

	% \questionboxed[3]{Señala la opción que responda correctamente a cada una de las siguientes preguntas:

	% 	\begin{multicols}{2}
	% 		\begin{parts}\large
	% 			\part En el número 1.829, ¿qué número ocupa la posición de las centésimas?

	% 			\begin{oneparcheckboxes}
	% 				\choice 1 \CorrectChoice 2 \choice 6 \choice 8 \choice 9
	% 			\end{oneparcheckboxes}

	% 			\part En el número 2.087, ¿qué número ocupa la posición de las décimas?

	% 			\begin{oneparcheckboxes}
	% 				\CorrectChoice 0 \choice 2 \choice 7 \choice 8 \choice 9
	% 			\end{oneparcheckboxes}

	% 			\part En el número 5.928, ¿qué número ocupa la posición de las décimas?

	% 			\begin{oneparcheckboxes}
	% 				\choice 5 \choice 2 \choice 6 \choice 8 \CorrectChoice 9
	% 			\end{oneparcheckboxes}

	% 			\part En el número 3.284, ¿qué número ocupa la posición de las milésimas?

	% 			\begin{oneparcheckboxes}
	% 				\choice 2 \choice 3 \CorrectChoice 4  \choice 8 \choice 9
	% 			\end{oneparcheckboxes}

	% 			\part En el número 1.285, ¿qué número ocupa la posición de las décimas?

	% 			\begin{oneparcheckboxes}
	% 				\choice 1 \CorrectChoice 2 \choice 5 \choice 8 \choice 9
	% 			\end{oneparcheckboxes}

	% 			\part En el número 1.823, ¿qué número ocupa la posición de las milésimas?

	% 			\begin{oneparcheckboxes}
	% 				\choice 1 \choice 2 \CorrectChoice 3 \choice 6 \choice 8
	% 			\end{oneparcheckboxes}
	% 		\end{parts}
	% 	\end{multicols}
	% }

	% \questionboxed[6]{Escribe los siguientes números

	% 	\begin{multicols}{2}
	% 		\begin{parts}\large
	% 			\part Veinticinco enteros ocho décimas                  \\ \hfill \fillin[$25.8$][1cm]
	% 			\part Seis enteros ciento veintiocho milésimas          \\ \hfill \fillin[$6.128$][1cm]
	% 			\part Catorce enteros veintinueve centésimas            \\ \hfill \fillin[$14.29$][1cm]
	% 			\part Cuarenta enteros dos décimas                      \\ \hfill \fillin[$40.2$][1cm]
	% 			\part Tres enteros cincuenta y ocho centésimas          \\ \hfill \fillin[$3.58$][1cm]
	% 			\part Cuatro enteros sesenta y nueve milésimas          \\ \hfill \fillin[$4.069$][1cm]
	% 			\part Siete enteros cuatro décimas                      \\ \hfill \fillin[$ 7.4$][1cm]
	% 			% \part Dos enteros siete décimas                         \\ \hfill \fillin[$2.7$][1cm]
	% 			% \part Cuatro enteros ocho milésimas                     \\ \hfill \fillin[$4.008$][1cm]
	% 			% \part Siete enteros setenta y siete centésimas          \\ \hfill \fillin[$7.77$][1cm]
	% 			% \part Once enteros ochenta y nueve centésimas           \\ \hfill \fillin[$11.89$][1cm]
	% 			\part Treinta y ocho enteros nueve décimas              \\ \hfill \fillin[$38.9$][1cm]
	% 		\end{parts}
	% 	\end{multicols}
	% }

	% \subsection*{\ifprintanswers{Notación desarrollada 1 }\else{}\fi}
	% \subsection*{\ifprintanswers{Notación desarrollada 2  }\else{}\fi}
	% \subsection*{\ifprintanswers{Posicionamiento decimal y Notación desarrollada }\else{}\fi}

	\questionboxed[4]{Señala la opción que responda correctamente a cada una de las siguientes preguntas:

		\begin{multicols}{2}
			\begin{parts}
				\part En el número 3658, ¿qué número ocupa la posición de las decenas?

				\begin{oneparcheckboxes}
					\choice 3 \CorrectChoice 5 \choice 6 \choice 8 \choice 9
				\end{oneparcheckboxes}

				\part En el número 17542, ¿qué número ocupa la posición de las unidades de millar?

				\begin{oneparcheckboxes}
					\choice 1 \CorrectChoice 7 \choice 5 \choice 4 \choice 2
				\end{oneparcheckboxes}

				\part En el número 5984, ¿qué número ocupa la posición de las centenas?

				\begin{oneparcheckboxes}
					\choice 4 \choice 2 \choice 5 \choice 8 \CorrectChoice 9
				\end{oneparcheckboxes}

				\part En el número 7841, ¿qué número ocupa la posición de las decenas?

				\begin{oneparcheckboxes}
					\choice 1 \choice 7 \choice 8 \CorrectChoice 4 \choice 2
				\end{oneparcheckboxes}

				\part En el número 3918, ¿qué número ocupa la posición de las centenas?

				\begin{oneparcheckboxes}
					\choice 3 \choice 1 \choice 6 \choice 8 \CorrectChoice 9
				\end{oneparcheckboxes}

				\part En el número 3621, ¿qué número ocupa la posición de las decenas?

				\begin{oneparcheckboxes}
					\CorrectChoice 2 \choice 3 \choice 6 \choice 8 \choice 1
				\end{oneparcheckboxes}

				\part En el número 51362, ¿qué número ocupa la posición de las decenas de millar?

				\begin{oneparcheckboxes}
					\choice 3 \CorrectChoice 5 \choice 6 \choice 1 \choice 2
				\end{oneparcheckboxes}

				\part En el número 7584, ¿qué número ocupa la posición de las decenas?

				\begin{oneparcheckboxes}
					\choice 3 \choice 5 \choice 7 \CorrectChoice 8 \choice 4
				\end{oneparcheckboxes}

				% \part En el número 9654, ¿qué número ocupa la posición de las centenas?

				% \begin{oneparcheckboxes}
				%    \choice 3 \choice 5 \CorrectChoice 6 \choice 4 \choice 9
				% \end{oneparcheckboxes}

				% \part En el número 240679, ¿qué número ocupa la posición de las centenas de millar?

				% \begin{oneparcheckboxes}
				%    \choice 0 \choice 6 \CorrectChoice 2 \choice 7 \choice 9 \choice 4
				% \end{oneparcheckboxes}
				% \part En el número 41589, ¿qué número ocupa la posición de las decenas de millar?
				% \part En el número 8459, ¿qué número ocupa la posición de las centenas?
				% \part En el número 10562, ¿qué número ocupa la posición de las centenas?
				% \part En el número 24781, ¿qué número ocupa la posición de las decenas de millar?
				% \part En el número 7856, ¿qué número ocupa la posición de las decenas?
			\end{parts}
		\end{multicols}
	}

	\addcontentsline{toc}{subsection}{Tablas de multiplicar}

	\subsection*{Tablas de multiplicar}
		% \subsection*{\ifprintanswers{Tabla del 1  }\else{}\fi}
	% \subsection*{\ifprintanswers{Tabla del 2 }\else{}\fi}
	% \subsection*{\ifprintanswers{Tabla del 3 }\else{}\fi}
	% \subsection*{\ifprintanswers{Tabla del 4 }\else{}\fi}
	% \subsection*{\ifprintanswers{Tabla del 5 }\else{}\fi}

	
	% \section*{\ifprintanswers{Tablas de multiplicar 2        }\else{}\fi}
	% \subsection*{\ifprintanswers{Tabla del 6                    }\else{}\fi}
	% \subsection*{\ifprintanswers{Tabla del 7                    }\else{}\fi}
	% \subsection*{\ifprintanswers{Tabla del 8                    }\else{}\fi}
	% \subsection*{\ifprintanswers{Tabla del 9                    }\else{}\fi}
	% \subsection*{\ifprintanswers{Tabla del 10                   }\else{}\fi}

	\questionboxed[8]{Reponde las siguientes tablas de multiplicar:

		\begin{multicols}{4}
			\begin{parts}\Large
				\part $5 \times 9=$ \fillin[$45$][0cm]
				\part $5 \times 6=$ \fillin[$30$][0cm]
				\part $6 \times 8=$ \fillin[$48$][0cm]
				\part $6 \times 9=$ \fillin[$54$][0cm]
				\part $3 \times 6=$ \fillin[$18$][0cm]
				\part $2 \times 7=$ \fillin[$14$][0cm]
				\part $4 \times 7=$ \fillin[$28$][0cm]
				\part $3 \times 8=$ \fillin[$24$][0cm]
				\part $2 \times 9=$ \fillin[$18$][0cm]
				\part $4 \times 4=$ \fillin[$16$][0cm]
				\part $7 \times 7=$ \fillin[$49$][0cm]
				\part $7 \times 5=$ \fillin[$35$][0cm]
				\part $5 \times 4=$ \fillin[$20$][0cm]
				\part $8 \times 7=$ \fillin[$56$][0cm]
				\part $7 \times 6=$ \fillin[$42$][0cm]
				\part $9 \times 7=$ \fillin[$63$][0cm]
			\end{parts}
		\end{multicols}
	}

	\questionboxed[8]{Completa las siguientes tablas de multiplicar:

		\begin{multicols}{4}
			\begin{parts}\Large
				\part $\fillin[6][0.5cm] \times 6= 36$
				\part $\fillin[8][0.5cm] \times 8= 64$
				\part $\fillin[7][0.5cm] \times 8= 56$
				\part $5 \times \fillin[10][0.5cm]=50$
				\part $4 \times \fillin[8][0.5cm]=32$
				\part $8 \times \fillin[5][0.5cm]= 40$
				\part $\fillin[6][0.5cm] \times 4= 24$
				\part $7 \times \fillin[7][0.5cm]= 49$
				\part $\fillin[8][0.5cm] \times 3= 24$
				\part $9 \times \fillin[8][0.5cm]= 72$
				\part $\fillin[9][0.5cm] \times 5= 45$
				\part $6 \times \fillin[7][0.5cm]= 42$
				\part $\fillin[9][0.5cm] \times 9= 81$
				\part $4 \times \fillin[9][0.5cm]= 36$
				\part $\fillin[7][0.5cm] \times 4= 28$
				\part $\fillin[9][0.5cm] \times 3= 21$
			\end{parts}
		\end{multicols}
	}

	\addcontentsline{toc}{section}{Unidad 2}
	\section*{Unidad 2}
	\addcontentsline{toc}{subsection}{Sumas}
	\subsection*{Sumas}                       
	% \subsection*{\ifprintanswers{Sumando con 1, 2 y 3           }\else{}\fi}
	% \subsection*{\ifprintanswers{Sumando con 4, 5 y 6           }\else{}\fi}
	% \subsection*{\ifprintanswers{Sumando con 7, 8 y 9           }\else{}\fi}
	% \subsection*{\ifprintanswers{Sumando números entre 10 y 20  }\else{}\fi}
	% \subsection*{\ifprintanswers{Sumando números entre 20 y 50  }\else{}\fi}

	% \section*{\ifprintanswers{Sumas 2                        }\else{}\fi}
	% \subsection*{\ifprintanswers{Sumas hasta el 100             }\else{}\fi}
	% \subsection*{\ifprintanswers{Sumas hasta el 500             }\else{}\fi}
	% \subsection*{\ifprintanswers{Sumas con acarreos 1           }\else{}\fi}
	% \subsection*{\ifprintanswers{Sumas con acarreos 2           }\else{}\fi}
	% \subsection*{\ifprintanswers{Sumas con acarreos 3           }\else{}\fi}

	\questionboxed[8]{Realiza las siguientes sumas:

		\begin{multicols}{4}
			\begin{parts}
				\part $9 + 8=$ \fillin[17][0cm]

				\part
				\ifprintanswers{\opadd[hfactor=decimal,resultstyle=\color{red},carryadd=true]{17}{18}}
				\else{\opadd[hfactor=decimal,resultstyle=\color{white},carryadd=false]{17}{18}\\[0.5cm]}\fi

				\part
				\ifprintanswers{\opadd[hfactor=decimal,resultstyle=\color{red},carryadd=true]{155}{93}}
				\else{\opadd[hfactor=decimal,resultstyle=\color{white},carryadd=false]{155}{93}}\fi

				\part $5 + 7=$ \fillin[12][0cm]

				\part
				\ifprintanswers{\opadd[hfactor=decimal,resultstyle=\color{red},carryadd=true]{26}{19}}
				\else{\opadd[hfactor=decimal,resultstyle=\color{white},carryadd=false]{26}{19}\\[0.5cm]}\fi

				\part
				\ifprintanswers{\opadd[hfactor=decimal,resultstyle=\color{red},carryadd=true]{271}{128}}
				\else{\opadd[hfactor=decimal,resultstyle=\color{white},carryadd=false]{271}{128}}\fi

				\part $8 + 7=$ \fillin[15][0cm]

				\part
				\ifprintanswers{\opadd[hfactor=decimal,resultstyle=\color{red},carryadd=true]{37}{28}}
				\else{\opadd[hfactor=decimal,resultstyle=\color{white},carryadd=false]{37}{28}\\[0.5cm]}\fi

				\part
				\ifprintanswers{\opadd[hfactor=decimal,resultstyle=\color{red},carryadd=true]{182}{149}}
				\else{\opadd[hfactor=decimal,resultstyle=\color{white},carryadd=false]{182}{149}}\fi

				\part $4 + 9=$ \fillin[13][0cm]

				\part
				\ifprintanswers{\opadd[hfactor=decimal,resultstyle=\color{red},carryadd=true]{44}{25}}
				\else{\opadd[hfactor=decimal,resultstyle=\color{white},carryadd=false]{44}{25}\\[0.5cm]}\fi

				\part
				\ifprintanswers{\opadd[hfactor=decimal,resultstyle=\color{red},carryadd=true]{482}{398}}
				\else{\opadd[hfactor=decimal,resultstyle=\color{white},carryadd=false]{482}{398}}\fi
			\end{parts}
		\end{multicols}
	}

	\addcontentsline{toc}{subsection}{Restas}
	\subsection*{Restas}	% \subsection*{\ifprintanswers{Restando con 1, 2 y 3          }\else{}\fi}
	% \subsection*{\ifprintanswers{Restando con 4, 5, 6, 7, 8 y 9 }\else{}\fi}
	% \subsection*{\ifprintanswers{Restas como sumas 1            }\else{}\fi}
	% \subsection*{\ifprintanswers{Restas como suma 2             }\else{}\fi}
	% \subsection*{\ifprintanswers{Restas como suma 3             }\else{}\fi}


	% \section*{\ifprintanswers{Restas 2                       }\else{}\fi}
	% \subsection*{\ifprintanswers{Restas sin transformación 1    }\else{}\fi}
	% \subsection*{\ifprintanswers{Restas sin transformación 2    }\else{}\fi}
	% \subsection*{\ifprintanswers{Restas con transformación 1    }\else{}\fi}
	% \subsection*{\ifprintanswers{Restas con transformación 2    }\else{}\fi}
	% \subsection*{\ifprintanswers{Restas con transformación 3    }\else{}\fi}

	\questionboxed[8]{Realiza las siguientes restas:

		\begin{multicols}{4}
			\begin{parts}
				\part $9 - 3=$ \fillin[6][0cm]\\[0.2cm]
				\part $15 - \fillin[8][0.5cm]= 7$\\[0.2cm]

				\part
				\ifprintanswers{\opsub[hfactor=decimal,resultstyle=\color{red},carrysub=true]{47}{24}}
				\else{\opsub[hfactor=decimal,resultstyle=\color{white},carrysub=false]{47}{24}}\fi\\[0.2cm]

				\part
				\ifprintanswers{\opsub[hfactor=decimal,resultstyle=\color{red},carrysub=true]{155}{93}}
				\else{\opsub[hfactor=decimal,resultstyle=\color{white},carrysub=false]{155}{93}}\fi

				\part $7 - 4=$ \fillin[3][0cm]\\[0.2cm]
				\part $12 - \fillin[7][0.5cm]= 5$\\[0.2cm]

				\part
				\ifprintanswers{\opsub[hfactor=decimal,resultstyle=\color{red},carrysub=true]{37}{25}}
				\else{\opsub[hfactor=decimal,resultstyle=\color{white},carrysub=false]{37}{25}}\fi\\[0.2cm]

				\part
				\ifprintanswers{\opsub[hfactor=decimal,resultstyle=\color{red},carrysub=true]{145}{118}}
				\else{\opsub[hfactor=decimal,resultstyle=\color{white},carrysub=false]{145}{118}}\fi

				\part $8 - 8=$ \fillin[0][0cm]\\[0.2cm]
				\part $18 - \fillin[14][0.5cm]= 4$\\[0.2cm]

				\part
				\ifprintanswers{\opsub[hfactor=decimal,resultstyle=\color{red},carrysub=true]{82}{50}}
				\else{\opsub[hfactor=decimal,resultstyle=\color{white},carrysub=false]{82}{50}}\fi\\[0.2cm]

				\part
				\ifprintanswers{\opsub[hfactor=decimal,resultstyle=\color{red},carrysub=true]{482}{398}}
				\else{\opsub[hfactor=decimal,resultstyle=\color{white},carrysub=false]{482}{398}}\fi

				\part $11 - 4=$ \fillin[7][0cm]\\[0.2cm]
				\part $25 - \fillin[20][0.5cm]= 5$\\[0.2cm]

				\part
				\ifprintanswers{\opsub[hfactor=decimal,resultstyle=\color{red},carrysub=true]{71}{45}}
				\else{\opsub[hfactor=decimal,resultstyle=\color{white},carrysub=false]{71}{45}}\fi\\[0.2cm]

				\part
				\ifprintanswers{\opsub[hfactor=decimal,resultstyle=\color{red},carrysub=true]{1090}{845}}
				\else{\opsub[hfactor=decimal,resultstyle=\color{white},carrysub=false]{1090}{845}}\fi
			\end{parts}
		\end{multicols}
	}

	\addcontentsline{toc}{section}{Unidad 3}
	\section*{Unidad 3}
	\addcontentsline{toc}{subsection}{Multiplicaciones}
	\subsection*{Multiplicaciones}

	% \section*{\ifprintanswers{Restas 3                                   }\else{}\fi}
	% \subsection*{\ifprintanswers{Restas con transformación 1                }\else{}\fi}
	% \subsection*{\ifprintanswers{Restas con transformación 2                }\else{}\fi}
	% \subsection*{\ifprintanswers{Minuendos múltiplos de 10                  }\else{}\fi}
	% \subsection*{\ifprintanswers{Minuendos con ceros intermedios            }\else{}\fi}
	% \subsection*{\ifprintanswers{Repaso de restas                           }\else{}\fi}


	% \section*{\ifprintanswers{Multiplicaciones                           }\else{}\fi}
	% \subsection*{\ifprintanswers{Multiplicaciones con una cifra 1           }\else{}\fi}
	% \subsection*{\ifprintanswers{Multiplicaciones con una cifra 2           }\else{}\fi}
	% \subsection*{\ifprintanswers{Multiplicaciones con una cifra 3           }\else{}\fi}
	% \subsection*{\ifprintanswers{Multiplicaciones con una cifra 4           }\else{}\fi}
	% \subsection*{\ifprintanswers{Multiplicaciones con dos cifras            }\else{}\fi}

	\questionboxed[6]{Realiza las siguientes multiplicaciones:

		\begin{multicols}{4}
			\begin{parts}
				\part
				\ifprintanswers{  \opmul[hfactor=decimal,resultstyle=\color{red},displayintermediary=None]{43}{7} }
				\else{           \opmul[hfactor=decimal,resultstyle=\color{white},displayintermediary=None]{43}{7} }
				\fi\\[1cm]

				\part
				\ifprintanswers{  \opmul[hfactor=decimal,resultstyle=\color{red},displayintermediary=None]{1863}{6} }
				\else{           \opmul[hfactor=decimal,resultstyle=\color{white},displayintermediary=None]{1863}{6} }
				\fi\\[1cm]

				\part
				\ifprintanswers{  \opmul[hfactor=decimal,resultstyle=\color{red},displayintermediary=None]{152}{4} }
				\else{           \opmul[hfactor=decimal,resultstyle=\color{white},displayintermediary=None]{152}{4} }
				\fi\\[1cm]

				\part
				\ifprintanswers{  \opmul[hfactor=decimal,resultstyle=\color{red},displayintermediary=None]{2145}{5} }
				\else{           \opmul[hfactor=decimal,resultstyle=\color{white},displayintermediary=None]{2145}{5} }
				\fi\\[1cm]

				\part
				\ifprintanswers{  \opmul[hfactor=decimal,resultstyle=\color{red},displayintermediary=None]{512}{9} }
				\else{           \opmul[hfactor=decimal,resultstyle=\color{white},displayintermediary=None]{512}{9} }
				\fi\\[1cm]

				\part
				\ifprintanswers{  \opmul[hfactor=decimal,resultstyle=\color{red},displayintermediary=None]{34}{28} }
				\else{           \opmul[hfactor=decimal,resultstyle=\color{white},displayintermediary=None]{34}{28} }
				\fi\\[1cm]

				\part
				\ifprintanswers{  \opmul[hfactor=decimal,resultstyle=\color{red},displayintermediary=None]{321}{8} }
				\else{           \opmul[hfactor=decimal,resultstyle=\color{white},displayintermediary=None]{321}{8} }
				\fi\\[1cm]

				\part
				\ifprintanswers{  \opmul[hfactor=decimal,resultstyle=\color{red},displayintermediary=None]{45}{54} }
				\else{           \opmul[hfactor=decimal,resultstyle=\color{white},displayintermediary=None]{45}{54} }
				\fi\\[1cm]
			\end{parts}
		\end{multicols}
	}


	\addcontentsline{toc}{subsection}{Divisiones}
	\subsection*{Divisiones}
	% \subsection*{\ifprintanswers{Divisiones del 1 al 5                      }\else{}\fi}
	% \subsection*{\ifprintanswers{Divisiones del 6 al 10                     }\else{}\fi}
	% \subsection*{\ifprintanswers{Divisiones sin residuos                    }\else{}\fi}
	% \subsection*{\ifprintanswers{Divisiones con residuo 1                   }\else{}\fi}
	% \subsection*{\ifprintanswers{Divisiones con residuo 2                   }\else{}\fi}

	\questionboxed[8]{Realiza las siguientes divisiones:

		\begin{multicols}{4}
			\begin{parts}
				\part \ifprintanswers{\large\opidiv{20}{4}\\[2em]}
				\else{          \Large  \quad $4 \overline{) \ 20\ }$\\[4em]}
				\fi

				\part \ifprintanswers{\large\opidiv{193}{7}\\[2em]}
				\else{          \Large  \quad $7 \overline{) \ 193\ }$\\[4em]}
				\fi

				\part \ifprintanswers{\large\opidiv{10}{2}\\[2em]}
				\else{          \Large  \quad $2 \overline{) \ 10\ }$\\[4em]}
				\fi

				\part \ifprintanswers{\large\opidiv{432}{9}\\[2em]}
				\else{          \Large  \quad $9 \overline{) \ 432\ }$\\[4em]}
				\fi

				\part \ifprintanswers{\large\opidiv{23}{6}\\[2em]}
				\else{          \Large  \quad $6 \overline{) \ 283\ }$\\[4em]}
				\fi

				\part \ifprintanswers{\large\opidiv{644}{8}\\[2em]}
				\else{          \Large  \quad $8 \overline{) \ 644\ }$\\[4em]}
				\fi

				\part \ifprintanswers{\large\opidiv{95}{5}\\[2em]}
				\else{          \Large  \quad $5 \overline{) \ 95\ }$\\[4em]}
				\fi

				\part \ifprintanswers{\large\opidiv{656}{7}\\[2em]}
				\else{          \Large  \quad $7 \overline{) \ 656\ }$\\[4em]}
				\fi
			\end{parts}
		\end{multicols}
	}

	\addcontentsline{toc}{subsection}{Introducción a las fracciones}
	\subsection*{Introducción a las fracciones}
		% \subsection*{\ifprintanswers{Clasificación de fracciones                }\else{}\fi}

	\questionboxed[8]{Clasifica las siguientes fracciones en propias, impropias o mixtas:

		\begin{multicols}{4}
			\begin{parts}
				% \part $\dfrac{5}{6}$   \fillin[Propia][1cm]   \\[1em]
				\part $5\dfrac{5}{11}$ \fillin[Mixta][1cm]    \\[1em]
				\part $\dfrac{7}{3}$   \fillin[Impropia][1cm] \\[1em]
				% \part $\dfrac{3}{4}$   \fillin[Propia][1cm]   \\[1em]
				\part $1\dfrac{2}{3}$  \fillin[Mixta][1cm]    \\[1em]
				\part $\dfrac{7}{5}$   \fillin[Impropia][1cm] \\[1em]
				\part $\dfrac{7}{8}$   \fillin[Propia][1cm]   \\[1em]
				\part $3\dfrac{2}{9}$  \fillin[Mixta][1cm]    \\[1em]
				\part $\dfrac{3}{2}$   \fillin[Impropia][1cm] \\[1em]
				\part $4\dfrac{1}{4}$ \fillin[Mixta][1cm]   \\[1em]
			\end{parts}
		\end{multicols}
	}

	% \subsection*{\ifprintanswers{Representación de fracciones               }\else{}\fi}

	\questionboxed[5]{Escribe sobre la línea la fracción que representa cada imagen:

		\begin{multicols}{5}
			\begin{parts}
				\part \includegraphics[width=45px]{../images/imagen_frac01.png} \fillin[\fbox{$\dfrac{10}{20}$}][0in] \\[1em]
				\part \includegraphics[width=45px]{../images/imagen_frac02.png} \fillin[\fbox{$\dfrac{5}{6}$}][0in] \\[1em]
				\part \includegraphics[width=45px]{../images/imagen_frac03.png} \fillin[\fbox{$\dfrac{11}{16}$}][0in] \\[1em]
				\part \includegraphics[width=45px]{../images/imagen_frac04.png} \fillin[\fbox{$\dfrac{14}{16}$}][0in] \\[1em]
				\part \includegraphics[width=45px]{../images/imagen_frac05.png} \fillin[\fbox{$\dfrac{17}{20}$}][0in] \\[1em]
				\part \includegraphics[width=45px]{../images/imagen_frac06.png} \fillin[\fbox{$\dfrac{9}{16}$}][0in] \\[1em]
				\part \includegraphics[width=45px]{../images/imagen_frac07.png} \fillin[\fbox{$\dfrac{15}{20}$}][0in] \\[1em]
				\part \includegraphics[width=45px]{../images/imagen_frac08.png} \fillin[\fbox{$\dfrac{7}{8}$}][0in] \\[1em]
				\part \includegraphics[width=45px]{../images/imagen_frac09.png} \fillin[\fbox{$\dfrac{13}{20}$}][0in] \\[1em]
				\part \includegraphics[width=45px]{../images/imagen_frac11.png} \fillin[\fbox{$\dfrac{1}{20}$}][0in] \\[1em]
			\end{parts}
		\end{multicols}
	}


	% \subsection*{\ifprintanswers{Nombre de fracciones                       }\else{}\fi}

	\questionboxed[5]{Escribe la fracción que corresponda en cada inciso:

		\begin{parts}
			\part ¿Cómo se escribe numéricamente la fracción \textbf{ocho quintos}?    \fillin[$\dfrac{8}{5}$][0in]  \\
			\part ¿Cómo se escribe numéricamente la fracción \textbf{seis onceavos}?   \fillin[$\dfrac{6}{11}$][0in] \\
			\part ¿Cómo se escribe numéricamente la fracción \textbf{dos séptimos}?    \fillin[$\dfrac{2}{7}$][0in]  \\
			\part ¿Cómo se escribe numéricamente la fracción \textbf{once medios}?     \fillin[$\dfrac{11}{2}$][0in] \\
			\part ¿Cómo se escribe numéricamente la fracción \textbf{diez décimos}?    \fillin[$\dfrac{10}{10}$][0in]\\
		\end{parts}
	}

	% \subsection*{\ifprintanswers{Conversión de fracciones mixtas a impropias}\else{}\fi}

	\questionboxed[3]{Convierte la siguientes fracciones mixtas a impropias:

		\begin{multicols}{3}
			\begin{parts}\large
				\part $4\dfrac{2}{3}= $ \fillin[$\dfrac{14}{3}$][0in]
				\part $2\dfrac{3}{10}= $ \fillin[$\dfrac{23}{10}$][0in]
				\part $5\dfrac{1}{5}= $ \fillin[$\dfrac{26}{5}$][0in]
			\end{parts}
		\end{multicols}
	}

	% \subsection*{\ifprintanswers{Conversión de fracciones impropias a mixtas}\else{}\fi}

	\questionboxed[3]{Convierte la siguientes fracciones impropias a mixtas:

		\begin{multicols}{3}
			\begin{parts}\large
				\part $\dfrac{13}{3}= $ \fillin[$4\dfrac{1}{3}$][0in]
				\part $\dfrac{63}{10}= $ \fillin[$6\dfrac{3}{10}$][0in]
				\part $\dfrac{51}{5}= $ \fillin[$10\dfrac{1}{5}$][0in]
			\end{parts}
		\end{multicols}
	}

	\addcontentsline{toc}{subsection}{Operaciones con fracciones}
	\subsection*{Operaciones con fracciones}
		% \subsection*{\ifprintanswers{Suma de fracciones                         }\else{}\fi}
	% \subsection*{\ifprintanswers{Resta de fracciones                        }\else{}\fi}
	% \subsection*{\ifprintanswers{Multiplicación de fracciones               }\else{}\fi}
	% \subsection*{\ifprintanswers{División de fracciones                     }\else{}\fi}
	% \subsection*{\ifprintanswers{Operaciones de fracciones mixtas           }\else{}\fi}

	\questionboxed[8]{Realiza las siguientes operaciones.

		\begin{multicols}{2}
			\begin{parts}\large
				\part $\dfrac{3}{5}+\dfrac{4}{5}=$ \fillin[$\dfrac{7}{5} = 1\dfrac{2}{5}$][0in] \\[2em]
				\part $\dfrac{13}{6}-\dfrac{5}{6}=$ \fillin[$\dfrac{8}{6}=\dfrac{4}{3}$][0in] \\[2em]
				\part $\dfrac{12}{7}-\dfrac{5}{7}=$ \fillin[$\dfrac{7}{7}=1$][0in] \\[2em]
				\part $1\dfrac{1}{8}+1\dfrac{7}{8}=$ \fillin[$2\dfrac{8}{8} = 3$][0in] \\[2em]
				\part $\dfrac{3}{5}\times\dfrac{2}{3}=$ \fillin[$\dfrac{6}{15}$][0in]   \\[2em]
				\part $\dfrac{7}{8}\times\dfrac{3}{4}=$ \fillin[$\dfrac{21}{32}$][0in] \\[2em]
				\part $\dfrac{3}{5} \divisionsymbol\dfrac{2}{3}=$ \fillin[$\dfrac{9}{10}$][0in] \\[2em]
				\part $\dfrac{7}{8} \divisionsymbol\dfrac{3}{4}=$ \fillin[$\dfrac{28}{24}$][0in]	\\[2em]
			\end{parts}
		\end{multicols}
	}
\end{questions}
\end{document}