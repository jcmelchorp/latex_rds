\documentclass[12pt,addpoints]{evalua}
\grado{1$^\circ$ de Secundaria}
\cicloescolar{2024-2025}
\materia{Matemáticas 1 \normalfont \color{darkgray} \small con adecuación curricular a \\Matemáticas 3$^\circ$ de Primaria.}
\unidad{1, 2 y 3}
\title{Examen de la Unidad}
\aprendizajes{\tiny%
   \item Expresa oralmente la sucesión numérica hasta cuatro cifras, en español y hasta donde sea posible, en su lengua materna, de manera ascendente y descendente a partir de un número natural dado.\\[-2em]
   \item Representa, con apoyo de material concreto y modelos gráficos, fracciones: medios, cuartos, octavos, dieciseisavos, para expresar el resultado de mediciones y repartos en situaciones vinculadas a su contexto.\\[-2em]
   \item Resuelve situaciones problemáticas vinculadas a su contexto que implican sumas de números naturales de hasta tres cifras utilizando el algoritmo convencional.\\[-2em]
   \item Resuelve problemas de suma o resta vinculados a su contexto, que impliquen el uso de fracciones (medios, cuartos, octavos, dieciseisavos), con el apoyo de material concreto o representaciones gráficas.\\[-2em]
   \item Resuelve multiplicaciones cuyo producto es un número natural de tres cifras, mediante diversos procedimientos (suma de multiplicaciones parciales, multiplicaciones por 10, 20, 30, entre otros); además, divisiones (reparto y agrupamiento), mediante diversos procedimientos, en particular con la multiplicación; representa la división como: a ÷ b = c.\\[-2em]
   \item A partir de retículas de triángulos, cuadrados o puntos, construye, analiza y clasifica figuras geométricas a partir de sus lados y su simetría, en particular a los triángulos; explica los criterios utilizados para la clasificación.\\[-2em]
   \item Resuelve situaciones problemáticas vinculadas a su contexto que impliquen, medición, estimación y comparación, de longitudes, masas y capacidades, con el uso del metro, kilogramo, litro y medios y cuartos de estas unidades; en el caso de la longitud, el decímetro y centímetro.\\[-2em]
      }
\author{Prof.: Julio César Melchor Pinto}
\begin{document}
\begin{questions}
	\question[10]
	% UNIDAD 1                  
	% \section*{\ifprintanswers{Escritura de cantidades}\else{}\fi}
	% \subsection*{\ifprintanswers{Escritura de cantidades 1 }\else{}\fi}
	% \subsection*{\ifprintanswers{Escritura de cantidades 2 }\else{}\fi}
	% \subsection*{\ifprintanswers{Escritura de cantidades 3 }\else{}\fi}
	% \subsection*{\ifprintanswers{Escritura de cantidades 4 }\else{}\fi}
	% \subsection*{\ifprintanswers{Escritura de cantidades 5 }\else{}\fi}
	% \section*{\ifprintanswers{Sistema decimal 1}\else{}\fi}
	% \subsection*{\ifprintanswers{Posicionamiento decimal 1 }\else{}\fi}
	% \subsection*{\ifprintanswers{Posicionamiento decimal 2 }\else{}\fi}
	% \subsection*{\ifprintanswers{Notación desarrollada 1 }\else{}\fi}
	% \subsection*{\ifprintanswers{Notación desarrollada 2 }\else{}\fi}
	% \subsection*{\ifprintanswers{Notación desarrollada 3 }\else{}\fi}
	% \section*{\ifprintanswers{Sistema decimal 2}\else{}\fi}
	% \subsection*{\ifprintanswers{Posicionamiento decimal 1 }\else{}\fi}
	% \subsection*{\ifprintanswers{Posicionamiento decimal 2 }\else{}\fi}
	% \subsection*{\ifprintanswers{Notación desarrollada 1 }\else{}\fi}
	% \subsection*{\ifprintanswers{Notación desarrollada 2  }\else{}\fi}
	% \subsection*{\ifprintanswers{Posicionamiento decimal y Notación desarrollada }\else{}\fi}
	% \section*{\ifprintanswers{Tablas de multiplicar 1}\else{}\fi}
	% \subsection*{\ifprintanswers{Tabla del 1  }\else{}\fi}
	% \subsection*{\ifprintanswers{Tabla del 2 }\else{}\fi}
	% \subsection*{\ifprintanswers{Tabla del 3 }\else{}\fi}
	% \subsection*{\ifprintanswers{Tabla del 4 }\else{}\fi}
	% \subsection*{\ifprintanswers{Tabla del 5 }\else{}\fi}

	% UNIDAD 2

	% \section*{\ifprintanswers{Tablas de multiplicar 2        }\else{}\fi}
	% \subsection*{\ifprintanswers{Tabla del 6                    }\else{}\fi}
	% \subsection*{\ifprintanswers{Tabla del 7                    }\else{}\fi}
	% \subsection*{\ifprintanswers{Tabla del 8                    }\else{}\fi}
	% \subsection*{\ifprintanswers{Tabla del 9                    }\else{}\fi}
	% \subsection*{\ifprintanswers{Tabla del 10                   }\else{}\fi}
	% \section*{\ifprintanswers{Sumas 1                        }\else{}\fi}
	% \subsection*{\ifprintanswers{Sumando con 1, 2 y 3           }\else{}\fi}
	% \subsection*{\ifprintanswers{Sumando con 4, 5 y 6           }\else{}\fi}
	% \subsection*{\ifprintanswers{Sumando con 7, 8 y 9           }\else{}\fi}
	% \subsection*{\ifprintanswers{Sumando números entre 10 y 20  }\else{}\fi}
	% \subsection*{\ifprintanswers{Sumando números entre 20 y 50  }\else{}\fi}
	% \section*{\ifprintanswers{Sumas 2                        }\else{}\fi}
	% \subsection*{\ifprintanswers{Sumas hasta el 100             }\else{}\fi}
	% \subsection*{\ifprintanswers{Sumas hasta el 500             }\else{}\fi}
	% \subsection*{\ifprintanswers{Sumas con acarreos 1           }\else{}\fi}
	% \subsection*{\ifprintanswers{Sumas con acarreos 2           }\else{}\fi}
	% \subsection*{\ifprintanswers{Sumas con acarreos 3           }\else{}\fi}
	% \section*{\ifprintanswers{Restas 1                       }\else{}\fi}
	% \subsection*{\ifprintanswers{Restando con 1, 2 y 3          }\else{}\fi}
	% \subsection*{\ifprintanswers{Restando con 4, 5, 6, 7, 8 y 9 }\else{}\fi}
	% \subsection*{\ifprintanswers{Restas como sumas 1            }\else{}\fi}
	% \subsection*{\ifprintanswers{Restas como suma 2             }\else{}\fi}
	% \subsection*{\ifprintanswers{Restas como suma 3             }\else{}\fi}
	% \section*{\ifprintanswers{Restas 2                       }\else{}\fi}
	% \subsection*{\ifprintanswers{Restas sin transformación 1    }\else{}\fi}
	% \subsection*{\ifprintanswers{Restas sin transformación 2    }\else{}\fi}
	% \subsection*{\ifprintanswers{Restas con transformación 1    }\else{}\fi}
	% \subsection*{\ifprintanswers{Restas con transformación 2    }\else{}\fi}
	% \subsection*{\ifprintanswers{Restas con transformación 3    }\else{}\fi}

	% UNIDAD 3

	% \section*{\ifprintanswers{Restas 3                                   }\else{}\fi}
	% \subsection*{\ifprintanswers{Restas con transformación 1                }\else{}\fi}
	% \subsection*{\ifprintanswers{Restas con transformación 2                }\else{}\fi}
	% \subsection*{\ifprintanswers{Minuendos múltiplos de 10                  }\else{}\fi}
	% \subsection*{\ifprintanswers{Minuendos con ceros intermedios            }\else{}\fi}
	% \subsection*{\ifprintanswers{Repaso de restas                           }\else{}\fi}
	% \section*{\ifprintanswers{Multiplicaciones                           }\else{}\fi}
	% \subsection*{\ifprintanswers{Multiplicaciones con una cifra 1           }\else{}\fi}
	% \subsection*{\ifprintanswers{Multiplicaciones con una cifra 2           }\else{}\fi}
	% \subsection*{\ifprintanswers{Multiplicaciones con una cifra 3           }\else{}\fi}
	% \subsection*{\ifprintanswers{Multiplicaciones con una cifra 4           }\else{}\fi}
	% \subsection*{\ifprintanswers{Multiplicaciones con dos cifras            }\else{}\fi}
	% \section*{\ifprintanswers{Divisiones                                 }\else{}\fi}
	% \subsection*{\ifprintanswers{Divisiones del 1 al 5                      }\else{}\fi}
	% \subsection*{\ifprintanswers{Divisiones del 6 al 10                     }\else{}\fi}
	% \subsection*{\ifprintanswers{Divisiones sin residuos                    }\else{}\fi}
	% \subsection*{\ifprintanswers{Divisiones con residuo 1                   }\else{}\fi}
	% \subsection*{\ifprintanswers{Divisiones con residuo 2                   }\else{}\fi}
	% \section*{\ifprintanswers{Introducción a las fracciones              }\else{}\fi}
	% \subsection*{\ifprintanswers{Clasificación de fracciones                }\else{}\fi}
	% \subsection*{\ifprintanswers{Representación de fracciones               }\else{}\fi}
	% \subsection*{\ifprintanswers{Nombre de fracciones                       }\else{}\fi}
	% \subsection*{\ifprintanswers{Conversión de fracciones mixtas a impropias}\else{}\fi}
	% \subsection*{\ifprintanswers{Conversión de fracciones impropias a mixtas}\else{}\fi}
	% \section*{\ifprintanswers{Operaciones con fracciones                 }\else{}\fi}
	% \subsection*{\ifprintanswers{Suma de fracciones                         }\else{}\fi}
	% \subsection*{\ifprintanswers{Resta de fracciones                        }\else{}\fi}
	% \subsection*{\ifprintanswers{Multiplicación de fracciones               }\else{}\fi}
	% \subsection*{\ifprintanswers{División de fracciones                     }\else{}\fi}
	% \subsection*{\ifprintanswers{Operaciones de fracciones mixtas           }\else{}\fi}
\end{questions}
\end{document}