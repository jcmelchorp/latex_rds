Los sistemas de radar detectan objetos a partir de la reflexión de la
radiación electromagnética. En los aeropuertos se usan radares para rastrear
aviones grandes. En las estaciones meteorológicas también se usan para
monitorear las pequeñas gotas de agua que forman las tormentas.
Los sistemas de radar de los aeropuertos pueden identificar y rastrear
aviones, incluso cuando hay tormentas en el área. ¿Por qué es esto posible?

\begin{choices}
    \CorrectChoice Las longitudes de onda con las que funcionan los sistemas de radar de
    los aeropuertos son mucho más grandes que las longitudes de onda con las que
    funcionan las estaciones meteorológicas.
    \choice La radiación electromagnética con la que funcionan los sistemas de
    radar de los aeropuertos se refleja en el metal, mientras que la radiación
    electromagnética con la que funcionan las estaciones meteorológicas se refleja
    en el agua.
    \choice Las longitudes de onda con las que funcionan los sistemas de radar de
    los aeropuertos son mucho más pequeñas que las longitudes de onda con las que
    funcionan las estaciones meteorológicas.
\end{choices}