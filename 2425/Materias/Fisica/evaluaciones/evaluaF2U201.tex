\documentclass[12pt,addpoints,answers]{evalua}
\grado{2$^\circ$ de Secundaria}
\cicloescolar{2022-2023}
\materia{Ciencias y Tecnología: Física}
\unidad{2}
\title{Examen de la Unidad}
\aprendizajes{
    \item Describe, representa y experimenta la fuerza como la interacción
    entre
    objetos y reconoce distintos tipos de fuerza.
    \item Identifica y describe la presencia de fuerzas en interacciones
    cotidianas
    (fricción, flotación, fuerzas en equilibrio).
    \item Analiza la gravitación y su papel en
    la explicación del movimiento de los
    planetas y en la caída de los cuerpos
    (atracción) en la superficie terrestre.
    \item Analiza la energía mecánica (cinética y potencial) y describe
    casos
    donde se conserva.
}
\author{Prof.: Julio César Melchor Pinto}
\begin{document}
\begin{multicols}{2}
    \include*{../blocks/block003}
    \include*{../blocks/block001}
    \include*{../blocks/block000}
    \include*{../blocks/block002}
\end{multicols}
\include*{../blocks/block004}
\begin{questions}
    \question[20] \include*{../questions/question009}
    \newpage
    % \include*{../questions/question029}
    \question[20] \include*{../questions/question030}
    % \end{multicols}
    % \include*{../questions/question032}
    \newpage
    \question[20] \include*{../questions/question031}
    %\include*{../questions/question033}
    \newpage
    % \begin{multicols}{2}
    \question[20] \include*{../questions/question034}
    % \end{multicols}
    \newpage
    \question[20] \include*{../questions/question035}
\end{questions}
\end{document}