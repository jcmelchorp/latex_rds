\begin{sectionbox}{Riesgos de la electricidad en nuestro cuerpo}
Es importante destacar que la electricidad que interviene en el funcionamiento de nuestro cuerpo es de muy baja intensidad, mucho menor que la que obtenemos de las tomas
de corriente de nuestras casas. Es vital, entonces, tener presente que la interacción del
cuerpo con la electricidad puede generar efectos negativos que van desde hormigueos,
calambres leves y contracciones musculares (como las que experimentaron las ancas
de la rana de Galvani) hasta paros cardiacos, respiratorios, quemaduras severas
e incluso la muerte.

\renewcommand{\arraystretch}{1.2}
\begin{table}[H]
    \centering
    \caption{Efectos de la corriente eléctrica en el cuerpo humano}
    \label{tab:efectos}
    \begin{tabular}{|>{\centering}p{0.15\linewidth}|p{0.8\linewidth}|}
        \toprule
        \textbf{Intensidad de corriente (mA)} & \textbf{Efecto} \\\midrule                                                                                                                                                     
            0 - 0.5       & No se observan sensaciones ni efectos; el umbral de percepción se sitúa en 0.5 mA.                                                                              \\\hline
            0.5 - 10      & Calambres y movimientos reflejos musculares; el umbral de no soltar un conductor con corriente se ubica en 10 mA.                                               \\\hline
            10 - 25       & Contracciones musculares; agarrotamiento de brazos y piernas con dificultad para soltar objetos; aumento de la presión arterial y dificultades respiratorias.   \\\hline
            25 - 40       & Contracción muscular fuerte; irregularidades cardíacas; quemaduras; asfixia a partir de 4 s de exposición a esa corriente.                                      \\\hline
            40 - 100      & Efectos anteriores con mayor intensidad y gravedad; fibrilación y arritmias cardíacas.                                                                          \\\hline
            1 000         & Fibrilación y paro cardiaco; quemaduras muy graves; alto riesgo de muerte.                                                                                      \\\hline
            1 000 - 5 000 & Quemaduras muy graves; paro cardíaco con elevada probabilidad de muerte.                   \\           \hline                                                         
        \bottomrule
    \end{tabular}
\end{table}

Un choque eléctrico es el paso de corriente eléctrica a través del cuerpo humano,
y para que la corriente fluya por nuestro organismo es necesario que éste forme
parte de un circuito eléctrico cerrado y actué como conductor. En condiciones normales el cuerpo humano es un buen conductor de corriente eléctrica, por lo que los
electricistas y quienes trabajan con electricidad deben usar guantes de plástico y zapatos de goma que los aíslen de la corriente. La tabla \ref{tab:efectos} muestra los efectos de diferentes intensidades de corriente en el cuerpo humano.
\end{sectionbox}