\begin{infocard}{Energía}
    La   \textbf{Energía cinética}a de un cuerpo en movimiento depende de
    dos variables o magnitudes f\'isicas: su masa (m) y su rapidez (v). La ecuaci\'on
    que relaciona ambas variables y define a la energ\'ia cin\'etica ($E_C$) es:

    \[E_c=\frac{1}{2}mv^2 \]

    Las unidades de la energía se llama Joule (J). Como sabes, la unidad de fuerza es el
    newton (N), y 1 N equivale a 1 kg m/s$^2$, de manera que: $1 J = 1 kg m^2/s^2 = 1 Nm $
    \tcblower

    La \textbf{Energía potencial} gravitacional ($E_P$) involucra
    a la masa de un cuerpo (m), la altura a la que se encuentra con respecto al
    marco de referencia (h) y la aceleraci\'on de la gravedad (g):
    \[E_p=mgh\]
    \DrawLine

    La \textbf{Energía mecánica} depende de la energ\'ia
    cin\'etica y de la energ\'ia potencial de acuerdo con la siguiente expresi\'on:
    \begin{equation*}
        E_m=E_c+E_p
    \end{equation*}
\end{infocard}