Relaciona cada grupo de galaxias con su descripción.

% \begin{multicols}{2}
\begin{minipage}{0.67\textwidth}
    \begin{parts}
        \part Grupo formado por la Vía Láctea y unas 30 galaxias más.\dotfill\faIcon[regular]{square}
        \part Son cúmulos de galaxias.\dotfill\faIcon[regular]{square}
        \part Grupo formado por la Vía Láctea y otras 14 galaxias gigantes que integra una estructura en forma de anillo. \dotfill\faIcon[regular]{square}
        \part Grupo de galaxias cuyos tamaños típicos son de 2 a 3 Mpc.\dotfill\faIcon[regular]{square}
    \end{parts}
\end{minipage}%%
\begin{minipage}{0.35\textwidth}
    \begin{checkboxes}
        \choice Supercúmulo \\
        \choice Grupo local \\
        \choice Cúmulos de galaxias \\
        \choice Concilio de Gigantes
    \end{checkboxes}
\end{minipage}
% \begin{parts}\large
%     \part Grupo formado por la Vía Láctea y unas 30 galaxias más.\hfill\faIcon[regular]{square}
%     \part Son cúmulos de galaxias.\hfill\faIcon[regular]{square}
%     \part Grupo formado por la Vía Láctea y otras 14 galaxias gigantes que integra una estructura en forma de anillo. \hfill\faIcon[regular]{square}
%     \part Grupo de galaxias cuyos tamaños típicos son de 2 a 3 Mpc.\hfill\faIcon[regular]{square}
% \end{parts}

%\columnbreak
% \begin{checkboxes}\Large
%     \choice Supercúmulo	     \vspace{0.5cm}
%     \choice Concilio de Gigantes \vspace{0.5cm}
%     \choice Cúmulos de galaxias  \vspace{0.5cm}
%     \choice Grupo local
% \end{checkboxes}
% \end{multicols}