Con base en tu entendimiento de las fuerzas, contesta las siguientes preguntas
argumentando tu respuesta.

\begin{multicols}{2}
    \begin{parts}
        \part ¿Cómo identificas cuando un objeto cambia su estado de movimiento?

        \begin{solutionbox}{2.8cm}\footnotesize%
            Respuestas aceptadas:
            \begin{enumerate}
                \item Cuando se acelera (cuando hay un cambio de velocidad).
                \item Cuando existe una fuerza.
            \end{enumerate}
        \end{solutionbox}

        \part ¿Qué origina que un objeto cambie el estado de movimiento del punto
        anterior?

        \begin{solutionbox}{1.5cm}\footnotesize%
            La interacción con algo más, como cuando existe una fuerza.
        \end{solutionbox}

        \part ¿Por qué las naves y sondas espaciales pueden mantener su movimiento?

        \begin{solutionbox}{1.5cm}\footnotesize%
            Porque no interactuan con nada que modifique su inercia.
        \end{solutionbox}

        \part ¿Qué relación existe entre el plano inclinado y la cuña?

        \begin{solutionbox}{1.5cm}\footnotesize%
            La cuña son dos planos inclinados juntos.
        \end{solutionbox}
    \end{parts}
\end{multicols}