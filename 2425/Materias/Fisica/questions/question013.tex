En un experimento se lanza verticalmente hacia arriba una pelota de 200 gramos y se determina que alcanza una altura de 65 metros. Determine la energía potencial de la pelota.

\begin{solutionbox}{5.5cm}
    \begin{multicols}{2}
        Datos:

        E$_p$ = ?

        h = 65 m

        g=9.8 m/s$^2$

        m = 200 gr = 0.200 kg

        La energía potencial es:
        \[E_p=mgh\]

        \vspace{2cm}

        Sustituyendo nuestros datos en la fórmula:
        \[
            \begin{array}{rl}
                E_p & = (0.200 \text{ kg})(9.8 \text{ m/s$^2$})(65 \text{ m}) \\[1em]
                    & =127.4 \text{ J }
            \end{array}
        \]
        La energía potencial de la pelota es de 127.4 Joules.
    \end{multicols}
\end{solutionbox}
