Elige la respuesta correcta a cada enunciado.

\begin{parts}
    Es un sistema de estrellas, gas y polvo interestelar que orbita en torno a un centro de gravedad. (pág. \pageref{085a_a})

    \begin{oneparchoices}
        \choice Cúmulo
        \CorrectChoice Galaxia
        \choice Nebulosa
        \choice Pulsar
    \end{oneparchoices}

    Unidad conveniente para medir el tamaño de las galaxias. (pág. \pageref{085a_b})

    \begin{oneparchoices}
        \choice Metro
        \choice Millas aéreas
        \choice Kilómetro
        \CorrectChoice Año Luz
    \end{oneparchoices}

    Es la magnitud que mide un año luz. (pág. \pageref{085a_c})

    \begin{oneparchoices}
        \choice Tiempo
        \choice Masa
        \CorrectChoice Longitud
        \choice Energía
    \end{oneparchoices}

    Número aproximado de galaxias en el Universo.(pág. \pageref{085a_d})

    \begin{oneparchoices}
        \choice miles
        \CorrectChoice billones
        \choice millones
        \choice trillones
    \end{oneparchoices}

    Proporción detectable de una galaxia por medio de las ondas electromagnéticas. (pág. \pageref{085a_e})

    \begin{oneparchoices}
        \CorrectChoice 10\%
        \choice 20\%
        \choice 30\%
        \choice 40\%
    \end{oneparchoices}

    Masa que no es detectable por medio de las ondas electromagnéticas de la cual se conoce su existencia
    por la manera en que afecta a la rotación de las estrellas más alejadas del núcleo galáctico. (pág. \pageref{085a_f})

    \begin{oneparchoices}
        \choice Energía oscura
        \CorrectChoice Materia oscura
        \choice Fuerza oscura
        \choice Masa oscura
    \end{oneparchoices}
\end{parts}