Señala si son verdaderas o falsas las siguientes afirmaciones.

\begin{parts}
    \part Cuando se viaja de norte a sur, o viceversa, la altura aparente de las estrellas cambia.

    \begin{choices}
        \CorrectChoice Verdadero
        \choice Falso
    \end{choices}

    \part La sombra que la Tierra proyecta sobre la Luna en los eclipses lunares es un argumento sobre la redondez de la Tierra.

    \begin{choices}
        \CorrectChoice Verdadero
        \choice Falso
    \end{choices}


    \part La Tierra no rota sobre su propio eje porque nosotros no percibimos que nos estamos moviendo.

    \begin{choices}
        \choice Verdadero
        \CorrectChoice Falso
    \end{choices}

    \part En un eclipse solar se observa que la Luna pasa delante del Sol y que ambos tienen un tamaño en apariencia iguales.
    De ello se concluye que el Sol está a la misma distancia que la Luna.

    \begin{choices}
        \choice Verdadero
        \CorrectChoice Falso
    \end{choices}

    \part El hecho de que en el mar primero desaparece el casco y luego la vela de un navío es un argumento sobre la redondez de la Tierra.

    \begin{choices}
        \CorrectChoice Verdadero
        \choice Falso
    \end{choices}

\end{parts}
