Señala si son verdaderos o falsos los siguientes enunciados.
\begin{parts}
    El hipotálamo, entre otras funciones, controla los mecanismos que regulan la temperatura corporal.
        {\footnotesize
            \begin{choices}
                \choice Verdadero
                \choice Falso
            \end{choices}
        }

    La vasodilatación es un mecanismo que emplea el cuerpo para conservar el calor.
        {\footnotesize
            \begin{choices}
                \choice Verdadero
                \choice Falso
            \end{choices}
        }

    La vasoconstricción es un mecanismo destinado a la dispersión de la temperatura corporal.
        {\footnotesize
            \begin{choices}
                \choice Verdadero
                \choice Falso
            \end{choices}
        }

    La disminución de la temperatura del cuerpo por debajo de los 35 °C puede ocasionar la pérdida de conciencia e incluso la muerte.
        {\footnotesize
            \begin{choices}
                \choice Verdadero
                \choice Falso
            \end{choices}
        }

    La sudoración y el jadeo son mecanismos que el cuerpo emplea para conservar su temperatura.
        {\footnotesize
            \begin{choices}
                \choice Verdadero
                \choice Falso
            \end{choices}
        }

    El aumento de la temperatura corporal del cuerpo por encima de los 37 °C puede ocasionar la muerte.
        {\footnotesize
            \begin{choices}
                \choice Verdadero
                \choice Falso
            \end{choices}
        }
\end{parts}