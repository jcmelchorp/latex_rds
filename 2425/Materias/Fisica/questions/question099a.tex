[20] Elige la respuesta correcta para cada pregunta.
\begin{parts}
    ¿Cuál de las siguientes opciones describe la forma en que los astrónomos conciben al Universo según la teoría de la gran explosión?

    \begin{choices}
        \choice El Universo es un fluido homogéneo y estático que siempre ha existido.
        \choice El Universo es un fluido heterogéneo, estático y de inmensas proporciones.
        \choice El Universo nació cuando la estrella primigenia agotó su combustible y explotó dando lugar a todo lo que existe.
        \CorrectChoice El Universo es un fluido homogéneo y en expansión, constituido de radiación electromagnética y materia.
    \end{choices}

    Según la teoría de la gran explosión, actualmente el Universo se encuentra en expansión. ¿Cuál de las siguientes proposiciones permite deducir esto?

    \begin{choices}
        \choice En el principio el Universo era un lugar inhóspito, estático y frio.
        \CorrectChoice En el principio el Universo estaba concentrado en un punto de densidad y temperatura prácticamente infinitas.
        \choice El Universo siempre se ha encontrado en expansión, y actualmente se está deteniendo.
        \choice El Universo siempre ha estado expandiéndose a la misma velocidad.
    \end{choices}

    \newpage

    ¿En qué consiste el evento conocido como inflación?

    \begin{choices}
        \choice Es el periodo de tiempo en el que aumentó más rápidamente la entropía del Universo.
        \choice Así llaman los astrónomos al evento en el que prevén tendrá lugar el final del Universo.
        \CorrectChoice Fue un periodo muy breve, sucedido inmediatamente después de la gran explosión, en el que el Universo se expandió muy rápidamente.
        \choice Se le llama así al per\'iodo de tiempo en el que aumentó más rápidamente la temperatura del Universo.
    \end{choices}

    ¿Cuál de las siguientes opciones caracteriza al periodo de tiempo conocido como Universo temprano?

    \begin{choices}
        \choice La temperatura del Universo aumentó drásticamente, dando lugar a las primeras supernovas.
        \CorrectChoice La temperatura del Universo aumentó drásticamente, dando lugar a las estrellas de neutrones.
        \choice La temperatura del Universo descendió a 1012 °C, lo que permitió que la materia comenzara a agruparse, formando los primeros átomos de hidrógeno y helio.
        \choice La temperatura del Universo se mantuvo estable, lo que dio lugar a que la materia se agrupara y formara elementos pesados, como Uranio, Plutonio y Oro.
    \end{choices}

    ¿Cuál de las siguientes opciones caracteriza al periodo conocido como Universo actual?

    \begin{choices}
        \choice La temperatura del Universo disminuyó drásticamente, dando lugar a las nubes moleculares.
        \choice La temperatura del Universo aumentó drásticamente, dando lugar a las estrellas de neutrones.
        \choice La temperatura del Universo descendió a 1012 °C, lo que permitió que la materia comenzara a agruparse, formando los primeros átomos de hidrógeno y helio.
        \CorrectChoice Durante esta etapa tuvieron lugar diferencias de temperatura (y por tanto de densidad) que condujeron a la acumulación de materia, lo cual originó las galaxias.
    \end{choices}

    \newpage
    ¿Cuál de las siguientes opciones representa una prueba a favor de la teoría de la gran explosión?
    \begin{choices}
        \choice La teoría especial de la relatividad de Albert Einstein.
        \choice La existencia de elementos pesados, como plutonio y oro.
        \choice La existencia de elementos ligeros, como hidrógeno y helio.
        \CorrectChoice La existencia de la radiación de microondas que llena todo el Universo actual, mejor conocida como radiación cósmica de fondo.
    \end{choices}

    ¿Cuál de las siguientes opciones representa una limitación de la teoría de la gran explosión?
    \begin{choices}
        \CorrectChoice No explica por qué el Universo comenzó con una entropía tan baja.
        \choice No explica por qué la radiación cósmica de fondo es isotrópica.
        \choice No explica cómo se formaron los elementos más pesados, como el plutonio o el oro.
        \choice No explica cómo se formaron los elementos más ligeros, como el hidrogeno y el helio.
    \end{choices}
    ¿Cuál de las siguientes teorías plantea que la materia se crea continuamente, y por lo cual el Universo permanece estable en el espacio y en el tiempo, a pesar de su proceso de expansión?
    \begin{choices}
        \choice Teoría de la gran explosión.
        \choice Teoría del decaimiento fotónico.
        \CorrectChoice Teoría del estado estacionario.
        \choice Teoría del gran colapso gravitacional.
    \end{choices}
    ¿Qué teoría afirma que no existe la radiación cósmica de fondo, argumentando que el fenómeno observado es consecuencia del proceso de dispersión de la luz?
    \begin{choices}
        \choice La teoría de la gran explosión
        \CorrectChoice La teoría del estado estacionario
        \choice La teoría del decaimiento fotónico
        \choice La teoría del gran colapso gravitacional
    \end{choices}

    % \newpage

    ¿Qué teoría plantea que el fenómeno de corrimiento al rojo se debe a que la luz pierde energía al viajar por el espacio, y que por lo tanto el Universo no se encuentra en expansión?
    \begin{choices}
        \choice La teoría del gran colapso
        \choice La teoría de la gran explosión
        \choice La teoría del estado estacionario
        \CorrectChoice La teoría del decaimiento fotónico
    \end{choices}
    ¿Cuál de las siguientes teorías plantea que si el Universo está en expansión, en un momento anterior debió ocupar un espacio muy reducido, cuya densidad y temperatura eran prácticamente infinitas?
    \begin{choices}
        \CorrectChoice La teoría de la gran explosión
        \choice La teoría del estado estacionario
        \choice La teoría del decaimiento fotónico
        \choice La teoría del gran colapso gravitacional
    \end{choices}
    ¿Por qué se dice que los aceleradores de partículas, como el Gran Colisionador de Hadrones, nos ayudan a entender cómo era el Universo en los primeros instantes de su existencia?
    \begin{choices}
        \choice Porque en los experimentos realizados ahí se demuestra que la teoría de la relatividad de Einstein es correcta.
        \choice Porque en su interior se crean de nuevo las partículas elementales que se encontraban extintas actualmente.
        \choice Porque en su interior se crean diminutos hoyos negros supermasivos, como los que tienen las galaxias en sus centros.
        \CorrectChoice Porque en ellos se reproducen las condiciones de temperatura y presión que caracterizaron los primeros instantes de vida del Universo.
    \end{choices}
\end{parts}