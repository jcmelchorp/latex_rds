El parsec (pc) puede definirse a partir del año luz como: $1 \text{ pc} = 3.26 \text{ años luz}$.
Si la distancia $d$ que recorre la luz es igual a la velocidad $v$ de la luz por el tiempo $t$ que tarda en recorrerla, entonces:
\[d=vt\]

\begin{parts}
    \part \textbf{¿A cuántos metros equivale un parsec?}

    Considera que un año tiene 365 días y que la velocidad de la luz es $3 \times 10^{8} \text{ m/s}$.

    \begin{solutionbox}{7cm}
        Usando la fórmula $d=vt$, donde $d$ es la distancia, $v$ es la velocidad y $t$ es el tiempo, la distancia $d$ que hay en un año luz es:
        \begin{align*}
            d & = vt                                                                                                                                                                                                                                                                                                                                                                                                                              \\
              & = {\textstyle \left(3 \times 10^{8} \frac{\text{m}}{\text{s}} \right) \left(1 \text{ a\~no}\right)}                                                                                                                                                                                                                                                                                                                               \\
              & = {\textstyle \left(3 \times 10^{8} \frac{\text{m}}{\text{\cancel{s}}} \right) \left(1 \text{ \cancel{a\~no}}\right) \cdot \left(\frac{365 \text{ \cancel{día}}}{1 \text{ \cancel{a\~no}}}\right) \cdot \left(\frac{24 \text{ \cancel{hora}}}{1 \text{ \cancel{día}}}\right) \cdot \left(\frac{60 \text{ \cancel{min}}}{1 \text{ \cancel{hora}}}\right) \cdot \left(\frac{60 \text{ \cancel{s}}}{1 \text{ \cancel{min}}}\right) } \\
              & = 9.46 \times 10^{15} \text{ m}
        \end{align*}
        Si 1 año luz equivale a $9.46 \times 10^{15} \text{ m}$, entonces $1 pc = 3.26 \text{ años luz} \cdot 9.46 \times 10^{15} \text{ m}=3.08 \times 10^{16} \text{ m}$
    \end{solutionbox}

    \part La galaxia M31 está a $650$ kpc de la Vía Láctea y se acerca a ella a una velocidad
    de unos $350$ km/s. Si la fórmula de cinemática para el tiempo es:
    \[t=\dfrac{d}{v}\]
    \textbf{¿En cuánto tiempo \comillas{chocará} con ella?}

    Considea como el kiloparsec, $1 \text{ kpc} = 10^{3} \text{ pc}$, y el megaparsec, 1 Mpc = 10$^6$ pc.%Resuelvan en equipo.

    \begin{solutionbox}{6cm}
        \begin{multicols}{2}
            Sabemos que $1 \text{ pc}=3.08 \times 10^{13} \text{ km}$, entonces
            \begin{align*}
                650 \text{ kpc} & =650 \times 10^{3} \text{ pc}                            \\
                                & =650 \times 10^{3} \times 3.08 \times 10^{13} \text{ km} \\
                                & =2.002 \times 10^{19} \text{ km}
            \end{align*}

            \columnbreak

            Usando la fórmula $t=\dfrac{d}{v}$, el tiempo $t$ en segundos es:
            \begin{align*}
                t & =\dfrac{2.002 \times 10^{19} \text{ \cancel{km}}}{350 \text{ \cancel{km}/s}} \\
                  & =5.72 \times 10^{16} \text{ s}                                               \\
                  & =1,812.5 \text{ millones de años}
            \end{align*}
        \end{multicols}
    \end{solutionbox}
\end{parts}