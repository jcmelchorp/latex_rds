Elige la respuesta correcta.

\begin{multicols}{2}
    \begin{parts}
        \part La relación de proporcionalidad entre la velocidad con la que se
        alejan las galaxias y la distancia a la que se encuentran.
        %(pág. \pageref{086a_c})

        \begin{choices}
            \choice Ley de Hook
            \choice Ley de Bubble
            \CorrectChoice Ley de Hubble
            \choice Ley de Hobbs
        \end{choices}

        \part Grupo formado por la Vía Láctea y otras 14 galaxias gigantes
        que integra una estructura en forma de anillo.

        \begin{choices}
            \choice Supercúmulo
            \CorrectChoice Concilio de Gigantes
            \choice Cúmulos de galaxias
            \choice Grupo local
        \end{choices}

        \part Grupo formado por la Vía Láctea y unas 30 galaxias más.

        \begin{choices}
            \choice Supercúmulo
            \choice Concilio de Gigantes
            \choice Cúmulos de galaxias
            \CorrectChoice Grupo local
        \end{choices}

        \part Instrumento gracias al cual es posible observar cuerpos celestes muy lejanos.

        \begin{choices}
            \choice Microscopio
            \choice Estetoscopio
            \CorrectChoice Telescopio
            \choice Astrolabio
        \end{choices}

        \part Células receptoras de luz capaces de percibir colores, pero para que funcionen es necesario que haya suficiente luz.

        \begin{choices}
            \choice Bastones
            \choice Esferas
            \CorrectChoice Conos
            \choice Rizos
        \end{choices}


        \part Pulso eléctrico que se propaga a través de la neurona.
        \begin{choices}
            \choice Potencial eléctrico
            \choice Potencial magnético
            \CorrectChoice Potencial de acción
            \choice Potencial neuronal
        \end{choices}

        \part Variación aparente de la posición de un objeto al cambiar la posición del observador.

        \begin{choices}
            \choice Eclipse
            \CorrectChoice Paralaje
            \choice Declinación
            \choice Movimiento

        \end{choices}

        \part Técnica gracias a la cual se puede comparar el cambio en la posición de una estrella al transcurrir cierto período de tiempo.

        \begin{choices}
            \choice Radiografía
            \CorrectChoice Fotografía
            \choice Radiometría
            \choice Espectroscopía
        \end{choices}

        \part Porcentaje de \emph{energía oscura} que hay en el Universo.

        \begin{choices}
            \choice 4.9\%
            \choice 26.8\%
            \choice 33.3\%
            \CorrectChoice 68.3\%
        \end{choices}

        \part Indica que el Universo se
        expande.%(pág. \pageref{086a_b})

        \begin{choices}
            \choice El corrimiento al rojo de la luz que emite nuestro Sol.
            \choice La Teoría de la Gravitación Universal
            \CorrectChoice El corrimiento al rojo de la luz que emiten las
            galaxias.
            \choice La Teoría de la Relatividad General
        \end{choices}

        % \part Perturbación eléctrica que se genera cuando una neurona recibe un estímulo.
        % \begin{choices}
        %     \choice Impulso eléctrico
        %     \choice Impulso magnético
        %     \choice Impulso celular
        %     \CorrectChoice Impulso nervioso
        % \end{choices}

        % \part Grupo de galaxias cuyos tamaños típicos son de 2 a 3 Mpc.

        % \begin{choices}
        %     \CorrectChoice Cúmulos de galaxias
        %     \choice Supercúmulo
        %     \choice Concilio de Gigantes
        %     \choice Grupo local
        % \end{choices}

        % \part Porcentaje de \emph{materia oscura} que hay en el Universo.

        % \begin{choices}
        %     \choice 4.9\%
        %     \CorrectChoice 26.8\%
        %     \choice 33.3\%
        %     \choice 68.3\%
        % \end{choices}

        % \part Grupo formado por cúmulos de galaxias.

        % \begin{choices}
        %     \CorrectChoice Supercúmulo
        %     \choice Concilio de Gigantes
        %     \choice Cúmulos de galaxias
        %     \choice Grupo local
        % \end{choices}

        % \part Antigüedad estimada del Universo.

        % \begin{choices}
        %     \choice 14,800 millones de años
        %     \choice 10,800 millones de años
        %     \choice 15,800 millones de años
        %     \CorrectChoice 13,800 millones de años
        % \end{choices}

        % \part Longitud del diámetro del Universo.

        % \begin{choices}
        %     \choice Un millón de años luz.
        %     \CorrectChoice Cien mil millones de años luz.
        %     \choice Un billón de años luz.
        %     \choice Mil millones de años luz.
        % \end{choices}

        % \part Porcentaje de \emph{materia ordinaria} que hay en el Universo.

        % \begin{choices}
        %     \CorrectChoice 4.9\%
        %     \choice 26.8\%
        %     \choice 33.3\%
        %     \choice 68.3\%
        % \end{choices}
    \end{parts}
\end{multicols}