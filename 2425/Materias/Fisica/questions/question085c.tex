Considera que la velocidad de la luz es de $3 \times 10^{8} \text{ m/s}$ y que un año tiene 365.25 días.

\begin{parts}
    \part ¿Cuántos segundos hay en un año?

    \begin{solutionbox}{1.6cm}
        \[ 1 \text{ año} = 365.25 \text{ días} \times 24 \text{ horas} \times 60 \text{ minutos} \times 60 \text{ segundos} = 31,557,600 \text{ segundos}=3.15576 \times 10^{7} \text{ segundos}\]
    \end{solutionbox}

    Si sabemos que $v=\dfrac{d}{t}$ ¿Cuántos metros recorre la luz en un año?, ¿a cuántos kilómetros equivale?

    \begin{solutionbox}{1.6cm}
        \[ d=vt=\left(3 \times 10^{8} \text{ m/s}\right)\left(3.15576 \times 10^{7} \text{ s}\right)=9.47 \times 10^{15} \text{ m}=9.47 \times 10^{12} \text{ km}\]
    \end{solutionbox}

    Después del Sol, la estrella más cercana a la Tierra es Próxima Centauri, que está a $3.99 \times 10^{13} \text{ km}$. ¿Cuánto tiempo tarda la luz de Próxima Centauri en llegar a la Tierra?

    \begin{solutionbox}{2cm}
        \[ t=\dfrac{d}{v}=\dfrac{9.47 \times 10^{15} \text{ m}}{3 \times 10^{8} \text{ m/s}}=133 \text{ millones de segundos} = 4.21 \text{ años}\]
    \end{solutionbox}
\end{parts}
