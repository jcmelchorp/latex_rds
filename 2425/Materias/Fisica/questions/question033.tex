Elige  o  para indicar si las siguientes afirmaciones son o  aportaciones de
Newton a la ciencia.

\begin{multicols}{2}
    \begin{parts}



        % \part Los objetos pesan porque son atraídos por la Tierra.

        % \begin{oneparcheckboxes}\footnotesize%
        %     \choice Sí
        %     \choice No
        % \end{oneparcheckboxes}

        % \part El movimiento de los objetos terrestres y celestes es regido por las
        % mismas leyes.

        % \begin{oneparcheckboxes}\footnotesize%
        %     \choice Sí
        %     \choice No
        % \end{oneparcheckboxes}

        % \part Los objetos se mueven según su naturaleza.

        % \begin{oneparcheckboxes}\footnotesize%\footnotesize%
        %     \choice Sí
        %     \choice No
        % \end{oneparcheckboxes}

        \part Un objeto cae con una velocidad proporcional a su peso.

        \begin{oneparcheckboxes}\footnotesize%
            \choice Sí
            \choice No
        \end{oneparcheckboxes}

        \part Cuando un objeto ejerce una fuerza de acción sobre otro, éste último
        ejerce una fuerza de reacción al mismo tiempo, de igual magnitud y en dirección
        opuesta sobre el primero.

        \begin{oneparcheckboxes}\footnotesize%
            \choice Sí
            \choice No
        \end{oneparcheckboxes}

        \part La fuerza de gravedad es una propiedad que tienen los cuerpos con
        masa de atraerse mutuamente

        \begin{oneparcheckboxes}\footnotesize%
            \choice Sí
            \choice No
        \end{oneparcheckboxes}

        % \part Los cuerpos celestes se encuentran en el mundo etéreo o supralunar y
        % se mueven en círculos, donde todo es perfecto, inmutable, infinito y eter.

        % \begin{oneparcheckboxes}\footnotesize%
        %     \choice Sí
        %     \choice No
        % \end{oneparcheckboxes}


        % \part La aceleración que experimenta un objeto al recibir una fuerza es
        % directamente proporcional a la magnitud de la fuerza aplicada e inversamente
        % proporcional a su masa, y tiene la misma dirección que la fuerza aplicada.

        % \begin{oneparcheckboxes}\footnotesize%
        %     \choice Sí
        %     \choice No
        % \end{oneparcheckboxes}

        \part La fuerza de gravedad que actúa entre dos cuerpos es siempre de
        atracción, es directamente proporcional al producto de sus masas e inversamente
        proporcional al cuadrado de su distancia.

        \begin{oneparcheckboxes}\footnotesize%
            \choice Sí
            \choice No
        \end{oneparcheckboxes}

        \part Todo cuerpo tiende a mantener su estado de reposo o de movimiento con velocidad constante, a menos que una fuerza que actúe sobre él le
        obligue a cambiar ese estado.

        \begin{oneparcheckboxes}\footnotesize%
            \choice Sí
            \choice No
        \end{oneparcheckboxes}


    \end{parts}
\end{multicols}