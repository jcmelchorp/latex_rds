\question[10] Relaciona con una l\'inea recta cada figura con la clasificación del par de ángulos (verde y azul) que se resaltan en ella.
\begin{center}
  \begin{minipage}{0.3\linewidth}
    \begin{parts}
      \part \includegraphics[height=2.5cm]{Images/poly_angle_01}
      \hfill{\color{cadmiumgreen}$\square$}
      \part \includegraphics[height=1.8cm]{Images/poly_angle_02}
      \hfill{\color{cadmiumgreen}$\square$}
      \part \includegraphics[height=3cm]{Images/poly_angle_03}
      \hfill{\color{cadmiumgreen}$\square$}
      \part \includegraphics[height=2.5cm]{Images/poly_angle_04}
      \hfill{\color{cadmiumgreen}$\square$}
      \part \includegraphics[height=2.8cm]{Images/poly_angle_05}
      \hfill{\color{cadmiumgreen}$\square$}
      \part \includegraphics[height=2.8cm]{Images/poly_angle_06}
      \hfill{\color{cadmiumgreen}$\square$}
    \end{parts}
  \end{minipage}\hspace{1cm}
  \begin{minipage}{0.4\linewidth}
    \checkboxchar{ {\color{cadmiumorange}
          $\Box$}
    }
    \begin{checkboxes}
      \choice Suplementarios                \vspace{2cm}
      \choice Correspondientes              \vspace{2cm}
      \choice Alternos externos             \vspace{2cm}
      \choice Adyacentes no suplementarios  \vspace{2cm}
      \choice Opuestos por el vértice       \vspace{2cm}
      \choice Alternos internos
    \end{checkboxes}
  \end{minipage}
\end{center}