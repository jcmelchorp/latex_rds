% This file was converted to LaTeX by Writer2LaTeX ver. 1.0.1
% see http://writer2latex.sourceforge.net for more info
\documentclass{article}
\usepackage[ascii]{inputenc}
\usepackage[T1]{fontenc}
\usepackage[spanish]{babel}
\usepackage{amsmath}
\usepackage{amssymb,amsfonts,textcomp}
\usepackage{array}
\usepackage{hhline}
\newcommand\normalsubformula[1]{\text{\mathversion{normal}$#1$}}
\title{}
\begin{document}
\clearpage\setcounter{page}{1}
\bigskip


\bigskip


\bigskip


\bigskip


\bigskip


\bigskip


\bigskip


\bigskip

Proped\'eutico\\


\bigskip

Matem\'atica B\'asica\\


\bigskip

Secci\'on 6\\


\bigskip

Trabajo \#3\\


\bigskip

{\textquotedblleft}Trabajo especial grupal{\textquotedblright}\\


\bigskip

Joan Zoquier 2013-1833\


\bigskip

Jorge Ysabel 2013-1780\


\bigskip

Jos\'e Beltre 2013-1734\


\bigskip

Jos\'e Lluberes 2013-1576\


\bigskip

Daniel Caama\~no 2013-1773\


\bigskip

Valentina S\'anchez 2013-1858\\


\bigskip


\bigskip

Ing. Endy Pe\~na\\\

\bigskip

\clearpage\section[Contenido]{Contenido}


\bigskip


\bigskip

N\'umeros Racionales \ \ \ \ \ \ \ \ \ \ \ \ \ \ \ \ \ \ \ \ \ \ \ \ \ \ \ \ \ \ \ \ \ \ \ \ \ \ \ \ \ \ \ \ \ \ \ \ \ \ \ \ \ \ \ \ \ \ \ \ \ \ \ \ \ \ \ \ \ \ \ \ \ \ \ \ \ \ \ \ \ \ \ \ \ \ \ \ \ \ \ \ \ \ \ 3

Conjunto de los n\'umeros Irracionales \ \ \ \ \ \ \ \ \ \ \ \ \ \ \ \ \ \ \ \ \ \ \ \ \ \ \ \ \ \ \ \ \ \ \ \ \ \ \ \ \ \ \ \ \ \ \ \ \ \ \ \ \ \ \ \ \ \ \ \ \ \ \ \ 3

Las Fracciones \ \ \ \ \ \ \ \ \ \ \ \ \ \ \ \ \ \ \ \ \ \ \ \ \ \ \ \ \ \ \ \ \ \ \ \ \ \ \ \ \ \ \ \ \ \ \ \ \ \ \ \ \ \ \ \ \ \ \ \ \ \ \ \ \ \ \ \ \ \ \ \ \ \ \ \ \ \ \ \ \ \ \ \ \ \ \ \ \ \ \ \ \ \ \ \ \ \ \ \ \ \ \ \ \ 4

Clasificaci\'on

Adici\'on y sustracci\'on

Multiplicaci\'on

Divisi\'on

Potenciaci\'on \ \ \ \ \ \ \ \ \ \ \ \ \ \ \ \ \ \ \ \ \ \ \ \ \ \ \ \ \ \ \ \ \ \ \ \ \ \ \ \ \ \ \ \ \ \ \ \ \ \ \ \ \ \ \ \ \ \ \ \ \ \ \ \ \ \ \ \ \ \ \ \ \ \ \ \ \ \ \ \ \ \ \ \ \ \ \ \ \ \ \ \ \ \ \ \ \ \ \ \ \ \ \ \ \ \ \ 5

Propiedades

Signos de una potencia

Potencias de exponente negativo

Potencias de fracciones

Potencias fraccionarias de exponente negativo

Radicaci\'on \ \ \ \ \ \ \ \ \ \ \ \ \ \ \ \ \ \ \ \ \ \ \ \ \ \ \ \ \ \ \ \ \ \ \ \ \ \ \ \ \ \ \ \ \ \ \ \ \ \ \ \ \ \ \ \ \ \ \ \ \ \ \ \ \ \ \ \ \ \ \ \ \ \ \ \ \ \ \ \ \ \ \ \ \ \ \ \ \ \ \ \ \ \ \ \ \ \ \ \ \ \ \ \ \ \ \ \ \ 8

Valor absoluto \ \ \ \ \ \ \ \ \ \ \ \ \ \ \ \ \ \ \ \ \ \ \ \ \ \ \ \ \ \ \ \ \ \ \ \ \ \ \ \ \ \ \ \ \ \ \ \ \ \ \ \ \ \ \ \ \ \ \ \ \ \ \ \ \ \ \ \ \ \ \ \ \ \ \ \ \ \ \ \ \ \ \ \ \ \ \ \ \ \ \ \ \ \ \ \ \ \ \ \ \ \ \ 8

Inecuaciones \ \ \ \ \ \ \ \ \ \ \ \ \ \ \ \ \ \ \ \ \ \ \ \ \ \ \ \ \ \ \ \ \ \ \ \ \ \ \ \ \ \ \ \ \ \ \ \ \ \ \ \ \ \ \ \ \ \ \ \ \ \ \ \ \ \ \ \ \ \ \ \ \ \ \ \ \ \ \ \ \ \ \ \ \ \ \ \ \ \ \ \ \ \ \ \ \ \ \ \ \ \ \ \  \ \ 8

Inecuaciones de primer grado

Polinomios \ \ \ \ \ \ \ \ \ \ \ \ \ \ \ \ \ \ \ \ \ \ \ \ \ \ \ \ \ \ \ \ \ \ \ \ \ \ \ \ \ \ \ \ \ \ \ \ \ \ \ \ \ \ \ \ \ \ \ \ \ \ \ \ \ \ \ \ \ \ \ \ \ \ \ \ \ \ \ \ \ \ \ \ \ \ \ \ \ \ \ \ \ \ \ \ \ \ \ \ \ \ \ \ \ \ \ \ \ 9

Tipos

Valor num\'erico de un polinomio

Productos notables \ \ \ \ \ \ \ \ \ \ \ \ \ \ \ \ \ \ \ \ \ \ \ \ \ \ \ \ \ \ \ \ \ \ \ \ \ \ \ \ \ \ \ \ \ \ \ \ \ \ \ \ \ \ \ \ \ \ \ \ \ \ \ \ \ \ \ \ \ \ \ \ \ \ \ \ \ \ \ \ \ \ \ \ \ \ \ \ \ \ \ \ \ \ 11

An\'alisis combinatorio \ \ \ \ \ \ \ \ \ \ \ \ \ \ \ \ \ \ \ \ \ \ \ \ \ \ \ \ \ \ \ \ \ \ \ \ \ \ \ \ \ \ \ \ \ \ \ \ \ \ \ \ \ \ \ \ \ \ \ \ \ \ \ \ \ \ \ \ \ \ \ \ \ \ \ \ \ \ \ \ \ \ \ \ \ \ \ \ \ \ 12

Permutaciones \ \ \ \ \ \ \ \ \ \ \ \ \ \ \ \ \ \ \ \ \ \ \ \ \ \ \ \ \ \ \ \ \ \ \ \ \ \ \ \ \ \ \ \ \ \ \ \ \ \ \ \ \ \ \ \ \ \ \ \ \ \ \ \ \ \ \ \ \ \ \ \ \ \ \ \ \ \ \ \ \ \ \ \ \ \ \ \ \ \ \ \ \ \ \ \ \ \ \ \ \ \ 12

Combinaciones \ \ \ \ \ \ \ \ \ \ \ \ \ \ \ \ \ \ \ \ \ \ \ \ \ \ \ \ \ \ \ \ \ \ \ \ \ \ \ \ \ \ \ \ \ \ \ \ \ \ \ \ \ \ \ \ \ \ \ \ \ \ \ \ \ \ \ \ \ \ \ \ \ \ \ \ \ \ \ \ \ \ \ \ \ \ \ \ \ \ \ \ \ \ \ \ \ \ \ \ \ 13



Logaritmos \ \ \ \ \ \ \ \ \ \ \ \ \ \ \ \ \ \ \ \ \ \ \ \ \ \ \ \ \ \ \ \ \ \ \ \ \ \ \ \ \ \ \ \ \ \ \ \ \ \ \ \ \ \ \ \ \ \ \ \ \ \ \ \ \ \ \ \ \ \ \ \ \ \ \ \ \ \ \ \ \ \ \ \ \ \ \ \ \ \ \ \ \ \ \ \ \ \ \ \ \ \ \ \ \ \ \ \ 13

Logaritmos decimales

Logaritmos neperianos o naturales

Derivadas \ \ \ \ \ \ \ \ \ \ \ \ \ \ \ \ \ \ \ \ \ \ \ \ \ \ \ \ \ \ \ \ \ \ \ \ \ \ \ \ \ \ \ \ \ \ \ \ \ \ \ \ \ \ \ \ \ \ \ \ \ \ \ \ \ \ \ \ \ \ \ \ \ \ \ \ \ \ \ \ \ \ \ \ \ \ \ \ \ \ \ \ \ \ \ \ \ \ \ \ \ \ \ \ \ \ \ \ \ \ 14

Bibliograf\'ia \ \ \ \ \ \ \ \ \ \ \ \ \ \ \ \ \ \ \ \ \ \ \ \ \ \ \ \ \ \ \ \ \ \ \ \ \ \ \ \ \ \ \ \ \ \ \ \ \ \ \ \ \ \ \ \ \ \ \ \ \ \ \ \ \ \ \ \ \ \ \ \ \ \ \ \ \ \ \ \ \ \ \ \ \ \ \ \ \ \ \ \ \ \ \ \ \ \ \ \ \ \ \ \ \ \ \ 17

\bigskip
\bigskip

\clearpage\section[N\'umeros racionales]{N\'umeros racionales}

\bigskip

Los n\'umeros racionales, son el conjunto de n\'umeros fraccionarios y n\'umeros enteros representados por medio de fracciones. Este conjunto est\'a situado en la recta real num\'erica pero a diferencia de los n\'umeros naturales que son consecutivos, por ejemplo a 4 le sigue 5 y a este a su vez le sigue el 6, y los n\'umeros negativos cuya consecuci\'on se da as\'i, a -9 le sigue -8 y a este a su vez le sigue -7; los n\'umeros racionales no poseen consecuci\'on pues entre cada n\'umero racional existen infinitos n\'umeros que solo podr\'ian ser escritos durante toda la eternidad.

Todos los n\'umeros fraccionarios son n\'umeros racionales, y sirven para representar medidas. Pues a veces es m\'as conveniente expresar un n\'umero de esta manera que convertirlo a decimal exacto o peri\'odico, debido a la gran cantidad de decimales que se podr\'ian obtener.

Al conjunto de los n\'umeros racionales se lo denota con la letra Q, que sirve para recogerlos como subgrupo dentro de los n\'umeros reales y junto a los n\'umeros enteros cuya denotaci\'on es la letra Z.

Existe una clasificaci\'on de los n\'umeros racionales dependiendo de su expresi\'on decimal, estos son:

Los n\'umeros racionales limitados, cuya representaci\'on decimal tiene un n\'umero determinado y fijo de cifras, por ejemplo 1/8 es igual a 0,125.

Los n\'umeros racionales peri\'odicos, de los cuales sus decimales tienen un n\'umero ilimitado de cifras.

A su vez los n\'umeros racionales peri\'odicos se dividen en dos, los peri\'odicos puros, cuyo patr\'on se encuentra inmediatamente despu\'es de la coma, por ejemplo 0,6363636363{\dots} y los peri\'odicos mixtos, de los cuales el patr\'on se encuentra despu\'es de un n\'umero determinado de cifras, por ejemplo 5,48176363636363{\dots} Ejemplos de n\'umeros racionales

Los n\'umeros racionales son n\'umeros fraccionarios, es decir que podr\'iamos escribir cualquier cociente entre dos n\'umeros enteros y llamarlo n\'umero racional, aqu\'i un ejemplo 5/7


\bigskip

\section[Conjunto de los n\'umeros irracionales]{Conjunto de los n\'umeros irracionales}

\bigskip

Los n\'umeros irracionales tienen como definici\'on que son n\'umeros que poseen infinitas cifras decimales no peri\'odicas, que por lo tanto no pueden ser expresados como fracciones.

Para distinguir los n\'umeros irracionales de los racionales, debemos tomar en cuenta que los n\'umeros racionales si se pueden escribir de manera fraccionada o racional, por ejemplo: 18/5 que es igual a 3,6 por lo tanto es un n\'umero racional a diferencia de la ra\'iz cuadrada de dos en cuyo resultado se obtienen infinito n\'umero de cifras decimales, y su fraccionamiento resulta imposible.


\bigskip


\bigskip

Notaci\'on de los n\'umeros irracionales

La representaci\'on gr\'afica de los n\'umeros irracionales se la hace con la letra I may\'uscula. Se la utiliza de esta manera para diferenciarla de los n\'umeros imaginarios, cuya representaci\'on es la i min\'uscula. Pero el s\'imbolo no se representa en las ecuaciones al no constituir una estructura algebraica, y para no crear confusi\'on, en ocasiones se los puede ver como R/Q como la representaci\'on de n\'umeros irracionales por definici\'on. Ejemplos de n\'umeros irracionales

\section[Fracciones]{Fracciones}

\bigskip

Una fracci\'on \ es la expresi\'on de una cantidad dividida entre otra cantidad; es decir que representa un cociente no efectuado de n\'umeros. Por razones hist\'oricas tambi\'en se les llama fracci\'on com\'un, fracci\'on vulgar o fracci\'on decimal. El conjunto matem\'atico que contiene a las fracciones es el conjunto de los n\'umeros racionales, denotado.

De manera m\'as general, se puede extender el concepto de fracci\'on a un cociente cualquiera de expresiones matem\'aticas (no necesariamente n\'umeros).

\subsection[Clasificaci\'on de las fracciones]{Clasificaci\'on de las fracciones}

\bigskip

Seg\'un la relaci\'on entre el numerador y el denominador:

Fracci\'on mixta: suma abreviada de un entero y una fracci\'on propia: 1 {\textonequarter}

Fracci\'on propia: fracci\'on en que el denominador es mayor que el numerador: 5/8

Fracci\'on impropia: fracci\'on en donde el numerador es mayor que el denominador: 5/2

Fracci\'on reducible: fracci\'on en la que el numerador y el denominador no son primos entre s\'i y puede ser simplificada: 4/2

Fracci\'on irreducible: fracci\'on en la que el numerador y el denominador son primos entre s\'i, y por tanto no puede ser simplificada: 1/2

Fracci\'on inversa: fracci\'on obtenida a partir de otra dada, en la que se han invertido el numerador y el denominador: 3/4,4/3

Fracci\'on aparente o entera: fracci\'on que representa cualquier n\'umero perteneciente al conjunto de los enteros.

Fracci\'on compuesta: fracci\'on cuyo numerador o denominador (o los dos) contiene a su vez fracciones.

Seg\'un la escritura del denominador:

Fracci\'on equivalente: la que tiene el mismo valor que otra dada: 2/1=2, 4/2=2

Fracci\'on homog\'enea: fracciones que tienen el mismo denominador: \ \ 3/4,5/4

Fracci\'on heterog\'enea: fracciones que tienen diferentes denominadores: 3/4,5/6

Fracci\'on decimal: el denominador es una potencia de diez: 1/10, 2/100... entero positivo y n un natural.

\subsection[Adicci\'on y Sustracci\'on de Fracciones]{Adicci\'on y Sustracci\'on de Fracciones}

\bigskip

Para sumar y restar fracciones hay que distinguir entre:

Fracciones con igual denominador

En este caso para sumar o restar fracciones se mantiene constante el denominador y se suman o restan sus numeradores.

Sumamos sus numeradores y mantenemos el denominador:

Veamos ahora un ejemplo de sustracci\'on otro ejemplo:

Fracciones con distinto denominador

En este caso para sumar o restar fracciones:

Lo primero que hay que hacer es buscar un denominador com\'un a todas ellas.

Luego sustituir las fracciones originales por fracciones equivalentes con este denominador com\'un.

Y ?`c\'omo se calcula este denominador com\'un? utilizaremos el m\'etodo del m\'inimo com\'un m\'ultiplo (MCM).

Una vez obtenido el denominador com\'un hay que calcular las fracciones equivalentes. Para cada fracci\'on haremos lo siguiente.

Sustituimos su denominador por el denominador com\'un.

Calculamos su numerador de la siguiente manera: dividimos el denominador com\'un por el denominador original de cada fracci\'on. El resultado obtenido lo multiplicamos por el numerador original, obteniendo el numerador de la fracci\'on equivalente.

\subsection[Multiplicaci\'on de fracciones]{Multiplicaci\'on de fracciones}

\bigskip

El producto de dos fracciones es otra fracci\'on que tiene:

Por numerador el producto de los numeradores.

Por denominador el producto de los denominadores.

\subsection[Divisi\'on de fracciones]{Divisi\'on de fracciones}

\bigskip

El cociente de dos fracciones es otra fracci\'on que tiene:

Por numerador el producto de los extremos.

Por denominador el producto de los medios


\bigskip

\section[Potenciaci\'on]{Potenciaci\'on}
La potenciaci\'on es una forma abreviada de escribir un producto formado por varios factores iguales.

7 {\textbullet} 7 {\textbullet} 7 {\textbullet} 7 = 74

Base

La base de una potencia es el n\'umero que multiplicamos por s\'i mismo, en este caso el 7.

Exponente

El exponente de una potencia indica el n\'umero de veces que multiplicamos la base, en el ejemplo es el 4.


\bigskip

\subsection[Propiedades]{Propiedades}

\bigskip

Un n\'umero elevado a 0 es igual a 1.

 $a^{0}=1$

 $6^{0}=1$

Un n\'umero elevado a 1 es igual a s\'i mismo.

 $a^{1}=a$

 $6^{1}=6$

Producto de potencias con la misma base:

Es otra potencia con la misma base y cuyo exponente es la suma de los exponentes.

 $a^{m}\cdot a^{n}=a^{(m+n)}$

 $3^{5}\cdot 3^{2}=3^{(5+2)}=37$

Divisi\'on de potencias con la misma base:

Es otra potencia con la misma base y cuyo exponente es la diferencia de los exponentes.

 $\frac{a^{m}}{a^{n}}=a^{(m-n)}$

 $\frac{3^{5}}{3^{2}}=3^{(5-2)}=33$


\bigskip

Potencia de una potencia:

Es otra potencia con la misma base y cuyo exponente es el producto de los exponentes.

 ${(a^{m})}^{n}=a^{(m\cdot n)}$

 ${(3^{5})}^{3}=3^{15}$


\bigskip

Producto de potencias con el mismo exponente:

Es otra potencia con el mismo exponente y cuya base es el producto de las bases.

 $a^{n}\cdot b^{n}={(a\cdot b)}^{n}$

 $2^{5}\cdot 4^{5}=8^{5}$

Cociente de potencias con el mismo exponente:

Es otra potencia con el mismo exponente y cuya base es el cociente de las bases.

 $\frac{a^{n}}{b^{n}}={(\frac{a}{b})}^{n}$

 $\frac{6^{4}}{3^{4}}=2^{4}$

\subsection[Signo de una potencia de base entera]{Signo de una potencia de base entera}

\bigskip

Para determinar el signo de la potencia de un n\'umero entero tendremos en cuenta que:

\subsection[Las potencias de exponente par son siempre positivas. \ ]{Las potencias de exponente par son siempre positivas. \ }

\bigskip

 $2^{6}=64$

 ${(-2)}^{6}=64$

\subsection[Las potencias de exponente impar tiene el mismo signo de la base. \ ]{Las potencias de exponente impar tiene el mismo signo de la base. \ }

\bigskip

 $2^{3}=8$

 ${(-2)}^{3}=-8$


\bigskip

\subsection[Potencias de exponente negativo]{Potencias de exponente negativo}

\bigskip

La potencia de un n\'umero entero con exponente negativo es igual al inverso del n\'umero elevado a exponente positivo. \ \ \ \ 

\subsection[Potencias de fracciones:]{Potencias de fracciones:}

\bigskip

Para elevar una fracci\'on a una potencia se eleva tanto el numerador como el denominador al exponente. \ \ 

\subsection[Potencias fraccionarias de exponente negativo:]{Potencias fraccionarias de exponente negativo:}

\bigskip

Una potencia fraccionaria de exponente negativo es igual a la inversa de la fracci\'on elevada a exponente positivo. \ \ \ \ \ \ 


\bigskip

\section[Radicaci\'on]{Radicaci\'on}
En matem\'atica, la radicaci\'on de orden n de un n\'umero a es cualquier n\'umero b tal que , donde n se llama \'indice u orden, a se denomina radicando, y b es una ra\'iz en\'esima, por lo que se suele conocer tambi\'en con ese nombre. La notaci\'on a seguir tiene varias formas:

 $\sqrt[n]{x}=x^{1/n}$


\bigskip

Para todo n natural, a y b reales positivos, se tiene la equivalencia:

 $a=b^{n}\rightarrow b=\sqrt[n]{a}$

Dentro de los n\'umeros reales \ positivos, siempre puede encontrarse una \'unica ra\'iz en\'esima tambi\'en positiva. Si el n\'umero a es negativo entonces s\'olo existir\'a una ra\'iz real cuando el \'indice n sea impar1 . La ra\'iz en\'esima de un n\'umero negativo no es un n\'umero real (no est\'a definida dentro de los n\'umeros reales) cuando el \'indice n es par.

Dentro de los n\'umeros complejos , para cada n\'umero z siempre es posible encontrar exactamente n ra\'ices en\'esimas diferentes.

La ra\'iz de orden dos se llama ra\'iz cuadrada y, por ser la m\'as frecuente, se escribe sin super\'indice:  $\sqrt{x}$ en vez de  $\sqrt[2]{x}$.La ra\'iz de orden tres se llama ra\'iz c\'ubica.


\bigskip


\bigskip

\section[Valor Absoluto]{Valor Absoluto}

\bigskip

El valor absoluto de un n\'umero entero es el n\'umero natural que resulta al suprimir su signo.

El valor absoluto lo escribiremos entre barras verticales.

{\textbar}$-$5{\textbar} = 5

{\textbar}5{\textbar} = 5

1) \ {\textbar}5{\textbar} = 5 \ \ \ \ \ \ \ \ \ \ \ \ \ {\textbar}-5 {\textbar}= 5 \ \ \ \ \ \ \ \ \ \ {\textbar}0{\textbar} = 0

\section[Inecuaciones]{Inecuaciones}

\bigskip

Las inecuaciones son desigualdades algebraicas en la que sus dos miembros se relacionan por uno de estos signos:

{\textless} menor que 2x $-$ 1 {\textless} 7

${\leq}$ menor o igual que 2x $-$ 1 ${\leq}$ 7

{\textgreater} mayor que 2x $-$ 1 {\textgreater} 7

${\geq}$ mayor o igual que 2x $-$ 1 ${\geq}$ 7 \ \ 

La soluci\'on de una inecuaci\'on es el conjunto de valores de la variable que la verifica.

La soluci\'on de la inecuaci\'on se expresa mediante:

1. Una representaci\'on gr\'afica.

2. Un intervalo.

2x ${\geq}$ 8 \ \ \ \ x ${\geq}$ 4 \ 

[4, ${\infty}$)

\subsection[Inecuaciones de primer grado]{Inecuaciones de primer grado}

\bigskip

Inecuaciones de primer grado Inecuaciones de primer grado con una inc\'ognita

\ 1{\textordmasculine} Quitar corchetes y par\'entesis.

2{\textordmasculine} Quitar denominadores.

3{\textordmasculine} Agrupar los t\'erminos en x a un lado de la desigualdad y los t\'erminos independientes en el otro.

4{\textordmasculine} Efectuar las operaciones

5{\textordmasculine} Si el coeficiente de la x es negativo multiplicamos por $-$1, por lo que cambiar\'a el sentido de la desigualdad.

6{\textordmasculine} Despejamos la inc\'ognita.

7{\textordmasculine} Expresar la soluci\'on de forma gr\'afica y con un intervalo. \ \ \ \ \ 

\ \ \ \ [3, +${\infty}$)


\bigskip

\section[Polinomios]{Polinomios}

\bigskip

Un polinomio es una expresi\'on algebraica de la forma:

P(x) = an xn + an $-$ 1 xn $-$ 1 + an $-$ 2 xn $-$ 2+ .. + a11 + a0


\bigskip

Siendo:

An, An$-$1 ... a1, Ao, N\'umeros, llamados coeficientes

n un n\'umero natural

x la variable o indeterminada

An es el coeficiente principal

Ao es el t\'ermino independiente

Grado de un Polinomio

El grado de un polinomio P(x) es el mayor exponente al que se encuentra elevada la variable x.

Seg\'un su grado los polinomios pueden ser de: \ 

PRIMER GRADO P(x) = 3x + 2

SEGUNDO GRADO P(x) = ${2x}^{2}+3x+2$

TERCER GRADO P(x) = $x^{3}-{2x}^{2}+3x+2$


\bigskip

Tipos de polinomios:

\subsection[1{}- Polinomio nulo]{1- Polinomio nulo}
Es aquel polinomio que tiene todos sus coeficientes nulos.

P(x) =  ${0x}^{2}+0x+0$


\bigskip

\subsection[2{}- Polinomio homog\'eneo]{2- Polinomio homog\'eneo}
Es aquel polinomio en el que todos sus t\'erminos o monomios son del mismo grado.

 $P(x)={2x}^{2}+3xy$

3- Polinomio heterog\'eneo

Es aquel polinomio en el que no todos sus t\'erminos no son del mismo grado.

 $P(x)={2x}^{3}+{3x}^{2}-3$

\subsection[4{}- Polinomio completo]{4- Polinomio completo}
Es aquel polinomio que tiene todos los t\'erminos desde el t\'ermino independiente hasta el t\'ermino de mayor grado.

 $P(x)={2x}^{3}+{3x}^{2}+5x-3$

\subsection[5{}- Polinomio incompleto]{5- Polinomio incompleto}
Es aquel polinomio que no tiene todos los t\'erminos desde el t\'ermino independiente hasta el t\'ermino de mayor grado.

 $P(x)={2x}^{3}+5x-3$

\subsection[6{}- Polinomio ordenado]{6- Polinomio ordenado}
Un polinomio est\'a ordenado si los monomios que lo forman est\'an escritos de mayor a menor grado.

 $P(x)={2x}^{3}+5x-3$

\subsection[7{}- Polinomios iguales]{7- Polinomios iguales}
Dos polinomios son iguales si verifican:

Los dos polinomios tienen el mismo grado.

 $P(x)={2x}^{3}+5x-3$

 $Q(x)={5x}^{3}-2x-7$

\subsection[8{}- Polinomios semejantes]{8- Polinomios semejantes}
Es el resultado que obtenemos al sustituir la variable x por un n\'umero cualquiera.

 $P(x)={2x}^{3}+{5x}^{(-3)};x=1$


\bigskip

\subsection[Valor num\'erico de un polinomio]{Valor num\'erico de un polinomio}
Es el resultado que obtenemos al sustituir la variable x por un n\'umero cualquiera.

 $P(x)={2x}^{3}+5x-3;x=1$

 $P(1)=2\cdot 1^{3}+5\cdot 1-3=2+5-3=4$

\section[Productos notables]{Productos notables}
Sabemos que se llama producto al resultado de una multiplicaci\'on. Tambi\'en sabemos que los valores que se multiplican se llaman factores.

Se llama productos notables a ciertas expresiones algebraicas que se encuentran frecuentemente y que es preciso saber factorizarlas a simple vista; es decir, sin necesidad de hacerlo paso por paso.

Se les llama productos notables (tambi\'en productos especiales) precisamente porque son muy utilizados en los ejercicios.

\ A continuaci\'on veremos algunas expresiones algebraicas y del lado derecho de la igualdad se muestra la forma de factorizarlas (mostrada como un producto notable).


\bigskip

{}-Cuadrado de la suma de dos cantidades o binomio cuadrado

 $a^{2}+2ab+b^{2}={(a+b)}^{2}$

\ \ El cuadrado de la suma de dos cantidades es igual al cuadrado de la primera cantidad, m\'as el doble de la primera cantidad multiplicada por la segunda, m\'as el cuadrado de la segunda cantidad.


\bigskip


\bigskip

{}-Cuadrado de la diferencia de dos cantidades

 $a^{2}\text{--}2ab+b^{2}={(a\text{--}b)}^{2}$

\ \ El cuadrado de la diferencia de dos cantidades es igual al cuadrado de la primera cantidad, menos el doble de la primera cantidad multiplicada por la segunda, m\'as el cuadrado de la segunda cantidad.


\bigskip

{}-Producto de la suma por la diferencia de dos cantidades (o producto de dos binomios conjugados)

 $(a+b)(a\text{--}b)=a^{2}\text{--}b^{2}$

\ \ El producto de la suma por la diferencia de dos cantidades es igual al cuadrado de la primera cantidad, menos el cuadrado de la segunda \ 


\bigskip

{}-Producto de dos binomios con un t\'ermino com\'un, de la forma

 $x^{2}+(a+b)x+ab=(x+a)(x+b)$


\bigskip


\bigskip

\ \ {}-Producto de dos binomios con un t\'ermino com\'un, de la forma

 $x^{2}\text{--}(a+b)x+ab=(x\text{--}a)(x\text{--}b)$


\bigskip

{}-Producto de dos binomios con un t\'ermino com\'un, de la forma

 ${mnx}^{2}+ab+(mb+na)x=(mx+a)(nx+b)$

\ \ En este caso, vemos que el t\'ermino com\'un (x) tiene distinto coeficiente en cada binomio (mx y nx).


\bigskip

\ \ {}-Cubo de una suma

 $a^{3}+{3a}^{2b}+{3ab}^{2}+b^{3}={(a+b)}^{3}$

\ \ Entonces, para entender de lo que hablamos, cuando nos encontramos con una expresi\'on de la forma a3 + 3a2b + 3ab2 + b3debemos identificarla de inmediato y saber que podemos factorizarla como (a + b)3.


\bigskip


\bigskip

{}-Cubo de una diferencia

 $a^{3}\text{--}{3a}^{2b}+{3ab}^{2}\text{--}b^{3}={(a\text{--}b)}^{3}$

\ \ Entonces, para entender de lo que hablamos, cuando nos encontramos con una expresi\'on de la forma a3 -- 3a2b + 3ab2 -- b3debemos identificarla de inmediato y saber que podemos factorizarla como (a -- b)3.


\bigskip

\section[An\'alisis Combinatorio]{An\'alisis Combinatorio}
An\'alisis Combinatorio: Es la rama de la matem\'atica que estudia los diversos arreglos o selecciones que podemos formar con los elementos de un conjunto dado, los cuales nos permite resolver muchos problemas pr\'acticos. Por ejemplo podemos averiguar cu\'antos n\'umeros diferentes de tel\'efonos, placas o loter\'ias se pueden formar utilizando un conjunto dado de letras y d\'igitos.

Adem\'as el estudio y comprensi\'on del an\'alisis combinatorio no va a servir de andamiaje para poder resolver y comprender problemas sobre probabilidades

Principios fundamentales del An\'alisis Combinatorio: En la mayor\'ia de los problemas de an\'alisis combinatorio se observa que una operaci\'on o actividad aparece en forma repetitiva y es necesario conocer las formas o maneras que se puede realizar dicha operaci\'on. Para dichos casos es \'util conocer determinadas t\'ecnicas o estrategias de conteo que facilitar\'an el c\'alculo se\~nalado.

\section[Permutaciones]{Permutaciones}
Se llama permutaciones de m elementos (m = n) a las diferentes agrupaciones de esos m elementos de forma que:

S\'i entran todos los elementos.

S\'i importa el orden.

No se repiten los elementos. \ 

\section[Combinaciones]{Combinaciones}
Las combinaciones de orden m de n objetos: son los grupos de m objetos que se pueden formar entre los n, de modo que dos cualesquiera difieran en alg\'un objeto. A diferencia de las variaciones, en las combinaciones, no importa el orden de sucesi\'on de los elementos.

\section[Logaritmos]{Logaritmos}
El logaritmo de un n\'umero, en una base dada, es el exponente al cual se debe elevar la base para obtener el n\'umero.

 $logx=y\rightarrow a^{y}=x$

Se lee {\textquotedblleft}logaritmo de x en base a es igual a y{\textquotedblright}, pero debe cumplir con la condici\'on general de que a (la base) sea mayor que cero y a la vez distinta de uno:

 $\begin{matrix}a>0\\a\neq 1\end{matrix}$

Para aclarar el concepto, podr\'iamos decir que logaritmo es solo otra forma de expresar la potenciaci\'on, como en este ejemplo:

 $3^{2}=9\rightarrow log9=2$

Que leeremos: logaritmo de 9 en base 3es igual a 2

Esto significa que una potencia se puede expresar como logaritmo y un logaritmo se puede expresar como potencia.

Cuando la base no aparece expresada se supone que \'esta es 10:

 $log0,001=y$, el10que indica la base, no se coloca, se supone, as\'i:

 $log0,001=y\rightarrow 10^{y}=0,001\rightarrow 10^{y}=10^{-3}\rightarrow y=-3$

Aqu\'i, otra nota importante, para no olvidar: Los logaritmos que tienen base e se llaman logaritmos neperianos o naturales. Para representarlos se escribe ln o bien L. La base e est\'a impl\'icita, no se escribe.

\subsection[Logaritmos decimales:]{Logaritmos decimales:}
Son los que tienen base 10. Se representan por log (x) (ya vimos que la base 10 no se escribe, queda impl\'icita).

\subsection[Logaritmos neperianos o naturales:]{Logaritmos neperianos o naturales:}
Son los que tienen base e. Se representan por ln (x) o L(x) (ya vimos que la base e tampoco se escribe, se subentiende cuando aparece ln).

Algunos ejemplos de logaritmos neperianos son:

ln 1 = 0; puesto que e0 = 1

ln e2 = 2; puesto que e2 = e2

ln e$-$1 = $-$1; puesto que e$-$1 = e$-$1

El n\'umero e tiene gran importancia en las Matem\'aticas. No es racional (no es cociente de dos n\'umeros enteros) y su valor, con seis cifras decimales, es: e = 2,718281...

Propiedades de los logaritmos:

No existe el logaritmo de un n\'umero con base negativa.

 $\log _{-x}y=\mathit{ne}$ \newline
No existe el logaritmo de un n\'umero negativo.

 $\log -x=\mathit{ne}$ \newline
No existe el logaritmo de cero.

 $\log 0=\mathit{ne}$ \newline
El logaritmo de 1 es cero.

 $\log 1=0$ \newline
El logaritmo de a en base a es uno.

 $loga=1$

El logaritmo en base a de una potencia en base a es igual al exponente.

 $loga^{n}=n$

El logaritmo de un producto es igual a la suma de los logaritmos de los factores:

 $log(x\normalsubformula{\text{*}}y)=logx+logy$

 $log4\normalsubformula{\text{*}}8=log4+log8=2+3=5$

El logaritmo de un cociente es igual al logaritmo del dividendo menos el logaritmo del divisor:

 ${log}_{a}(\frac{x}{y})=logx\text{--}logy$

El logaritmo de una potencia es igual al producto del exponente por el logaritmo de la base:

 ${log}_{a}(x^{n})={nlog}_{a}x$

El logaritmo de una ra\'iz es igual al cociente entre el logaritmo del radicando y el \'indice de la ra\'iz:

 ${log}_{2}(4\sqrt{8})=\frac{1}{4}{log}_{2}8=\frac{1}{4}\normalsubformula{\text{*}}3=\frac{3}{4}$

Cambio de base:

 ${log}_{a}x=\frac{({log}_{b}x)}{({log}_{b}a)}$


\bigskip

\section[Derivadas]{Derivadas}

La derivada de una funci\'on en un punto x0 surge del problema de calcular la tangente a la gr\'afica de la funci\'on en el punto de abscisa x0, y fue Fermat el primero que aport\'o la primera idea al tratar de buscar los m\'aximos y m\'inimos de algunas funciones. En dichos puntos las tangentes han de ser paralelas al eje de abscisas, por lo que el \'angulo que forman con \'este es de cero grados. En estas condiciones, Fermat buscaba aquellos puntos en los que las tangentes fueran horizontales

La derivada de una funci\'on es una medida de la rapidez con la que cambia el valor de dicha funci\'on matem\'atica, seg\'un cambie el valor de su variable independiente.

Un ejemplo habitual aparece al estudiar el movimiento: si una funci\'on representa la posici\'on de un objeto con respecto al tiempo, su derivada es la velocidad de dicho objeto

Sean a, b, e y k constantes (n\'umeros reales) y consideremos a: u(x) y v(x) como funciones existen las siguientes propiedades.

Derivada de una constante

 $f(x)=k$ \ \ \ \ \ \ \ \ \ \ \  $f'(x)=0$

Derivada de x

 $\begin{matrix}f(x)=x\\f'(x)=1\end{matrix}$

Derivada de la funci\'on lineal

\  $\begin{matrix}f(x)=ax+b\\f'(x)=a\end{matrix}$

Derivada de una potencia

 $\begin{matrix}f(X)=u^{k}\\f'(x)=k\normalsubformula{\text{*}}u^{(k-1)}\normalsubformula{\text{*}}u'\end{matrix}$

Derivada de una ra\'iz cuadrada

 $\begin{matrix}f(x)=\sqrt{u}\\f'(x)=\frac{(u')}{(2\normalsubformula{\text{*}}\sqrt{u})}\end{matrix}$

Derivada de una ra\'iz

\  $\begin{matrix}f(x)=k\sqrt{u}\\f'(x)=\frac{(u')}{(k\normalsubformula{\text{*}}k\sqrt{u^{(k-1)}})}\end{matrix}$

Derivada de una suma

 $\begin{matrix}f(x)=u\pm v\\f'(x)=u'\pm v'\end{matrix}$

Derivada de una constante por una funci\'on

 $\begin{matrix}f(x)=k\normalsubformula{\text{*}}u\\f'(x)=k\normalsubformula{\text{*}}u'\end{matrix}$

Derivada de un producto

 $\begin{matrix}f(x)=u\normalsubformula{\text{*}}v\\f'(x)=u'\normalsubformula{\text{*}}v+u\normalsubformula{\text{*}}v'\end{matrix}$

Derivada de una constante partida por una funci\'on

 $\begin{matrix}f(x)=\frac{k}{v}\\f'(x)=\frac{(-k\normalsubformula{\text{*}}v')}{v^{2}}\end{matrix}$

Derivada de un cociente

 $\begin{matrix}f(x)=\frac{u}{v}\\f'(x)=\frac{(u'\normalsubformula{\text{*}}v\text{--}u\normalsubformula{\text{*}}v)}{v^{2}}\end{matrix}$

Derivada de la funci\'on exponencial

 $\begin{matrix}f(x)=a^{u}\\f'(x)=u'\normalsubformula{\text{*}}a^{u}\normalsubformula{\text{*}}{ln}_{a}\end{matrix}$

Derivada de la funci\'on exponencial de base e

\  $\begin{matrix}f(x)=e^{u}\\f'(x)=u'\normalsubformula{\text{*}}e^{u}\end{matrix}$

Derivada de un logaritmo

 $\begin{matrix}f(x)={log}_{a}u\\f'(x)=\frac{(u')}{(u\normalsubformula{\text{*}}lna)}=(u\frac{'}{u})\normalsubformula{\text{*}}{log}_{a}e\end{matrix}$

Como  ${log}_{a}e=\frac{(lne)}{(lna)}=\frac{1}{(lna)}$, tambi\'en se puede expresar as\'i:

 $\begin{matrix}f(x)={log}_{a}u\\f'(x)=(\frac{(u')}{(u)})\normalsubformula{\text{*}}(\frac{(1)}{(lna)})\end{matrix}$

Derivada del logaritmo neperiano

 $\begin{matrix}f(x)=lnu\\f'(x)=\frac{(u')}{(u)}\end{matrix}$

Derivada del seno

 $\begin{matrix}f(x)=senu\\f'(x)=u'\normalsubformula{\text{*}}cosu\end{matrix}$

Derivada del coseno

 $\begin{matrix}f(x)=cosu\\f'(x)=-u'\normalsubformula{\text{*}}senu\end{matrix}$

Derivada de la tangente

 $\begin{matrix}f(x)=tanu\\f'(x)=(\frac{u}{({cos}^{2}u)})=u'\normalsubformula{\text{*}}{sec}^{2}u=u'\normalsubformula{\text{*}}(1+{tan}^{2}u)\end{matrix}$

Derivada de la cotangente

 $\begin{matrix}f(x)=cotu\\f'(x)=-(\frac{(u')}{({sen}^{2}u)})=-u'\normalsubformula{\text{*}}{csc}^{2}u=-u'\normalsubformula{\text{*}}(1+{cot}^{2}u)\end{matrix}$

Derivada de la secante

 $\begin{matrix}f(x)=secu\\f'(x)=\frac{(u'\normalsubformula{\text{*}}senu)}{({cos}^{2}u)}=u'\normalsubformula{\text{*}}secu\normalsubformula{\text{*}}tanu\end{matrix}$

Derivada de la cosecante

 $\begin{matrix}f(x)=cscu\\f'(x)=-(\frac{(u'\normalsubformula{\text{*}}cosu)}{({sen}^{2}u)})=-u'\normalsubformula{\text{*}}cscu\normalsubformula{\text{*}}cotu\end{matrix}$

Derivada del arco seno

 $\begin{matrix}f(x)=arcsenu\\f'(x)=\frac{(u')}{\sqrt{(1-u^{2})}}\end{matrix}$

Derivada del arco coseno

 $\begin{matrix}f(x)=arccosu\\f'(x)=-(\frac{(u')}{\sqrt{(1-u^{2})}})\end{matrix}$

Derivada del arco tangente

\  $\begin{matrix}f(x)=arctanu\\f'(x)=\frac{(u')}{(1+u^{2})}\end{matrix}$

Derivada del arco cotangente

\  $\begin{matrix}f(x)=arccotu\\f'(x)=-(\frac{(u')}{(1+u^{2})})\end{matrix}$

Derivada del arco secante

 $\begin{matrix}f(x)=arcsecu\\f'(x)=\frac{(u')}{(u\normalsubformula{\text{*}}\sqrt{u^{2}}-1)}\end{matrix}$

Derivada del arco cosecante

 $\begin{matrix}f(x)=arccscu\\f'(x)=-(\frac{(u')}{(u\normalsubformula{\text{*}}\sqrt{u^{2}}-1)})\end{matrix}$

Derivada de la funci\'on potencial-exponencial

 $\begin{matrix}f(x)=u^{v}\\f'(x)=v\normalsubformula{\text{*}}u^{[v-1]}\normalsubformula{\text{*}}u'+u^{v}\normalsubformula{\text{*}}v'\normalsubformula{\text{*}}lnu\end{matrix}$

Regla de la cadena

 $(g\normalsubformula{\text{*}}f)'(x)=g'[f(x)]\normalsubformula{\text{*}}f'(x)$

Derivadas impl\'icitas

 $y'=\frac{(-F'_{x})}{(F'_{y})}$


\bigskip


\bigskip


\bigskip


\bigskip


\bigskip

\section[Bibliograf\'ia]{Bibliograf\'ia}

\bigskip

\'Indice detallado de:

Valentina S\'anchez

Jorge Ysabel

Daniel Caama\~no

Jos\'e Lluberes

Jos\'e Beltre

Joan Marcos Zoquier
\end{document}
