Relaciona con una l\'inea recta cada una de las siguientes {\color{cadmiumorange}palabras} con su {\color{cadmiumgreen}definici\'on}.

\begin{minipage}{0.5\linewidth}
    \begin{parts}
        \part Se define como la resistencia de un fluido para moverse.
        \hfill{\color{cadmiumgreen}$\square$}
        \part Es la capacidad de un material para oponerse al paso de una corriente el\'ectrica.
        \hfill{\color{cadmiumgreen}$\square$}
        \part Es la masa por unidad de volumen de una sustancia.
        \hfill{\color{cadmiumgreen}$\square$}
        \part Es la temperatura en la que un líquido cambia al estado gaseoso.
        \hfill{\color{cadmiumgreen}$\square$}
        \part De esta propiedad depende la cantidad de material que se va a estudiar.
        \hfill{\color{cadmiumgreen}$\square$}
        \part Medida de la cantidad de materia que hay en un objeto.
        \hfill{\color{cadmiumgreen}$\square$}
        \part Propiedad en la que su valor es independiente de la cantidad de sustancia analizada.
        \hfill{\color{cadmiumgreen}$\square$}
        \part Cantidad máxima de una sustancia que puede disolverse en otra.
        \hfill{\color{cadmiumgreen}$\square$}
        \part Temperatura en la que un sólido cambia al estado líquido.
        \hfill{\color{cadmiumgreen}$\square$}
        \part Espacio que ocupa un material.
        \hfill{\color{cadmiumgreen}$\square$}
    \end{parts}
\end{minipage}
\begin{minipage}{0.4\linewidth}
    \checkboxchar{ {\color{cadmiumorange}
                $\Box$}
    }
    \begin{checkboxes}
        \choice Masa \vspace{0.5cm}
        \choice Intensiva\vspace{0.5cm}
        \choice Temperatura de ebullici\'on    \vspace{0.5cm}
        \choice Volumen    \vspace{0.5cm}
        \choice Temperatura de fusi\'on    \vspace{0.5cm}
        \choice Viscosidad  \vspace{0.5cm}
        \choice Extensiva \vspace{0.5cm}
        \choice Densidad \vspace{0.5cm}
        \choice Solubilidad \vspace{0.5cm}
        \choice Resistividad
    \end{checkboxes}

\end{minipage}