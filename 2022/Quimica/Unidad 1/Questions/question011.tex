\question[15] Completa el siguiente mapa mental sobre la materia y sus propiedades.

\begin{tikzpicture}[ every annotation/.style = {draw,
        fill = white, font = \HUGE}]
  \path[mindmap,concept color=colorrds!50,text=black]
  node[concept] {La materia \\{\small(Todo lo que existe y ocupa un lugar en el espacio)}}
  [clockwise from=-45]
  child[concept color=green!20!black,text=white] {
      node[concept] {se clasifica en...}
        [clockwise from=-45]
      child { node[concept] {Mezclas}
            [clockwise from=285]
          child[concept color=green!50!black] {
              node[concept] {heterogéneas}
                [clockwise from=15]
              child { node[concept] {Coloides} }
              child { node[concept] {Emulsiones} }
              child { node[concept] {Suspensi\'on} }
              child { node[concept] {Disoluci\'on saturada} }
            }
          child[concept color=green!50!black] {
              node[concept] {homogéneas}
                [clockwise from=275]
              child { node[concept] {Disoluci\'on insaturada} }
              child { node[concept] {Aleación} }
            }
        }
      child { node[concept] {Sustancias puras}
            [clockwise from=-115]
          child[concept color=green!50!black] {
              node[concept] {Compuestos}
            }
          child[concept color=green!50!black] {
              node[concept] {Sustancias elementales }
            }
        }
    }
  % child[concept color=red] { node[concept] {technical} }
  % child[concept color=orange] { node[concept] {theoretical} };
\end{tikzpicture}