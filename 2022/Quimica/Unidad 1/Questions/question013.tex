En una muestra de sangre de 2 mL se encontraron 0.0011 mg de mercurio.
La muestra pertenece a un trabajador de una mina de mercurio y se requiere calcular
la concentración de mercurio a fin de tomar las medidas necesarias para prevenir daños
graves en su salud. En la tabla se muestran los niveles tóxicos de concentración de
mercurio en seres humanos medida en miligramos por litro (mg/L).
\begin{center}
    \begin{frame}{}
        \centering
        \begin{tabular}{lr}
            \hline
            \rowcolor{colorrds}
            \color{white} Contaminante & \color{white}Concentración (mg/L) \\ \hline
            Límite máximo permitido    & 0.049                             \\ \hline
            Aparición de síntomas      & 0.49                              \\ \hline
            Problemas graves           & 1.29                              \\ \hline
        \end{tabular}
    \end{frame}
\end{center}

\begin{parts}
    \part La concentración de mercurio en la muestra de sangre es:
    \begin{choices}
        \choice 0.55 mg/L
        \choice 0.055 mg/L
        \choice 5.5 mg/L
        \choice 55 mg/L
    \end{choices}

    \part ¿La concentración de mercurio en la sangre del trabajador representa un riesgo para su salud?

    \begin{choices}
        \choice No, puesto que la concentración es menor que el límite permitido.
        \choice Sí, porque la concentración de mercurio es mayor que el límite permitido.
        \choice No, porque la concentración está en el límite permitido y el organismo aún no se afecta.
        \choice Sí, puesto que la concentración es mayor que el límite permitido y es probable que aparezcan los primeros síntomas de intoxicación.
    \end{choices}
\end{parts}
