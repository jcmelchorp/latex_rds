\documentclass{exam-n}
\begin{document}
\begin{questions}
    \question Analiza las siguientes situaciones y responde a cada una de las preguntas.
    \begin{parts}
        \part La autopista 10 en Haradh, Arabia Saudita, tiene más de 200
        km en línea recta. Un automóvil viaja por esta carretera con
        velocidad constante de 120 km/h durante 200 km.
        \begin{table}[!h]
            \label{tab:arabia}
            \centering
            \begin{tabular}{|c|c|}
                \hline
                Distancia [km] & Tiempo [h] \\
                \hline
                50             &            \\
                75             &            \\
                100            &            \\
                125            &            \\
                150            &            \\
                175            &            \\
                200            &            \\
                \hline
            \end{tabular}
            \caption{}
        \end{table}
        \begin{itemize}
            \item Completa la tabla \ref{tab:arabia} que registra la velocidad del automóvil.
            \item ¿Cómo es la variación de los datos de la tabla? ¿Por qué?
            \item ¿En cuántos minutos recorre 1 km? Explica tu procedimiento.
            \item ¿En cuánto tiempo se recorrieron 120 km?
            \item ¿Y en cuánto tiempo se recorrerán 230 km?
        \end{itemize}
    \end{parts}

    \question Analicen los rectángulos de la figura 2.1. La tabla \ref{tab:rectangulos} muestra las medidas de los lados de cada uno. Complétenla.
    \begin{table}[!h]
        \label{tab:rectangulos}
        \centering
        \begin{tabular}{|l|c|c|c|}
            \hline
                         & Lado 1 (u) & Lado 2 (u) & Área (u$^2$) \\
            \hline
            Rectángulo 1 & 2          & 24         & 48           \\
            Rectángulo 2 & 3          &            &              \\
            Rectángulo 3 & 5          &            &              \\
            Rectángulo 4 & 6          &            &              \\
            \hline
        \end{tabular}
        \caption{}
    \end{table}
    \begin{itemize}
        \item Completa la tabla \ref{tab:rectangulos} que muestra la medida de los lados de un conjunto de rect\'angulos.
        \item ¿Cómo es la variación de los datos de la tabla? ¿Por qué?
        \item Si se añade el rectángulo cuyo lado 1 mide 4 u, ¿cuál es la medida del otro lado? ¿Cuál es su área? Expliquen su procedimiento.
        \item ¿Cómo es el área de los rectángulos?
    \end{itemize}
\end{questions}

\end{document}