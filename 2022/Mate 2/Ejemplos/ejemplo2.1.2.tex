{\color{brown}\textbf{Ejemplo 2}}

Cinthia va a la escuela en bicicleta desde su casa.
Ella calcula que llega al colegio en 45 minutos cuando va a una velocidad promedio de 0.75 kilómetros por minuto.\\
\textbf{¿Cuánto tiempo tardará si cambia la velocidad a 0.5 kilómetros por minuto?}\\

Asignemos $t$ al tiempo que demora en ir de su casa a la escuela y $v$ a la velocidad promedio de su bicicleta.
\begin{center}
    $t$ = tiempo \quad $v$ = velocidad.
\end{center}
Observa que la relación entre el tiempo y la velocidad es una relación inversa. Planteamos esa relación inversa entre $x$ e $y$.
\[v=k \times \frac{1}{t}\]
Reemplazamos los valores que nos dan en la situación, 45 minutos a una velocidad de 0.75 kilómetros por minuto. Es decir $x=45$ e $y=0.75$.
\[0.75=k\times \frac{1}{45} \Rightarrow k=33.75\]
Por tanto, la relación proporcional es:
\[v=33.75 \times \frac{1}{t}\]
Como nos piden calcular la cantidad de minutos que tarda en llegar a la escuela a una velocidad de 0.5 kilómetros por minuto, sabemos que \[0.5=33.75 \times \frac{1}{t} \Rightarrow t=67.5 \text{ minutos}\]
Por tanto, Cinthia tardará 67.5 minutos en llegar a la escuela a una velocidad de 0.5 kilómetros por minuto.\\