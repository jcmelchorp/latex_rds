\question[10] Escribe el \textbf{n\'umero decimal} equivalente a cada una de las siguientes fracciones.
Expresa tu resultado como un decimal exacto o utilizando la notación para decimales peri\'odicos segun sea el caso.
\begin{multicols}{2}
    \begin{parts}
        % {\printanswers
        % \subfile{Questions/Parts/question001a}
        \subfile{Questions/Parts/question001b}
        % }
        %\subfile{Questions/Parts/question001c}
        % \subfile{Questions/Parts/question001i}
        \subfile{Questions/Parts/question001d}
        % \subfile{Questions/Parts/question001g}
        % \subfile{Questions/Parts/question001h}
        % \subfile{Questions/Parts/question001j}
        % \subfile{Questions/Parts/question001e}
        % \subfile{Questions/Parts/question001k}
        % \subfile{Questions/Parts/question001f}
    \end{parts}
\end{multicols}