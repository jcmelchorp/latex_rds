[20] Un ciclista se estaba moviendo hacia la izquierda con una velocidad de $14$ m/s. Después de una ráfaga
de viento constante que dura $3.5$ s, el ciclista se mueve hacia la izquierda con una velocidad de $21$ m/s.

\textbf{¿Cuál es la aceleración del ciclista?}

\begin{minipage}[c]{\linewidth}
    \begin{solutionbox}{4.5cm}
        \begin{minipage}[t]{0.3\textwidth}
            \centering
            \underline{Datos:}
            \begin{align*}
                t   & =  3.5 \text{ s}  \\
                v_i & = 14  \text{ m/s} \\
                v_f & = 21  \text{ m/s}
            \end{align*}
        \end{minipage}%
        \begin{minipage}[t]{0.3\textwidth}
            \centering
            \underline{F\'ormula:}
            \begin{equation*}
                a=\frac{v_f-v_i}{t}
            \end{equation*}
        \end{minipage}
        \begin{minipage}[t]{0.3\textwidth}
            \centering
            \underline{Sustituci\'on y resultado:}
            \begin{align*}
                a & =\frac{21 \text{ m/s}-14 \text{ m/s}}{3.5 \text{ s}} \\
                  & = \frac{7 \text{ m/s}}{3.5 \text{ s}}                \\
                  & =2 \text{ m/s}^2
            \end{align*}
        \end{minipage}
    \end{solutionbox}
\end{minipage}