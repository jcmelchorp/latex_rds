Relaciona con una l\'inea recta el {\color{cadmiumgreen}enunciado} con las unidades de  {\color{cadmiumorange}tiempo}   que las representa.
\begin{minipage}{0.8\linewidth}
    \begin{parts}
        \part 1,825 días o 60 meses son un\dots
        \hfill{\color{cadmiumgreen}$\square$}
        \part La Tierra completa su per\'iodo de rotaci\'on en esta unidad de tiempo.
        \hfill{\color{cadmiumgreen}$\square$}
        \part Hay tortugas que llegan a vivir más de uno, el árbol más viejo del mundo hace poco que cumplió 5 y nuestro país es tan jóven que hace 9 años cumplió 2. ¿De qué unidad de tiempo estamos hablando?
        \hfill{\color{cadmiumgreen}$\square$}
        \part Los hay de 28, ocasionalmente 29, 30 y 31 días, pero siempre son 12.
        \hfill{\color{cadmiumgreen}$\square$}
        \part La Tierra completa su per\'iodo de traslaci\'on en esta unidad de tiempo.
        \hfill{\color{cadmiumgreen}$\square$}
        \part 87,600 horas o conforman una\dots
        \hfill{\color{cadmiumgreen}$\square$}
    \end{parts}
\end{minipage}%
\begin{minipage}{0.2\linewidth}
    \checkboxchar{ {\color{cadmiumorange}
                $\Box$}
    }
    \begin{checkboxes}
        \choice d\'ecada \vspace{0.2cm}
        \choice a\~no    \vspace{0.2cm}
        \choice d\'ia    \vspace{0.2cm}
        \choice siglo    \vspace{0.2cm}
        \choice lustro   \vspace{0.2cm}
        \choice mes
    \end{checkboxes}
\end{minipage}
