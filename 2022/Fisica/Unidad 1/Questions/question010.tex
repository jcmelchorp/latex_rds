Una persona en patineta se estaba moviendo hacia la derecha con una velocidad de $8$ m/s.
Después de una ráfaga de viento constante de $5$ s, la persona se mueve hacia la derecha con una velocidad de $5$ m/s.
Suponiendo que la aceleración es contante,

\textbf{¿Cuál fue la aceleración de la persona durante el periodo de $5$ s?}

\begin{minipage}[c]{\linewidth}
    \begin{solutionbox}{5cm}
        \begin{minipage}[t]{0.3\textwidth}
            \centering
            \underline{Datos:}
            \begin{align*}
                t   & =  5 \text{ s}   \\
                v_i & = 8  \text{ m/s} \\
                v_f & = 5  \text{ m/s}
            \end{align*}
        \end{minipage}%
        \begin{minipage}[t]{0.3\textwidth}
            \centering
            \underline{F\'ormula:}
            \begin{equation*}
                a=\frac{v_f-v_i}{t}
            \end{equation*}
        \end{minipage}
        \begin{minipage}[t]{0.3\textwidth}
            \centering
            \underline{Sustituci\'on y resultado:}
            \begin{align*}
                a & =\frac{5 \text{ m/s}-8 \text{ m/s}}{5 \text{ s}} \\
                  & = \frac{-3 \text{ m/s}}{5 \text{ s}}             \\
                  & =-0.6 \text{ m/s}^2
            \end{align*}
        \end{minipage}
    \end{solutionbox}
\end{minipage}