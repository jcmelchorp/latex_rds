Anita es maestra y tiene 160 niñas y 120 niños en su grupo. Ella quiere dividir al grupo en equipos del mismo tamaño,
en donde cada equipo tenga el mismo número de niñas y el mismo número de niños. Si Anita crea el mayor número de
equipos posible.

\part ¿cuántos niños habrá en cada equipo?

\begin{solutionbox}{5cm}
    \begin{minipage}[c]{0.2\linewidth}
        \begin{tabular}{ cc|c }
            160 & 120 & \circled{2} \\
            80  & 60  & \circled{2} \\
            40  & 30  & \circled{2} \\
            20  & 15  & \circled{5} \\
            4   & 3   & 2           \\
            2   & 3   & 2           \\
            1   & 3   & 3           \\
            1   & 1   &             \\
        \end{tabular}
    \end{minipage}%
    \begin{minipage}[c]{0.7\linewidth}
        El mayor número de estudiantes que puede haber en cada fila es el n\'umero m\'as grande que puede
        dividir de forma entera a 160 y a 120; es decir, el m\'aximo com\'un divisor (MCD) de 160 y 120.
        $\Rightarrow$\\ MCD(160,120)$=2^3\times5=40 \therefore$ el mayor número de estudiantes que puede haber en cada fila es:

    \end{minipage}
\end{solutionbox}
