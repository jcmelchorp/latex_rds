\question Analiza cada una de las siguientes situaciones y responde a las preguntas.

``Dos luces del estadio local están parpadeando. Ambas acaban de parpadear al mismo tiempo.
Una de las luces parpadea cada 12 segundos y la otra parpadea cada 18 segundos.''
\begin{parts}
    \part¿Cuánto tiempo pasará antes de que ambas luces vuelvan a parpadear al mismo tiempo?
    \begin{solutionbox}{5cm}
        \begin{minipage}[c]{0.2\linewidth}
            \begin{tabular}{ cc|c }
                12 & 18 & \circled{2} \\
                6  & 9  & \circled{3} \\
                2  & 3  & 2           \\
                1  & 3  & 3           \\
                1  & 1  &
            \end{tabular}
        \end{minipage}%
        \begin{minipage}[c]{0.7\linewidth}
            El momento en que las dos luces coincidir\'an ser\'a el n\'umero m\'as pequeño que sea m\'ultiplo de 12 y 18; es decir, el m\'inimo com\'un m\'ultiplo (MCM) de 12 y 18.
            $\Rightarrow$\\ El MCM(12,18)$=2\times3\times2\times3=36 \therefore$ el intervalo de tiempo m\'as pr\'oximo a coincidir es de 36 segundos.
        \end{minipage}
    \end{solutionbox}
    \newpage
    ``Hay 76 niños y 92 niñas en el equipo de matemáticas de Cirilo.
    Para la siguiente competencia de matemáticas, Cirilo quiere acomodar a los estudiantes en filas iguales con solo niñas o solo niños en cada fila.''



    \part ¿Cuál es el mayor número de estudiantes que puede haber en cada fila?

    \begin{solutionbox}{5cm}
        \begin{minipage}[c]{0.2\linewidth}
            \begin{tabular}{ cc|c }
                76 & 92 & \circled{2} \\
                38 & 46 & \circled{2} \\
                19 & 23 & 19          \\
                1  & 23 & 23          \\
                   & 1  &
            \end{tabular}
        \end{minipage}%
        \begin{minipage}[c]{0.7\linewidth}
            El mayor número de estudiantes que puede haber en cada fila es el n\'umero m\'as grande que puede dividir de forma entera a 76 y a 92; es decir, el m\'aximo com\'un divisor (MCD) de 76 y 92.
            $\Rightarrow$\\ El MCD(76,92)$=2^2=4 \therefore$ el mayor número de estudiantes que puede haber en cada fila es 4.
        \end{minipage}
    \end{solutionbox}
    \part ¿Cuántas filas estar\'an ocupadas por niñas?
    \begin{solutionbox}{4cm}
        \begin{minipage}[c]{\linewidth}
            Ya que hay 92 ni\~nas repartidas en filas de 4 estudiantes, la cantidad de filas ocupadas por ni\~nas es:
            \begin{align*}
                \dfrac{92}{4} & = 23
            \end{align*}
            Las filas ocupadas por ni\~nas son 23.
        \end{minipage}

    \end{solutionbox}

    \part ¿Cuántas filas estar\'an ocupadas por niños?
    \begin{solutionbox}{4cm}
        \begin{minipage}[c]{\linewidth}
            Ya que hay 76 ni\~nos repartidos en filas de 4 estudiantes, la cantidad de filas ocupadas por ni\~nos es:
            \begin{align*}
                \dfrac{76}{4} & = 19
            \end{align*}
            Las filas ocupadas por ni\~nos son 19.
        \end{minipage}

    \end{solutionbox}
\end{parts}