\question[10] Relaciona con una l\'inea recta los {\color{cadmiumorange}conceptos} con su {\color{cadmiumgreen}significado} que las representa.

\begin{minipage}{0.45\linewidth}
    \begin{parts}
        \part Este tipo de ondas necesitan un medio físico para propagarse. \hfill{\color{cadmiumgreen}$\square$}
        \part Número de ondas que pasan por un punto en un tiempo dado. \hfill{\color{cadmiumgreen}$\square$}
        \part El tiempo que le toma a una onda pasar por un punto. \hfill{\color{cadmiumgreen}$\square$}
        \part En este tipo de ondas, la perturbación se produce en la misma dirección en que viajan. \hfill{\color{cadmiumgreen}$\square$}
        \part Es la distancia entre dos crestas o dos valles. \hfill{\color{cadmiumgreen}$\square$}
        \part Máximo desplazamiento de una onda. \hfill{\color{cadmiumgreen}$\square$}
        \part En este tipo de ondas, la perturbación se produce perpendicularmente a la dirección en que viajan. \hfill{\color{cadmiumgreen}$\square$}
        \part Cambio de velocidad de una onda al pasar de un medio físico a otro. \hfill{\color{cadmiumgreen}$\square$}
        \part Cambio de dirección de una onda cuando choca con un medio en el que no puede viajar. \hfill{\color{cadmiumgreen}$\square$}
        \part Este tipo de ondas se propagan en el vacío, por ejemplo, la luz. \hfill{\color{cadmiumgreen}$\square$}
    \end{parts}
\end{minipage}
\begin{minipage}{0.4\linewidth}
    \checkboxchar{ {\color{cadmiumorange}
                $\Box$}
    }
    \begin{checkboxes}
        \choice Onda transversal \vspace{0.6cm}
        \choice Reflexión \vspace{0.6cm}
        \choice Onda longitudinal \vspace{0.6cm}
        \choice Longitud de onda \vspace{0.6cm}
        \choice Refracci\'on \vspace{0.6cm}
        \choice Onda mec\'anica \vspace{0.6cm}
        \choice Per\'iodo \vspace{0.6cm}
        \choice Frecuencia de onda \vspace{0.6cm}
        \choice Amplitud \vspace{0.6cm}
        \choice Onda electromagnética \vspace{0.6cm}
    \end{checkboxes}


\end{minipage}
