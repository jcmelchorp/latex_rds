\usepackage{verbatim}

\begin{comment}
:Title: Fancy chapter headings
:Grid: 2x2

An example of how you can use TikZ and absolute positioning to create fancy chapter headings.

:Source: `TeXblog: Fancy chapter headings with TikZ`__

__ http://texblog.net/latex-archive/layout/fancy-chapter-tikz/

\end{comment}

%\usepackage{kpfonts}
\usepackage[explicit]{titlesec}
\newcommand*\chapterlabel{}
\titleformat{\chapter}
{\gdef\chapterlabel{}
    \comfortaa\huge\bfseries
}
{\gdef\chapterlabel{\thechapter\ }}{0pt}
{\begin{tikzpicture}[remember picture,overlay]
        \node[yshift=-3cm] at (current page.north west)
        {\begin{tikzpicture}[remember picture, overlay]
                \draw[fill=colorrds!20] (0,0) rectangle
                (\paperwidth,3cm);
                \node[anchor=east,xshift=.9\paperwidth,rectangle,
                    rounded corners=10pt,inner sep=11pt,
                    fill=colorrds]
                {\color{white}\textbf{\chapterlabel#1}};
            \end{tikzpicture}
        };
    \end{tikzpicture}
}
\titlespacing*{\chapter}{0pt}{50pt}{-60pt}


\newcommand*\sectionlabel{}
\titleformat{\section}
{\gdef\sectionlabel{}
    \comfortaa\Large\bfseries
}
{\gdef\sectionlabel{\thesection\ }}{0pt}
{\begin{tikzpicture}[remember picture,overlay]
        \node[yshift=-2cm] at (current page.north west)
        {\begin{tikzpicture}[remember picture, overlay]
                \draw[fill=blue] (0,0) rectangle
                (\paperwidth,2cm);
                \node[anchor=east,xshift=.9\paperwidth,rectangle,
                    rounded corners=10pt,inner sep=11pt,
                    fill=blue!35]
                {\color{white}\sectionlabel};
                % \node[anchor=east,xshift=.9\paperwidth,rectangle,
                %     rounded corners=10pt,inner sep=11pt,
                %     fill=blue!35]
                % {\color{green}\sectionlabel};
            \end{tikzpicture}
        };
    \end{tikzpicture}
}
\titlespacing*{\section}{0pt}{50pt}{0pt}
% \newcommand*\subsectionlabel{}
% \titleformat{\subsection}
% {\gdef\subsectionlabel{}
% %\normalfont%\sffamily\Huge\bfseries\scshape
% \comfortaa\bfseries
% }
{\gdef\sectionlabel{}
    %\normalfont%\sffamily\Huge\bfseries\scshape
    \comfortaa\Large\bfseries
}
%{\gdef\subsectionlabel{\thesubsection\ }}{0pt}

