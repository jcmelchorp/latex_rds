{\color{brown}\textbf{Ejemplo 1}}
Marcos sale diariamente con su bicicleta y recorre todo el contorno del parque de su barrio. Él sabe que tarda aproximadamente 6 minutos en dar 3 vueltas al parque.
Si Marcos quiere dar 12 vueltas al parque,
\textbf{¿cuánto tiempo tardará?}\\

Asignemos $x$ a la cantidad de vueltas e $y$ el tiempo necesario para dar las $x$ vueltas.

\begin{center}
    $x$ = n\'umero de vueltas. \quad $y$ = tiempo para dar las vueltas.
\end{center}
Planteamos la relación directa entre $x$ e $y$.
\[y=kx\]
Reemplazamos los valores que nos dan en la situación; 6 minutos para dar 3 vueltas. Es decir, $x=3$ e $y=6$.
\[6=k(3) \Rightarrow k=2\]
Por tanto, la relación proporcional es $y=2x$.
Como nos piden calcular la cantidad de minutos que necesita Marcos para dar 12 vueltas, sabemos que $x=12$.
Reemplazamos:
\[y=2(12) \Rightarrow y=24\]
Por tanto, Marcos tardará 24 minutos en dar 12 vueltas alrededor del parque de su barrio.\\