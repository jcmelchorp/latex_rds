Manuel canjea sus estampillas por canicas. Cada día canjea dos estampillas más
que el día anterior.
El canje se realiza de la siguiente forma: por cada estampilla le entregan dos
canicas.
Para ordenar y contar las canicas que recibirá, él elaboró la Tabla
\ref{tab:estampillas}:

\begin{table}[H]
    \centering
    \caption{}
    \label{tab:estampillas}
    \begin{tabular}{c|c|c|c|c}
        Día         & 1 & 2 & 3  & 4  \\ \hline
        Estampillas & 1 & 3 & 5  & 7  \\ \hline
        Canicas     & 2 & 6 & 10 & 14
    \end{tabular}
\end{table}

Si Manuel suma la cantidad de canicas que recibió cada día,
\textbf{¿cuántas canicas en total tendrá Manuel por el canje de sus estampillas
    al término de 365 días?}\\

\begin{solutionbox}{6.5cm}
    La regla de recurrencia para la serie de canicas es:
    \[a_n=4(n-1)+2\]
    Calculando el $365^{vo}$ término de la serie
    \[a_{365}=4(365-1)+2=1458\]
    Utilizando la suma de los términos de una serie:
    \[s_{365}=\dfrac{365(2+1458)}{2}=266,450\]
    Manuel tendrá 266,450 canicas al cabo de 365 días.
\end{solutionbox}