\begin{multicols}{2}
    \textbf{¿Cuál es el \'area del triángulo de la figura \ref{fig:area_isoseles_03}?}

    \begin{figure}[H]
        \centering
        \includegraphics[width=0.4\linewidth]{../images/area_isoseles_03.png}
        \caption{}
        \label{fig:area_isoseles_03}
    \end{figure}
\end{multicols}

\vspace{-0.5cm}
\begin{solutionbox}{7cm} \footnotesize
    \begin{minipage}{0.2\textwidth}
        \begin{figure}[H]
            \centering
            \includegraphics[width=0.9\textwidth]{../images/area_isoseles_03a.png}
            \caption{}
            \label{fig:area_isoseles_03a}
        \end{figure}
    \end{minipage}%
    \begin{minipage}{0.7\textwidth}
        Para determinar el área del triángulo debemos saber la base y la altura. Llamemos $x$ a la longitud (ver Figura \ref{fig:area_isoseles_03a}).
        Cuando tenemos un triángulo rectángulo, podemos usar el teorema de Pitágoras para obtener la longitud del cateto.
        La ecuación para el teorema de Pitágoras es:
        \[c^2=a^2+b^2\]
    \end{minipage}

    \vspace{-0.5cm}
    \begin{multicols}{2}
        En este caso, $a=4$, $b=x$ y $c=5$. Entonces,

        \begin{align*}
            4^2+x^2 & =5^2   \\
            16+x^2  & =25    \\
            x^2     & =25-16 \\
            x^2     & =9     \\
            x       & =3
        \end{align*}

        La altura del triángulo es 3 y el área del triángulo es:

        \begin{align*}
            A & =\frac{1}{2}bx             \\
            A & =\frac{1}{2}\cdot 8\cdot 3 \\
            A & =12 \text{ u}^2
        \end{align*}
    \end{multicols}
\end{solutionbox}
