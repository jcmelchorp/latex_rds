\documentclass[12pt,addpoints]{repaso}
\grado{3}
\nivel{Secundaria}
\cicloescolar{2025-2026}
\materia{Matemáticas}
\unidad{1}
\title{Practica la Unidad}
\aprendizajes{
      \item Resuelve problemas que impliquen la suma, resta, la multiplicación y la división de números enteros.
      \item Identifica y ubica números negativos en una recta numérica.
      \item Identifica y factoriza expresiones algebraicas. 
      \item Aplica las leyes de los exponentes para simplificar expresiones algebraicas y resuelve problemas que involucren exponentes.
      \item Identifica y completa sucesiones aritméticas, calcula la diferencia común y formula el término general.
}     
\author{Melchor Pinto, J.C.}
\begin{document}
\INFO%
\begin{multicols}{2}
	\tableofcontents
\end{multicols}
\vfill
% \afterpage{\blankpage}
\begin{questions}

      \section{Cálculos numéricos}

      \questionboxed[5]{Realiza las siguientes operaciones de \textit{cálculo numérico}:
  
      \begin{parts}
                  \begin{multicols}{2}
	\subsection{Suma de números}
    
      \part $849.332+242.25+469.381=$ \fillin[$1560.963$][0in]
                        
                        \part $687+547+464=$ \fillin[$1698$][0in]
                        
                        \part $344.12+34.25+729.12=$ \fillin[$1107.49$][0in]
                        
                        \part $3\dfrac{3}{5}+2\dfrac{5}{8}=$ \fillin[$6\dfrac{9}{40}$][0in]
                        % \addcontentsline{toc}{subsection}{Resta de números}
	
                        \subsection{Resta de números}
     
      \part $82.48-28.19=$ \fillin[$54.29$][0in]
                        
                        \part $4\dfrac{4}{5}-1\dfrac{1}{2}=$ \fillin[$3\dfrac{3}{10}$][0in]
                        
                        \part $45.487-29.229=$ \fillin[$16.258$][0in]
                        
                        \part $2\dfrac{2}{3}-2\dfrac{2}{5}=$ \fillin[$\dfrac{4}{15}$][0in]
                        % \addcontentsline{toc}{subsection}{Multiplicación de números}
	
                        \subsection{Multiplicación de números}
    
      \part $4.5\times2.3=$ \fillin[$10.35$][0in]
                        
                        \part $\dfrac{7}{8}\times\dfrac{6}{5}=$ \fillin[$1\dfrac{1}{20}$][0in]
                        
                        \part $26.37\times13=$ \fillin[$343.81$][0in]
                        
                        \part $1\dfrac{1}{4}\times 1\dfrac{2}{3}=$ \fillin[$2\dfrac{1}{12}$][0in]
                        % \addcontentsline{toc}{subsection}{División de números}
	
                        \subsection{División de números}
      
      \part $922\divisionsymbol1.2=$ \fillin[$768.333$][0in]
                        
                        \part $0.1\divisionsymbol0.02=$ \fillin[$5$][0in]
                        
                        \part $180\divisionsymbol 0.09=$ \fillin[$2000$][0in]
                        
                        \part $25.25\divisionsymbol 0.5=$ \fillin[$50.5$][0in]

                        \subsection{Resolución de problemas}
     
      \part Natalia al vender su carro en \$135,450 pesos, obtiene una ganancia de \$25,400 pesos, ¿Cuánto le costó su carro?
                       
                        \begin{solutionbox}{2cm}
                              El costo del carro fue de
                              \[\$135,450-\$25,400 =\$110,050
                              \]
                        \end{solutionbox}

                        % \part Si el millar de hojas de papel tiene un costo de \$655 pesos, ¿cuál es el precio por una sola hoja?
                        % \begin{solutionbox}{2cm}
                        %       El precio por una sola hoja es
                        %       \[
                        %             \dfrac{\$655}{1000}=\$0.655
                        %       \]
                        % \end{solutionbox}

                        % \part Un campesino vendió arroz y le pagaron \$3,565 pesos por 23 sacos. ¿A cómo vendió cada saco?
                        % \begin{solutionbox}{2cm}
                        %       El precio por cada saco es
                        %       \[
                        %             \dfrac{\$3,565}{23}=\$155
                        %       \]
                        % \end{solutionbox}

                  \end{multicols}
            \end{parts}
      }
      % \newpage

      % \addcontentsline{toc}{section}{Factorización}
	\section{Factorización}
      % \addcontentsline{toc}{subsection}{Término común}
	\subsection{Término común}
     
      \questionboxed[6]{Factoriza las siguientes expresiones algebraicas:
   
      \begin{multicols}{2}
                  \begin{parts}
                        \part $mno-mnp=$ \fillin[$mn(o-p)$][0in]\\
                        \part $a^4-a^6+7a^3+11a=$ \fillin[$a(a^3-a^5+7a^2+11)$][0in]\\
                        \part $6x-11xy+19xz=$ \fillin[$x(6-11y+19z)$][0in]\\
                        \part $x^6+x^4+x^2=$ \fillin[$x^2(x^4+x^2+1)$][0in]\\
                        \part $xyz-xy+xz=$ \fillin[$x(yz-y+z)$][0in]\\
                        \part $a^4-a^2+a^6=$ \fillin[$a^2(a^2-1+a^4)$][0in]\\
                        \part $x^2y^4-xy=$ \fillin[$xy(y^3-1)$][0in]\\
                        \part $x^3y^4-x^2y^5=$ \fillin[$x^2y^4(xy-y^2)$][0in]\\
                  \end{parts}
            \end{multicols}
      }

      % \addcontentsline{toc}{subsection}{Diferencia de cuadrados}
	\subsection{Diferencia de cuadrados}
   
      \questionboxed[6]{ Factoriza las siguientes diferencias de cuadrados:

            \begin{multicols}{2}
                  \begin{parts}
                        \part $x^2-9=$ \fillin[$(x+3)(x-3)$][0in]\\
                        \part $x^2-225=$ \fillin[$(x+15)(x-15)$][0in]\\
                        \part $x^2-256=$ \fillin[$(x+16)(x-16)$][0in]\\
                        \part $x^2-1=$ \fillin[$(x+1)(x-1)$][0in]\\
                        \part $x^2-289=$ \fillin[$(x+17)(x-17)$][0in]\\
                        \part $9x^2-4y^2=$ \fillin[$(3x+2y)(3x-2y)$][0in]\\
                        \part $64x^2-25=$ \fillin[$(8x+5)(8x-5)$][0in]\\
                        \part $4x^2-1=$ \fillin[$(2x+1)(2x-1)$][0in]\\
                  \end{parts}
            \end{multicols}
      }

      % \addcontentsline{toc}{subsection}{Trinomio cuadrado perfecto}
	\subsection{Trinomio cuadrado perfecto}
 
      \questionboxed[6]{Factoriza las siguientes expresiones algebraicas:
   
      \begin{multicols}{2}
                  \begin{parts}
                        \part $4x^2+12x+9=$ \fillin[$(2x+3)^2$][0in]\\
                        \part $x^2-30x+225=$ \fillin[$(x-15)^2$][0in]\\
                        \part $4x^2-36x+81=$ \fillin[$(2x-9)^2$][0in]\\
                        \part $4x^2-4x+1=$ \fillin[$(2x-1)^2$][0in]\\
                        \part $x^2+4x+4=$ \fillin[$(x+2)^2$][0in]\\
                        \part $x^2+22x+121=$ \fillin[$(x+11)^2$][0in]\\

                  \end{parts}
            \end{multicols}
      }

      % \addcontentsline{toc}{subsection}{Trinomios de la forma x²+bx+c}
	\subsection{Trinomios de la forma x²+bx+c}
     
      \questionboxed[6]{Factoriza las siguientes expresiones algebraicas:
 
      \begin{multicols}{2}
                  \begin{parts}
                        \part $x^2-10x+24=$ \fillin[$(x-6)(x-4)$][0in]\\
                        \part $x^2+3x+2=$ \fillin[$(x+2)(x+1)$][0in]\\
                        \part $x^2+x-42=$ \fillin[$(x+7)(x-6)$][0in]\\
                        \part $x^2-8x+15=$ \fillin[$(x-7)(x+2)$][0in]\\
                        \part $x^2-13x+40=$ \fillin[$(x-5)(x-8)$][0in]\\
                        \part $x^2-7x-30=$ \fillin[$(x-10)(x+3)$][0in]\\
                  \end{parts}
            \end{multicols}
      }

      % \addcontentsline{toc}{subsection}{Trinomios de la forma ax²+bx+c}
	\subsection{Trinomios de la forma ax²+bx+c}
     
      \questionboxed[6]{Factoriza las siguientes expresiones algebraicas:
      
      \begin{multicols}{2}
                  \begin{parts}
                        \part $6x^2+27x+21=$ \fillin[$3(2x+7)(x+1)$][0in]\\
                        \part $2x^2-17x+21=$ \fillin[$(2x-3)(x-7)$][0in]\\
                        \part $6x^2-5x-6=$ \fillin[$(2x-3)(3x+2)$][0in]\\
                        \part $2x^2-5x+2=$ \fillin[$(2x-1)(x-2)$][0in]\\
                        \part $15x^2+34x+15=$ \fillin[$(3x+5)(5x+3)$][0in]\\
                        \part $8x^2+14x+5=$ \fillin[$(4x+5)(2x+1)$][0in]\\
                  \end{parts}
            \end{multicols}
      }

      % \addcontentsline{toc}{section}{Leyes de los exponentes}
	\section{Leyes de los exponentes}
    
      \subsection{Suma de exponentes}

	\questionboxed[4]{Realiza las siguientes operaciones con exponentes:

		\begin{multicols}{3}
			\begin{parts}
				\part $(-5a^4)(-3a^2)=$ \fillin[$15a^6$][0in]

				% \begin{solutionbox}{1cm}
				% 	$(-5a^4)(-3a^2) = 15a^6$
				% \end{solutionbox}

				\part $(5x^3)(-x^{11})=$ \fillin[$-5x^{14}$][0in]

				% \begin{solutionbox}{1cm}
				% 	$(5x^3)(-x^{11})= -5x^{14}$
				% \end{solutionbox}

				\part $x^4x^{12}x^7=$ \fillin[$x^{23}$][0in]


				% \begin{solutionbox}{1cm}
				% 	$x^4x^{12}x^7= x^{23}$
				% \end{solutionbox}

				\part $(-2a^3)(-a)=$ \fillin[$-2a^4$][0in]

				% \begin{solutionbox}{1cm}
				% 	$(-2a^3)(-a)= -2a^4$
				% \end{solutionbox}

				\part $(5y^5)(7y^4)=$  \fillin[$35y^9$][0in]

				% \begin{solutionbox}{1cm}
				% 	$(5y^5)(7y^4)=35y^9$
				% \end{solutionbox}

				\part $(-3a^4)(8a^2)=$ \fillin[$-24a^6$][0in]

				% \begin{solutionbox}{1cm}
					% $(-3a^4)(8a^2) = -24a^6$
				% \end{solutionbox}

				\part $4x^2\cdot x^5\cdot 5x^8=$ \fillin[$20x^{15}$][0in]

				% \begin{solutionbox}{1cm}
				% 	$4x^2\cdot x^5\cdot 5x^8 = 20x^{15}$
				% \end{solutionbox}

				\part $x^2y^3z^4 \cdot x^5z^4=$ \fillin[$x^7y^3z^8$][0in]

				% \begin{solutionbox}{1cm}
				% 	$x^2y^3z^4 \cdot x^5z^4 = x^7y^3z^8$
				% \end{solutionbox}

				\part $7x^2\cdot 3x^4 \cdot 6x^2=$  \fillin[$126x^8$][0in]

				% \begin{solutionbox}{1cm}
				% 	$7x^2\cdot 3x^4 \cdot 6x^2 = 126x^8$
				% \end{solutionbox}
			\end{parts}
		\end{multicols}
	}

	\subsection{Resta de exponentes}

	\questionboxed[3]{Realiza las siguientes operaciones con exponentes:

		\begin{multicols}{3}
			\begin{parts}
				\part $\dfrac{18x^{15}}{6x^{12}}=$ \fillin[$3x^3$][0in]

				% \begin{solutionbox}{1.5cm}
				% 	$\dfrac{18x^{15}}{6x^{12}}=3x^3$
				% \end{solutionbox}

				\part $\dfrac{6x^{7}}{2x^{2}}=$ \fillin[$3x^5$][0in]

				% \begin{solutionbox}{1.5cm}
				% 	$\dfrac{6x^{7}}{2x^{2}}=3x^5$
				% \end{solutionbox}

				\part $\dfrac{a^3b^{9}c^5}{a^2b^{5}c^4}=$ \fillin[$ab^4c$][0in]

				% \begin{solutionbox}{1.5cm}
				% 	$\dfrac{a^3b^{9}c^5}{a^2b^{5}c^4}=ab^4c$
				% \end{solutionbox}

				\part $\dfrac{x^{13}y^{18}z^{4}}{x^{11}y^{9}z^{4}}=$ \fillin[$x^2y^9$][0in]

				% \begin{solutionbox}{1.5cm}
				% 	$\dfrac{x^{13}y^{18}z^{4}}{x^{11}y^{9}z^{4}} = x^2y^9$
				% \end{solutionbox}

				\part $\dfrac{21x^{23}}{7x^{11}}=$ \fillin[$3x^{12}$][0in]

				% \begin{solutionbox}{1.5cm}
				% 	$\dfrac{21x^{23}}{7x^{11}}=3x^{12}$
				% \end{solutionbox}

				\part $\dfrac{25x^{8}}{5x^{3}}=$ \fillin[$5x^5$][0in]

				% \begin{solutionbox}{1.5cm}
				% 	 $\dfrac{25x^{8}}{5x^{3}}=5x^5$
				% \end{solutionbox}

				\part $\dfrac{x^{3}y^{12}z^{13}}{x^{3}y^{12}z^{13}}=$ \fillin[$1$][0in]

				% \begin{solutionbox}{1.5cm}
				% 	$\dfrac{x^{3}y^{12}z^{13}}{x^{3}y^{12}z^{13}} = 1$
				% \end{solutionbox}

				\part $\dfrac{81a^5b^{12}c^9}{9a^3b^{7}c^5}=$ \fillin[$9a^2b^5c^4$][0in]

				% \begin{solutionbox}{1.5cm}
				% 	$\dfrac{81a^5b^{12}c^9}{9a^3b^{7}c^5} = 9a^2b^5c^4$
				% \end{solutionbox}

				\part $\dfrac{5x^{8}}{25x^{3}}=$ \fillin[$\dfrac{x^5}{5}$][0in]

				% \begin{solutionbox}{1.5cm}
				% 	$\dfrac{5x^{8}}{25x^{3}}=\dfrac{x^5}{5}$
				% \end{solutionbox}
			\end{parts}
		\end{multicols}
	}	

	\subsection{Multiplicación de exponentes}

	\questionboxed[4]{Realiza las siguientes operaciones con exponentes:

	\begin{multicols}{3}
		\begin{parts}
			\part $(a^3b^2c^4)^3=$ \fillin[$a^9b^6c^{12}$][0in]

			% \begin{solutionbox}{1cm}
			% 	$(a^3b^2c^4)^3 = a^9b^6c^{12}$
			% \end{solutionbox}
			\part $(x^9y^5z^2)^5=$ \fillin[$x^{45}y^{25}z^{10}$][0in]

			\part $(a^4b^5)^4=$ \fillin[$a^{16}b^{20}$][0in]

			\part $\left(x^9 y^5\right)^11=$  \fillin[$x^{99}y^{55}$][0in]

			\part $\left(x^4 y^5\right)^6=$  \fillin[$x^{24}y^{30}$][0in]

			% \begin{solutionbox}{1cm}
			% 	$\left(x^4 y^5\right)^6 = x^{24}y^{30}$
			% \end{solutionbox}

			\part $(x^7y^8z^4w^5)^6=$ \fillin[$x^{42}y^{48}z^{24}w^{30}$][0in]

			\part $(a^3b^7c^5d^4)^4=$ \fillin[$a^{12}b^{28}c^{20}d^{16}$][0in]

			\part $\left(a^3 b^5 c^{11} \right)^7=$  \fillin[$a^{21}b^{35}c^{77}$][0in]

			% \begin{solutionbox}{1cm}
			% 	$\left(a^3 b^5 c^{11} \right)^7 = a^{21}b^{35}c^{77}$
			% \end{solutionbox}

			\part $(a^4b^4c^5d^{11})^5=$ \fillin[$a^{20}b^{20}c^{25}d^{55}$][0in]

		\end{parts}
	\end{multicols}
}	

      % \addcontentsline{toc}{subsection}{División de exponentes}
	\subsection{División de exponentes}

      \questionboxed[4]{Simplifica las siguientes expresiones algebraicas con exponentes:
            
      \begin{multicols}{2}
                  \begin{parts}
                        \part $\sqrt{x^4}=$ \fillin[$x^2$][0in]\\
                        \part $\sqrt[6]{x^6y^{12}}=$ \fillin[$xy^2$][0in]\\
                        \part $\sqrt[3]{x^6y^{12}z^{18}}=$ \fillin[$xy^2z^6$][0in]\\
                        \part $\sqrt[4]{x^{12}y^{8}z^{16}}=$ \fillin[$x^3y^2z^4$][0in]\\
                        \part $\sqrt{x^{20}y^{12}z^{6}}=$ \fillin[$x^{10}y^6z^3$][0in]\\
                        \part $\sqrt[5]{a^{15}b^{20}}=$ \fillin[$a^3b^4$][0in]\\
                  \end{parts}
            \end{multicols}
      }

      % \addcontentsline{toc}{subsection}{Exponentes negativos}
	\subsection{Exponentes negativos}
      
      \questionboxed[4]{Convierte las expresiones algebraicas usando exponentes positivos:
           
      \begin{multicols}{2}
                  \begin{parts}
                        \part $\dfrac{5}{x^{-8}}=$ \fillin[$5x^8$][0in]\\
                        \part $5x^{-7}=$ \fillin[$\dfrac{5}{x^7}$][0in]\\
                        \part $y^{-5}=$ \fillin[$\dfrac{1}{y^5}$][0in]\\
                        \part $3y^{-9}=$ \fillin[$\dfrac{3}{y^9}$][0in]\\
                        \part $\dfrac{1}{x^{-7}}=$ \fillin[$x^7$][0in]\\
                        \part $\dfrac{2}{y^{-2}}=$ \fillin[$2y^2$][0in]\\
                  \end{parts}
            \end{multicols}
      }

      % \addcontentsline{toc}{section}{Números negativos}
	\section{Números negativos}
   
      % \addcontentsline{toc}{subsection}{Ubicación en la recta numérica}
	\subsection{Ubicación en la recta numérica}
   
      \questionboxed[4]{Escribe el número que representa el punto indicado en la recta numérica de cada uno de los siguientes incisos.

            \begin{multicols}{2}
                  \begin{parts}
                        \part \includegraphics[width=140px]{../images/recta_num_-171.png} \\[-0.5em]   \fillin[$-171$][1.5in]
                        \part \includegraphics[width=140px]{../images/recta_num_-3.1.png} \\[-0.5em]   \fillin[$-3.1$][1.5in]
                        \part \includegraphics[width=140px]{../images/recta_num_-2.png}   \\[-0.5em] \fillin[$-2$][1.5in]
                        \part \includegraphics[width=140px]{../images/recta_num_-3.9.png} \\[-0.5em]   \fillin[$-3.9$][1.5in]
                        \part \includegraphics[width=140px]{../images/recta_num_-23.png}   \\[-0.5em] \fillin[$-23$][1.5in]
                        % \part \includegraphics[width=140px]{../images/recta_num_1.723.png}\\[-0.5em]  \fillin[$1.723$][1.5in]
                        % \part \includegraphics[width=140px]{../images/recta_num_0.4.png}  \\[-0.5em]  \fillin[$0.4.$][1.5in]
                        \part \includegraphics[width=140px]{../images/recta_num_-83.png}  \\[-0.5em]  \fillin[$-83$][1.5in]
                        \part \includegraphics[width=140px]{../images/recta_num_-182.png}\\[-0.5em]  \fillin[$-182$][1.5in]
                        \part \includegraphics[width=140px]{../images/recta_num_-9.png}  \\[-0.5em]  \fillin[$-9$][1.5in]
                  \end{parts}
            \end{multicols}
      }

      % \addcontentsline{toc}{subsection}{Comparación de negativos}
	\subsection{Comparación de negativos}
  
      \questionboxed[4]{Escribe sobre la línea el símbolo de mayor que ($>$), menor que ($<$), o igual ($=$) según corresponda.

            \begin{multicols}{2}
                  \begin{parts}
                        \part $-51$ \fillin[$>$][0.5in] $-55$\\[0.25em]
                        % \part $-77$ \fillin[$>$][0.5in] $-177$\\[0.25em]
                        \part $-100$ \fillin[$<$][0.5in] $-99$\\[0.25em]
                        \part $-182$ \fillin[$>$][0.5in] $-189$\\[0.25em]
                        \part $-97$ \fillin[$<$][0.5in] $-96.2$\\[0.25em]
                        \part $-36$ \fillin[$>$][0.5in] $-39$\\[0.25em]
                        \part $-3.5$ \fillin[$<$][0.5in] $-2.2$\\[0.25em]
                        % \part $-12$ \fillin[$<$][0.5in] $-11$\\[0.25em]
                        % \part $-10.001$ \fillin[$>$][0.5in] $-100.01$\\[0.25em]
                        % \part $-0.99$ \fillin[$>$][0.5in] $1.01$
                  \end{parts}
            \end{multicols}
      }


      % \addcontentsline{toc}{subsection}{Suma y resta con negativos}
	\subsection{Suma y resta con negativos}
  
      \questionboxed[4]{Realiza las siguientes sumas y restas con números negativos:

            \begin{multicols}{2}
                  \begin{parts}
                        \part $-223+67=$ \fillin[$-156$][0in]
                        \part $(16)-(-14)=$ \fillin[$30$][0in]
                        \part $-(-15)-(-14)=$ \fillin[$-1$][0in]
                        \part $-235+304=$ \fillin[$69$][0in]
                        \part $198-189=$ \fillin[$9$][0in]
                        \part $-201.1-9.4=$ \fillin[$-210.5$][0in]
                        \part $201.1-9.4=$ \fillin[$191.7$][0in]
                        \part $-201.1+9.4=$ \fillin[$-191.7$][0in]
                  \end{parts}
            \end{multicols}
      }

      % \addcontentsline{toc}{subsection}{Multiplicación y división con negativos}
	\subsection{Multiplicación y división con negativos}
  
      \questionboxed[4]{Realiza las siguientes multiplicaciones y divisiones con números negativos:
           
      \begin{multicols}{2}
                  \begin{parts}
                        \part $(31)\divisionsymbol (-62)=$ \fillin[$-\dfrac{1}{2}$][0in]
                        \part $(-15)(-14)=$ \fillin[$210$][0in]
                        \part $(-7)(20)=$ \fillin[$-140$][0in]
                        \part $(50)\divisionsymbol (0.5)=$ \fillin[$100$][0in]
                        \part $(-5)(5)(-5)(-5)=$ \fillin[$-625$][0in]
                        \part $(-220)\divisionsymbol (0.2)=$ \fillin[$-1100$][0in]
                  \end{parts}
            \end{multicols}
      }


      % \addcontentsline{toc}{subsection}{Jerarquía de operaciones}
	\subsection{Jerarquía de operaciones}
  
      \questionboxed[4]{Usando la jerarquía de operaciones, realiza la siguiente operación

            \begin{multicols}{2}
                  \begin{parts}
                        \part $9+6\times 4-5=$ \fillin[$28$][0in]
                        \part $7+2^2 \times 6+2^2-6=$ \fillin[$29$][0in]
                        \part $10\times 12-14\divisionsymbol 2+15=$ \fillin[$128$][0in]
                        \part $6^3 \divisionsymbol 8 \divisionsymbol 9 = $ \fillin[$3$][0in]
                        \part $8\times 3 +70\divisionsymbol 7-7=$ \fillin[$27$][0in]
                        \part $16 \times 15 \divisionsymbol 5 +12=$ \fillin[$60$][0in]
                  \end{parts}
            \end{multicols}
      }

      % \addcontentsline{toc}{section}{Sucesiones aritméticas}
	\section{Sucesiones aritméticas}
   
      % \addcontentsline{toc}{subsection}{Completando la sucesión}
	\subsection{Completando la sucesión}
    
      \questionboxed[4]{Escribe los términos faltantes de las siguientes sucesiones aritméticas:
         
      \begin{multicols}{2}
                  \begin{parts}
                        \part $ -8,-13,-18,$\fillin[$-23$][0.3in],\fillin[$-28$][0.3in],\fillin[$-33$][0.3in],\ldots
                        \part $-57,-65,-73,$\fillin[$-81$][0.3in],\fillin[$-89$][0.3in],\fillin[$-97$][0.3in],\ldots
                        \part $-14,-17,-20,$\fillin[$-23$][0.3in],\fillin[$-26$][0.3in],\fillin[$-29$][0.3in],\ldots
                        \part $-19,-15,-11,$\fillin[$-7$][0.3in],\fillin[$-3$][0.3in],\fillin[$1$][0.3in],\ldots
                  \end{parts}
            \end{multicols}
      }

      % \addcontentsline{toc}{subsection}{Diferencia de una sucesión}
	\subsection{Diferencia de una sucesión}
   
      \questionboxed[4]{Determina la diferencia de las siguientes sucesiones aritméticas:
           
      \begin{multicols}{2}
                  \begin{parts}
                        \part $-23,-15,-7,1,9,17,\ldots$ \hfill d=\fillin[$8$][1cm]\\
                        \part $-15,-10,-5,0,5,\ldots$ \hfill d=\fillin[$5$][1cm]\\
                        \part $-8,-13,-18,-23,-28,-33,\ldots$ \hfill d=\fillin[$-5$][1cm]\\
                        \part $-19,-15,-11,-7,-3,1,\ldots$ \hfill d=\fillin[$4$][1cm]\\
                        \part $7,9,11,13,15,17,\ldots$ \hfill d=\fillin[$2$][1cm]\\
                        \part $-4,-2,0,2,4,6,\ldots$ \hfill d=\fillin[$2$][1cm]\\
                  \end{parts}
            \end{multicols}
      }

      % \addcontentsline{toc}{subsection}{Término enésimo}
	\subsection{Término enésimo}
 
      \questionboxed[4]{Encuentra el \textit{n-ésimo} término de la siguientes sucesiones aritméticas:
           
      \begin{multicols}{2}
                  \begin{parts}
                        \part Calcula el término número 44 de la siguiente sucesión aritmética: $-3n-15$
                       
                        \begin{solutionbox}{1.5cm}
                              \[-3(44)-15=-132-15=-147\]
                        \end{solutionbox}

                        \part Calcula el término número 47 de la siguiente sucesión aritmética: $-5,0,5,10,15,\ldots$
                       
                        \begin{solutionbox}{1.5cm}
                              \[5(47)-5=235-5=225\]
                        \end{solutionbox}

                        \part Calcula el término número 28 de la siguiente sucesión aritmética: $-69,-72,-75,-78,-81,\ldots$
                       
                        \begin{solutionbox}{1.5cm}
                              \[-3(28)-66=-84-66=-150\]
                        \end{solutionbox}

                        \part Calcula el término número 15 de la siguiente sucesión aritmetica: $11,18,25,32,39,\ldots$
                       
                        \begin{solutionbox}{1.5cm}
                              \[7(15)+4=105+4=109\]
                        \end{solutionbox}

                        \part Calcula el término número 25 de la siguiente sucesión aritmética: $2n-6$
                       
                        \begin{solutionbox}{1.5cm}
                              \[2(25)-6=50-6=44\]
                        \end{solutionbox}

                        \part Calcula el término número 22 de la siguiente sucesión aritmética: $7,2,-3,-8,-13,\ldots$
                     
                        \begin{solutionbox}{1.5cm}
                              \[-5(22)+12=-110+12=-98\]
                        \end{solutionbox}
                  \end{parts}
            \end{multicols}
      }
      
      % \addcontentsline{toc}{subsection}{Término general}
	\subsection{Término general}
 
      \questionboxed[4]{Determina el término general de las siguientes sucesiones aritméticas:
           
      \begin{multicols}{2}
                  \begin{parts}
                        \part $3,9,15,21,27,\ldots$ \fillin[$6n-3$][1in]
                        \part $-69,-72,-75,-78,-81,\ldots$ \fillin[$-3n-66$][1in]
                        \part $40,35,30,25,20,\ldots$ \fillin[$5-5n$][1in]
                        \part $-2,-6,-10,-14,-18,\ldots$ \fillin[$-4n+2$][1in]
                        \part $-2,1,4,7,10,\ldots$ \fillin[$3n-5$][1in]
                        \part $-57,-65,-73,-81,-89,\ldots$ \fillin[$-8n-49$][1in]
                  \end{parts}
            \end{multicols}
      }

      % \addcontentsline{toc}{subsection}{Suma de una sucesión aritmética}
	\subsection{Suma de una sucesión aritmética}

      \questionboxed[10]{Calcula la suma de los primeros $n$ términos de las siguientes sucesiones aritméticas:
       
      \begin{multicols}{2}
                  \begin{parts}
                        \part Calcula la suma de los primeros 41 términos de la siguiente sucesión aritmética: $40,51,62,73,84,\ldots$
                       
                        \begin{solutionbox}{2cm}
                              \[a_{41}=40+11(41-1)=40+440=480\]
                              \[S_{41}=\dfrac{41(40+480)}{2}=10,660\]
                        \end{solutionbox}

                        \part Calcula la suma de los primeros 37 términos de la siguiente sucesión aritmética: $15,25,35,45,55,\ldots$
                        
                        \begin{solutionbox}{2cm}
                              \[a_{37}=15+10(37-1)=15+360=375\]
                              \[S_{37}=\dfrac{37(15+375)}{2}=7,215\]
                        \end{solutionbox}

                        \part Calcula la suma de los primeros 23 términos de la siguiente sucesión aritmética: $-5,0,5,10,15,\ldots$
                       
                        \begin{solutionbox}{2cm}
                              \[a_{23}=-5+5(23-1)=-5+110=105\]
                              \[S_{23}=\dfrac{23(-5+105)}{2}=1,150\]
                        \end{solutionbox}

                        \part Calcula la suma de los primeros 25 términos de la siguiente sucesión aritmética: $11,18,25,32,39,\ldots$
                      
                        \begin{solutionbox}{2cm}
                              \[a_{25}=11+7(25-1)=11+168=179\]
                              \[S_{25}=\dfrac{25(11+179)}{2}=2,375\]
                        \end{solutionbox}
                  \end{parts}
            \end{multicols}
      }
\end{questions}
\end{document}