\documentclass[12pt,addpoints]{repaso}
\grado{3}
\nivel{Primaria}
\cicloescolar{2025-2026}
\materia{Matemáticas}
\unidad{1}
\title{Practica la Unidad}
\aprendizajes{\scriptsize%
\item Expresa oralmente la sucesión numérica hasta cuatro cifras, en español y hasta donde sea posible, en su lengua materna, de manera ascendente y descendente a partir de un número natural dado.
\item Representa, con apoyo de material concreto y modelos gráficos, fracciones: medios, cuartos, octavos, dieciseisavos, para expresar el resultado de mediciones y repartos en situaciones vinculadas a su contexto.
\item Resuelve situaciones problemáticas vinculadas a su contexto que implican sumas, restas, multiplicación y división de números naturales de hasta tres cifras utilizando el algoritmo convencional. %y que impliquen, medición, estimación y comparación, de longitudes, masas y capacidades, con el uso del metro, kilogramo, litro y medios y cuartos de estas unidades; en el caso de la longitud, el decímetro y centímetro.
\item Resuelve problemas de suma, resta, multiplicación y división vinculados a su contexto, que impliquen el uso de fracciones (medios, cuartos, octavos, dieciseisavos), con el apoyo de material concreto o representaciones gráficas.
   }
\author{Olavarrieta Díaz, L.}
\begin{document}
\INFO
\begin{questions}
	% UNIDAD 1                  
	\section{Escritura de cantidades}
	% \subsection*{Escritura de cantidades 1 }
	% \subsection*{Escritura de cantidades 2 }
	% \subsection*{Escritura de cantidades 3 }
	% \subsection*{Escritura de cantidades 4 }
	% \subsection*{Escritura de cantidades 5 }
	\questionboxed[6]{Escribe sobre la línea \textbf{el número} que corresponde a cada una de las siguientes cantidades:

		\begin{multicols}{3}
			\begin{parts}
				\part \fillin[ 2][1.5cm] Dos
				\part \fillin[19][1.5cm] Diecinueve
				\part \fillin[32][1.5cm] Treinta y dos
				%\part \fillin[16][1.5cm] Dieciséis
				\part \fillin[21][1.5cm] Veintiuno
				\part \fillin[67][1.5cm] Sesenta y siete
				\part \fillin[51][1.5cm] Cincuenta y uno
			\end{parts}
		\end{multicols}
	}

	\questionboxed[6]{Escribe sobre la línea \textbf{el número} que corresponde a cada una de las siguientes cantidades:

		\begin{multicols}{3}
			\begin{parts}
				\part \fillin[ 5][1.5cm] Cinco
				\part \fillin[43][1.5cm] Cuarenta y tres
				\part \fillin[11][1.5cm] Once
				\part \fillin[89][1.5cm] Ochenta y nueve
				\part \fillin[18][1.5cm] Dieciocho
				%\part \fillin[22][1.5cm] Veintidos
				% \part \fillin[][1.5cm] Veintinueve
				\part \fillin[76][1.5cm] Setenta y seis
			\end{parts}
		\end{multicols}
	}

	\questionboxed[6]{Escribe sobre la línea \textbf{el número} que corresponde a cada una de las siguientes cantidades:

		\begin{multicols}{3}
			\begin{parts}\large
				\part \fillin[ 9][1.5cm] Nueve
				\part \fillin[13][1.5cm] Trece
				%\part \fillin[15][1.5cm] Quince
				\part \fillin[12][1.5cm] Doce
				\part \fillin[27][1.5cm] Veintisiete
				\part \fillin[60][1.5cm] Sesenta
				\part \fillin[75][1.5cm] Setenta y cinco
			\end{parts}
		\end{multicols}
	}

	\questionboxed[8]{Escribe sobre la línea \textbf{el número} que corresponde a cada una de las siguientes cantidades:

		\begin{multicols}{2}
			\begin{parts}\large
				\part \fillin[  65][1.5cm] Sesenta y cinco.
				\part \fillin[ 109][1.5cm] Ciento nueve.
				\part \fillin[ 254][1.5cm] Doscientos cincuenta y cuatro.
				\part \fillin[ 314][1.5cm] Trescientos catorce.
				\part \fillin[ 431][1.5cm] Cuatrocientos treinta y uno.
				\part \fillin[2024][1.5cm] Dos mil veinticuatro.
				\part \fillin[1849][1.5cm] Mil ochocientos cuarenta y nueve.
				% \part \fillin[1310][1.1cm] Mil trescientos diez.                      
				\part \fillin[ 703][1.5cm] Setecientos tres.
			\end{parts}
		\end{multicols}
	}
	% \section*{\ifprintanswers{Sistema decimal 1}

	% \subsection*{Notación desarrollada 1 }
	% \subsection*{Notación desarrollada 2 }
	% \subsection*{Notación desarrollada 3 }

	\questionboxed[12]{Escribe la \textbf{notación desarrollada} de cada uno de los siguientes números:

		\begin{multicols}{3}
			\begin{parts}\Large
				\part $28=$ \fillin[$20+8$][1in]
				\part $84=$ \fillin[$80+4$][1in]
				\part $77=$ \fillin[$70+7$][1in]
				\part $936 =$ \fillin[$900+30+6$][1in]

				\part $11 =$ \fillin[$10+1$][1in]
				\part $48 =$ \fillin[$40+8$][1in]
				\part $96=$ \fillin[$90+6$][1in]
				\part $215 =$ \fillin[$200+10+5$][1in]

				\part $39=$ \fillin[$30+9$][1in]
				\part $57 =$ \fillin[$50+7$][1in]
				\part $79=$ \fillin[$70+9$][1in]
				\part $105 =$ \fillin[$100+5$][1in]
			\end{parts}
		\end{multicols}
	}

	\questionboxed[6]{Escribe la \textbf{notación desarrollada} de cada uno de los siguientes números:

		\begin{multicols}{2}
			\begin{parts}\Large
				\part $6215 =$ \fillin[$6000+200+10+5$][2.4in]
				\part $5454 =$ \fillin[$5000+400+50+4$][2.4in]
				\part $6451 =$ \fillin[$6000+400+50+1$][2.4in]
				\part $15984=$ \fillin[$10000+5000+900+80+4$][2.4in]
				\part $4936 =$ \fillin[$4000+900+30+6$][2.4in]
				\part $27545=$ \fillin[$20000+7000+500+40+5$][2.4in]
			\end{parts}
		\end{multicols}
	}

	\questionboxed[6]{Escribe la \textbf{notación desarrollada} de cada uno de los siguientes números:

		\begin{multicols}{2}
			\begin{parts}\Large
				\part $7018 =$ \fillin[$7000+10+8$][2.4in]
				\part $7140 =$ \fillin[$7000+100+40$][2.4in]
				\part $19679=$ \fillin[$10000+9000+600+70+9$][2.4in]
				\part $26324=$ \fillin[$20000+6000+300+20+4$][2.4in]
				\part $5717 =$ \fillin[$5000+700+10+7$][2.4in]
				\part $31126=$ \fillin[$30000+1000+100+20+6$][2.4in]
			\end{parts}
		\end{multicols}
	}
	% \subsection*{Posicionamiento decimal 1 }
	% \subsection*{Posicionamiento decimal 2 }

	
	

\questionboxed[8]{Señala la opción que responda correctamente a cada una de las siguientes preguntas:

		\begin{multicols}{2}
			\begin{parts}
				\part ¿Qué lugar ocupa el 8 en 6418?     \fillin[C][0.5cm]
				\part ¿Qué lugar ocupa el 4 en 206418?   \fillin[A][0.5cm]
				\part ¿Qué lugar ocupa el 6 en 87264?    \fillin[B][0.5cm]
				\part ¿Qué lugar ocupa el 4 en 1684?     \fillin[C][0.5cm]
				\part ¿Qué lugar ocupa el 1 en 6138?     \fillin[A][0.5cm]
				\part ¿Qué lugar ocupa el 4 en 198114?   \fillin[C][0.5cm]
				\part ¿Qué lugar ocupa el 8 en 149778?   \fillin[C][0.5cm]
				\part ¿Qué lugar ocupa el 7 en 46878?    \fillin[B][0.5cm]
			\end{parts}

			\columnbreak%

			\begin{choices}\Large
				\choice {\color{red}centenas.}
				\choice {\color{blue}decenas.}
				\choice {\color{Goldenrod!65!black}unidades.}
			\end{choices}
		\end{multicols}
	}

	

	\questionboxed[8]{Señala la opción que responda correctamente a cada una de las siguientes preguntas:

		\begin{multicols}{2}
			\begin{parts}\large
				\part ¿Qué lugar ocupa el 6 en 6418?     \fillin[C][0.5cm]
				\part ¿Qué lugar ocupa el 2 en 206418?   \fillin[A][0.5cm]
				\part ¿Qué lugar ocupa el 2 en 87264?    \fillin[D][0.5cm]
				\part ¿Qué lugar ocupa el 1 en 1681?     \fillin[F][0.5cm]
				\part ¿Qué lugar ocupa el 1 en 6138?     \fillin[D][0.5cm]
				\part ¿Qué lugar ocupa el 8 en 198114?   \fillin[C][0.5cm]
				\part ¿Qué lugar ocupa el 7 en 46878?    \fillin[E][0.5cm]
				\part ¿Qué lugar ocupa el 4 en 149778?   \fillin[B][0.5cm]
			\end{parts}

			\columnbreak%

			\begin{choices}\Large
				\choice {\color{red}centenas de millar.}
				\choice {\color{blue}decenas de millar.}
				\choice {\color{Goldenrod}unidades de millar.}
				\choice {\color{red}centenas.}
				\choice {\color{blue}decenas.}
				\choice {\color{Goldenrod}unidades.}
			\end{choices}
		\end{multicols}
	}



	% \section*{\ifprintanswers{Sistema decimal 2}
	% \subsection*{Posicionamiento decimal 1 }
	% \subsection*{Posicionamiento decimal 2 }

	% \questionboxed[3]{Señala la opción que responda correctamente a cada una de las siguientes preguntas:

	% 	\begin{multicols}{2}
	% 		\begin{parts}\large
	% 			\part En el número 1.829, ¿qué número ocupa la posición de las centésimas?

	% 			\begin{oneparcheckboxes}
	% 				\choice 1 \CorrectChoice 2 \choice 6 \choice 8 \choice 9
	% 			\end{oneparcheckboxes}

	% 			\part En el número 2.087, ¿qué número ocupa la posición de las décimas?

	% 			\begin{oneparcheckboxes}
	% 				\CorrectChoice 0 \choice 2 \choice 7 \choice 8 \choice 9
	% 			\end{oneparcheckboxes}

	% 			\part En el número 5.928, ¿qué número ocupa la posición de las décimas?

	% 			\begin{oneparcheckboxes}
	% 				\choice 5 \choice 2 \choice 6 \choice 8 \CorrectChoice 9
	% 			\end{oneparcheckboxes}

	% 			\part En el número 3.284, ¿qué número ocupa la posición de las milésimas?

	% 			\begin{oneparcheckboxes}
	% 				\choice 2 \choice 3 \CorrectChoice 4  \choice 8 \choice 9
	% 			\end{oneparcheckboxes}

	% 			\part En el número 1.285, ¿qué número ocupa la posición de las décimas?

	% 			\begin{oneparcheckboxes}
	% 				\choice 1 \CorrectChoice 2 \choice 5 \choice 8 \choice 9
	% 			\end{oneparcheckboxes}

	% 			\part En el número 1.823, ¿qué número ocupa la posición de las milésimas?

	% 			\begin{oneparcheckboxes}
	% 				\choice 1 \choice 2 \CorrectChoice 3 \choice 6 \choice 8
	% 			\end{oneparcheckboxes}
	% 		\end{parts}
	% 	\end{multicols}
	% }

	% \questionboxed[6]{Escribe sobre la línea \textbf{el número} que corresponde a cada una de las siguientes cantidades:

	% 	\begin{multicols}{2}
	% 		\begin{parts}\large
	% 			\part Veinticinco enteros ocho décimas                  \\ \hfill \fillin[$25.8$][1cm]
	% 			\part Seis enteros ciento veintiocho milésimas          \\ \hfill \fillin[$6.128$][1cm]
	% 			\part Catorce enteros veintinueve centésimas            \\ \hfill \fillin[$14.29$][1cm]
	% 			\part Cuarenta enteros dos décimas                      \\ \hfill \fillin[$40.2$][1cm]
	% 			\part Tres enteros cincuenta y ocho centésimas          \\ \hfill \fillin[$3.58$][1cm]
	% 			\part Cuatro enteros sesenta y nueve milésimas          \\ \hfill \fillin[$4.069$][1cm]
	% 			\part Siete enteros cuatro décimas                      \\ \hfill \fillin[$ 7.4$][1cm]
	% 			% \part Dos enteros siete décimas                         \\ \hfill \fillin[$2.7$][1cm]
	% 			% \part Cuatro enteros ocho milésimas                     \\ \hfill \fillin[$4.008$][1cm]
	% 			% \part Siete enteros setenta y siete centésimas          \\ \hfill \fillin[$7.77$][1cm]
	% 			% \part Once enteros ochenta y nueve centésimas           \\ \hfill \fillin[$11.89$][1cm]
	% 			\part Treinta y ocho enteros nueve décimas              \\ \hfill \fillin[$38.9$][1cm]
	% 		\end{parts}
	% 	\end{multicols}
	% }

	% \subsection*{Notación desarrollada 1 }
	% \subsection*{Notación desarrollada 2  }
	% \subsection*{Posicionamiento decimal y Notación desarrollada }

	\questionboxed[8]{Señala la opción que responda correctamente a cada una de las siguientes preguntas:

		\begin{multicols}{2}
			\begin{parts}\large
				\part En el número 3658, ¿qué número ocupa la posición de las decenas?

				\begin{oneparcheckboxes}
					\choice 3 \CorrectChoice 5 \choice 6 \choice 8 \choice 9
				\end{oneparcheckboxes}

				\part En el número 17542, ¿qué número ocupa la posición de las unidades de millar?

				\begin{oneparcheckboxes}
					\choice 1 \CorrectChoice 7 \choice 5 \choice 4 \choice 2
				\end{oneparcheckboxes}

				\part En el número 5984, ¿qué número ocupa la posición de las centenas?

				\begin{oneparcheckboxes}
					\choice 4 \choice 2 \choice 5 \choice 8 \CorrectChoice 9
				\end{oneparcheckboxes}

				\part En el número 7841, ¿qué número ocupa la posición de las decenas?

				\begin{oneparcheckboxes}
					\choice 1 \choice 7 \choice 8 \CorrectChoice 4 \choice 2
				\end{oneparcheckboxes}

				\part En el número 3918, ¿qué número ocupa la posición de las centenas?

				\begin{oneparcheckboxes}
					\choice 3 \choice 1 \choice 6 \choice 8 \CorrectChoice 9
				\end{oneparcheckboxes}

				\part En el número 3621, ¿qué número ocupa la posición de las decenas?

				\begin{oneparcheckboxes}
					\CorrectChoice 2 \choice 3 \choice 6 \choice 8 \choice 1
				\end{oneparcheckboxes}

				\part En el número 51362, ¿qué número ocupa la posición de las decenas de millar?

				\begin{oneparcheckboxes}
					\choice 3 \CorrectChoice 5 \choice 6 \choice 1 \choice 2
				\end{oneparcheckboxes}

				\part En el número 7584, ¿qué número ocupa la posición de las decenas?

				\begin{oneparcheckboxes}
					\choice 3 \choice 5 \choice 7 \CorrectChoice 8 \choice 4
				\end{oneparcheckboxes}

				% \part En el número 9654, ¿qué número ocupa la posición de las centenas?

				% \begin{oneparcheckboxes}
				%    \choice 3 \choice 5 \CorrectChoice 6 \choice 4 \choice 9
				% \end{oneparcheckboxes}

				% \part En el número 240679, ¿qué número ocupa la posición de las centenas de millar?

				% \begin{oneparcheckboxes}
				%    \choice 0 \choice 6 \CorrectChoice 2 \choice 7 \choice 9 \choice 4
				% \end{oneparcheckboxes}
				% \part En el número 41589, ¿qué número ocupa la posición de las decenas de millar?
				% \part En el número 8459, ¿qué número ocupa la posición de las centenas?
				% \part En el número 10562, ¿qué número ocupa la posición de las centenas?
				% \part En el número 24781, ¿qué número ocupa la posición de las decenas de millar?
				% \part En el número 7856, ¿qué número ocupa la posición de las decenas?
			\end{parts}
		\end{multicols}
	}

	% \section*{\ifprintanswers{Tablas de multiplicar 1}
	% \subsection*{Tabla del 1  }
	% \subsection*{Tabla del 2 }
	% \subsection*{Tabla del 3 }
	% \subsection*{Tabla del 4 }
	% \subsection*{Tabla del 5 }

	% UNIDAD 2

	% \section*{\ifprintanswers{Tablas de multiplicar 2        }
	% \subsection*{Tabla del 6                    }
	% \subsection*{Tabla del 7                    }
	% \subsection*{Tabla del 8                    }
	% \subsection*{Tabla del 9                    }
	% \subsection*{Tabla del 10                   }

	% \questionboxed[8]{Reponde las siguientes tablas de multiplicar:

	% 	\begin{multicols}{4}
	% 		\begin{parts}\Large
	% 			\part $5 \times 9=$ \fillin[$45$][0cm]
	% 			\part $5 \times 6=$ \fillin[$30$][0cm]
	% 			\part $6 \times 8=$ \fillin[$48$][0cm]
	% 			\part $6 \times 9=$ \fillin[$54$][0cm]
	% 			\part $3 \times 6=$ \fillin[$18$][0cm]
	% 			\part $2 \times 7=$ \fillin[$14$][0cm]
	% 			\part $4 \times 7=$ \fillin[$28$][0cm]
	% 			\part $3 \times 8=$ \fillin[$24$][0cm]
	% 			\part $2 \times 9=$ \fillin[$18$][0cm]
	% 			\part $4 \times 4=$ \fillin[$16$][0cm]
	% 			\part $7 \times 7=$ \fillin[$49$][0cm]
	% 			\part $7 \times 5=$ \fillin[$35$][0cm]
	% 			\part $5 \times 4=$ \fillin[$20$][0cm]
	% 			\part $8 \times 7=$ \fillin[$56$][0cm]
	% 			\part $7 \times 6=$ \fillin[$42$][0cm]
	% 			\part $9 \times 7=$ \fillin[$63$][0cm]
	% 		\end{parts}
	% 	\end{multicols}
	% }

	% \questionboxed[8]{Completa las siguientes tablas de multiplicar:

	% 	\begin{multicols}{4}
	% 		\begin{parts}\Large
	% 			\part $\fillin[6][0.5cm] \times 6= 36$
	% 			\part $\fillin[8][0.5cm] \times 8= 64$
	% 			\part $\fillin[7][0.5cm] \times 8= 56$
	% 			\part $5 \times \fillin[10][0.5cm]=50$
	% 			\part $4 \times \fillin[8][0.5cm]=32$
	% 			\part $8 \times \fillin[5][0.5cm]= 40$
	% 			\part $\fillin[6][0.5cm] \times 4= 24$
	% 			\part $7 \times \fillin[7][0.5cm]= 49$
	% 			\part $\fillin[8][0.5cm] \times 3= 24$
	% 			\part $9 \times \fillin[8][0.5cm]= 72$
	% 			\part $\fillin[9][0.5cm] \times 5= 45$
	% 			\part $6 \times \fillin[7][0.5cm]= 42$
	% 			\part $\fillin[9][0.5cm] \times 9= 81$
	% 			\part $4 \times \fillin[9][0.5cm]= 36$
	% 			\part $\fillin[7][0.5cm] \times 4= 28$
	% 			\part $\fillin[9][0.5cm] \times 3= 21$
	% 		\end{parts}
	% 	\end{multicols}
	% }

	% \section*{\ifprintanswers{Sumas 1                        }
	% \subsection*{Sumando con 1, 2 y 3           }
	% \subsection*{Sumando con 4, 5 y 6           }
	% \subsection*{Sumando con 7, 8 y 9           }
	% \subsection*{Sumando números entre 10 y 20  }
	% \subsection*{Sumando números entre 20 y 50  }

	% \section*{\ifprintanswers{Sumas 2                        }
	% \subsection*{Sumas hasta el 100             }
	% \subsection*{Sumas hasta el 500             }
	% \subsection*{Sumas con acarreos 1           }
	% \subsection*{Sumas con acarreos 2           }
	% \subsection*{Sumas con acarreos 3           }

	\questionboxed[3]{Realiza las siguientes sumas:

	\begin{multicols}{4}
		\begin{parts}
			\part $9 + 8=$ \fillin[17][0cm]\\[0.2cm]
			\part $5 + 4=$ \fillin[9][0cm]\\[0.2cm]

			\part
			\ifprintanswers{\opadd[hfactor=decimal,resultstyle=\color{red},carryadd=true]{17}{18}}
			\else{\opadd[hfactor=decimal,resultstyle=\color{white},carryadd=false]{17}{18}}\fi\\[0.2cm]

\columnbreak%

			\part $1 + 1=$ \fillin[2][0cm]\\[0.2cm]
			\part $5 + 7=$ \fillin[12][0cm]\\[0.2cm]

			\part
			\ifprintanswers{\opadd[hfactor=decimal,resultstyle=\color{red},carryadd=true]{15}{9}}
			\else{\opadd[hfactor=decimal,resultstyle=\color{white},carryadd=false]{15}{9}}\fi\\[0.2cm]
			
			\columnbreak%
			% \part
			% \ifprintanswers{\opadd[hfactor=decimal,resultstyle=\color{red},carryadd=true]{26}{9}}
			% \else{\opadd[hfactor=decimal,resultstyle=\color{white},carryadd=false]{26}{9}\\[0.5cm]}\fi\\[0.2cm]

			% \part
			% \ifprintanswers{\opadd[hfactor=decimal,resultstyle=\color{red},carryadd=true]{27}{12}}
			% \else{\opadd[hfactor=decimal,resultstyle=\color{white},carryadd=false]{27}{12}}\fi\\[0.2cm]

			\part $0 + 7=$ \fillin[7][0cm]\\[0.2cm]
			\part $8 + 7=$ \fillin[15][0cm]\\[0.2cm]

			\part
			\ifprintanswers{\opadd[hfactor=decimal,resultstyle=\color{red},carryadd=true]{10}{9}}
			\else{\opadd[hfactor=decimal,resultstyle=\color{white},carryadd=false]{10}{9}}\fi\\[0.2cm]
		
			\columnbreak%
			% \part
			% \ifprintanswers{\opadd[hfactor=decimal,resultstyle=\color{red},carryadd=true]{37}{8}}
			% \else{\opadd[hfactor=decimal,resultstyle=\color{white},carryadd=false]{37}{8}\\[0.5cm]}\fi\\[0.2cm]

			% \part
			% \ifprintanswers{\opadd[hfactor=decimal,resultstyle=\color{red},carryadd=true]{48}{14}}
			% \else{\opadd[hfactor=decimal,resultstyle=\color{white},carryadd=false]{48}{14}}\fi\\[0.2cm]

			\part $1 + 9=$ \fillin[10][0cm]\\[0.2cm]
			\part $4 + 9=$ \fillin[13][0cm]\\[0.2cm]

			\part
			\ifprintanswers{\opadd[hfactor=decimal,resultstyle=\color{red},carryadd=true]{21}{19}}
			\else{\opadd[hfactor=decimal,resultstyle=\color{white},carryadd=false]{21}{19}}\fi\\[0.2cm]
			% \part
			% \ifprintanswers{\opadd[hfactor=decimal,resultstyle=\color{red},carryadd=true]{44}{25}}
			% \else{\opadd[hfactor=decimal,resultstyle=\color{white},carryadd=false]{44}{25}\\[0.5cm]}\fi\\[0.2cm]

			% \part
			% \ifprintanswers{\opadd[hfactor=decimal,resultstyle=\color{red},carryadd=true]{82}{38}}
			% \else{\opadd[hfactor=decimal,resultstyle=\color{white},carryadd=false]{82}{38}}\fi\\[0.2cm]
		\end{parts}
	\end{multicols}
}

	\questionboxed[8]{Realiza las siguientes sumas:

		\begin{multicols}{4}
			\begin{parts}
				\part \ifprintanswers{\large  \quad   \opadd[hfactor=decimal,resultstyle=\color{red},carryadd=true,carrysub=false]{37854}{18581} }
				\else{          \large  \quad  \opadd[hfactor=decimal,resultstyle=\color{white},carryadd=false,carrysub=false]{37854}{18581} }
				\fi
				\part \ifprintanswers{\large  \quad   \opadd[hfactor=decimal,resultstyle=\color{red},carryadd=true,carrysub=false]{3234}{24156} }
				\else{          \large  \quad  \opadd[hfactor=decimal,resultstyle=\color{white},carryadd=false,carrysub=false]{3234}{24156} }
				\fi
				\part \ifprintanswers{\large  \quad   \opadd[hfactor=decimal,resultstyle=\color{red},carryadd=true,carrysub=false]{30985}{19562} }
				\else{          \large  \quad  \opadd[hfactor=decimal,resultstyle=\color{white},carryadd=false,carrysub=false]{30985}{19562} }
				\fi
				\part \ifprintanswers{\large  \quad   \opadd[hfactor=decimal,resultstyle=\color{red},carryadd=true,carrysub=false]{2849}{2415} }
				\else{          \large  \quad  \opadd[hfactor=decimal,resultstyle=\color{white},carryadd=false,carrysub=false]{2849}{2415} }
				\fi
				\part \ifprintanswers{\large  \quad   \opadd[hfactor=decimal,resultstyle=\color{red},carryadd=true,carrysub=false]{31085}{19001} }
				\else{          \large  \quad  \opadd[hfactor=decimal,resultstyle=\color{white},carryadd=false,carrysub=false]{31085}{19001} }
				\fi
				\part \ifprintanswers{\large  \quad   \opadd[hfactor=decimal,resultstyle=\color{red},carryadd=true,carrysub=false]{35701}{25484} }
				\else{          \large  \quad  \opadd[hfactor=decimal,resultstyle=\color{white},carryadd=false,carrysub=false]{35701}{25484} }
				\fi
				\part \ifprintanswers{\large  \quad   \opadd[hfactor=decimal,resultstyle=\color{red},carryadd=true,carrysub=false]{45668}{19624} }
				\else{          \large  \quad  \opadd[hfactor=decimal,resultstyle=\color{white},carryadd=false,carrysub=false]{45668}{19624} }
				\fi
				\part \ifprintanswers{\large  \quad   \opadd[hfactor=decimal,resultstyle=\color{red},carryadd=true,carrysub=false]{58718}{3652} }
				\else{          \large  \quad  \opadd[hfactor=decimal,resultstyle=\color{white},carryadd=false,carrysub=false]{58718}{3652} }
				\fi
			\end{parts}
		\end{multicols}
	}

	% \section*{\ifprintanswers{Restas 1                       }
	% \subsection*{Restando con 1, 2 y 3          }
	% \subsection*{Restando con 4, 5, 6, 7, 8 y 9 }
	% \subsection*{Restas como sumas 1            }
	% \subsection*{Restas como suma 2             }
	% \subsection*{Restas como suma 3             }


	% \section*{\ifprintanswers{Restas 2                       }
	% \subsection*{Restas sin transformación 1    }
	% \subsection*{Restas sin transformación 2    }
	% \subsection*{Restas con transformación 1    }
	% \subsection*{Restas con transformación 2    }
	% \subsection*{Restas con transformación 3    }

\questionboxed[3]{Realiza las siguientes restas:

	\begin{multicols}{4}
		\begin{parts}
			\part $9 - 3=$ \fillin[6][0cm]\\[0.2cm]
			% \part $15 - \fillin[8][0.5cm]= 7$\\[0.2cm]

			\part
			\ifprintanswers{\opsub[hfactor=decimal,resultstyle=\color{red},carrysub=true]{17}{6}}
			\else{\opsub[hfactor=decimal,resultstyle=\color{white},carrysub=false]{17}{6}}\fi\\[0.2cm]

			\part
			\ifprintanswers{\opsub[hfactor=decimal,resultstyle=\color{red},carrysub=true]{15}{4}}
			\else{\opsub[hfactor=decimal,resultstyle=\color{white},carrysub=false]{15}{4}}\fi\\[0.2cm]

			% \part
			% \ifprintanswers{\opsub[hfactor=decimal,resultstyle=\color{red},carrysub=true]{34}{22}}
			% \else{\opsub[hfactor=decimal,resultstyle=\color{white},carrysub=false]{34}{22}}\fi\\[0.2cm]

			\part $7 - 4=$ \fillin[3][0cm]\\[0.2cm]
			% \part $12 - \fillin[7][0.5cm]= 5$\\[0.2cm]

			\part
			\ifprintanswers{\opsub[hfactor=decimal,resultstyle=\color{red},carrysub=true]{27}{5}}
			\else{\opsub[hfactor=decimal,resultstyle=\color{white},carrysub=false]{27}{5}}\fi\\[0.2cm]

			\part
			\ifprintanswers{\opsub[hfactor=decimal,resultstyle=\color{red},carrysub=true]{14}{11}}
			\else{\opsub[hfactor=decimal,resultstyle=\color{white},carrysub=false]{14}{11}}\fi\\[0.2cm]

			% \part
			% \ifprintanswers{\opsub[hfactor=decimal,resultstyle=\color{red},carrysub=true]{48}{29}}
			% \else{\opsub[hfactor=decimal,resultstyle=\color{white},carrysub=false]{48}{29}}\fi\\[0.2cm]

			\part $8 - 8=$ \fillin[0][0cm]\\[0.2cm]
			% \part $18 - \fillin[14][0.5cm]= 4$\\[0.2cm]

			\part
			\ifprintanswers{\opsub[hfactor=decimal,resultstyle=\color{red},carrysub=true]{8}{5}}
			\else{\opsub[hfactor=decimal,resultstyle=\color{white},carrysub=false]{8}{5}}\fi\\[0.2cm]

			\part
			\ifprintanswers{\opsub[hfactor=decimal,resultstyle=\color{red},carrysub=true]{16}{9}}
			\else{\opsub[hfactor=decimal,resultstyle=\color{white},carrysub=false]{16}{9}}\fi\\[0.2cm]

			% \part
			% \ifprintanswers{\opsub[hfactor=decimal,resultstyle=\color{red},carrysub=true]{57}{43}}
			% \else{\opsub[hfactor=decimal,resultstyle=\color{white},carrysub=false]{57}{43}}\fi\\[0.2cm]

			\part $11 - 4=$ \fillin[7][0cm]\\[0.2cm]
			% \part $25 - \fillin[20][0.5cm]= 5$\\[0.2cm]

			\part
			\ifprintanswers{\opsub[hfactor=decimal,resultstyle=\color{red},carrysub=true]{17}{5}}
			\else{\opsub[hfactor=decimal,resultstyle=\color{white},carrysub=false]{17}{5}}\fi\\[0.2cm]

			\part
			\ifprintanswers{\opsub[hfactor=decimal,resultstyle=\color{red},carrysub=true]{10}{8}}
			\else{\opsub[hfactor=decimal,resultstyle=\color{white},carrysub=false]{10}{8}}\fi\\[0.2cm]

			% \part
			% \ifprintanswers{\opsub[hfactor=decimal,resultstyle=\color{red},carrysub=true]{63}{18}}
			% \else{\opsub[hfactor=decimal,resultstyle=\color{white},carrysub=false]{63}{18}}\fi\\[0.2cm]
		\end{parts}
	\end{multicols}
}


	\questionboxed[12]{Realiza las siguientes restas:

		\begin{multicols}{4}
			\begin{parts}
				\part \ifprintanswers{\large  \quad   \opsub[hfactor=decimal,resultstyle=\color{red},carryadd=true,carrysub=true]{4000}{2267} }
				\else{          \large  \quad  \opsub[hfactor=decimal,resultstyle=\color{white},carryadd=false,carrysub=false]{4000}{2267} }
				\fi
				\part \ifprintanswers{\large  \quad   \opsub[hfactor=decimal,resultstyle=\color{red},carryadd=true,carrysub=true]{800}{744} }
				\else{          \large  \quad  \opsub[hfactor=decimal,resultstyle=\color{white},carryadd=false,carrysub=false]{800}{744} }
				\fi
				\part \ifprintanswers{\large  \quad   \opsub[hfactor=decimal,resultstyle=\color{red},carryadd=true,carrysub=true]{3500}{308} }
				\else{          \large  \quad  \opsub[hfactor=decimal,resultstyle=\color{white},carryadd=false,carrysub=false]{3500}{308} }
				\fi
				\part \ifprintanswers{\large  \quad   \opsub[hfactor=decimal,resultstyle=\color{red},carryadd=true,carrysub=true]{3000}{189} }
				\else{          \large  \quad  \opsub[hfactor=decimal,resultstyle=\color{white},carryadd=false,carrysub=false]{3000}{189} }
				\fi
				\part \ifprintanswers{\large  \quad   \opsub[hfactor=decimal,resultstyle=\color{red},carryadd=true,carrysub=true]{1200}{966} }
				\else{          \large  \quad  \opsub[hfactor=decimal,resultstyle=\color{white},carryadd=false,carrysub=false]{1200}{966} }
				\fi
				\part \ifprintanswers{\large  \quad   \opsub[hfactor=decimal,resultstyle=\color{red},carryadd=true,carrysub=true]{3300}{2117} }
				\else{          \large  \quad  \opsub[hfactor=decimal,resultstyle=\color{white},carryadd=false,carrysub=false]{3300}{2117} }
				\fi
				\part \ifprintanswers{\large  \quad   \opsub[hfactor=decimal,resultstyle=\color{red},carryadd=true,carrysub=true]{2000}{1251} }
				\else{          \large  \quad  \opsub[hfactor=decimal,resultstyle=\color{white},carryadd=false,carrysub=false]{2000}{1251} }
				\fi
				\part \ifprintanswers{\large  \quad   \opsub[hfactor=decimal,resultstyle=\color{red},carryadd=true,carrysub=true]{2400}{2023} }
				\else{          \large  \quad  \opsub[hfactor=decimal,resultstyle=\color{white},carryadd=false,carrysub=false]{2400}{2023} }
				\fi
				\part \ifprintanswers{\large  \quad   \opsub[hfactor=decimal,resultstyle=\color{red},carryadd=true,carrysub=false]{2400}{211} }
				\else{          \large  \quad  \opsub[hfactor=decimal,resultstyle=\color{white},carryadd=false,carrysub=false]{2400}{211} }
				\fi
				\part \ifprintanswers{\large  \quad   \opsub[hfactor=decimal,resultstyle=\color{red},carryadd=true,carrysub=false]{1500}{1044} }
				\else{          \large  \quad  \opsub[hfactor=decimal,resultstyle=\color{white},carryadd=false,carrysub=false]{1500}{1044} }
				\fi
				\part \ifprintanswers{\large  \quad   \opsub[hfactor=decimal,resultstyle=\color{red},carryadd=true,carrysub=false]{2000}{1105} }
				\else{          \large  \quad  \opsub[hfactor=decimal,resultstyle=\color{white},carryadd=false,carrysub=false]{2000}{1105} }
				\fi
				\part \ifprintanswers{\large  \quad   \opsub[hfactor=decimal,resultstyle=\color{red},carryadd=true,carrysub=false]{1600}{669} }
				\else{          \large  \quad  \opsub[hfactor=decimal,resultstyle=\color{white},carryadd=false,carrysub=false]{1600}{669} }
				\fi
			\end{parts}
		\end{multicols}
	}
\end{questions}
\end{document}