\documentclass[12pt,addpoints]{repaso}
\grado{3}
\nivel{Primaria}
\cicloescolar{2025-2026}
\materia{Matemáticas}
\unidad{2}
\title{Practica la Unidad}
\aprendizajes{\scriptsize%
\item Expresa oralmente la sucesión numérica hasta cuatro cifras, en español y hasta donde sea posible, en su lengua materna, de manera ascendente y descendente a partir de un número natural dado.
\item Representa de diferentes formas, interpreta, ordena, lee y escribe números naturales de hasta cuatro cifras; identifica regularidades en los números que representan decenas, centenas y unidades de millar.
\item Notación desarrollada: Representa números naturales de hasta cuatro cifras utilizando la notación desarrollada y el valor posicional de cada dígito.
\item Resuelve situaciones problemáticas vinculadas a su contexto que implican sumas o restas de números naturales de hasta tres cifras (y aproximación a cuatro) utilizando el algoritmo convencional.
\item Establece la relación entre la suma y la resta como operaciones inversas para verificar resultados y resolver problemas de su entorno.
\item Resuelve multiplicaciones cuyo producto es de hasta tres cifras, mediante diversos procedimientos (suma iterada, arreglos rectangulares, tablas de multiplicar) y el algoritmo convencional.
\item Resuelve situaciones problemáticas vinculadas a su contexto que impliquen repartos con divisores de un solo dígito y dividendos de hasta dos o tres cifras, reconociendo la división como la operación inversa de la multiplicación.
   }
\author{Melchor Pinto, JC}
\begin{document}
\INFO\begin{multicols}{2}
	\tableofcontents
\end{multicols}
\vfill
\newpage
% \afterpage{\blankpage}
\begin{questions}
	\section{Unidad 1}
	\subsection{Escritura de cantidades}

	\questionboxed[6]{Escribe sore la línea los siguientes números

		\begin{multicols}{2}
			\begin{parts}
				\part  \fillin[$14505$][2cm] Catorce mil quinientos cinco.
				\part  \fillin[$20624$][2cm] Veinte mil seiscientos veinticuatro.
				\part  \fillin[$3742$][2cm] Tres mil setecientos cuarenta y dos.
				\part  \fillin[$10189$][2cm] Diez mil ciento ochenta y nueve.
				\part  \fillin[$13990$][2cm] Trece mil novecientos noventa.
				\part  \fillin[$11300$][2cm] Once mil trescientos.
				\part  \fillin[$14400$][2cm] Catorce mil cuatrocientos.
				\part  \fillin[$12881$][2cm] Doce mil ochocientos ochenta y uno.
				\part  \fillin[$10711$][2cm] Diez mil setecientos once.
				\part  \fillin[$11704$][2cm] Once mil setecientos cuatro.
				\part  \fillin[$10298$][2cm] Diez mil doscientos noventa y ocho.
				\part  \fillin[$1422$][2cm] mil cuatrocientos veintidos.
			\end{parts}
		\end{multicols}
	}

	% \subsection*{\ifprintanswers{Notación desarrollada 1                        }\else{}\fi}
	% \subsection*{\ifprintanswers{Notación desarrollada 2                        }\else{}\fi}
\subsection{Notación desarrollada}

	\questionboxed[5]{Escribe la notación desarrollada de cada uno de los siguientes números:

		\begin{multicols}{2}
			\begin{parts}\large
				\part $15984=$ \fillin[$10000+5000+900+80+4$][2.4in]
				\part $4936 =$ \fillin[$4000+900+30+6$][2.4in]
				\part $27545=$ \fillin[$20000+7000+500+40+5$][2.4in]
				\part $6215 =$ \fillin[$6000+200+10+5$][2.4in]
				\part $5454 =$ \fillin[$5000+400+50+4$][2.4in]
				\part $6451 =$ \fillin[$6000+400+50+1$][2.4in]
				\part $19679=$ \fillin[$10000+9000+600+70+9$][2.4in]
				\part $26324=$ \fillin[$20000+6000+300+20+4$][2.4in]
				\part $5717 =$ \fillin[$5000+700+10+7$][2.4in]
				\part $31126=$ \fillin[$30000+1000+100+20+6$][2.4in]
				\part $4818 =$ \fillin[$4000+800+10+8$][2.4in]
				\part $7145 =$ \fillin[$7000+100+40+5$][2.4in]
			\end{parts}
		\end{multicols}
	}

	\subsection{Posicionamiento decimal}

	\questionboxed[5]{Señala la opción que responda correctamente a cada una de las siguientes preguntas:

		\begin{multicols}{2}
			\begin{parts}\large
				\part En el número 3658, ¿qué número ocupa la posición de las decenas?

				\begin{oneparcheckboxes}
					\choice 3 \CorrectChoice 5 \choice 6 \choice 8 \choice 9
				\end{oneparcheckboxes}

				\part En el número 17542, ¿qué número ocupa la posición de las unidades de millar?

				\begin{oneparcheckboxes}
					\choice 1 \CorrectChoice 7 \choice 5 \choice 4 \choice 2
				\end{oneparcheckboxes}

				\part En el número 5984, ¿qué número ocupa la posición de las centenas?

				\begin{oneparcheckboxes}
					\choice 4 \choice 2 \choice 5 \choice 8 \CorrectChoice 9
				\end{oneparcheckboxes}

				\part En el número 7841, ¿qué número ocupa la posición de las decenas?

				\begin{oneparcheckboxes}
					\choice 1 \choice 7 \choice 8 \CorrectChoice 4 \choice 2
				\end{oneparcheckboxes}

				\part En el número 3918, ¿qué número ocupa la posición de las centenas?

				\begin{oneparcheckboxes}
					\choice 3 \choice 1 \choice 6 \choice 8 \CorrectChoice 9
				\end{oneparcheckboxes}

				\part En el número 3621, ¿qué número ocupa la posición de las decenas?

				\begin{oneparcheckboxes}
					\CorrectChoice 2 \choice 3 \choice 6 \choice 8 \choice 1
				\end{oneparcheckboxes}

				\part En el número 51362, ¿qué número ocupa la posición de las decenas de millar?

				\begin{oneparcheckboxes}
					\choice 3 \CorrectChoice 5 \choice 6 \choice 1 \choice 2
				\end{oneparcheckboxes}

				\part En el número 7584, ¿qué número ocupa la posición de las decenas?

				\begin{oneparcheckboxes}
					\choice 3 \choice 5 \choice 7 \CorrectChoice 8 \choice 4
				\end{oneparcheckboxes}

				\part En el número 9654, ¿qué número ocupa la posición de las centenas?

				\begin{oneparcheckboxes}
					\choice 3 \choice 5 \CorrectChoice 6 \choice 4 \choice 9
				\end{oneparcheckboxes}

				\part En el número 240679, ¿qué número ocupa la posición de las centenas de millar?

				\begin{oneparcheckboxes}
					\choice 0 \choice 6 \CorrectChoice 2 \choice 7 \choice 9 \choice 4
				\end{oneparcheckboxes}
				% \part En el número 41589, ¿qué número ocupa la posición de las decenas de millar?
				% \part En el número 8459, ¿qué número ocupa la posición de las centenas?
				% \part En el número 10562, ¿qué número ocupa la posición de las centenas?
				% \part En el número 24781, ¿qué número ocupa la posición de las decenas de millar?
				% \part En el número 7856, ¿qué número ocupa la posición de las decenas?
			\end{parts}
		\end{multicols}
	}

	\subsection{Suma de números}

	\questionboxed[5]{Realiza las siguientes sumas:

	\begin{multicols}{4}
		\begin{parts}\large
			\part $9 + 8=$ \fillin[17][0cm]\\[0.2cm]
			\part $5 + 4=$ \fillin[9][0cm]\\[0.2cm]

			\part
			\ifprintanswers{\opadd[hfactor=decimal,resultstyle=\color{red},carryadd=true]{27}{18}}
			\else{\opadd[hfactor=decimal,resultstyle=\color{white},carryadd=false]{27}{18}}\fi\\[0.2cm]

\columnbreak%

			\part $1 + 1=$ \fillin[2][0cm]\\[0.2cm]
			\part $5 + 7=$ \fillin[12][0cm]\\[0.2cm]

			\part
			\ifprintanswers{\opadd[hfactor=decimal,resultstyle=\color{red},carryadd=true]{35}{9}}
			\else{\opadd[hfactor=decimal,resultstyle=\color{white},carryadd=false]{35}{9}}\fi\\[0.2cm]
			
			\columnbreak%
			% \part
			% \ifprintanswers{\opadd[hfactor=decimal,resultstyle=\color{red},carryadd=true]{26}{9}}
			% \else{\opadd[hfactor=decimal,resultstyle=\color{white},carryadd=false]{26}{9}\\[0.5cm]}\fi\\[0.2cm]

			% \part
			% \ifprintanswers{\opadd[hfactor=decimal,resultstyle=\color{red},carryadd=true]{27}{12}}
			% \else{\opadd[hfactor=decimal,resultstyle=\color{white},carryadd=false]{27}{12}}\fi\\[0.2cm]

			\part $0 + 7=$ \fillin[7][0cm]\\[0.2cm]
			\part $8 + 7=$ \fillin[15][0cm]\\[0.2cm]

			\part
			\ifprintanswers{\opadd[hfactor=decimal,resultstyle=\color{red},carryadd=true]{11}{9}}
			\else{\opadd[hfactor=decimal,resultstyle=\color{white},carryadd=false]{11}{9}}\fi\\[0.2cm]
		
			\columnbreak%
			% \part
			% \ifprintanswers{\opadd[hfactor=decimal,resultstyle=\color{red},carryadd=true]{37}{8}}
			% \else{\opadd[hfactor=decimal,resultstyle=\color{white},carryadd=false]{37}{8}\\[0.5cm]}\fi\\[0.2cm]

			% \part
			% \ifprintanswers{\opadd[hfactor=decimal,resultstyle=\color{red},carryadd=true]{48}{14}}
			% \else{\opadd[hfactor=decimal,resultstyle=\color{white},carryadd=false]{48}{14}}\fi\\[0.2cm]

			\part $1 + 9=$ \fillin[10][0cm]\\[0.2cm]
			\part $4 + 9=$ \fillin[13][0cm]\\[0.2cm]

			\part
			\ifprintanswers{\opadd[hfactor=decimal,resultstyle=\color{red},carryadd=true]{31}{19}}
			\else{\opadd[hfactor=decimal,resultstyle=\color{white},carryadd=false]{31}{19}}\fi\\[0.2cm]
			% \part
			% \ifprintanswers{\opadd[hfactor=decimal,resultstyle=\color{red},carryadd=true]{44}{25}}
			% \else{\opadd[hfactor=decimal,resultstyle=\color{white},carryadd=false]{44}{25}\\[0.5cm]}\fi\\[0.2cm]

			% \part
			% \ifprintanswers{\opadd[hfactor=decimal,resultstyle=\color{red},carryadd=true]{82}{38}}
			% \else{\opadd[hfactor=decimal,resultstyle=\color{white},carryadd=false]{82}{38}}\fi\\[0.2cm]
		\end{parts}
	\end{multicols}
}

	\questionboxed[8]{Realiza las siguientes sumas:

		\begin{multicols}{4}
			\begin{parts}
				\part \ifprintanswers{\large  \quad   \opadd[hfactor=decimal,resultstyle=\color{red},carryadd=true,carrysub=false]{38475}{18815} }
				\else{          \large  \quad  \opadd[hfactor=decimal,resultstyle=\color{white},carryadd=false,carrysub=false]{38475}{18815} }
				\fi
				\part \ifprintanswers{\large  \quad   \opadd[hfactor=decimal,resultstyle=\color{red},carryadd=true,carrysub=false]{3234}{24156} }
				\else{          \large  \quad  \opadd[hfactor=decimal,resultstyle=\color{white},carryadd=false,carrysub=false]{3234}{24156} }
				\fi
				% \part \ifprintanswers{\large  \quad   \opadd[hfactor=decimal,resultstyle=\color{red},carryadd=true,carrysub=false]{30985}{19562} }
				% \else{          \large  \quad  \opadd[hfactor=decimal,resultstyle=\color{white},carryadd=false,carrysub=false]{30985}{19562} }
				% \fi
				% \part \ifprintanswers{\large  \quad   \opadd[hfactor=decimal,resultstyle=\color{red},carryadd=true,carrysub=false]{2849}{2415} }
				% \else{          \large  \quad  \opadd[hfactor=decimal,resultstyle=\color{white},carryadd=false,carrysub=false]{2849}{2415} }
				% \fi
				% \part \ifprintanswers{\large  \quad   \opadd[hfactor=decimal,resultstyle=\color{red},carryadd=true,carrysub=false]{31085}{19001} }
				% \else{          \large  \quad  \opadd[hfactor=decimal,resultstyle=\color{white},carryadd=false,carrysub=false]{31085}{19001} }
				% \fi
				% \part \ifprintanswers{\large  \quad   \opadd[hfactor=decimal,resultstyle=\color{red},carryadd=true,carrysub=false]{35701}{25484} }
				% \else{          \large  \quad  \opadd[hfactor=decimal,resultstyle=\color{white},carryadd=false,carrysub=false]{35701}{25484} }
				% \fi
				\part \ifprintanswers{\large  \quad   \opadd[hfactor=decimal,resultstyle=\color{red},carryadd=true,carrysub=false]{46865}{24196} }
				\else{          \large  \quad  \opadd[hfactor=decimal,resultstyle=\color{white},carryadd=false,carrysub=false]{46865}{24196} }
				\fi
				\part \ifprintanswers{\large  \quad   \opadd[hfactor=decimal,resultstyle=\color{red},carryadd=true,carrysub=false]{71858}{5362} }
				\else{          \large  \quad  \opadd[hfactor=decimal,resultstyle=\color{white},carryadd=false,carrysub=false]{71858}{5362} }
				\fi
			\end{parts}
		\end{multicols}
	}

	% \section*{\ifprintanswers{Restas 1                       }
	% \subsection*{Restando con 1, 2 y 3          }
	% \subsection*{Restando con 4, 5, 6, 7, 8 y 9 }
	% \subsection*{Restas como sumas 1            }
	% \subsection*{Restas como suma 2             }
	% \subsection*{Restas como suma 3             }


	% \section*{\ifprintanswers{Restas 2                       }
	% \subsection*{Restas sin transformación 1    }
	% \subsection*{Restas sin transformación 2    }
	% \subsection*{Restas con transformación 1    }
	% \subsection*{Restas con transformación 2    }
	% \subsection*{Restas con transformación 3    }


	\subsection{Resta de números}

\questionboxed[5]{Realiza las siguientes restas:

	\begin{multicols}{4}
		\begin{parts}\large
			\part $9 - 3=$ \fillin[6][0cm]\\[0.2cm]
			% \part $15 - \fillin[8][0.5cm]= 7$\\[0.2cm]

			\part
			\ifprintanswers{\opsub[hfactor=decimal,resultstyle=\color{red},carrysub=true]{17}{6}}
			\else{\opsub[hfactor=decimal,resultstyle=\color{white},carrysub=false]{17}{6}}\fi\\[0.2cm]

			\part
			\ifprintanswers{\opsub[hfactor=decimal,resultstyle=\color{red},carrysub=true]{15}{4}}
			\else{\opsub[hfactor=decimal,resultstyle=\color{white},carrysub=false]{15}{4}}\fi\\[0.2cm]

			% \part
			% \ifprintanswers{\opsub[hfactor=decimal,resultstyle=\color{red},carrysub=true]{34}{22}}
			% \else{\opsub[hfactor=decimal,resultstyle=\color{white},carrysub=false]{34}{22}}\fi\\[0.2cm]

			\part $7 - 4=$ \fillin[3][0cm]\\[0.2cm]
			% \part $12 - \fillin[7][0.5cm]= 5$\\[0.2cm]

			\part
			\ifprintanswers{\opsub[hfactor=decimal,resultstyle=\color{red},carrysub=true]{37}{5}}
			\else{\opsub[hfactor=decimal,resultstyle=\color{white},carrysub=false]{37}{5}}\fi\\[0.2cm]

			\part
			\ifprintanswers{\opsub[hfactor=decimal,resultstyle=\color{red},carrysub=true]{24}{15}}
			\else{\opsub[hfactor=decimal,resultstyle=\color{white},carrysub=false]{24}{15}}\fi\\[0.2cm]

			% \part
			% \ifprintanswers{\opsub[hfactor=decimal,resultstyle=\color{red},carrysub=true]{48}{29}}
			% \else{\opsub[hfactor=decimal,resultstyle=\color{white},carrysub=false]{48}{29}}\fi\\[0.2cm]

			\part $8 - 8=$ \fillin[0][0cm]\\[0.2cm]
			% \part $18 - \fillin[14][0.5cm]= 4$\\[0.2cm]

			\part
			\ifprintanswers{\opsub[hfactor=decimal,resultstyle=\color{red},carrysub=true]{8}{5}}
			\else{\opsub[hfactor=decimal,resultstyle=\color{white},carrysub=false]{8}{5}}\fi\\[0.2cm]

			\part
			\ifprintanswers{\opsub[hfactor=decimal,resultstyle=\color{red},carrysub=true]{16}{9}}
			\else{\opsub[hfactor=decimal,resultstyle=\color{white},carrysub=false]{16}{9}}\fi\\[0.2cm]

			% \part
			% \ifprintanswers{\opsub[hfactor=decimal,resultstyle=\color{red},carrysub=true]{57}{43}}
			% \else{\opsub[hfactor=decimal,resultstyle=\color{white},carrysub=false]{57}{43}}\fi\\[0.2cm]

			\part $11 - 4=$ \fillin[7][0cm]\\[0.2cm]
			% \part $25 - \fillin[20][0.5cm]= 5$\\[0.2cm]

			\part
			\ifprintanswers{\opsub[hfactor=decimal,resultstyle=\color{red},carrysub=true]{14}{5}}
			\else{\opsub[hfactor=decimal,resultstyle=\color{white},carrysub=false]{14}{5}}\fi\\[0.2cm]

			\part
			\ifprintanswers{\opsub[hfactor=decimal,resultstyle=\color{red},carrysub=true]{10}{7}}
			\else{\opsub[hfactor=decimal,resultstyle=\color{white},carrysub=false]{10}{7}}\fi\\[0.2cm]

			% \part
			% \ifprintanswers{\opsub[hfactor=decimal,resultstyle=\color{red},carrysub=true]{63}{18}}
			% \else{\opsub[hfactor=decimal,resultstyle=\color{white},carrysub=false]{63}{18}}\fi\\[0.2cm]
		\end{parts}
	\end{multicols}
}


	\questionboxed[8]{Realiza las siguientes restas:

		\begin{multicols}{4}
			\begin{parts}
				\part \ifprintanswers{\large  \quad   \opsub[hfactor=decimal,resultstyle=\color{red},carryadd=true,carrysub=true]{811}{744} }
				\else{          \large  \quad  \opsub[hfactor=decimal,resultstyle=\color{white},carryadd=false,carrysub=false]{811}{744} }
				\fi
				\part \ifprintanswers{\large  \quad   \opsub[hfactor=decimal,resultstyle=\color{red},carryadd=true,carrysub=true]{4090}{2867} }
				\else{          \large  \quad  \opsub[hfactor=decimal,resultstyle=\color{white},carryadd=false,carrysub=false]{4090}{2867} }
				\fi
				% \part \ifprintanswers{\large  \quad   \opsub[hfactor=decimal,resultstyle=\color{red},carryadd=true,carrysub=true]{3500}{308} }
				% \else{          \large  \quad  \opsub[hfactor=decimal,resultstyle=\color{white},carryadd=false,carrysub=false]{3500}{308} }
				% \fi
				% \part \ifprintanswers{\large  \quad   \opsub[hfactor=decimal,resultstyle=\color{red},carryadd=true,carrysub=true]{3000}{189} }
				% \else{          \large  \quad  \opsub[hfactor=decimal,resultstyle=\color{white},carryadd=false,carrysub=false]{3000}{189} }
				% \fi
				% \part \ifprintanswers{\large  \quad   \opsub[hfactor=decimal,resultstyle=\color{red},carryadd=true,carrysub=true]{1200}{966} }
				% \else{          \large  \quad  \opsub[hfactor=decimal,resultstyle=\color{white},carryadd=false,carrysub=false]{1200}{966} }
				% \fi
				% \part \ifprintanswers{\large  \quad   \opsub[hfactor=decimal,resultstyle=\color{red},carryadd=true,carrysub=true]{3300}{2117} }
				% \else{          \large  \quad  \opsub[hfactor=decimal,resultstyle=\color{white},carryadd=false,carrysub=false]{3300}{2117} }
				% \fi
				% \part \ifprintanswers{\large  \quad   \opsub[hfactor=decimal,resultstyle=\color{red},carryadd=true,carrysub=true]{2000}{1251} }
				% \else{          \large  \quad  \opsub[hfactor=decimal,resultstyle=\color{white},carryadd=false,carrysub=false]{2000}{1251} }
				% \fi
				% \part \ifprintanswers{\large  \quad   \opsub[hfactor=decimal,resultstyle=\color{red},carryadd=true,carrysub=true]{2400}{2023} }
				% \else{          \large  \quad  \opsub[hfactor=decimal,resultstyle=\color{white},carryadd=false,carrysub=false]{2400}{2023} }
				% \fi
				% \part \ifprintanswers{\large  \quad   \opsub[hfactor=decimal,resultstyle=\color{red},carryadd=true,carrysub=false]{2400}{211} }
				% \else{          \large  \quad  \opsub[hfactor=decimal,resultstyle=\color{white},carryadd=false,carrysub=false]{2400}{211} }
				% \fi
				% \part \ifprintanswers{\large  \quad   \opsub[hfactor=decimal,resultstyle=\color{red},carryadd=true,carrysub=false]{1500}{1044} }
				% \else{          \large  \quad  \opsub[hfactor=decimal,resultstyle=\color{white},carryadd=false,carrysub=false]{1500}{1044} }
				% \fi
				\part \ifprintanswers{\large  \quad   \opsub[hfactor=decimal,resultstyle=\color{red},carryadd=true,carrysub=false]{2000}{1105} }
				\else{          \large  \quad  \opsub[hfactor=decimal,resultstyle=\color{white},carryadd=false,carrysub=false]{2000}{1105} }
				\fi
				\part \ifprintanswers{\large  \quad   \opsub[hfactor=decimal,resultstyle=\color{red},carryadd=true,carrysub=false]{1900}{669} }
				\else{          \large  \quad  \opsub[hfactor=decimal,resultstyle=\color{white},carryadd=false,carrysub=false]{1900}{669} }
				\fi
			\end{parts}
		\end{multicols}
	}




	% UNIDAD 2

	% \section*{\ifprintanswers{Tablas de multiplicar 2        }\else{}\fi}
	% \subsection*{\ifprintanswers{Tabla del 6                    }\else{}\fi}
	% \subsection*{\ifprintanswers{Tabla del 7                    }\else{}\fi}
	% \subsection*{\ifprintanswers{Tabla del 8                    }\else{}\fi}
	% \subsection*{\ifprintanswers{Tabla del 9                    }\else{}\fi}
	% \subsection*{\ifprintanswers{Tabla del 10                   }\else{}\fi}
\newpage
\section{Unidad 2}

	\subsection{Tablas de multiplicar}
	\questionboxed[15]{Reponde las siguientes tablas de multiplicar:

		\begin{multicols}{4}
			\begin{parts}\Large
				\part $7 \times 6=$ \fillin[$42$][0cm]
				\part $9 \times 7=$ \fillin[$63$][0cm]
				\part $6 \times 8=$ \fillin[$48$][0cm]
				\part $6 \times 9=$ \fillin[$54$][0cm]
				\part $4 \times 4=$ \fillin[$16$][0cm]
				\part $7 \times 7=$ \fillin[$49$][0cm]
				\part $7 \times 5=$ \fillin[$35$][0cm]
				\part $5 \times 4=$ \fillin[$20$][0cm]
				\part $8 \times 7=$ \fillin[$56$][0cm]
				\part $3 \times 6=$ \fillin[$18$][0cm]
				\part $5 \times 9=$ \fillin[$45$][0cm]
				\part $5 \times 6=$ \fillin[$30$][0cm]
				\part $3 \times 8=$ \fillin[$24$][0cm]
				\part $2 \times 9=$ \fillin[$18$][0cm]
				\part $2 \times 7=$ \fillin[$14$][0cm]
				\part $4 \times 7=$ \fillin[$28$][0cm]
				\part $3 \times 9=$ \fillin[$27$][0cm]
				\part $5 \times 8=$ \fillin[$40$][0cm]
				\part $6 \times 6=$ \fillin[$36$][0cm]
				\part $7 \times 8=$ \fillin[$56$][0cm]
				\part $9 \times 4=$ \fillin[$36$][0cm]
				\part $8 \times 8=$ \fillin[$64$][0cm]
				\part $9 \times 9=$ \fillin[$81$][0cm]
				\part $6 \times 7=$ \fillin[$42$][0cm]
			\end{parts}
		\end{multicols}
	}

	\questionboxed[15]{Completa las siguientes tablas de multiplicar:

		\begin{multicols}{4}
			\begin{parts}\Large
				\part $\fillin[9][0.5cm] \times 9= 81$
				\part $4 \times \fillin[9][0.5cm]= 36$
				\part $\fillin[7][0.5cm] \times 4= 28$
				\part $\fillin[9][0.5cm] \times 3= 27$
				\part $8 \times \fillin[5][0.5cm]= 40$
				\part $\fillin[6][0.5cm] \times 4= 24$
				\part $7 \times \fillin[7][0.5cm]= 49$
				\part $\fillin[8][0.5cm] \times 3= 24$
				\part $9 \times \fillin[8][0.5cm]= 72$
				\part $10 \times \fillin[10][0.5cm]=100$
				\part $5 \times \fillin[7][0.5cm]=35$
				\part $6 \times \fillin[7][0.5cm]= 42$	
				\part $\fillin[1][0.5cm] \times 9= 9$
				\part $\fillin[6][0.5cm] \times 5= 30$
				\part $\fillin[7][0.5cm] \times 3= 21$
				
				\part $9 \times \fillin[10][0.5cm]=90$

				\part $\fillin[7][0.5cm] \times 8= 56$
				\part $5 \times \fillin[10][0.5cm]=50$
				\part $4 \times \fillin[6][0.5cm]=24$
				\part $1 \times \fillin[7][0.5cm]= 7$	
				\part $\fillin[9][0.5cm] \times 5= 45$
				\part $\fillin[6][0.5cm] \times 6= 36$
				\part $\fillin[8][0.5cm] \times 8= 64$
				\part $\fillin[6][0.5cm] \times 9= 54$
				\part	$3 \times \fillin[10][0.5cm]=30$
				\part $4 \times \fillin[8][0.5cm]=32$
				\part $6 \times \fillin[0][0.5cm]= 0$				
			\end{parts}
		\end{multicols}
	}
% \section*{\ifprintanswers{Multiplicaciones                           }\else{}\fi}
% \subsection*{\ifprintanswers{Multiplicaciones con una cifra 1           }\else{}\fi}
% \subsection*{\ifprintanswers{Multiplicaciones con una cifra 2           }\else{}\fi}
% \subsection*{\ifprintanswers{Multiplicaciones con una cifra 3           }\else{}\fi}
% \subsection*{\ifprintanswers{Multiplicaciones con una cifra 4           }\else{}\fi}
% \subsection*{\ifprintanswers{Multiplicaciones con dos cifras            }\else{}\fi}

\subsection{multiplicación de números}

\questionboxed[12]{Realiza las siguientes multiplicaciones:

	\begin{multicols}{3}
		\begin{parts}\large
			\part \ifprintanswers{\large  \quad   \opmul[hfactor=decimal,resultstyle=\color{red},displayintermediary=None]{354}{7} }
			\else{          \large  \quad  \opmul[hfactor=decimal,resultstyle=\color{white},displayintermediary=None]{354}{7} }  \\[2em]
			\fi
			\part \ifprintanswers{\large  \quad   \opmul[hfactor=decimal,resultstyle=\color{red},displayintermediary=None]{289}{9} }
			\else{          \large  \quad  \opmul[hfactor=decimal,resultstyle=\color{white},displayintermediary=None]{289}{9} }  \\[2em]
			\fi
			\part \ifprintanswers{\large  \quad   \opmul[hfactor=decimal,resultstyle=\color{red},displayintermediary=None]{2081}{5} }
			\else{          \large  \quad  \opmul[hfactor=decimal,resultstyle=\color{white},displayintermediary=None]{2081}{5} }  \\[2em]
			\fi
			\part \ifprintanswers{\large  \quad   \opmul[hfactor=decimal,resultstyle=\color{red},displayintermediary=None]{4918}{6} }
			\else{          \large  \quad  \opmul[hfactor=decimal,resultstyle=\color{white},displayintermediary=None]{4918}{6} }  \\[2em]
			\fi
			\part \ifprintanswers{\large  \quad   \opmul[hfactor=decimal,resultstyle=\color{red},displayintermediary=None]{255}{24} }
			\else{          \large  \quad  \opmul[hfactor=decimal,resultstyle=\color{white},displayintermediary=None]{255}{24} }  \\[2em]
			\fi  \\[2em]
			\part \ifprintanswers{\large  \quad   \opmul[hfactor=decimal,resultstyle=\color{red},displayintermediary=None]{6767}{67} }
			\else{          \large  \quad  \opmul[hfactor=decimal,resultstyle=\color{white},displayintermediary=None]{6767}{67} }  \\[2em]
			\fi
		\end{parts}
	\end{multicols}
}


% \section*{\ifprintanswers{Divisiones                                 }\else{}\fi}
% \subsection*{\ifprintanswers{Divisiones del 1 al 5                      }\else{}\fi}
% \subsection*{\ifprintanswers{Divisiones del 6 al 10                     }\else{}\fi}
% \subsection*{\ifprintanswers{Divisiones sin residuos                    }\else{}\fi}
% \subsection*{\ifprintanswers{Divisiones con residuo 1                   }\else{}\fi}
% \subsection*{\ifprintanswers{Divisiones con residuo 2                   }\else{}\fi}

\subsection{División de números}

\questionboxed[16]{Realiza las siguientes divisiones:

	\begin{multicols}{4}
		\begin{parts}
			\part \ifprintanswers{\large\opidiv{123}{6}} \\[0.5em]
			\else{          \Large  \quad $6 \overline{) \ 123\ }$} \\[4em]
			\fi

			\part \ifprintanswers{\large\opidiv{200}{3}} \\[0.5em]
			\else{          \Large  \quad $3 \overline{) \ 200\ }$} \\[4em]
			\fi

			\part \ifprintanswers{\large\opidiv{399}{8}} \\[0.5em]
			\else{          \Large  \quad $8 \overline{) \ 399\ }$} \\[4em]
			\fi

			\part \ifprintanswers{\large\opidiv{193}{7}} \\[0.5em]
			\else{          \Large  \quad $7 \overline{) \ 193\ }$} \\[4em]
			\fi

			\part \ifprintanswers{\large\opidiv{283}{6}} \\[0.5em]
			\else{          \Large  \quad $6 \overline{) \ 283\ }$} \\[4em]
			\fi

			\part \ifprintanswers{\large\opidiv{432}{9}} \\[0.5em]
			\else{          \Large  \quad $9 \overline{) \ 432\ }$} \\[4em]
			\fi

			\part \ifprintanswers{\large\opidiv{644}{8}} \\[0.5em]
			\else{          \Large  \quad $8 \overline{) \ 644\ }$} \\[4em]
			\fi

			\part \ifprintanswers{\large\opidiv{656}{7}} \\[0.5em]
			\else{          \Large  \quad $7 \overline{) \ 656\ }$} \\[4em]
			\fi
		\end{parts}
	\end{multicols}
}
\end{questions}
\end{document}	