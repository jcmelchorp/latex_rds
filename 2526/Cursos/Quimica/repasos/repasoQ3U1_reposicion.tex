\documentclass[12pt,addpoints]{repaso}
\grado{3}
\nivel{Secundaria}
\cicloescolar{2025-2026}
\materia{Ciencias y Tecnología: Química}
\unidad{1}
\title{Practica la reposición a la Unidad}
\aprendizajes{\footnotesize%
\item Reconoce los aportes de saberes de diferentes pueblos y culturas en la satisfacción de necesidades humanas en diversos ámbitos (medicina, construcción, artesanías, textiles y alimentos).\\[-1.2em]
\item Indaga en fuentes de consulta orales y escritas, las aportaciones de mujeres y hombres en el desarrollo del conocimiento científico y tecnológico, para valorar su influencia en la sociedad actual.\\[-1.2em]
\item Formula hipótesis para diferenciar propiedades extensivas e intensivas, mediante actividades experimentales y, con base en el análisis de resultados, elabora conclusiones.\\[-1.2em]
\item Reconoce la importancia del uso de instrumentos de medición, para identificar y diferenciar propiedades de sustancias y materiales cotidianos.\\[-1.2em]
\item Reconoce intercambios de energía entre el sistema y sus alrededores durante procesos físicos.\\[-1.2em]
\item Describe los componentes de una mezcla (soluto - disolvente; fase dispersa y fase dispersante) mediante actividades experimentales y las clasifica en homogéneas y heterogéneas en materiales de uso cotidiano.\\[-1.2em]
\item Deduce métodos para separar mezclas mediante actividades experimentales con base en las propiedades físicas de las sustancias involucradas, así como su funcionalidad en actividades humanas.\\[-1.2em]
\item Analiza la concentración de sustancias de una mezcla expresadas en porcentaje en masa y porcentaje en volumen en productos de higiene personal, alimentos, limpieza, entre otros, para la toma de decisiones orientadas al cuidado de la salud y al consumo responsable.\\[-1.2em]
\item Sistematiza la información de diferentes fuentes de consulta, orales y escritas, acerca de la concentración de contaminantes (partes por millón, -ppm-) en aire, agua y suelo.\\[-1.2em]
\item Indaga situaciones problemáticas relacionadas con la degradación y contaminación en la comunidad, vinculadas con el uso de productos y procesos químicos.\\[-1.2em]
     %  \item Formula hipótesis para diferenciar propiedades extensivas e intensivas, mediante actividades experimentales y, con base en el análisis de resultados, elabora conclusiones.\\[-1.5em]
     %  \item Reconoce la importancia del uso de instrumentos de medición, para identificar y diferenciar propiedades de sustancias y materiales cotidianos.\\[-1.5em]
     %  \item Describe los componentes de una mezcla (soluto-disolvente; fase dispersa y fase dispersante) mediante actvidades experimentales y las clasifica en homogéneas y heterogéneas en materiales de uso cotidiano.\\[-1.5em]
     %  \item Deduce métodos para separar mezclas (evaporación, decantación, filtración, extracción, sublimación, cromatografía y cristalización) mediante actividades experimentales con base en las propiedades físicas de las sustancias.\\[-1.5em]
      }
\author{Melchor Pinto, J.C.}
\begin{document}
\INFO
\begin{multicols}{2}
	\tableofcontents
\end{multicols}
\vfill
% \afterpage{\blankpage}
\begin{questions}
	
	\section{L1 Aportaciones de culturas en la satisfacción de necesidades}

	\questionboxed[5]{Elige si son verdaderas o falsas las siguientes afirmaciones.

		\begin{multicols}{2}
			\begin{parts}
				\part Solamente las sociedades modernas han aportado conocimientos que ayudan a la satisfacción de las necesidades humanas.

				\begin{oneparchoices}
					\choice Verdadero \CorrectChoice Falso
				\end{oneparchoices}

				\part El Homo sapiens ``domesticó'' el fuego hace aproximadamente 1.6 millones de años.

				\begin{oneparchoices}
					\choice Verdadero \CorrectChoice Falso
				\end{oneparchoices}

				\part Los conocimientos empíricos de los pueblos prehispánicos sobre plantas medicinales y hongos pueden ayudarnos a resolver problemas y necesidades actuales.

				\begin{oneparchoices}
					\CorrectChoice Verdadero \choice Falso
				\end{oneparchoices}

				\part El jabón es un invento moderno del siglo XIX que ayuda a mejorar nuestra calidad de vida.

				\begin{oneparchoices}
					\choice Verdadero \CorrectChoice Falso
				\end{oneparchoices}

				\part Todas las culturas de los cinco continentes han aportado conocimientos y avances tecnológicos en beneficio de la humanidad.

				\begin{oneparchoices}
					\CorrectChoice Verdadero \choice Falso
				\end{oneparchoices}

				\part El conocimiento empírico es igual al conocimiento científico.

				\begin{oneparchoices}
					\choice Verdadero \CorrectChoice Falso
				\end{oneparchoices}

				\part La saponificación es el proceso químico que nos permite obtener jabones.

				\begin{oneparchoices}
					\CorrectChoice Verdadero \choice Falso
				\end{oneparchoices}

				\part Existe evidencia de que el jabón se producía en Babilonia hace 6000 años.

				\begin{oneparchoices}
					\choice Verdadero \CorrectChoice Falso
				\end{oneparchoices}

				\part La expectativa de vida ha incrementado en los últimos 150 años gracias al descubrimiento de medicamentos y al desarrollo de los procesos de sanidad.

				\begin{oneparchoices}
					\CorrectChoice Verdadero \choice Falso
				\end{oneparchoices}

				\part Las aportaciones de las culturas originarias en la satisfacción de necesidades también se ven reflejadas en la arquitectura y en la construcción.

				\begin{oneparchoices}
					\CorrectChoice Verdadero \choice Falso
				\end{oneparchoices}
			\end{parts}
		\end{multicols}
	}

	% \addcontentsline{toc}{section}{L3 Propiedades de los materiales}
	\section{L3 Propiedades de los materiales}

	\questionboxed[5]{Señala si los siguientes procesos son \textit{físicos} o \textit{químicos}.

		\begin{multicols}{2}
			\begin{parts}
				\part Romper un tazón de cerámica.

				\begin{oneparchoices}
					\CorrectChoice Físico \choice Químico
				\end{oneparchoices}

				\part Digerir y absorber los alimentos.

				\begin{oneparchoices}
					\choice Físico \CorrectChoice Químico
				\end{oneparchoices}

				\part Disolver azucar en una taza de té.

				\begin{oneparchoices}
					\CorrectChoice Físico \choice Químico
				\end{oneparchoices}

				\part Encender fuegos artificiales.

				\begin{oneparchoices}
					\choice Físico \CorrectChoice Químico
				\end{oneparchoices}

				\part Hornear un pastel de vainilla.

				\begin{oneparchoices}
					\choice Físico \CorrectChoice Químico
				\end{oneparchoices}

				\part Apretar una lata de aluminio.

				\begin{oneparchoices}
					\CorrectChoice Físico \choice Químico
				\end{oneparchoices}

				\part Mezclar pigmentos de colores.

				\begin{oneparchoices}
					\CorrectChoice Físico \choice Químico
				\end{oneparchoices}

				\part Cocinar un huevo estrellado.

				\begin{oneparchoices}
					\choice Físico \CorrectChoice Químico
				\end{oneparchoices}
			\end{parts}
		\end{multicols}
	}

	\questionboxed[5]{Selecciona si las afirmaciones son verdaderas o falsas.

		\begin{parts}
			\part Las propiedades químicas del PVC no se pueden determinar debido a que es un material que presenta demasiada dureza.

			\begin{oneparchoices}
				\choice Verdadero \CorrectChoice Falso
			\end{oneparchoices}

			\part El lustre y el brillo son propiedades físicas mecánicas que predominan con mayor frecuencia en los metales.

			\begin{oneparchoices}
				\choice Verdadero \CorrectChoice Falso
			\end{oneparchoices}

			\part El aroma, o incluso el sabor, de un material orgánico se clasifican como propiedades físicas cualitativas.

			\begin{oneparchoices}
				\CorrectChoice Verdadero \choice Falso
			\end{oneparchoices}
		\end{parts}
	}

	% \addcontentsline{toc}{section}{L4 Medición e identificación de sustancias}
	\section{L4 Medición e identificación de sustancias}

	\questionboxed[5]{Selecciona la opción que  resuelve correctamente cada uno de los siguientes problemas:

		\begin{parts}
			\part La máxima masa de glucosa que se disuelve en 0.1L de agua es 90.9 g a 25°C. ¿Cuál es la solubilidad en g/L?

			\begin{oneparchoices}
				\choice 90.9 g/L \choice 9.09 g/L \CorrectChoice 909 g/L \choice 0.909 g/L
			\end{oneparchoices}

			\part La máxima masa de fructosa que se disuelve en 1L de agua es 3750 g a 20°C. ¿Cuál es la solubilidad en g/dL?

			\begin{oneparchoices}
				\choice 37.5 g/dL \choice 20 g/dL \CorrectChoice 375 g/dL \choice 37500 g/dL
			\end{oneparchoices}

			\part Si la solubilidad del cloruro de magnesio es de 54.2 g/100 mL a 20°C, ¿cuál sería su solubilidad en g/L?

			\begin{oneparchoices}
				\CorrectChoice 542 g/L \choice 20 g/L \choice 54.2 g/L \choice 5.42 g/L
			\end{oneparchoices}

			\part  Si la solubilidad del amoniaco es de 89.9 g/100 ml a 0 °C, ¿cuál sería su solubilidad en g/L?

			\begin{oneparchoices}
				\CorrectChoice 899 g/L \choice 20 g/L \choice 8990 g/L \choice 89.9 g/L
			\end{oneparchoices}

			\part  La máxima masa de dióxido de carbono que se disuelve en 1L de agua es 1.45g a 25°C. ¿Cuál es la solubilidad en g/dL?

			\begin{oneparchoices}
				\choice 1.45 g/dL     \choice 145 g/dL     \choice 145 g/dL     \CorrectChoice 0.145 g/dL
			\end{oneparchoices}

		\end{parts}

	}

	% \questionboxed[5]{Selecciona si las afirmaciones son verdaderas o falsas.
	%       \begin{parts}
	%             \part La acetona cambia al estado líquido siempre con el mismo calor latente.

	%             \begin{oneparchoices}
	%                   \choice Verdadero \choice Falso
	%             \end{oneparchoices}

	%             \part El cambio de estado gaseoso a líquido de un material es un proceso de sublimación.

	%             \begin{oneparchoices}
	%                   \choice Verdadero \choice Falso
	%             \end{oneparchoices}

	%             \part El calor de vaporización es una propiedad extensiva característica de cada material.

	%             \begin{oneparchoices}
	%                   \choice Verdadero \choice Falso
	%             \end{oneparchoices}

	%             \part La condensación de un material ocurre cuando pasa de estado líquido a gaseoso.

	%             \begin{oneparchoices}
	%                   \choice Verdadero \choice Falso
	%             \end{oneparchoices}

	%             \part El calor de fusión es la energía que se proporciona a un material para que pase de estado sólido a líquido

	%             \begin{oneparchoices}
	%                   \choice Verdadero \choice Falso
	%             \end{oneparchoices}
	%       \end{parts}
	% }



	\questionboxed[5]{Relaciona cada enunciado con la propiedad física que representa.

		\begin{multicols}{2}
			\begin{parts}\raggedleft
				\part Espacio que ocupa un material.               \fillin[D][0.2in]
				\part Cantidad de materia de un material.          \fillin[B][0.2in]
				\part Masa por unidad de volumen.                  \fillin[E][0.2in]
				\part Depende de la cantidad total del sistema.    \fillin[A][0.2in]
				\part Es independiente a la cantidad de sustancia. \fillin[C][0.2in]

			\end{parts}
			\columnbreak

			\begin{choices}
				\choice Extensiva
				\choice Masa
				\choice Intensiva
				\choice Volumen
				\choice Densidad
			\end{choices}
		\end{multicols}
	}

	% \addcontentsline{toc}{section}{L6 Mezclas}
	\section{L6 Mezclas}

	\questionboxed[10]{Calcula la concentración de contaminantes en las siguientes muestras de agua potable y escribe el resultado en el cuadro de texto.

		\begin{table}[H]
			\centering\renewcommand{\arraystretch}{1.5}
			\begin{tabular}{p{2cm}>{\centering\columncolor{DarkOliveGreen!10}}p{3cm}>{\centering}p{2.8cm}>{\columncolor{Sepia!10}}p{3cm}p{0.1cm}}
				          & \bfseries Masa del agua potable & \bfseries Masa del cloro residual & \bfseries Concentración de nitratos \\
				Muestra 1 & 1 000 g                         & 0.006 g                           & \fillin[6][0.6in] ppm               \\ \hline
				Muestra 2 & 10 000 g                        & 0.6 g                             & \fillin[60][0.6in] ppm              \\ \hline
				Muestra 3 & 50 000 g                        & 1 g                               & \fillin[20][0.6in] ppm              \\ \hline
				Muestra 4 & 100 000 g                       & 12 g                              & \fillin[120][0.6in] ppm             \\ \bottomrule
			\end{tabular}
		\end{table}

	}

	\questionboxed[10]{ A partir de la información que se presenta, coloca los datos que faltan en la tabla.

		\begin{table}[H]
			\centering\renewcommand{\arraystretch}{1.5}
			\begin{tabular}{p{4cm}>{\centering\columncolor{DarkOliveGreen!10}}p{1.2cm}p{1.5cm}>{\columncolor{Sepia!10}}p{1.5cm}p{0.1cm}}
				\bfseries Sustancia  & \bfseries ppm & \bfseries \%             & \bfseries mg/l       \\    \hline
				Dióxido de azufre    & 0.13          & \fillin[0.000013][0.6in] & \fillin[0.13][0.6in] \\    \hline
				Dióxido de nitrógeno & 0.21          & \fillin[0.000021][0.6in] & \fillin[0.21][0.6in] \\    \hline
				Monóxido de carbono  & 11            & \fillin[0.0011][0.6in]   & \fillin[11][0.6in]   \\    \hline
				Ozono                & 0.11          & \fillin[0.000011][0.6in] & \fillin[0.11][0.6in] \\    \hline
			\end{tabular}
		\end{table}

	}

	\questionboxed[10]{Calcula la concentración de contaminantes en las siguientes muestras de agua potable y escribe el resultado en el cuadro de texto.

		\begin{table}[H]
			\centering\renewcommand{\arraystretch}{1.5}
			\begin{tabular}{p{2cm}>{\centering\columncolor{DarkOliveGreen!10}}p{3cm}>{\centering}p{2.8cm}>{\columncolor{Sepia!10}}p{3cm}p{0.1cm}}
				          & \bfseries Masa del agua potable & \bfseries Masa del cloro residual & \bfseries Concentración de nitratos \\
				Muestra 1 & 1 000 g                         & 0.016 g                           & \fillin[0.00016][0.6in] m/m         \\ \hline
				Muestra 2 & 10 000 g                        & 0.4 g                             & \fillin[0.004][0.6in] m/m           \\ \hline
				Muestra 3 & 50 000 g                        & 5 g                               & \fillin[0.01][0.6in] m/m            \\ \hline
				Muestra 4 & 100 000 g                       & 150 g                             & \fillin[0.15][0.6in] m/m            \\ \bottomrule
			\end{tabular}
		\end{table}

	}

	% \addcontentsline{toc}{section}{L7 Métodos de separación de mezclas}
	\section{L7 Métodos de separación de mezclas}

	\questionboxed[5]{Elige la respuesta correcta

		\begin{multicols}{2}
			\begin{parts}
				% \part¿Qué propiedad del cobre se expande cuando éste se calienta al rojo vivo?

				% \begin{choices}
				%       \choice Masa \choice Peso \choice Volumen \choice Fuerza
				% \end{choices}

				% \part ¿Qué nombre recibe la reducción térmica de una lámina de plomo al transferir energía en forma de calor a sus alrededores?

				% \begin{choices}
				%       \choice Calor específico \choice Dilatación \choice Fusibilidad \choice Conductividad
				% \end{choices}

				% \part ¿En qué estado de agregación es mayor el cambio de volumen del pentano al incrementar su temperatura?

				% \begin{choices}
				%       \choice Plasma \choice Líquido \choice Sólido \choice Gaseoso
				% \end{choices}

				% \part ¿Cómo se conoce a la energía en forma de calor requerida para elevar un grado Celsius la temperatura de un gramo de níquel?

				% \begin{choices}
				%       \choice Calor específico \choice  Dilatación \choice Fusibilidad \choice Conductividad
				% \end{choices}

				\part ¿Cuál de los siguientes materiales es una mezcla heterogénea?

				\begin{choices}
					\choice Acero \choice Plata \CorrectChoice Tierra \choice Metano
				\end{choices}

				\part ¿Qué método de separación de mezclas usarías para separar una muestra de arena que está suspendida en un líquido?

				\begin{choices}
					\choice Destilación \choice Cromatografía \choice Magnetismo \CorrectChoice Decantación
				\end{choices}

				\part ¿En qué propiedad de las sustancias se basa la decantación?

				\begin{choices}
					\choice Dureza \choice Temperatura \CorrectChoice Densidad \choice Conductividad
				\end{choices}

				\part ¿Qué método de separación usarías para separar los componentes de una mezcla heterogénea de níquel y agua?

				\begin{choices}
					\choice Destilación \choice Magnetismo \choice Cristalización \CorrectChoice Filtración
				\end{choices}

			\end{parts}

		\end{multicols}
	}


	\questionboxed[5]{Elige el método de separación que debe de usarse en cada mezcla.

		\begin{multicols}{2}
			\begin{parts}
				\part Una mezcla de aire.

				\begin{oneparchoices}\footnotesize
					\choice Extracción  \choice Destilación \\ \choice Filtración  \CorrectChoice Cromatografía
				\end{oneparchoices}

				\part Una mezcla de azufre y agua.

				\begin{oneparchoices}\footnotesize
					\CorrectChoice Filtración  \choice Destilación \\ \choice Cromatografía  \choice Extracción
				\end{oneparchoices}

				\part Una mezcla de sal, azufre y agua (recuerda que la sal se disuelve en agua pero el azufre no).

				\begin{oneparchoices}\footnotesize
					\choice Extracción y tamizado
					\choice Destilación y filtración
					\choice Cromatografía y evaporación \\
					\CorrectChoice Filtración y evaporación
				\end{oneparchoices}

				\part Una muestra de gasolina.

				\begin{oneparchoices}\footnotesize
					\choice Cromatografía  \choice Filtración \\ \CorrectChoice Destilación  \choice Extracción
				\end{oneparchoices}

				\part Una mezcla homogénea de líquidos.

				\begin{oneparchoices}\footnotesize
					\CorrectChoice Destilación  \choice Cromatografía \\ \choice Extracción  \choice Filtración
				\end{oneparchoices}

				\part Una mezcla de tinta negra.

				\begin{oneparchoices}\footnotesize
					\CorrectChoice Cromatografía  \choice Filtración \\ \choice Destilación  \choice Extracción.
				\end{oneparchoices}

				\part Una mezcla de agua y sal.

				\begin{oneparchoices}\footnotesize
					\CorrectChoice Evaporación  \choice Cromatografía \\ \choice Filtración  \choice Destilación
				\end{oneparchoices}

				\part Una mezcla de agua y arena.

				\begin{oneparchoices}\footnotesize
					\CorrectChoice Filtración  \choice Cromatografía \\ \choice Extracción  \choice Decantación
				\end{oneparchoices}

				\part Una mezcla de vinagre y aceite de olivo.

				\begin{oneparchoices}\footnotesize
					\choice Extracción  \CorrectChoice Decantación \\ \choice Cromatografía  \choice Filtración
				\end{oneparchoices}

				\part Una mezcla de pan molido y clips.

				\begin{oneparchoices}\footnotesize
					\choice Extracción  \CorrectChoice Filtración \\ \choice Decantación  \choice Cromatografía
				\end{oneparchoices}
			\end{parts}
		\end{multicols}
	}

	\questionboxed[5]{Relaciona los métodos que se utilizaría para separar las siguientes mezclas.

		\begin{multicols}{2}
			\begin{parts}\raggedleft\footnotesize
				\part Tierra y sal                                      \fillin[C][0.15in]
				\part Dos líquidos con diferente densidad               \fillin[E][0.15in]
				\part Aire y polvo                                      \fillin[B][0.15in]
				\part Sólidos de diferente tamaño y que no se disuelven \fillin[A][0.15in]
				\part Limadura de hierro y arena                        \fillin[D][0.15in]
			\end{parts}
			\columnbreak

			\begin{choices}
				\choice Tamización
				\choice Filtración
				\choice Filtración y evaporación
				\choice Filtración e imantación
				\choice Decantación
			\end{choices}
		\end{multicols}
	}

	\questionboxed[5]{Indica si se trata de una mezcla homogénea o heterogénea.

		\begin{multicols}{3}
			\begin{parts}
				\part Perfume

				\begin{choices}
					\CorrectChoice Homogénea \choice Heterogénea
				\end{choices}

				\part Café

				\begin{choices}
					\CorrectChoice Homogénea \choice Heterogénea
				\end{choices}

				\columnbreak%

				\part Aceite trifásico

				\begin{choices}
					\choice Homogénea \CorrectChoice Heterogénea
				\end{choices}

				\part Acero

				\begin{choices}
					\CorrectChoice Homogénea \choice Heterogénea
				\end{choices}

				\columnbreak%

				\part Vinagre y aceite

				\begin{choices}
					\choice Homogénea \CorrectChoice Heterogénea
				\end{choices}

				\part Granito

				\begin{choices}
					\choice  Homogénea \CorrectChoice Heterogénea
				\end{choices}
			\end{parts}
		\end{multicols}
	}

	\questionboxed[5]{Relaciona los métodos que se utilizaría para separar las siguientes mezclas.

		\begin{multicols}{2}
			\begin{parts}\raggedleft
				\part Tinta negra                          \fillin[B][0.15in]
				\part Agua con sal                         \fillin[D][0.15in]
				\part Azufre en polvo y limadura de hierro \fillin[E][0.15in]
				\part Sal fina y pedazos de roca           \fillin[C][0.15in]
				\part Petróleo                             \fillin[A][0.15in]
			\end{parts}
			\columnbreak

			\begin{choices}
				\choice Destilación
				\choice Cromatografía
				\choice Tamizado
				\choice Evaporación
				\choice Magnetización
			\end{choices}
		\end{multicols}
	}

	% \addcontentsline{toc}{section}{L8 Concentración de mezclas}
	\section{L8 Concentración de mezclas}

	\questionboxed[5]{Elige la respuesta correcta.

		\begin{multicols}{2}
			\begin{parts}
				\part  ¿Cómo se determina la concentración de una disolución?

				\begin{choices}
					\choice $\text{Concentración} = \dfrac{\text{Masa de disolvente}}{\text{Volumen de soluto}}$        \\
					\choice $\text{Concentración} = \dfrac{\text{Volumen de soluto}}{\text{Masa de disolvente}}$        \\
					\CorrectChoice $\text{Concentración} = \dfrac{\text{Masa de soluto}}{\text{Volumen de disolvente}}$ \\
					\choice $\text{Concentración} = \dfrac{\text{Volumen de disolvente}}{\text{Masa de soluto}}$        \\
				\end{choices}

				\part ¿De qué manera es posible cambiar las propiedades de una mezcla?

				\begin{choices}
					\choice Manteniendo las proporciones de sus solutos.
					\CorrectChoice Modificando las proporciones de sus componentes.
					\choice Modificando todos sus componentes.
					\choice Manteniendo todos sus componentes.
				\end{choices}

				\part  ¿Qué es una disolución?

				\begin{choices}
					\choice Una mezcla heterogénea de dos o más sustancias distintas.
					\choice Una mezcla heterogénea de dos o más sustancias idénticas.
					\CorrectChoice Una mezcla homogénea de dos o más sustancias distintas.
					\choice Una mezcla homogénea de dos o más sustancias idénticas.
				\end{choices}



				\part ¿Qué concentración tiene una disolución de hidróxido de sodio preparada con 0.20 g de esta sustancia y 25 mL de disolvente?

				\begin{oneparchoices}
					\choice 0.080 g/mL
					\choice 1.250 g/mL\\
					\CorrectChoice 0.008 g/mL
					\choice 125.0 g/mL
				\end{oneparchoices}


				\part  ¿Con qué masa se prepararon 1 000 mL de una disolución de ácido acético a una concentración de 0.75 g/mL?

				\begin{oneparchoices}
					\choice 133.3 g
					\choice 7.500 g\\
					\choice  13.33 g
					\CorrectChoice 750.0 g
				\end{oneparchoices}
			\end{parts}
		\end{multicols}
	}


	% \addcontentsline{toc}{section}{L9 Concentracion de contaminantes del medio ambiente}
	\section{L9 Concentracion de contaminantes del medio ambiente}

	\questionboxed[5]{Selecciona si las afirmaciones son verdaderas o falsas.

		\begin{parts}
			\part Un conductímetro permite identificar contaminantes en el suelo y agua con base en la conductividad térmica de las sustancias.

			\begin{oneparchoices}
				\choice Verdadero \CorrectChoice Falso
			\end{oneparchoices}

			\part La cromatografía líquida de alta eficacia funciona únicamente para separar sustancias nocivas de ríos y lagos.

			\begin{oneparchoices}
				\choice Verdadero \CorrectChoice Falso
			\end{oneparchoices}

			\part Una fase del tratamiento de aguas residuales consiste en pasar los contaminantes sólidos por un filtro para separarlos del líquido.

			\begin{oneparchoices}
				\choice Verdadero \CorrectChoice Falso
			\end{oneparchoices}

			\part Las estaciones de monitoreo detectan y determinan la concentración de partículas suspendidas en la atmósfera.

			\begin{oneparchoices}
				\CorrectChoice Verdadero \choice Falso
			\end{oneparchoices}
		\end{parts}
	}
	\questionboxed[5]{Selecciona las respuestas correctas a cada pregunta.

		\begin{multicols}{2}
			\begin{parts}
				\part ¿Cuáles son los principales contaminantes del aire?

				\begin{oneparcheckboxes}
					\choice Residuos de cobre
					\choice Vapor de agua \\
					\CorrectChoice Dióxido de azufre
					\choice Residuos de plomo
					\CorrectChoice Monóxido de carbono
				\end{oneparcheckboxes}


				\part ¿Qué factores disminuyen la cantidad de oxígeno en el agua?

				\begin{oneparcheckboxes}
					\choice El consumo doméstico
					\choice El sobrepastoreo
					\CorrectChoice La presión atmosférica\\
					\CorrectChoice La actividad humana\\
					\choice El exceso de nutrientes
				\end{oneparcheckboxes}


				\part  ¿Qué sustancias son capaces de modificar la toxicidad del agua y suelos?

				\begin{oneparcheckboxes}
					\CorrectChoice Residuos de cobre
					\choice Vapor de agua \\
					\choice Dióxido de azufre
					\choice Dióxido de carbono
					\CorrectChoice Residuos de plomo
				\end{oneparcheckboxes}


				\part ¿Cuáles son las principales causas de degradación del suelo en México?

				\begin{oneparcheckboxes}
					\choice El consumo doméstico
					\CorrectChoice El sobrepastoreo
					\choice La presión atmosférica\\
					\CorrectChoice La actividad  humana\\
					\choice El exceso de nutrientes
				\end{oneparcheckboxes}
			\end{parts}
		\end{multicols}
	}

	% \addcontentsline{toc}{section}{L10 Habitos de consumo y su impacto}
	\section{L10 Habitos de consumo y su impacto}

	\questionboxed[5]{Selecciona si las afirmaciones son verdaderas o falsas.

		\begin{multicols}{2}
			\begin{parts}
				\part Debido al consumismo se acumulan bienes y servicios no esenciales.

				\begin{oneparchoices}
					\CorrectChoice Verdadero \choice Falso
				\end{oneparchoices}

				\part A pesar de que se consume un exceso de recursos naturales el impacto del consumismo en la generación de residuos es mínimo.

				\begin{oneparchoices}
					\choice Verdadero \CorrectChoice Falso
				\end{oneparchoices}

				\part Los consumidores responsables saben de las consecuencias del consumo a nivel ambiental, social y económico.

				\begin{oneparchoices}
					\CorrectChoice Verdadero \choice Falso
				\end{oneparchoices}

				\part Ser un consumidor responsable implica respetar a la naturaleza.

				\begin{oneparchoices}
					\CorrectChoice Verdadero \choice Falso
				\end{oneparchoices}

				\part Se estima que 2/3 de la comida en el mundo se pudre por no ser consumida.

				\begin{oneparchoices}
					\choice Verdadero \CorrectChoice Falso
				\end{oneparchoices}

				\part Según la ONU, con 25\% de la comida que se desperdicia se podría alimentar a 870 millones de personas con hambre.

				\begin{oneparchoices}
					\CorrectChoice Verdadero \choice Falso
				\end{oneparchoices}

				\part La mayoría de la energía mundial la consumen las personas en sus hogares.

				\begin{oneparchoices}
					\choice Verdadero \CorrectChoice Falso
				\end{oneparchoices}

				\part Cada año se destruye1 millón de hectáreas de bosques por el consumo excesivo de los recursos naturales.

				\begin{oneparchoices}
					\choice Verdadero \CorrectChoice Falso
				\end{oneparchoices}

				\part El consumo responsable solamente implica el realizar grandes acciones como protestas.

				\begin{oneparchoices}
					\choice Verdadero \CorrectChoice Falso
				\end{oneparchoices}

				\part Para reducir el impacto del consumo de productos es importante el informarse para poder optar por opciones sostenibles.

				\begin{oneparchoices}
					\CorrectChoice Verdadero \choice Falso
				\end{oneparchoices}

				\part Seguir el punto 12 de los ODS de la ONU nos ayuda a garantizar formas de consumo y producción sostenibles.

				\begin{oneparchoices}
					\CorrectChoice Verdadero \choice Falso
				\end{oneparchoices}

				\part Reducir la cantidad de desechos que producimos es parte del consumo responsable.

				\begin{oneparchoices}
					\CorrectChoice Verdadero \choice Falso
				\end{oneparchoices}

			\end{parts}
		\end{multicols}

	}
\end{questions}
\end{document}