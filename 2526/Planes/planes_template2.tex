\documentclass[11pt]{article}
\usepackage[utf8]{inputenc}
\usepackage[T1]{fontenc}
\usepackage[spanish]{babel}
\usepackage[margin=0.5in, landscape]{geometry}
\usepackage{graphicx}
\usepackage{multicol}
\usepackage{array}
\usepackage{tabularx}
\usepackage{hhline}
\usepackage{longtable}
\usepackage{lastpage}
\usepackage{calc}
\usepackage{fancybox}
\usepackage{tikz}

% Deshabilitar la numeración de páginas y la cabecera/pie de página
\pagestyle{empty}

% Definir el estilo de las líneas
\renewcommand{\arraystretch}{1}
\setlength{\extrarowheight}{0pt}

% Definir estilos de TikZ para las cajas
\tikzset{
    titlebox/.style={
        rectangle,
        rounded corners=5pt,
        draw=blue!50,
        fill=blue!10,
        line width=1pt,
        inner sep=5pt,
        minimum width=1.0\textwidth
    },
    daybox/.style={
        rectangle,
        rounded corners=5pt,
        draw=blue!30,
        fill=blue!5,
        line width=0.5pt,
        inner sep=5pt,
        text width=0.95\textwidth
    }
}

\begin{document}

\small % Usar una fuente más pequeña para que quepa todo

\begin{tabularx}{\textwidth}{p{0.25\textwidth}|X|p{0.25\textwidth}}
                \hhline{|-|-|-|}
    \multicolumn{1}{p{0.25\textwidth}}{
        \includegraphics[width=0.2\textwidth]{logo.png} \\
        \textbf{Escuela Rafael Díaz Serdán} \\
        30PES0329R turno matutino \\
        \textbf{Planeación didáctica} \\
        Educación para la vida Ciclo escolar 2024-2025 } &
    \multicolumn{2}{p{0.75\textwidth}}{
        \begin{tabularx}{\linewidth}{|l|X|l|X|l|X|}
            \hhline
            \textbf{Periodo:} & \underline{\hspace{\fill}} & \textbf{Grado y grupo:} & \underline{\hspace{\fill}} & \textbf{Sesiones:} & \underline{\hspace{\fill}} \\
            \hhline{|-|-|-|-|-|-|}
            \textbf{Profesor:} & \underline{\hspace{\fill}} & \textbf{Campo formativo:} & \underline{\hspace{\fill}} & \multicolumn{2}{c|}{} \\
            \hhline{|-|-|----|}
            \textbf{Disciplina:} & \underline{\hspace{\fill}} & \textbf{Lección:} & \underline{\hspace{\fill}} & \multicolumn{2}{c|}{} \\
            \hhline{|-|-|----|}
            \textbf{Tema:} & \underline{\hspace{\fill}} & \textbf{Contenido:} & \underline{\hspace{\fill}} & \multicolumn{2}{c|}{} \\
            \hhline{|-|-|----|}
            \textbf{Ejes articuladores:} & \underline{\hspace{\fill}} & \multicolumn{4}{c|}{} \\
            \hhline{|---|---|}
        \end{tabularx}
    }  \\
                \hhline{|-|-|-|}
\end{tabularx}

\vspace{0.2cm}

\begin{tabularx}{\textwidth}{|p{0.6\textwidth}|X|}
    \hline
    \textbf{Contenido programático:}
    \begin{tabularx}{\textwidth}{lX}
        \textbf{INICIO:} & \underline{\hspace{\fill}} \\
        \textbf{DESARROLLO:} & \underline{\hspace{\fill}} \\
        \textbf{CIERRE:} & \underline{\hspace{\fill}} 
    \end{tabularx}
    
    \vspace{0.2cm}
    \hrule
    \vspace{0.2cm}
    
    \textbf{Proceso de desarrollo de aprendizaje (PDA):} \underline{\hspace{\fill}}
    \vspace{1cm}
    
    \begin{tabularx}{\textwidth}{|p{0.45\textwidth}|X|}
        \hline
        \textbf{Instrumento y estrategia de evaluación:} & \\
        \underline{\hspace{\fill}} & \\
        \hhline{|-|-|}
        \multicolumn{2}{|c|}{\textbf{Nombre y firma del docente \hspace{2cm} Nombre y firma de quien autoriza este documento}} \\
        \hline
    \end{tabularx}
    &
    \multicolumn{1}{|p{0.4\textwidth}|}{
    \textbf{Observaciones:}
    \begin{tabularx}{\textwidth}{|X|}
        \hline
        \begin{tikzpicture}
            \node[daybox, anchor=west] {\textbf{Lunes:}\newline\underline{\hspace{\fill}}\newline\underline{\hspace{\fill}}};
        \end{tikzpicture} \\
        \hline
        \begin{tikzpicture}
            \node[daybox, anchor=west] {\textbf{Martes:}\newline\underline{\hspace{\fill}}\newline\underline{\hspace{\fill}}};
        \end{tikzpicture} \\
        \hline
        \begin{tikzpicture}
            \node[daybox, anchor=west] {\textbf{Miércoles:}\newline\underline{\hspace{\fill}}\newline\underline{\hspace{\fill}}};
        \end{tikzpicture} \\
        \hline
        \begin{tikzpicture}
            \node[daybox, anchor=west] {\textbf{Jueves:}\newline\underline{\hspace{\fill}}\newline\underline{\hspace{\fill}}};
        \end{tikzpicture} \\
        \hline
        \begin{tikzpicture}
            \node[daybox, anchor=west] {\textbf{Viernes:}\newline\underline{\hspace{\fill}}\newline\underline{\hspace{\fill}}};
        \end{tikzpicture} \\
        \hline
    \end{tabularx}
    } \\
    \hhline{|-|-|}
\end{tabularx}

\end{document}
