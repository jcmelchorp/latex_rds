\documentclass{article}
\usepackage[spanish]{babel}
\usepackage[T1]{fontenc}
\usepackage[many]{tcolorbox}
\usepackage{tabularx}
\usepackage[landscape, margin=0.5in]{geometry}
\usepackage{ragged2e}
\usepackage{multicol}
\usepackage{jsonparse} % Load the json package
\usepackage{anyfontsize}
\usepackage{adjustbox}
\usepackage{graphicx}


% Configuración de tcolorbox para los títulos y bloques
\tcbset{
    titlebox/.style={
        colbacktitle=white,
        colback=blue!0.5!white,
        colframe=black,
        coltitle=black,
        boxrule=0.4pt,
        arc=2mm,
        boxsep=0mm,
        left=10pt,
        right=10pt,
        top=10pt,
        bottom=10pt,
        subtitle style={
            fontupper=\sffamily\large,
            fontlower=\huge,
            boxrule=0.4pt,
            boxsep=1mm,
            top=1mm,
            bottom=1mm,
            colback=blue!10,
            colupper=green!30!black,
        }
    },
    infobox/.style={
        coltitle=green!30!black,
        colframe=black!100,
        colback=blue!1!white,
        colbacktitle=blue!10,
        boxrule=0.4pt,
        arc=2mm,
        boxsep=1mm,
        frame style={black!100},
        left=2mm,
        right=2mm,
        top=2mm,
        bottom=2mm,
    },
    contentbox/.style={
        colframe=gray!40!white,
        colback=white,
        colbacktitle=gray!30!white,
        boxrule=1pt,
        arc=0mm,
        boxsep=1mm,
        frame style={gray!10!white},
        left=2mm,
        right=2mm,
        top=0mm,
        bottom=2mm,
        title code={
            %\tcbifempty{\tcbtitle}{}{\textbf{\tcbtitle\quad}}
        }
    }
}

\def\LOGO{%
    \begin{picture}(0,0)\unitlength=2cm
        \put (0.045,-0.47) {\includegraphics[width=1.55cm]{logo_shadow.png}}
        \put (0.04,-0.46) {\includegraphics[width=1.55cm]{logo.png}}
    \end{picture}
}

\pagestyle{empty}
% Formato para la tabla
\newcolumntype{Y}{>{\RaggedRight\arraybackslash}X}
\renewcommand{\arraystretch}{1}
\setlength{\columnsep}{2pt}
\setlength{\arrayrulewidth}{0.1pt}

\newcommand*\planTitle[0]{\small%
\begin{minipage}{0.065\textwidth}
\LOGO
\end{minipage}%
\begin{minipage}{0.3\textwidth}\centering
{\color{blue!40!black}\fontsize{17}{12}\selectfont Escuela Rafael Díaz Serdán}\\[-0.5em]
{\color{gray!40!black}\fontsize{6}{6}\selectfont \sffamily \hspace{15pt} 30PES0329R \hfill turno matutino \hspace{15pt}}\\[0.5em]
{\color{green!20!black} \fontsize{12}{12}\selectfont \sffamily Planeación didáctica semanal}\\
%{\color{red!40!black}\fontsize{8}{8}\selectfont Ciclo escolar {\sffamily\bfseries  2024-2025}}\\
{\color{black!80}Profesor: Julio César Melchor Pinto}
\end{minipage}%
% Bloque de información general
    \begin{tabularx}{0.6\textwidth}{rYrl}
        {\sffamily Período de realización:} &  del {01/01/01}   al { 01/01/01} & {\sffamily Trimestre:} &  { Unidad 1} \\
        {\sffamily Campo formativo:} & { Saberes y Pensamiento Científico} & {\sffamily Disciplina:} &  { Matemáticas 1} \\
        {\sffamily Tema:}  & { Sucesiones Aritméticas y Geométricas} & {\sffamily Grado y grupo:} &  { 3 de Secundaria} \\
        {\sffamily Lección:} & { Término general de una serie} & {\sffamily Ciclo escolar:} & { 2025-2026} \\
        {\sffamily Ejes articuladores:} & \multicolumn{3}{l}{ Inclusión, Vida saludable, Pensamiento crítico} \\
    \end{tabularx}
}

\begin{document}
% Caja principal
\begin{tcolorbox}[titlebox,title=\RaggedLeft\planTitle]
    % Bloque de contenido programático
    {\sffamily \Large Contenido programático}
    \tcbsubtitle{INICIO}%\centering
    %\resizebox{0.95\linewidth}{!}{
    %  \adjustbox{minipage=[t][3.2cm][t]{\linewidth}}{
    \tcboxfit[colback=white,colframe=white,top=0mm,bottom=0mm,left=0mm,right=0mm,height=2.5cm]{
        La actividad de esta sección, además de tener un enfoque ambiental del que se puede obtener un provecho en la formación de valores en los estudiantes, es muy útil en cuanto a dar significado y representar en un contexto cercano el concepto de separación física de una mezcla. Un contenedor de basura es un ejemplo de una mezcla heterogénea de sustancias: plásticos, metales, materia orgánica, entre otros. Un proceso mecánico de separación de mezclas es la clasificación y separación de residuos del hogar en función de las características de cada residuo. Se recomienda que los alumnos midan el volumen de los residuos generados sin clasificar y lo contrasten contra el volumen generado al clasificar y separar la basura, para hacer una reflexión sobre el impacto en el volumen que se destina a la basura en rellenos sanitarios.
    }
    \tcbsubtitle{DESARROLLO}%\centering
    %\resizebox{0.95\linewidth}{!}{
    %   \adjustbox{minipage=[b][3.2cm][t]{\linewidth}}{
    \tcboxfit[colback=white,colframe=white,top=0mm,bottom=0mm,left=0mm,right=0mm,height=4cm]{
        La parte fundamental de esta lección es que los educandos sean capaces de identificar las propiedades de una mezcla de sustancias (estado de agregación, densidad, punto de fusión, tamaño de partícula, entre otras) así como las propiedades de cada sustancia por separado para poder identificar y aplicar el método más adecuado de separación. Es importante resaltar el estado de agregación de las mezclas de las sustancias y también indicar para qué tipo de mezclas es adecuado cada método de separación. En la actividad experimental (páginas 51 y 52), mezcla de agua, arena y grava, se sugiere preguntar previamente a los estudiantes ¿qué es más rápido y sencillo de separar, el agua de la arena o el agua de la grava? Existen múltiples formas de separar agua y grava; decantación, filtración con una coladera (tamiz grande) e incluso de forma mecánica. Después invítelos a reflexionar cómo separarían la arena del agua; también existen múltiples formas: decantación, filtración y destilación. Se debe hacer énfasis en “qué” se desea recuperar y “para qué”; por ejemplo, si se desea recuperar el agua, resulta más adecuado la destilación, porque se recuperaría en un contendor. Si lo que se desea recuperar es la arena, entonces el proceso de filtración resulta más adecuado, porque no requiere una inversión energética (proceso de calentamiento). El procedimiento de separación cromatográfico de la segunda actividad experimental (página 54) es muy útil y suele generar mucha atracción en el alumnado; sin embargo, se debe tener especial cuidado en la elección de los plumones, ya que deben ser solubles en agua. Además, de las cuatro marcas que se hacen en la hoja, una de estasdebe ser forzosamente del plumón elegido por el docente (plumóndel crimen), y se debe hacer la siguiente apreciación: si se desea queun equipo consiga identificar al plumón del crimen, entonces se debecolocar en otro espacio de nuevo el mismo plumón, dejando libres losotros dos espacios para utilizar plumones distintos, por otra parte, sepueden asignar a un equipo tres marcas de plumones que no seanel “plumón del crimen”. Una última recomendación es que los plumones deben ser de distintos proveedores y de colores semejantes, delo contrario, no tendrá sentido si el “plumón del crimen” es azul y secoloca una marca amarilla.
    }%}
    \tcbsubtitle{CIERRE}%\centering
    % \resizebox{0.95\linewidth}{!}{
    %  \adjustbox{minipage=[t][3.2cm][t]{\linewidth}}{
    \tcboxfit[colback=white,colframe=white,top=0mm,bottom=0mm,left=0mm,right=0mm,height=3cm]{
        La actividad de esta etapa pertenece a la denominada “enseñanza ambiciosa de la ciencia”, un método de enseñanza novedoso que permite la generación de conocimiento a partir de la experiencia propia de los educandos. Se le recomienda evaluar la creatividad y la participación por encima del resultado. Es aconsejable hacer una pequeña introducción sobre el acceso de la población al agua potable. Se pueden retomar los resultados obtenidos de la actividad en la cual se separó la mezcla de agua, grava y arena, y añadir sal a la mezcla. Otro aspecto importante que puede resaltarse para evaluar la comprensión de los estudiantes es que expliquen qué consideran como agua purificada o la forma en que pueden garantizar que la mezcla de agua ha sido purificada o no; sin embargo, estas reflexiones deben quedar en un análisis teórico, de indagación, y no formar parte de la evaluación ni del procedimiento experimental. Complemente la etapa de “Cierre” con la ficha “Aguas brillantes y cristalinas” del cuaderno de evidencias. Pida a los estudiantes que la resuelvan e invítelos a reflexionar acerca del uso de los métodos de separación de mezclas para obtener sustancias puras y seguras. Permítales que compartan sus experiencias y aprendizajes de estas actividades.
    }%}
\end{tcolorbox}\vfill
\begin{multicols}{3}\centering
    % Bloque de Proceso de desarrollo de aprendizaje (PDA)
    \begin{tcolorbox}[infobox, title={\sffamily \large Proceso de desarrollo de aprendizaje (PDA):}]
        %\vspace{4em} % Espacio en blanco para el PDA
    \tcboxfit[colback=white,colframe=white,top=0mm,bottom=0mm,left=0mm,right=0mm,height=1.5cm, fit basedim=11pt]{
            Calcular la diferencia común y determinar términos específicos en una sucesión aritmética.
        }
    \end{tcolorbox}%
    % Bloque de firmas
    \begin{tabularx}{0.95\linewidth}{|c|c|}
        \multicolumn{1}{p{0.45\linewidth}}{\centering \sffamily Elabora:}                 & \multicolumn{1}{p{0.45\linewidth}}{\centering \sffamily Autoriza:}                \\
        \hline
        \multicolumn{1}{Y|}{\vspace{3em}\centering \footnotesize \sffamily Nombre y firma} & \multicolumn{1}{Y}{\vspace{3em}\centering \footnotesize \sffamily Nombre y firma} \\
    \end{tabularx}
    % Bloque de Instrumento y estrategia de evaluación
    \begin{tcolorbox}[infobox, title={\sffamily \large Instrumento y estrategia de evaluación:}]
        %\vspace{4em} % Espacio en blanco para la evaluación
    \tcboxfit[colback=white,colframe=white,top=0mm,bottom=0mm,left=0mm,right=0mm,height=1.5cm, fit basedim=11pt]{
            Resolución de problemas en MeXmáticas, elaboración de diagramas con explicaciones detalladas.
        }
    \end{tcolorbox}%
\end{multicols}

\end{document}