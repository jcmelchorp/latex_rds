Cálculos numéricos
A.1.  Suma de números
A.2.  Resta de números
A.3.  Multiplicación de números
A.4.  División de números
A.5.  Resolución de problemas
Números negativos
A.6.  Ubicación en la recta numérica
A.7.  Comparación de negativos
A.8.  Suma y resta con negativos
A.9.  Multiplicación y división con negativos
A.10. Potencias con números negativos
Exponentes y notación científica
A.11. Suma de exponentes
A.12. Resta de exponentes
A.13. Multiplicación de exponentes
A.14. Notación científica 1
A.15. Notación científica 2
Plano cartesiano y la recta
A.16. Ubicación en el plano cartesiano
A.17. Cuadrantes en el plano cartesiano
A.18. Pendiente de una recta
A.19. Pendiente y ordenada
A.20. Ecuación de una recta
Porcentajes
A.21. Porcentajes a decimal
A.22. Decimal a porcentaje
A.23. Porcentaje de cantidades 1
A.24. Porcentaje de cantidades 2
A.25. Resolución de problemas

Círculo
A.26. Diámetro de un círculo
A.27. Radio de un círculo
A.28. Perímetro
A.29. Área
A.30. Resolución de problemas
Polígonos y circunferencias
A.31. Ángulos interiores
A.32. Ángulos centrales y exteriores
A.33. Ángulos centrales e inscritos
A.34. Arco de una circunferencia
A.35. Área de un sector circular
Figuras y cuerpos geométricos
A.36. Perímetro
A.37. Área
A.38. Área lateral y total
A.39. Volumen
A.40. Resolución de problemas
Monomios y polinomios
A.41. Lenguaje algebraico
A.42. Suma de monomios y polinomios
A.43. Resta de monomios y polinomios
A.44. Operaciones combinadas
A.45. Perímetro de figuras geométricas
Operaciones con monomios y polinomios
A.46. Suma y resta de exponentes
A.47. Multiplicación de exponentes
A.48. Multiplicación y división de monomios
A.49. Multiplicación de polinomios
A.50. Áreas de figuras geométricas
Sistema de unidades
A.51. Unidades de longitud
A.52. Unidades de masa
A.53. Unidades de capacidad
A.54. Unidades de área y volumen
A.55. Unidades de capacidad 2

Probabilidad y estadística
A.56. Mediana y moda
A.57. Promedio
A.58. Interpretación de gráficas
A.59. Eventos mutuamente excluyentes
A.60. Eventos dependientes e independientes
Razones y proporciones
A.61. Relaciones proporcionales
A.62. Constante de proporcionalidad
A.63. Regla de correspondencia
A.64. Proporción directa e inversa
A.65. Proporciones compuestas
Sucesiones aritméticas
A.66. Completando la sucesión
A.67. Diferencia de una sucesión
A.68. Término enésimo 1
A.69. Término general
A.70. Término enésimo 2
Ecuaciones lineales
A.71. Lenguaje algebraico
A.72. Sustitución de valores
A.73. Ecuaciones de primer grado 1
A.74. Ecuaciones de primer grado 2
A.75. Resolución de problemas
Sistemas de ecuaciones
A.76. Método de eliminación
A.77. Método de sustitución
A.78. Método de igualación
A.79. Sistema de ecuaciones 2x2
A.80. Sistema de ecuaciones 3x3