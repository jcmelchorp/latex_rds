Me preparo
1.	Aportaciones de diversas culturas en la satisfacción de necesidades
AI.1	Química en la vida
AI.2	Química en los alimentos
P. Materiales Prehispanicas
2.	Química y sociedad
AI.3	Química y sociedad
F1	Dosis que te hacen fuerte
P. Conocimiento Cientifico
3.	Propiedades de los materiales
AI.4	Materiales y sus propiedades
AI.5	Propiedades físicas cuantitativas
AI.6	Procesos y propiedades
AI.7	Estado de agregación de diferentes materiales
P. Materiales
4.	Medición e identificación de sustancias
AI.8	Propiedades extensivas e intensivas
AI.9	Medición e identificación de sustancias
F2	Propiedades que te mantienen saludable
P. Propiedades Intensivas
5.	Propiedades de las sustancias e intercambio de energía
AI.10	Energía y metales
AI.11	Interacciones térmicas
AI.12	Estados de agregación y materiales
AI.13	Estados de agregación
6.	Mezclas
AI.14	Mezclas y estados de agregación
AI.15	¿Cuál es la concentración?
7.	Métodos de separación de mezclas
AI.16	¿Cómo separar una mezcla heterogénea?
AI.17	Hasta que nos separen
AI.18	Distingue los métodos de separación
AI.19	Separación de mezclas
AI.20	¿Homogénea o heterogénea?
F3	Aguas brillantes y cristalinas
P. Mezclas
8.	Concentración en mezclas
AI.21	Disoluciones y cantidades
AI.22	Concentración en mezclas
F4	¿Y adónde va la basura que tiro?
P. Delincuente
9.	Concentración de contaminantes en el medio ambiente
AI.23	Sustancias contaminantes
F5	Agua contaminada que corre por aquí y por allá.
P. Toxicidad
10.	Hábitos de consumo y su impacto
AI.24	¿Cómo mejorar mis hábitos?
F6	El costo de estar a la moda
P. Reutilizar Agua
¿Qué aprendí?
Examen de Unidad 1
Calificaciones de la Unidad 1
Me preparo
1.	Sustancias elementales y compuestos
AI.1	Mezclas y sustancias puras
AI.2	¿Elemento, compuesto o mezcla?
AI.3	Elementos y compuestos
F7	Aire limpio, planeta sano
2.	Modelo corpuscular
AI.4	¿Elemento, compuesto o mezcla? Parte 2
3.	Elementos en todos lados
AI.5	Elementos y su abundancia
AI.6	Modelo corpuscular
4.	Modelos atómicos
AI.7	Estructura atómica
AI.8	Cantidad de la materia
P. Atomos
5.	Tabla periódica
AI.9	¿De qué elemento se trata?
AI.10	Identificación atómica
F8	La calidad del suelo, sostén para la vida
6.	Propiedades periódicas
AI.11	¿Cuántos electrones de valencia tiene?
AI.12	Tabla periódica y propiedades de los elementos
P. Periodica
7.	Enlaces químicos
AI.13	Protones y electrones de valencia
AI.14	Representaciones de las sustancias
8.	Sustancias moleculares y compuestos iónicos
AI.15	¿Cuál catión y anión forman el compuesto?
AI.16	¿Qué tipo de enlace tiene?
AI.17	Química de disoluciones acuosas
P. Disociacion
9.	Aprovechamiento de compuestos iónicos y moleculares
AI.18	Química de los plásticos
AI.19	¿Natural o sintético?
F9	Un poco de sal para preparar agua dulce
P. Polimero
10.	Agua, vitaminas y minerales
AI.20	¿Qué elementos son más abundantes en nuestro organismo?
AI.21	Agua, vitaminas y minerales
F10	Vitaminas y minerales para todos
P. Vitamina C
11.	Reacciones químicas
AI.22	Cambio físico y cambio químico
AI.23	Cambios en los procesos químicos
AI.24	Cambios químicos y cómo identificarlos
P. Cambio Quimico
12.	Conservación de la materia
AI.25	Conservación de la materia en cambios químicos
AI.26	Ecuaciones y reacciones
AI.27	La misma cantidad de un lado que del otro
AI.28	Ecuación química
AI.29	Érase una vez que
AI.30	Cantidad de la materia en cambios químicos
F11	La química de la vida
P. Hierro
13.	Reacciones químicas que solucionan problemas
F12	¿Qué tanto color dan las reacciones químicas?
P. Cambio
14.	Reacciones endotérmicas y exotérmicas
AI.31	Intercambios de energía en procesos
AI.32	Transferencia de energía en cambios químicos
AI.33	¿Qué tipo de reacción es?
AI.34	Energía química en tu entorno
AI.35	Costos y riesgos de la energía química
¿Qué aprendí?
Examen de Unidad 2
Calificaciones de la Unidad 2
Me preparo
1.	Diversidad cultural de los alimentos
AI.1	Conoce tus alimentos
P. Descomposicion
2.	Nutrimentos como fuentes de masa y energía
AI.2	Nutrimentos como fuentes de masa y energía
AI.3	Biomoléculas
F13	El mole, un festín de sabor
P. Proteinas
3.	Aporte energético de los alimentos
AI.4	¿Cuánta energía comes?
AI.5	Necesidades energéticas de cada individuo
F14	Los extras del pozolito
4.	Sustancias ácidas y básicas
AI.6	¿Ácido o base?
AI.7	¿Cuál es el pH?
P. Acidos
5.	Ácidos y bases en agua
AI.8	El modelo de Arrhenius
AI.9	Fórmulas, pH y acidez
AI.10	Los modelos científicos evolucionan
F15	Lo que disuelve la lluvia ácida
6.	Reacciones de neutralización
AI.11	Productos en la vida cotidiana
AI.12	Neutralizar para ayudar
F16	Hay vida en el mar
P. Alimentos Acidos
7.	Beneficios y riesgos de ácidos y bases
AI.13	Beneficios y riesgos de ácidos y bases
AI.14	¿Tus alimentos son ácidos?
AI.15	Cuida el medio ambiente
P. Arrehenius
8.	Reacciones de óxido-reducción
AI.16	¿Oxida o reduce?
AI.17	Reacciones de óxido-reducción
9.	Identificación y uso de reacciones redox
AI.18	La química de la fotosíntesis y de la respiración
AI.19	identificación y uso de reacciones redox
AI.20	Energía química y sus alternativas
F17	Cómo proteger el mundo de la corrosión
P. Corrosion
10.	Reacciones redox y el desarrollo sustentable
AI.21	Las redox en el medio ambiente
AI.22	Control de la rapidez de una reacción
F18	Alternativas de energía con redox
P. Reduccion
11.	Factores que influyen en la velocidad de una reacción
AI.23	¿Cómo modifico la rapidez de una reacción química?
AI.24	Factores de rapidez en las reacciones
AI.25	Energía y rapidez de las reacciones
AI.26	Catalizadores en las reacciones químicas
AI.27	Rapidez de reacción y conservación de alimentos
F19	Cómo acelerar un consumo responsable
12.	Beneficios de modificar la rapidez de una reacción química
AI.28	Rapidez de reacción y conservación
AI.29	La química en la medicina
AI.30	Implicaciones de la Química en la vida
F20	Acelera para contaminar menos
P. Velocidad Reaccion
¿Qué aprendí?
Examen de Unidad 3
Calificaciones de la Unidad 3