Me preparo
L.1. Aportaciones de diversas culturas en la satisfacción de necesidades
A.1 Química en la vida
A.2 Química en los alimentos
P. Materiales Prehispanicas
L.2. Química y sociedad
A.3 Química y sociedad
F.1 Dosis que te hacen fuerte
P. Conocimiento Cientifico
L.3. Propiedades de los materiales
A.4 Materiales y sus propiedades
A.5 Propiedades físicas cuantitativas
A.6 Procesos y propiedades
A.7 Estado de agregación de diferentes materiales
P. Materiales
L.4. Medición e identificación de sustancias
A.8 Propiedades extensivas e intensivas
A.9 Medición e identificación de sustancias
F.2 Propiedades que te mantienen saludable
P. Propiedades Intensivas
L.5. Propiedades de las sustancias e intercambio de energía
A.10 Energía y metales
A.11 Interacciones térmicas
A.12 Estados de agregación y materiales
A.13 Estados de agregación
L.6. Mezclas
A.14 Mezclas y estados de agregación
A.15 ¿Cuál es la concentración?
L.7. Métodos de separación de mezclas
A.16 ¿Cómo separar una mezcla heterogénea?
A.17 Hasta que nos separen
A.18 Distingue los métodos de separación
A.19 Separación de mezclas
A.20 ¿Homogénea o heterogénea?
F.3 Aguas brillantes y cristalinas
P. Mezclas
L.8. Concentración en mezclas
A.21 Disoluciones y cantidades
A.22 Concentración en mezclas
F.4 ¿Y adónde va la basura que tiro?
P. Delincuente
L.9. Concentración de contaminantes en el medio ambiente
A.23 Sustancias contaminantes
F.5 Agua contaminada que corre por aquí y por allá.
P. Toxicidad
L.10. Hábitos de consumo y su impacto
A.24 ¿Cómo mejorar mis hábitos?
F.6 El costo de estar a la moda
P. Reutilizar Agua
¿Qué aprendí?
Examen de Unidad 1
Calificaciones de la Unidad 1
Me preparo
L.1. Sustancias elementales y compuestos
A.1 Mezclas y sustancias puras
A.2 ¿Elemento, compuesto o mezcla?
A.3 Elementos y compuestos
F.7 Aire limpio, planeta sano
L.2. Modelo corpuscular
A.4 ¿Elemento, compuesto o mezcla? Parte 2
L.3. Elementos en todos lados
A.5 Elementos y su abundancia
A.6 Modelo corpuscular
L.4. Modelos atómicos
A.7 Estructura atómica
A.8 Cantidad de la materia
P. Atomos
L.5. Tabla periódica
A.9 ¿De qué elemento se trata?
A.10 Identificación atómica
F.8 La calidad del suelo, sostén para la vida
L.6. Propiedades periódicas
A.11 ¿Cuántos electrones de valencia tiene?
A.12 Tabla periódica y propiedades de los elementos
P. Periodica
L.7. Enlaces químicos
A.13 Protones y electrones de valencia
A.14 Representaciones de las sustancias
L.8. Sustancias moleculares y compuestos iónicos
A.15 ¿Cuál catión y anión forman el compuesto?
A.16 ¿Qué tipo de enlace tiene?
A.17 Química de disoluciones acuosas
P. Disociacion
L.9. Aprovechamiento de compuestos iónicos y moleculares
A.18 Química de los plásticos
A.19 ¿Natural o sintético?
F.9 Un poco de sal para preparar agua dulce
P. Polimero
L.10. Agua, vitaminas y minerales
A.20 ¿Qué elementos son más abundantes en nuestro organismo?
A.21 Agua, vitaminas y minerales
F.10 Vitaminas y minerales para todos
P. Vitamina C
L.11. Reacciones químicas
A.22 Cambio físico y cambio químico
A.23 Cambios en los procesos químicos
A.24 Cambios químicos y cómo identificarlos
P. Cambio Quimico
L.12. Conservación de la materia
A.25 Conservación de la materia en cambios químicos
A.26 Ecuaciones y reacciones
A.27 La misma cantidad de un lado que del otro
A.28 Ecuación química
A.29 Érase una vez que
A.30 Cantidad de la materia en cambios químicos
F.11 La química de la vida
P. Hierro
L.13. Reacciones químicas que solucionan problemas
F.12 ¿Qué tanto color dan las reacciones químicas?
P. Cambio
L.14. Reacciones endotérmicas y exotérmicas
A.31 Intercambios de energía en procesos
A.32 Transferencia de energía en cambios químicos
A.33 ¿Qué tipo de reacción es?
A.34 Energía química en tu entorno
A.35 Costos y riesgos de la energía química
¿Qué aprendí?
Examen de Unidad 2
Calificaciones de la Unidad 2
Me preparo
L.1. Diversidad cultural de los alimentos
A.1 Conoce tus alimentos
P. Descomposicion
L.2. Nutrimentos como fuentes de masa y energía
A.2 Nutrimentos como fuentes de masa y energía
A.3 Biomoléculas
F.13 El mole, un festín de sabor
P. Proteinas
L.3. Aporte energético de los alimentos
A.4 ¿Cuánta energía comes?
A.5 Necesidades energéticas de cada individuo
F.14 Los extras del pozolito
L.4. Sustancias ácidas y básicas
A.6 ¿Ácido o base?
A.7 ¿Cuál es el pH?
P. Acidos
L.5. Ácidos y bases en agua
A.8 El modelo de Arrhenius
A.9 Fórmulas, pH y acidez
A.10 Los modelos científicos evolucionan
F.15 Lo que disuelve la lluvia ácida
L.6. Reacciones de neutralización
A.11 Productos en la vida cotidiana
A.12 Neutralizar para ayudar
F.16 Hay vida en el mar
P. Alimentos Acidos
L.7. Beneficios y riesgos de ácidos y bases
A.13 Beneficios y riesgos de ácidos y bases
A.14 ¿Tus alimentos son ácidos?
A.15 Cuida el medio ambiente
P. Arrehenius
L.8. Reacciones de óxido-reducción
A.16 ¿Oxida o reduce?
A.17 Reacciones de óxido-reducción
L.9. Identificación y uso de reacciones redox
A.18 La química de la fotosíntesis y de la respiración
A.19 identificación y uso de reacciones redox
A.20 Energía química y sus alternativas
F.17 Cómo proteger el mundo de la corrosión
P. Corrosion
L.10. Reacciones redox y el desarrollo sustentable
A.21 Las redox en el medio ambiente
A.22 Control de la rapidez de una reacción
F.18 Alternativas de energía con redox
P. Reduccion
L.11. Factores que influyen en la velocidad de una reacción
A.23 ¿Cómo modifico la rapidez de una reacción química?
A.24 Factores de rapidez en las reacciones
A.25 Energía y rapidez de las reacciones
A.26 Catalizadores en las reacciones químicas
A.27 Rapidez de reacción y conservación de alimentos
F.19 Cómo acelerar un consumo responsable
L.12. Beneficios de modificar la rapidez de una reacción química
A.28 Rapidez de reacción y conservación
A.29 La química en la medicina
A.30 Implicaciones de la Química en la vida
F.20 Acelera para contaminar menos
P. Velocidad Reaccion
¿Qué aprendí?
Examen de Unidad 3
Calificaciones de la Unidad 3