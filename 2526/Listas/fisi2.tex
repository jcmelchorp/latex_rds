Me preparo
L.1. Conocimiento empírico
A.1. Conocimiento empírico
A.2. Tipos de conocimientos
A.3. Otras formas de conocimiento
L.2. El conocimiento científico
A.4. ¿Qué es la ciencia?
A.5. Método científico
F.1. Siete razones para dedicarse a la ciencia
L.3. Física y sociedad
A.6. ¿Qué es la Física?
A.7. División de la Física
F.2. Ciencia para todos
P. Tecnologia
L.4. Mediciones
A.8. Mediciones
P. Tiempo
L.5. Unidades fundamentales y derivadas de medida
A.9. Unidades fundamentales
L.6. Múltiplos y submúltiplos
A.10. Múltiplos
L.7. Instrumentos de medición
A.11. Es momento de medir
F.3. Contaminantes del aire
L.8. Materiales y sus propiedades
A.12. Propiedades de la materia
A.13. ¿Qué es la materia?
F.4. Litio, el oro blanco del siglo XXI
F.5. La isla de plástico
P. Densidad
L.9. Origen de las teorías sobre la estructura de la materia
A.14. Un modelo para la materia
L.10. La teoría atómica
A.15. Construyendo un modelo para el átomo
A.16. ¿Thomson, Rutherford o Bohr?
A.17. La estructura interna de la materia
F.6. Invierno nuclear
L.11. Estados de agregación de la materia y modelo cinético
A.18. ¿Cómo se organiza la materia?
A.19. Las facetas de la materia
A.20. La materia
L.12. Temperatura y equilibrio térmico
A.21. Conversaciones entre escalas de temperatura
A.22. ¿Cómo se mide la temperatura?
P. Temperatura
¿Qué aprendí?
Examen de Unidad 1
Calificaciones de la Unidad 1
Me preparo
L.1. Movimiento
A.1. ¿Trayectoria o desplazamiento?
A.2. Cambios en el movimiento
A.3. ¡Qué onda!
A.4. ¿Cómo se mueven las ondas?
F.7. A ruidos necios, oídos sordos
P. Desplazamiento
L.2. Velocidad y rapidez
A.5. ¿Quién fue el más rápido?
A.6. Explicaciones gráficas de velocidad
A.7. ¿Cómo se describe el movimiento?
A.8. ¿Cómo se ven la velocidad y el reposo?
A.9. Propagación de ondas
F.8. Alternativas del transporte urbano
P. Reposo
L.3. Movimiento acelerado
A.10. ¿Cuándo es rapidez y cuándo es velocidad?
A.11. ¿Cómo cambia su rapidez?
A.12. ¿Es aceleración o es velocidad?
A.13. ¿Se mueve o está quieto?
P. Aceleracion
L.4. Fuerzas e interacciones
A.14. Contacto y distancia
A.15. La medición de la fuerza
A.16. ¿Cómo se representan las fuerzas?
P. Fuerza
L.5. Suma de fuerzas y equilibrio
A.17. ¿Cómo se suman las fuerzas?
L.6. Fuerza de fricción
A.18. Fricción en el movimiento
F.9. La cuarta revolución industrial
L.7. Máquinas simples: palanca y rueda
A.19. ¿Qué máquinas son?
L.8. Otras máquinas simples
A.20. Máquinas ¿simples o compuestas?
A.21. Las ventajas de utilizar una polea
P. Masa Fuerza
L.9. Leyes de Newton
A.22. La medida de la inercia
A.23. Masa e inercia en el movimiento
A.24. Relación entre fuerza y aceleración
A.25. Acción y reacción
A.26. Tres ideas fundamentales sobre las fuerzas
F.10. ¡Abróchense los cinturones!
P. Movimiento Cuerpos
L.10. Ley de la Gravitación Universal
A.27. Importancia de las aportaciones de Newton
A.28. ¡A hombros de gigantes!
A.29. El peso en el Universo
L.11. Principio de Pascal
A.30. El barril de Pascal
F.11. La prensa hidráulica
P. Pascal
L.12. Principio de Arquímedes
A.31. Es momento de flotar
A.32. ¿Cómo se representa la fuerza de flotación?
P. Presion
L.13. Energía Mecánica
A.33. ¡Ponle energía!
A.34. ¿Cómo se conserva la energía mecánica?
P. Energia Mecanica
L.14. Calor como transferencia de energía.
A.35. Mecanismos de transferencia de calor
A.36. Energía térmica
L.15. Máquinas térmicas
A.37. ¿Cómo funcionan las máquinas térmicas?
A.38. Eficiencia ideal
A.39. Transformación de energía calorífica
F.12. Fuentes de energía y su impacto ambiental
P. Maquina Vapor
L.16. Energías renovables
A.40. Energía eléctrica y fuentes de energía
F.13. Biomímesis
P. Eolico
L.17. Energía solar
A.41. Aprovechamiento de la energía y consumo sustentable
A.42. Aprovechemos la energía solar
F.14. ¿Por qué usar un calentador solar?
P. Energia
¿Qué aprendí?
Examen de Unidad 2
Calificaciones de la Unidad 2
Me preparo
L.1. Electricidad
A.1. Cargas en movimiento
F.15. Energía eléctrica y medio ambiente
P. Electrostaticos
L.2. Electricidad, cuidados y precauciones 
A.2. Cuidados con el uso de la electricidad
F.16. ¡Cuidado con la electricidad!
P. Conductores
L.3. Electricidad y magnetismo 
A.3. ¿Qué sabes sobre el magnetismo?
A.4. Relación entre electricidad y magnetismo
P. Induccion
L.4. Luz visible 
A.5. ¡Qué onda con la luz!
F.17. Luces deslumbrantes
P. Descomposicion
L.5. Ondas electromagnéticas
A.6. Características del espectro electromagnético
A.7. Los efectos de la radiación electromagnética
F.18. Me mantengo alerta
P. Ondas Electromagneticas
L.6. El Universo 
A.8. Las galaxias
A.9. Grupos de galaxias
L.7. Origen y evolución del Universo 
A.10. Expansión del universo
A.11. Mecanismos de las estrellas
A.12. Los cuerpos cósmicos
A.13. Evolución del Universo
P. Universo
L.8. Descubrimiento del Universo
A.14. El Universo a simple vista
A.15. El telescopio y paralaje
A.16. Exploración de cuerpos celestes por medio de ondas electromagnéticas
L.9. Sistema Solar 
A.17. El Sol
A.18. Estructura del Sol
A.19. El Sistema Solar
P. Telescopio
L.10. Efecto invernadero, causas y consecuencias 
A.20. El efecto invernadero
L.11. Gases de efecto invernadero 
A.21. Gases naturales
L.12. Aumento de la temperatura del planeta 
A.22. Cambio Climático
F.19. ¿Calentamiento global o cambio climático?
L.13. Cuidado del ambiente 
A.23. ¿Soy sustentable?
F.20. Los guardianes del clima
P. Ambiente
¿Qué aprendí?
Examen de Unidad 3
Calificaciones de la Unidad 3