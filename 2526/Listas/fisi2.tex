Me preparo
1.	Conocimiento empírico
AI.1.	Conocimiento empírico
AI.2.	Tipos de conocimientos
AI.3.	Otras formas de conocimiento
2.	El conocimiento científico
AI.4.	¿Qué es la ciencia?
AI.5.	Método científico
F1.	Siete razones para dedicarse a la ciencia
3.	Física y sociedad
AI.6.	¿Qué es la Física?
AI.7.	División de la Física
F2.	Ciencia para todos
P.	Tecnologia
4.	Mediciones
AI.8.	Mediciones
P.	Tiempo
5.	Unidades fundamentales y derivadas de medida
AI.9.	Unidades fundamentales
6.	Múltiplos y submúltiplos
AI.10.	Múltiplos
7.	Instrumentos de medición
AI.11.	Es momento de medir
F3.	Contaminantes del aire
8.	Materiales y sus propiedades
AI.12.	Propiedades de la materia
AI.13.	¿Qué es la materia?
F4.	Litio, el oro blanco del siglo XXI
F5.	La isla de plástico
P.	Densidad
9.	Origen de las teorías sobre la estructura de la materia
AI.14.	Un modelo para la materia
10.	La teoría atómica
AI.15.	Construyendo un modelo para el átomo
AI.16.	¿Thomson, Rutherford o Bohr?
AI.17.	La estructura interna de la materia
F6.	Invierno nuclear
11.	Estados de agregación de la materia y modelo cinético
AI.18.	¿Cómo se organiza la materia?
AI.19.	Las facetas de la materia
AI.20.	La materia
12.	Temperatura y equilibrio térmico
AI.21.	Conversaciones entre escalas de temperatura
AI.22.	¿Cómo se mide la temperatura?
P.	Temperatura
¿Qué aprendí?
Examen de Unidad 1
Calificaciones de la Unidad 1
Me preparo
1.	Movimiento
AI.1.	¿Trayectoria o desplazamiento?
AI.2.	Cambios en el movimiento
AI.3.	¡Qué onda!
AI.4.	¿Cómo se mueven las ondas?
F7.	A ruidos necios, oídos sordos
P.	Desplazamiento
2.	Velocidad y rapidez
AI.5.	¿Quién fue el más rápido?
AI.6.	Explicaciones gráficas de velocidad
AI.7.	¿Cómo se describe el movimiento?
AI.8.	¿Cómo se ven la velocidad y el reposo?
AI.9.	Propagación de ondas
F8.	Alternativas del transporte urbano
P.	Reposo
3.	Movimiento acelerado
AI.10.	¿Cuándo es rapidez y cuándo es velocidad?
AI.11.	¿Cómo cambia su rapidez?
AI.12.	¿Es aceleración o es velocidad?
AI.13.	¿Se mueve o está quieto?
P.	Aceleracion
4.	Fuerzas e interacciones
AI.14.	Contacto y distancia
AI.15.	La medición de la fuerza
AI.16.	¿Cómo se representan las fuerzas?
P.	Fuerza
5.	Suma de fuerzas y equilibrio
AI.17.	¿Cómo se suman las fuerzas?
6.	Fuerza de fricción
AI.18.	Fricción en el movimiento
F9.	La cuarta revolución industrial
7.	Máquinas simples: palanca y rueda
AI.19.	¿Qué máquinas son?
8.	Otras máquinas simples
AI.20.	Máquinas ¿simples o compuestas?
AI.21.	Las ventajas de utilizar una polea
P.	Masa Fuerza
9.	Leyes de Newton
AI.22.	La medida de la inercia
AI.23.	Masa e inercia en el movimiento
AI.24.	Relación entre fuerza y aceleración
AI.25.	Acción y reacción
AI.26.	Tres ideas fundamentales sobre las fuerzas
F10.	¡Abróchense los cinturones!
P.	Movimiento Cuerpos
10.	Ley de la Gravitación Universal
AI.27.	Importancia de las aportaciones de Newton
AI.28.	¡A hombros de gigantes!
AI.29.	El peso en el Universo
11.	Principio de Pascal
AI.30.	El barril de Pascal
F11.	La prensa hidráulica
P.	Pascal
12.	Principio de Arquímedes
AI.31.	Es momento de flotar
AI.32.	¿Cómo se representa la fuerza de flotación?
P.	Presion
13.	Energía Mecánica
AI.33.	¡Ponle energía!
AI.34.	¿Cómo se conserva la energía mecánica?
P.	Energia Mecanica
14.	Calor como transferencia de energía.
AI.35.	Mecanismos de transferencia de calor
AI.36.	Energía térmica
15.	Máquinas térmicas
AI.37.	¿Cómo funcionan las máquinas térmicas?
AI.38.	Eficiencia ideal
AI.39.	Transformación de energía calorífica
F12.	Fuentes de energía y su impacto ambiental
P.	Maquina Vapor
16.	Energías renovables
AI.40.	Energía eléctrica y fuentes de energía
F13.	Biomímesis
P.	Eolico
17.	Energía solar
AI.41.	Aprovechamiento de la energía y consumo sustentable
AI.42.	Aprovechemos la energía solar
F14.	¿Por qué usar un calentador solar?
P.	Energia
¿Qué aprendí?
Examen de Unidad 2
Calificaciones de la Unidad 2
Me preparo
1.	Electricidad
AI.1.	Cargas en movimiento
F15.	Energía eléctrica y medio ambiente
P.	Electrostaticos
2.	Electricidad, cuidados y precauciones 
AI.2.	Cuidados con el uso de la electricidad
F16.	¡Cuidado con la electricidad!
P.	Conductores
3.	Electricidad y magnetismo 
AI.3.	¿Qué sabes sobre el magnetismo?
AI.4.	Relación entre electricidad y magnetismo
P.	Induccion
4.	Luz visible 
AI.5.	¡Qué onda con la luz!
F17.	Luces deslumbrantes
P.	Descomposicion
5.	Ondas electromagnéticas
AI.6.	Características del espectro electromagnético
AI.7.	Los efectos de la radiación electromagnética
F18.	Me mantengo alerta
P.	Ondas Electromagneticas
6.	El Universo 
AI.8.	Las galaxias
AI.9.	Grupos de galaxias
7.	Origen y evolución del Universo 
AI.10.	Expansión del universo
AI.11.	Mecanismos de las estrellas
AI.12.	Los cuerpos cósmicos
AI.13.	Evolución del Universo
P.	Universo
8.	Descubrimiento del Universo
AI.14.	El Universo a simple vista
AI.15.	El telescopio y paralaje
AI.16.	Exploración de cuerpos celestes por medio de ondas electromagnéticas
9.	Sistema Solar 
AI.17.	El Sol
AI.18.	Estructura del Sol
AI.19.	El Sistema Solar
P.	Telescopio
10.	Efecto invernadero, causas y consecuencias 
AI.20.	El efecto invernadero
11.	Gases de efecto invernadero 
AI.21.	Gases naturales
12.	Aumento de la temperatura del planeta 
AI.22.	Cambio Climático
F19.	¿Calentamiento global o cambio climático?
13.	Cuidado del ambiente 
AI.23.	¿Soy sustentable?
F20.	Los guardianes del clima
P.	Ambiente
¿Qué aprendí?
Examen de Unidad 3
Calificaciones de la Unidad 3