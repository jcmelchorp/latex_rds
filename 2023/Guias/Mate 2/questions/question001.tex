\question Analiza la tabla \ref{tab:masa} y responde las siguientes preguntas:
\renewcommand{\arraystretch}{1.1}

\begin{table}[H]
    \centering
    \begin{tabular}{clclcl}
        {\footnotesize \cellcolor[HTML]{FFCE93}Elemento    } &
        {\footnotesize \cellcolor[HTML]{E4E4C0}\% de átomos} &
        {\footnotesize \cellcolor[HTML]{FFCE93}Elemento    } &
        {\footnotesize \cellcolor[HTML]{E4E4C0}\% de átomos} &
        {\footnotesize \cellcolor[HTML]{FFCE93}Elemento    } &
        {\footnotesize \cellcolor[HTML]{E4E4C0}\% de átomos}
        \\
        H                                                    & 62 \%   & P           &
        0.22 \%                                              & Mg      & 0.007 \%      \\
        O                                                    & 24 \%   & S           &
        0.038 \%                                             & Si      & 0.0058 \%     \\
        C                                                    & 12 \%   & Na          &
        0.037  \%                                            & F       & 0.0012   \%   \\
        N                                                    & 1.1 \%  & K           &
        0.033 \%                                             & Fe      & 0.00067 \%    \\
        Ca                                                   & 0.22 \% & Cl          &
        0.024 \%                                             & Zn      & 0.00031 \%
    \end{tabular}
    \caption{Porcentaje de átomos de los elementos más abundantes en el cuerpo
        humano}
    \label{tab:masa}
\end{table}

\begin{parts}
    \subfile{../parts/question001a}
    %\newpage
    \subfile{../parts/question001b}
    \subfile{../parts/question001c}
\end{parts}