\question[10] Calcular la energía cinética de un automóvil compacto de 1340 kg que viaja a 145 km/h ¿cuánto cambia la energía, si el conductor reduce la velocidad de 145 km/h a 80 km/h ?.

\begin{solutionbox}{14cm}
    \begin{multicols}{2}
        Datos:

        Ec = ?

        m = 1340 kg

        v = 80 km/h

        v$_2$ = 145 km/h

        La energía cinética es:
        \[E_c=\frac{1}{2}mv^2\]

        Convirtiendo la velocidad de km/h a m/s:

        \[v=80\dfrac{\text{ km}}{\text{ h}}=80\left(\dfrac{1000 \text{ m}}{1 \text{ km}}\right)\left(\dfrac{1 \text{ h}}{3600 \text{ s}}\right)=22.\overline{2} \text{ m/s}\]

        \[v=145\dfrac{\text{ km}}{\text{ h}}=145\left(\dfrac{1000 \text{ m}}{1 \text{ km}}\right)\left(\dfrac{1 \text{ h}}{3600 \text{ s}}\right)=40.2\overline{7} \text{ m/s}\]

        \vspace{2cm}

        Calculando la energía cinética del auto a partir del reposo.
        \[
            \begin{array}{rl}
                E_c & = \dfrac{1}{2} (1340 \text{ kg})(40.2\overline{7} \text{ m/s})^2 \\[1em]
                    & = 0.5 (1340 \text{ kg})(1622.29 \text{ m$^2$/s$^2$})             \\[1em]
                    & =1,086,940.58 \text{ J }
            \end{array}
        \]

        Calculando la energía cinética del auto cuando se reduce su velocidad.

        \[
            \begin{array}{rl}
                E_c & = \dfrac{1}{2} (1340 \text{ kg})(22.\overline{2} \text{ m/s})^2 \\[1em]
                    & = 0.5 (1340 \text{ kg})(493.82 \text{ m$^2$/s$^2$})             \\[1em]
                    & =330,864.19 \text{ J }
            \end{array}
        \]

        Calculando la diferencia de energía:
        \[\Delta E = 1,086,940.58 \text{ J } - 330,864.19 \text{ J } = 756,076.38 \text{ J } \]

        Esta energía de 756,076.38 J (756.076 kJ) es equivalente al trabajo que está desarrollando el motor del auto para desplazarse con los cambios de velocidad señalados; en el segundo caso el signo menos nos indica en que cantidad se reduce la energía que suministra el motor al sistema, y en un momento determinado nos permite establecer la potencia que se requiere para mover todo el conjunto.
    \end{multicols}
\end{solutionbox}
