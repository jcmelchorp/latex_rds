\part[10] Antoine se encuentra en un balcón y lanza una pelota a su perro, que está a nivel del suelo.
La altura de la pelota (en metros sobre el suelo), $x$ segundos después de que Antoine la lanzó, está modelada por:
\begin{equation*}
    h(x)=-2x^2+4x+16
\end{equation*}
\textbf{¿Cuántos segundos después de ser lanzada la pelota llegará al suelo?}
\fullwidth{
    \begin{solutionbox}{5cm}
        \begin{multicols}{3}
            \[
                \begin{array}{rl}
                    -2x^2+4x+16 & =0  \\
                    a           & =-2 \\
                    b           & =4  \\
                    c           & =16 \\
                \end{array}
            \]
            \vspace{1cm}
            \[
                \begin{array}{rl}
                    x_{1,2} & = \dfrac{-b\pm\sqrt{b^2-4ac}}{2a}          \\[1.5em]
                    x_{1,2} & = \dfrac{-4\pm\sqrt{4^2-4(-2)(16)}}{2(-2)} \\[1.5em]
                    x_{1,2} & = \dfrac{-4\pm\sqrt{144}}{-4}              \\[1.5em]
                \end{array}
            \]
            \[
                \begin{array}{rl}
                    x_{1,2}        & = \dfrac{-4\pm 12}{-4}                           \\[1.5em]
                    \therefore x_1 & = \dfrac{-4-12}{-4}=\dfrac{-16}{-4}=4 \text{ y } \\[1.5em]
                    x_2            & = \dfrac{-4+12}{-4}=\dfrac{8}{-4}=-2             \\[1.5em]
                \end{array}
            \]
        \end{multicols}
    \end{solutionbox}
}