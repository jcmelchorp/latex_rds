\part[15] Aditi y Kavita tenían 40 monedas entre las dos. Aditi le dio 10 monedas a Kavita.
El producto de las monedas que tienen ahora es 375.
Sea $a$ la cantidad de monedas que tenía Aditi al principio.

\textbf{¿Cuál de las siguientes ecuaciones cuadráticas satisface $a$?}

\begin{oneparchoices}
    \choice $-a^2+60a+875=0$
    \choice $-a^2-60a+875=0$
    \choice $-a^2-60a-875=0$
    \CorrectChoice $-a^2+60a-875=0$
\end{oneparchoices}

\begin{solutionbox}{3.5cm}
    Sea $a$ la cantidad de monedas que tenía Aditi al principio, entonces Kavita tiene $40-a$. Si Aditi le da 10 monedas a Kavita, entonces Aditi tiene $a-10$ y Kavita tiene $40-a+10=50-a$. Con estas variables, su producto es:
    \[
        \begin{array}{rl}
            (a-10)(-a+50)        & =375 \\
            -a^2+50a+10a-500-375 & = 0  \\
            -a^2+60a-875         & = 0
        \end{array}
    \]
\end{solutionbox}

Si Aditi tenía menos de 30 monedas al principio.\\
\textbf{¿Con cuántas monedas empezó Aditi?}

\begin{solutionbox}{4.5cm}
    \begin{multicols}{3}
        \[
            \begin{array}{rl}
                -a^2+60a-875 & =0    \\
                a            & =-1   \\
                b            & =60   \\
                c            & =-875 \\
            \end{array}
        \]
        \vspace{1cm}
        \[
            \begin{array}{rl}
                x_{1,2} & = \dfrac{-b\pm\sqrt{b^2-4ac}}{2a}              \\[1.5em]
                x_{1,2} & = \dfrac{-60\pm\sqrt{60^2-4(-1)(-875)}}{2(-1)} \\[1.5em]
                x_{1,2} & = \dfrac{-60\pm\sqrt{3600-3500}}{-2}           \\[1.5em]
            \end{array}
        \]
        \[
            \begin{array}{rl}
                x_{1,2}        & = \dfrac{-60\pm 10}{-2}                            \\[1.5em]
                \therefore x_1 & = \dfrac{-60+10}{-2}=\dfrac{-50}{-2}=25 \text{ y } \\[1.5em]
                x_2            & = \dfrac{-60-10}{-2}=\dfrac{-70}{-2}=35            \\[1.5em]
            \end{array}
        \]
    \end{multicols}
\end{solutionbox}
