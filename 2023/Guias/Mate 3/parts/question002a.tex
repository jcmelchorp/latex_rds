\part[15] $x^2+x-42=0$

\fullwidth{
    \begin{solutionbox}{10cm}
        \begin{multicols}{2}
            Por factorización:
            \[
                \begin{array}{rl}
                    x^2+x-42   & = 0                       \\
                    (x+7)(x-6) & = 0                       \\
                    \therefore & x_1 =-7 \text{ y } x_2 =6
                \end{array}
            \]
            \\
            Por fórmula general:
            \[
                \begin{array}{rl}
                    y          & =	      x^2+x-42
                    \\
                    a          & =1
                    \\
                    b          & =1
                    \\
                    c          & =-42
                    \\
                    x_{1,2}    & = \dfrac{-b\pm\sqrt{b^2-4ac}}{2a}
                    \\[2em]
                    x_{1,2}    & = \dfrac{-1\pm\sqrt{1^2-4(1)(-42)}}{2(1)}
                    \\[2em]
                    x_{1,2}    & = \dfrac{-1\pm\sqrt{169}}{2}
                    \\[2em]
                    x_{1,2}    & = \dfrac{-1\pm 13}{2}
                    \\[2em]
                    \therefore & x_1 =\dfrac{12}{2}= \text{ y }  x_2
                    =\dfrac{-14}{2}=-7                                     \\[2em]
                \end{array}
            \]
        \end{multicols}
    \end{solutionbox}
}