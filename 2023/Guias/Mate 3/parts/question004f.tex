\part[15] El área de un rectángulo es 20cm$^2$. Su altura es 4 cm más que el triple del ancho.
Sea $A$ el ancho del rectángulo.\\
\textbf{¿Cuál de las siguientes ecuaciones cuadráticas satisface $a$?}

\begin{oneparchoices}
    \choice $3A^2+4A+20=0$
    \choice $3A^2-4A-20=0$
    \CorrectChoice $3A^2+4A-20=0$
    \choice $3A^2-4A+20=0$
\end{oneparchoices}

\begin{solutionbox}{3cm}
    Sea $A$ el ancho del rectángulo, entonces su altura es $3A+4$. Y el producto es:
    \[
        \begin{array}{rl}
            A(3A+4)     & =20 \\
            3A^2+4A     & =20 \\
            3A^2+4A -20 & =0
        \end{array}
    \]
\end{solutionbox}
}
\textbf{Determina el ancho del rectángulo $a$.}

\begin{solutionbox}{5cm}
    \begin{multicols}{3}
        \[
            \begin{array}{rl}
                3A^2+4A -20 & =0   \\
                a           & =3   \\
                b           & =4   \\
                c           & =-20 \\
            \end{array}
        \]
        \vspace{1cm}
        \[
            \begin{array}{rl}
                A_{1,2} & = \dfrac{-4\pm\sqrt{b^2-4ac}}{2a}         \\[1.5em]
                A_{1,2} & = \dfrac{-4\pm\sqrt{4^2-4(3)(-20)}}{2(3)} \\[1.5em]
                A_{1,2} & = \dfrac{-4\pm\sqrt{256}}{6}              \\[1.5em]
            \end{array}
        \]
        \[
            \begin{array}{rl}
                A_{1,2}        & = \dfrac{-4\pm 16}{6}                                       \\[1.5em]
                \therefore A_1 & = \dfrac{-4-16}{6}=\dfrac{-20}{6}=\dfrac{-10}{3} \text{ y } \\[1.5em]
                A_2            & = \dfrac{-4+16}{6}=\dfrac{12}{6}= 2                         \\[1.5em]
            \end{array}
        \]
    \end{multicols}
\end{solutionbox}
}