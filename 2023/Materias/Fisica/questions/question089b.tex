[10] Elige la respuesta correcta.
\begin{parts}
    \part Instrumento gracias al cual es posible observar cuerpos celestes muy lejanos.

    \begin{oneparchoices}
        \choice Microscopio
        \choice Estetoscopio
        \CorrectChoice Telescopio
        \choice Electroscopio
    \end{oneparchoices}

    \part Variación aparente de la posición de un objeto al cambiar la posición del observador.

    \begin{oneparchoices}
        \choice Eclipse
        \choice Declinación
        \choice Transformación
        \CorrectChoice Paralaje
    \end{oneparchoices}

    \part Aparato que sirve para medir ángulos muy pequeños que ayudó a medir la distancia a la cual se encuentran algunos objetos celestes.

    \begin{oneparchoices}
        \choice Vernier
        \CorrectChoice Micrómetro
        \choice Astrolabio
        \choice Transportador
    \end{oneparchoices}

    \part Técnica gracias a la cual se puede comparar el cambio en la posición de una estrella al transcurrir cierto periodo de tiempo.

    \begin{oneparchoices}
        \choice Radiografía
        \choice Radiometría
        \CorrectChoice Fotografía
        \choice Espectroscopía
    \end{oneparchoices}
\end{parts}