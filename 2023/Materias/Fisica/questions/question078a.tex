Lee el texto a continuación y escribe las palabras que faltan en los espacios vacíos dentro de cada afirmación.

\begin{parts}
    \part La luz \rule{3cm}{0.2mm} es una mezcla de todos los colores presentes en el arcoíris;
    éste es una manifestación de la \rule{3cm}{0.2mm} de la luz proveniente del Sol.

    \part El \rule{3cm}{0.2mm} de los objetos que el ojo humano distingue (por ejemplo, una
    manzana roja) es producto del \rule{3cm}{0.2mm} de una parte de la luz que les llega, y
    corresponde al \rule{3cm}{0.2mm} del objeto.

    \part Cuando la luz incide sobre una \rule{3cm}{0.2mm} opaca, como un espejo, la \rule{3cm}{0.2mm}
    del rayo incidente es igual a la del reflejado.

    \part Cuando un rayo de luz pasa de un medio a otro, oblicuamente (por ejemplo, del aire
    al agua), experimenta un cambio de \rule{3cm}{0.2mm} al cual se le conoce como \rule{3cm}{0.2mm}

    \part Los fenómenos de reflexión y \rule{3cm}{0.2mm} de la luz se pueden explicar si
    suponemos que ésta es una \rule{3cm}{0.2mm}.

    \part La luz es resultado de una combinación de oscilaciones eléctricas y \rule{3cm}{0.2mm}.

    \part Las cargas con movimiento acelerado generan ondas \rule{3cm}{0.2mm}.
    \part La luz se propaga en línea \rule{3cm}{0.2mm} y puede hacerlo incluso en el \rule{3cm}{0.2mm}.

    \part La \rule{3cm}{0.2mm} de la ondas electromagnéticas es proporcional a su \rule{3cm}{0.2mm}.

    \part La energía de las ondas electromagnéticas es \rule{3cm}{0.2mm} proporcional a su
    \rule{3cm}{0.2mm} de onda.

    \part Todas las ondas electromagnéticas se propagan con la misma \rule{3cm}{0.2mm}. En el vacío,
    ésta es de aproximadamente 300 000 km/s.

    \part A la clasificación de las ondas electromagnéticas según su frecuencia se le conoce como \rule{3cm}{0.2mm}.
\end{parts}