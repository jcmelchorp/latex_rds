Lee el texto a continuación y escribe las palabras que faltan en los espacios vacíos dentro de cada afirmación.

\begin{parts}
    \part La luz \fillin[visible][2cm] es una mezcla de todos los colores presentes en el arcoíris;
    éste es una manifestación de la \fillin[descomposición][3cm] de la luz proveniente del Sol.

    \part El \fillin[color][2cm] de los objetos que el ojo humano distingue (por ejemplo, una
    manzana roja) es producto del \fillin[reflejo][2cm] de una parte de la luz que les llega, y
    corresponde al \fillin[color][2cm] del objeto.

    \part Cuando un rayo de luz pasa de un medio a otro, oblicuamente (por ejemplo, del aire
    al agua), experimenta un cambio de \fillin[dirección][2cm] al cual se le conoce como \fillin[refracción][2cm]

    \part Los fenómenos de reflexión y \fillin[refracción][2cm] de la luz se pueden explicar si
    suponemos que ésta es una \fillin[onda][2cm].

    \part La luz es resultado de una combinación de oscilaciones eléctricas y \fillin[magnéticas][3cm].

    \part La luz se propaga en línea \fillin[recta][2cm] y puede hacerlo incluso en el \fillin[vacío][2cm].

    \part La \fillin[energía][2cm] de las ondas electromagnéticas es proporcional a su \fillin[frecuencia][2cm].

    \part La energía de las ondas electromagnéticas es \fillin[inversamente][3cm] proporcional a su
    \fillin[longitud][2cm] de onda.

    \part Todas las ondas electromagnéticas se propagan con la misma \fillin[velocidad][2cm]. En el vacío,
    ésta es de aproximadamente 300 000 km/s.

    \part A la clasificación de las ondas electromagnéticas según su frecuencia se le conoce como \fillin[espectro electromagnético][5cm].
\end{parts}