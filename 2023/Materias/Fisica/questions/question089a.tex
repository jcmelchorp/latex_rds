Coloca las palabras que completan las afirmaciones.
\begin{parts}
    \part Con el telescopio ha sido posible observar \rule{3cm}{0.2mm}
    celestes muy lejanos y calcular a que distancia se encuentran usando la paralaje,
    que es la variación en la posición aparente de un objeto respecto a otros cuando
    se observa desde dos \rule{3cm}{0.2mm} diferentes.


    \part Es posible medir esta diferencia por el ángulo de \rule{3cm}{0.2mm}, si se conoce la
    \rule{3cm}{0.2mm} que separa los puntos de observación, se puede estimar la distancia a la
    cual se encuentra el objeto observado.

\end{parts}