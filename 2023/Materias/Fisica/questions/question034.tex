Elige a qué ley pertenece cada ejemplo.

\begin{parts}

    \part  La aceleración que experimenta un objeto al recibir una fuerza es
    directamente proporcional a la fuerza aplicada e inversamente proporcional
    a su
    masa, y tiene lugar en la dirección de ella.

    \begin{choices}
        \choice Primera ley de Newton
        \choice Segunda ley de Newton
        \choice Tercera ley de Newton
        \choice Ley de la gravitación universal
    \end{choices}

    \part  Todo objeto tiende a mantener su estado de reposo o movimiento en
    línea recta con velocidad constante, mientras una fuerza no actúe sobre él.

    \begin{choices}
        \choice Primera ley de Newton
        \choice Segunda ley de Newton
        \choice Tercera ley de Newton
        \choice Ley de la gravitación universal
    \end{choices}

    \part  Esta ley establece que la fuerza gravitacional entre dos objetos es
    directamente proporcional a sus masas e inversamente proporcional al
    cuadrado
    de la distancia que hay entre los dos.

    \begin{choices}
        \choice Primera ley de Newton
        \choice Segunda ley de Newton
        \choice Tercera ley de Newton
        \choice Ley de la gravitación universal
    \end{choices}

    \part  Cuando un objeto ejerce una acción sobre otro, este último ejerce
    una reacción de igual magnitud y en dirección opuesta.

    \begin{choices}
        \choice Primera ley de Newton
        \choice Segunda ley de Newton
        \choice Tercera ley de Newton
        \choice Ley de la gravitación universal
    \end{choices}

    % \columnbreak

    \part  Si la fuerza gravitacional, al actuar sobre cualquier objeto, es
    directamente proporcional a su masa.

    \begin{choices}
        \choice Primera ley de Newton
        \choice Segunda ley de Newton
        \choice Tercera ley de Newton
        \choice Ley de la gravitación universal
    \end{choices}

    \part  Al empujar una caja que está sobre un suelo liso, ésta acelera.

    \begin{choices}
        \choice Primera ley de Newton
        \choice Segunda ley de Newton
        \choice Tercera ley de Newton
        \choice Ley de la gravitación universal
    \end{choices}

    \part  Si la Luna no fuera afectada por la Tierra, seguiría una trayectoria
    en línea recta a velocidad constante. ¿Cuál de las leyes del movimiento de
    Newton se aplica a esta situación?

    \begin{choices}
        \choice Primera ley de Newton
        \choice Segunda ley de Newton
        \choice Tercera ley de Newton
        \choice Ley de la gravitación universal
    \end{choices}

    \part  Un jet descarga un chorro de fluido hacia atrás a gran velocidad;
    sin embargo, la aeronave se mueve hacia adelante.

    \begin{choices}
        \choice Primera ley de Newton
        \choice Segunda ley de Newton
        \choice Tercera ley de Newton
        \choice Ley de la gravitación universal
    \end{choices}

\end{parts}