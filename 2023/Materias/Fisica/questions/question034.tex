Elige a qué ley pertenece cada ejemplo.
\begin{multicols}{2}
    \begin{parts}

        \part  La aceleración que experimenta un objeto al recibir una fuerza es
        directamente proporcional a la fuerza aplicada e inversamente proporcional
        a su
        masa, y tiene lugar en la dirección de ella.

        \begin{oneparchoices}\footnotesize%
            \choice 1° ley de Newton
            \choice 2° ley de Newton \\
            \choice 3° ley de Newton
            \choice Ley de la gravitación universal
        \end{oneparchoices}

        \part  Al empujar una caja que está sobre un suelo liso, ésta acelera.

        \begin{oneparchoices}\footnotesize%
            \choice 1° ley de Newton
            \choice 2° ley de Newton \\
            \choice 3° ley de Newton
            \choice Ley de la gravitación universal
        \end{oneparchoices}

        \part  Esta ley establece que la fuerza gravitacional entre dos objetos es
        directamente proporcional a sus masas e inversamente proporcional al
        cuadrado
        de la distancia que hay entre los dos.


        \begin{oneparchoices}\footnotesize%
            \choice 1° ley de Newton
            \choice 2° ley de Newton \\
            \choice 3° ley de Newton
            \choice Ley de la gravitación universal
        \end{oneparchoices}

        \part  Si la fuerza gravitacional, al actuar sobre cualquier objeto, es
        directamente proporcional a su masa.

        \begin{oneparchoices}\footnotesize%
            \choice 1° ley de Newton
            \choice 2° ley de Newton \\
            \choice 3° ley de Newton
            \choice Ley de la gravitación universal
        \end{oneparchoices}

        % \columnbreak

        \part  Cuando un objeto ejerce una acción sobre otro, este último ejerce
        una reacción de igual magnitud y en dirección opuesta.

        \begin{oneparchoices}\footnotesize%
            \choice 1° ley de Newton
            \choice 2° ley de Newton \\
            \choice 3° ley de Newton
            \choice Ley de la gravitación universal
        \end{oneparchoices}

        \part  Todo objeto tiende a mantener su estado de reposo o movimiento en
        línea recta con velocidad constante, mientras una fuerza no actúe sobre él.


        \begin{oneparchoices}\footnotesize%
            \choice 1° ley de Newton
            \choice 2° ley de Newton \\
            \choice 3° ley de Newton
            \choice Ley de la gravitación universal
        \end{oneparchoices}

        \part  Si la Luna no fuera afectada por la Tierra, seguiría una trayectoria
        en línea recta a velocidad constante. ¿Cuál de las leyes del movimiento de
        Newton se aplica a esta situación?

        \begin{oneparchoices}\footnotesize%
            \choice 1° ley de Newton
            \choice 2° ley de Newton \\
            \choice 3° ley de Newton
            \choice Ley de la gravitación universal
        \end{oneparchoices}

        \part  Un jet descarga un chorro de fluido hacia atrás a gran velocidad;
        sin embargo, la aeronave se mueve hacia adelante.

        \begin{oneparchoices}\footnotesize%
            \choice 1° ley de Newton
            \choice 2° ley de Newton \\
            \choice 3° ley de Newton
            \choice Ley de la gravitación universal
        \end{oneparchoices}

    \end{parts}
\end{multicols}