Elige la respuesta correcta.

\begin{multicols}{2}
    \begin{parts}
        \part La relación de proporcionalidad entre la velocidad con la que se
        alejan las galaxias y la distancia a la que se encuentran.
        %(pág. \pageref{086a_c})

        \begin{choices}
            \choice Ley de Hook
            \choice Ley de Faraday
            \CorrectChoice Ley de Hubble
            \choice Ley de Moore
        \end{choices}

        % \part Grupo formado por la Vía Láctea y otras 14 galaxias gigantes
        % que integra una estructura en forma de anillo.

        % \begin{choices}
        %     \choice Supercúmulo
        %     \choice Concilio de Gigantes
        %     \choice Cúmulos de galaxias
        %     \choice Grupo local
        % \end{choices}

        % \part Grupo formado por la Vía Láctea y unas 30 galaxias más.

        % \begin{choices}
        %     \choice Supercúmulo
        %     \choice Concilio de Gigantes
        %     \choice Cúmulos de galaxias
        %     \choice Grupo local
        % \end{choices}

        \part Indica que el Universo se expande. %(pág. \pageref{086a_b})

        \begin{choices}
            \choice El corrimiento al azul de la luz que emiten las galaxias.
            \CorrectChoice El corrimiento al rojo de la luz que emiten las
            galaxias.
            \choice Todas las galaxias se alejan de la Vía Láctea.
            \choice La Teoría de la Relatividad General
        \end{choices}

        % \part Grupo de galaxias cuyos tamaños típicos son de 2 a 3 Mpc.

        % \begin{choices}
        %     \choice Supercúmulo
        %     \choice Concilio de Gigantes
        %     \choice Cúmulos de galaxias
        %     \choice Grupo local
        % \end{choices}

        % \part Grupo formado por cúmulos de galaxias.

        % \begin{choices}
        %     \choice Supercúmulo
        %     \choice Concilio de Gigantes
        %     \choice Cúmulos de galaxias
        %     \choice Grupo local
        % \end{choices}

    \end{parts}
\end{multicols}