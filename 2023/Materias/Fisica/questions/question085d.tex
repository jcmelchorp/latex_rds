El parsec (pc) puede definirse a partir del año luz: 1 pc = 3.26 años luz. Como
no es muy diferente de él, en realidad lo práctico consiste en usar sus múltiplos,
como el kiloparsec, $1 \text{ kpc} = 10^{3} \text{ pc}$, o el megaparsec, 1 Mpc = 10 6 pc. El uso del
parsec en la astronomía es una cuestión más bien de tradición.

\begin{parts}
    \part ¿A cuántos kilómetros equivale un parsec?

    \begin{solutionbox}{1.6cm}
        Si 1 año luz equivale a $9.46 \times 10^{12} \text{ km}$, entonces $1 pc = 3.26 \text{ años luz} \cdot 9.46 \times 10^{12} \text{ km}=3.08 \times 10^{13} \text{ km}$
    \end{solutionbox}

    \part La galaxia M31 está a 650 kpc de la Vía Láctea y se acerca a ella a una velocidad
    de unos 300 km/s. ¿En cuánto tiempo \comillas{chocará} con ella? Resuelvan en equipo.

    \begin{solutionbox}{1.6cm}
        \[ 1.03 \times 10^{11} \text{ segundos} = 2,120,222,391 \text{ millones de años}\]
    \end{solutionbox}
\end{parts}