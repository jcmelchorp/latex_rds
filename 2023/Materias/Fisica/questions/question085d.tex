El parsec (pc) puede definirse a partir del año luz: 1 pc = 3.26 años luz. Como
no es muy diferente de él, en realidad lo práctico consiste en usar sus múltiplos,
como el kiloparsec, $1 \text{ kpc} = 10^{3} \text{ pc}$, o el megaparsec, 1 Mpc = 10$^6$ pc. El uso del
parsec en la astronomía es una cuestión más bien de tradición.

\begin{parts}
    \part ¿A cuántos metros equivale un parsec?

    \begin{solutionbox}{3cm}\footnotesize
        Usando la fórmula $d=vt$, donde $d$ es la distancia, $v$ es la velocidad y $t$ es el tiempo, la distancia $d$ que hay en un año luz es:
        \begin{align*}
            d & = (3 \times 10 ^{8} \text{ m/s } ) (365.25\times 24 \times 60 \times 60 \text{ s}) \\
              & = 9.46 \times 10^{15} \text{ m}
        \end{align*}
        Si 1 año luz equivale a $9.46 \times 10^{15} \text{ m}$, entonces $1 pc = 3.26 \text{ años luz} \cdot 9.46 \times 10^{15} \text{ m}=3.08 \times 10^{16} \text{ m}$
    \end{solutionbox}

    \part La galaxia M31 está a 650 kpc de la Vía Láctea y se acerca a ella a una velocidad
    de unos 350 km/s. ¿En cuánto tiempo \comillas{chocará} con ella? %Resuelvan en equipo.

    \begin{solutionbox}{4cm}\footnotesize
        \begin{multicols}{2}
            Sabemos que $1 \text{ pc}=3.08 \times 10^{13} \text{ km}$, entonces
            \begin{align*}
                650 \text{ kpc} & =650 \times 10^{3} \text{ pc}                            \\
                                & =650 \times 10^{3} \times 3.08 \times 10^{13} \text{ km} \\
                                & =2.002 \times 10^{19} \text{ km}
            \end{align*}

            \columnbreak

            Usando la fórmula $t=\dfrac{d}{v}$, el tiempo $t$ en segundos es:
            \begin{align*}
                t & =\dfrac{2.002 \times 10^{19} \text{ km}}{350 \text{ km/s}} \\
                  & =5.72 \times 10^{16} \text{ s}                             \\
                  & =1,812.5 \text{ millones de años}
            \end{align*}
        \end{multicols}
    \end{solutionbox}
\end{parts}