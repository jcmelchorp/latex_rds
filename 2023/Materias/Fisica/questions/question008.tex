\question[10] Un auto con masa de 1400 kg parte del reposo con movimiento uniforme acelerado hasta alcanzar una velocidad de 80 km/h. Determine la energía cinética del auto.

\begin{solutionbox}{8cm}
    \begin{multicols}{2}
        Datos:

        Ec = ?

        m = 1400 kg

        v = 80 km/h

        La energía cinética es:
        \[E_c=\frac{1}{2}mv^2\]


        Convirtiendo la velocidad de km/h a m/s:

        \[v=80\dfrac{\text{ km}}{\text{ h}}=80\left(\dfrac{1000 \text{ m}}{1 \text{ km}}\right)\left(\dfrac{1 \text{ h}}{3600 \text{ s}}\right)=22.\overline{2} \text{ m/s}\]

        \vspace{2cm}

        Sustituyendo nuestros datos en la fórmula:
        \[
            \begin{array}{rl}
                E_c & = \dfrac{1}{2} (1400 \text{ kg})(22.\overline{2} \text{ m/s})^2 \\[1em]
                    & = 0.5 (1400 \text{ kg})(493.82 \text{ m$^2$/s$^2$})             \\[1em]
                    & =345,679.01 \text{ J }
            \end{array}
        \]
        La energía cinética del auto cuando está partiendo del reposo y alcanza una velocidad de 80 km/h es de 345,679.01 J.
    \end{multicols}
\end{solutionbox}
