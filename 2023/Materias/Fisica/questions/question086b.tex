Escribe las palabras que completan las afirmaciones.
\begin{parts}
    En 1929 Edwin Hubble midió las distancias y los espectros de la luz de varias galaxias
    y descubrió que, en general, la longitud de onda de la luz que emitían se alarga como
    consecuencia de su \fillin[movimiento][2cm] respecto a nosotros, alejándose. (pág. \pageref{086b_a})

    El corrimiento al rojo permitió a Hubble determinar que la \fillin[velocidad][2cm]
    con la que se alejan las galaxias es proporcional a la \fillin[distancia][2cm] a la que se encuentran. (pág. \pageref{086b_b})

    La interpretación que se le dio al descubrimiento de \fillin[Hubble][2cm] es que el Universo se expande. (pág. \pageref{086b_c})
\end{parts}