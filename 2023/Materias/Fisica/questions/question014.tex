\question[10] Si la energía potencial de una pelota de golf al ser golpeada es de 80 J. ¿Cuál será su masa si alcanza una altura de 30m?

\begin{solutionbox}{7.5cm}
    \begin{multicols}{2}
        Datos:

        E$_p$ = 80 J

        h = 30 m

        g=9.8 m/s$^2$

        m = ?

        La energía potencial es:
        \[E_p=mgh\]

        \vspace{2cm}

        Despejando a "m"

        \[m=\dfrac{E_p}{gh}\]


        Sustituyendo nuestros datos en la fórmula:
        \[
            \begin{array}{rl}
                E_p & = \dfrac{80 \text{ J}}{(9.8 \text{ m/s$^2$})(30 \text{ m})}           \\[1em]
                E_p & = \dfrac{80 \text{ m$^2$/s$^2$}}{(9.8 \text{ m/s$^2$})(30 \text{ m})} \\[1em]
                    & =0.2721 \text{ kg }
            \end{array}
        \]
        La energía potencial de la pelota es de 127.4 Joules.
    \end{multicols}
\end{solutionbox}
