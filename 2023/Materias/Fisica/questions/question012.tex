Calcular la energía potencial contenida en una cascada de agua de 40 m de altura considerando que la cantidad de masa en movimiento es de 80,000 kg.

\begin{solutionbox}{5.5cm}
    \begin{multicols}{2}
        Datos:

        E$_p$ = ?

        h = 40 m

        g = 9.8 m/s$^2$

        m = 80,000 kg

        La energía potencial es:
        \[E_p=mgh\]

        \vspace{2cm}

        Sustituyendo nuestros datos en la fórmula:
        \[
            \begin{array}{rl}
                E_p & = (80,000 \text{ kg})(9.8 \text{ m/s$^2$})(40 \text{ m}) \\[1em]
                    & =31,360,000 \text{ J }
            \end{array}
        \]
        Por lo que la energía potencial gravitacional almacenada en la cascada es de 31.36 MJ (Mega Joules)
    \end{multicols}
\end{solutionbox}
