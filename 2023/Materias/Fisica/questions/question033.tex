\question Elige  o  para indicar si las siguientes afirmaciones son o  aportaciones de Newton a la ciencia.

\begin{parts}

    \part[5] Los objetos se mueven según su naturaleza.

    \begin{oneparchoices}
        \choice Sí
        \choice No
    \end{oneparchoices}

    \part[5] El estado normal de los objetos, a excepción de los objetos celestes, es el de reposo.

    \begin{oneparchoices}
        \choice Sí
        \choice No
    \end{oneparchoices}

    \part[5] Los objetos pesan porque son atraídos por la Tierra.

    \begin{oneparchoices}
        \choice Sí
        \choice No
    \end{oneparchoices}

    \part[5] Cuando un objeto ejerce una fuerza de acción sobre otro, éste último ejerce una fuerza de reacción al mismo tiempo, de igual magnitud y en dirección opuesta sobre el primero.

    \begin{oneparchoices}
        \choice Sí
        \choice No
    \end{oneparchoices}

    \part[5] Un objeto que está en su lugar propio  se mueve, a menos que se le someta a una fuerza.

    \begin{oneparchoices}
        \choice Sí
        \choice No
    \end{oneparchoices}

    \part[5] La aceleración que experimenta un objeto al recibir una fuerza es directamente proporcional a la magnitud de la fuerza aplicada e inversamente proporcional a su masa, y tiene la misma dirección que la fuerza aplicada.

    \begin{oneparchoices}
        \choice Sí
        \choice No
    \end{oneparchoices}

    \part[5] La fuerza de gravedad es una propiedad que tienen los cuerpos con masa de atraerse mutuamente

    \begin{oneparchoices}
        \choice Sí
        \choice No
    \end{oneparchoices}

    \part[5] Los cuerpos celestes se encuentran en el mundo etéreo o supralunar y se mueven en círculos, donde todo es perfecto, inmutable, infinito y eter.

    \begin{oneparchoices}
        \choice Sí
        \choice No
    \end{oneparchoices}

    \part[5] Los cuerpos celestes siguen leyes del movimiento distintas a la de los cuerpos terrestres.

    \begin{oneparchoices}
        \choice Sí
        \choice No
    \end{oneparchoices}

    \part[5] El movimiento de los objetos terrestres y celestes es regido por las mismas leyes.

    \begin{oneparchoices}
        \choice Sí
        \choice No
    \end{oneparchoices}

    \part[5] La fuerza de gravedad que actúa entre dos cuerpos es siempre de atracción, es directamente proporcional al producto de sus masas e inversamente proporcional al cuadrado de su distancia.

    \begin{oneparchoices}
        \choice Sí
        \choice No
    \end{oneparchoices}

    \part[5] Un objeto cae con una velocidad proporcional a su peso.

    \begin{oneparchoices}
        \choice Sí
        \choice No
    \end{oneparchoices}

    \part[5] Una flecha se mueve a causa de la brecha en el aire originada por su movimiento. La brecha en el aire causa un efecto de apriete en la parte trasera de la flecha a medida que el aire regresa para evitar que se forme el vacío.

    \begin{oneparchoices}
        \choice Sí
        \choice No
    \end{oneparchoices}

    \part[5] Todo cuerpo tiende a mantener su estado de reposo o de movimiento en línea recta con velocidad constante, a mes que una fuerza que actúe sobre él le obligue a cambiar ese estado.

    \begin{oneparchoices}
        \choice Sí
        \choice No
    \end{oneparchoices}


\end{parts}
