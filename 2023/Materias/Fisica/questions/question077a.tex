Relaciona el tipo de onda electromagnética  que está involucrada con la  explicación de cómo manifiesta su energía.

\begin{minipage}{0.3\textwidth}
    \begin{choices}
        \choice Rayos X
        \choice Luz visible
        \choice Radiación infraroja
        \choice Microondas
    \end{choices}
\end{minipage}\hfill
\begin{minipage}{0.7\textwidth}
    \begin{parts}
        \fillin[D][1cm] Poseen altas frecuencias y hacen vibrar las moléculas de agua, por lo que incrementan su temperatura. Se utilizan para calentar alimentos con altos contenidos de agua.
        \fillin[C][1cm] Es también conocida como radiación térmica, y es aplicada en la comunicación entre dispositivos electrónicos a corta distancia, como el control remoto de un televisor.
        \fillin[B][1cm] Puede ser aprovechada por los seres vivos; por ejemplo, para generar energía química mediante la fotosíntesis.
        \fillin[A][1cm] Poseen gran energía, por lo que pueden atravesar la materia blanda, pero no la dura. Esta propiedad permite generar imágenes de los huesos.
    \end{parts}
\end{minipage}%