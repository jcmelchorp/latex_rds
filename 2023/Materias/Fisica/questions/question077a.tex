Relaciona el tipo de onda electromagnética  que está involucrada con la  explicación de cómo manifiesta su energía.
\begin{multicols}{2}
    \begin{flushright}
        Rayos X $\square$\\                   \vspace{2cm}
        Luz visible $\square$\\               \vspace{2cm}
        Radiación infraroja $\square$\\       \vspace{2cm}
        Microondas $\square$\\                \vspace{2cm}
    \end{flushright}
    \vspace{1cm}
    \begin{checkboxes}
        \choice Poseen altas frecuencias y hacen vibrar las moléculas de agua, por lo que incrementan su temperatura. Se utilizan para calentar alimentos con altos contenidos de agua.
        \choice Puede ser aprovechada por los seres vivos; por ejemplo, para generar energía química mediante la fotosíntesis.
        \choice Es también conocida como radiación térmica, y es aplicada en la comunicación entre dispositivos electrónicos a corta distancia, como el control remoto de un televisor.
        \choice Poseen gran energía, por lo que pueden atravesar la materia blanda, pero no la dura. Esta propiedad permite generar imágenes de los huesos.
    \end{checkboxes}
\end{multicols}