 Coloca las palabras que completan las afirmaciones.
    \begin{center}
        \fbox{rayos X} \quad  \fbox{dañados} \quad \fbox{observar} \quad \fbox{\'oseos} \quad
        \fbox{placas} \quad \fbox{romper} \quad \fbox{formar} \quad
        \fbox{blandos} \quad
    \end{center}
    \begin{parts}
        \part En 1895 William Roentgen descubrió los rayos X al \rule{2cm}{0.2mm}
        una radiación muy penetrante, pero invisible, que era capaz de atravesar el cartón y algunas \rule{2cm}{0.2mm}
        delgadas de metal.

        \part Gracias al poder de penetración de los \rule{2cm}{0.2mm}
        es posible generar radiografías.

        \part Para generar imágenes del cuerpo humano se aprovecha que los rayos X atraviesan con mayor facilidad los tejidos
        \rule{2cm}{0.2mm} que los tejidos \rule{2cm}{0.2mm}.
        .
    \end{parts}
