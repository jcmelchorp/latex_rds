\question Elige las opciones que resuelvan cada problema sobre planos inclinados.

\begin{parts}
    \part[5] ¿Qué fuerza tendrías que aplicar para subir un sillón de 25 N
    de peso a una altura de 4 m si utilizas un plano inclinado de 5 m?

    % -
    \part[5] ¿De qué longitud tendrá que ser el plano inclinado por utilizar si deseas
    subir un peso de 200 N a una altura de 2 m, si tu máxima capacidad te permite
    aplicar una fuerza de 50 N?

    % \begin{oneparchoices}
    %     \CorrectChoice 8 m
    %     \choice  5,000 m
    %     \choice 0.5 m
    %     \choice 80 m
    % \end{oneparchoices}

    \part[5] ¿A qué altura se subió un objeto de 50 N si se aplicó una fuerza de 25 N y
    se utilizó un plano inclinado de 10 m?

    % \begin{oneparchoices}
    %     \choice 50 m
    %     \CorrectChoice  5 m
    %     \choice 500 m
    %     \choice 5.5 m
    % \end{oneparchoices}

    \part[5] ¿Qué fuerza se debe aplicar a una caja de 100 N de peso para subirla a
    un templete a una altura de 80 cm si se usa una rampa de 240 cm?

    \part[5] Se necesita subir una carga de 500 kg (4900 N) a una altura de 1.5 m
    deslizándola sobre una rampa inclinada, ¿qué longitud debe tener la rampa si
    sólo se puede aplicar una fuerza de 1633.33 N?

\end{parts}