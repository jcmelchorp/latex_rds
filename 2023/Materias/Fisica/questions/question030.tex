\question Resuelve los siguientes problemas sobre planos inclinados.

\begin{parts}
    \part[5] ¿Qué fuerza tendrías que aplicar para subir un sillón de 25 N
    de peso a una altura de 4 m si utilizas un plano inclinado de 5 m?

    \begin{solutionbox}{6cm}
        de la ecuación del plano inclinado se tiene:
        \[F_1 \times d_1= F_2 \times d_2\]
        donde $F_1=25$ N, $d_1=4$ m, $d_2=5m$
        $Rightarrow$
        \[
            \begin{array}{lr}
                25 \times 4            & =              F_2 \times 5 \\
                \dfrac{25 \times 4}{5} & =  F_2                      \\
                \dfrac{100}{5}         & =          F_2              \\
                20 = F_2
            \end{array}
        \]
    \end{solutionbox}

    \part[5] ¿De qué longitud tendrá que ser el plano inclinado por utilizar si deseas
    subir un peso de 200 N a una altura de 2 m, si tu máxima capacidad te permite
    aplicar una fuerza de 50 N?

    \begin{solutionbox}{6cm}
    \end{solutionbox}

    \part[5] ¿A qué altura se subió un objeto de 50 N si se aplicó una fuerza de 25 N y
    se utilizó un plano inclinado de 10 m?

    \begin{solutionbox}{6cm}
    \end{solutionbox}


    \part[5] ¿Qué fuerza se debe aplicar a una caja de 100 N de peso para subirla a
    un templete a una altura de 80 cm si se usa una rampa de 240 cm?
    \begin{solutionbox}{6cm}
        de la ecuación del plano inclinado se tiene:
        \[F_1 \times d_1= F_2 \times d_2\]
        donde $F_1=100$ N, $d_1=0.8$ m, $d_2=2.40$ m
        $Rightarrow$
        \[
            \begin{array}{lr}
                100 \times 0.8              & =                F_2 \times 2.40 \\
                \dfrac{100 \times 0.8}{2.4} & =  F_2                           \\
                \dfrac{100}{5}              & =               F_2              \\
                20 = F_2
            \end{array}
        \]
    \end{solutionbox}

    \part[5] Se necesita subir una carga de 500 kg (4900 N) a una altura de 1.5 m
    deslizándola sobre una rampa inclinada, ¿qué longitud debe tener la rampa si
    sólo se puede aplicar una fuerza de 1633.33 N?

    \begin{solutionbox}{6cm}
    \end{solutionbox}

\end{parts}