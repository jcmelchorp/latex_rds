\documentclass[12pt]{guia}
\grade{2$^\circ$ de Secundaria}
\cycle{2022-2023}
\subject{Física 2}
\guide{4}
\title{Energía mecánica}
%\title{El título de la guía}
\aprendizajes{
    \begin{itemize}[leftmargin=*,label=\small\color{colorrds}\faIcon{user-graduate}]
        \item Analiza la energía mecánica (cinética y potencial) y describe casos donde se conserva.
    \end{itemize}
}
\requisitos{
    \begin{itemize}
        \item Requisito 1
        \item Requisito 2
    \end{itemize}
}
\author{J. C. Melchor Pinto}

\begin{document}
\pagestyle{headandfoot}
\addpoints
\INFO
%\printanswers
\begin{opening}[¿Qué es la energía mecánica?]
    {La energía mecánica, E$_M$, es la suma de la energía potencial E$_p$ y
        la energía cinética E$_c$ en un sistema.

        \[ E_M=E_p+E_c\]

        El caso especial de la conservación de energía mecánica a menudo es más útil, y para ello debe considerarse a la energía mec\'anica como invariable. Es decir, un balance o intercambio entre estos dos tipos de energía (potencial y cinética). Cuando la energía se conserva, podemos establecer ecuaciones que igualen la suma de las diferentes formas de energía en un sistema.
    }
\end{opening}
\begin{questions}
    \include*{../questions/question018}
    \newpage
    \include*{../questions/question021}
    \newpage
    \include*{../questions/question022}
\end{questions}

% \vfill
% \puntuacion

\end{document}