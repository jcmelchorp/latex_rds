\documentclass[12pt,addpoints,answers]{guia}
\grado{2$^\circ$ de Secundaria}
\cicloescolar{2022-2023}
\materia{Física 2}
\guia{1}
\unidad{3}
\title{¿Qué es la energia?}
\aprendizajes{\item Analiza la energía mecánica (cinética y potencial) y describe casos donde se conserva.
    }
\author{JC Melchor Pinto}
\begin{document}
\pagestyle{headandfoot}
%\thispagestyle{plain}

\INFO
%\printanswers
%\pagestyle{headandfoot}
\begin{opening}[¿Qué es la energía?]
    {Si la energía nos parece un concepto tan familiar, ¿por qué resulta tan
        difícil definirlo? Tal vez porque la energía es un concepto abstracto: no
        es un objeto o una sustancia. A diferencia de la materia, no podemos
        ver ni tocar la energía y, sin embargo, es uno de los conceptos fundamentales
        de la ciencia, y quizá el más importante de toda la física. Te sorprenderá saber que incluso a Isaac Newton se le escapó
        el concepto de energía, y que más de un siglo después de su muerte los
        científicos aún cuestionaban su existencia.\\[1em]
        Sin embargo, lo que todas las formas de energía tienen en común es
        que pueden transformarse de una forma a otra; por ejemplo, la energía
        eléctrica puede provocar movimiento y transformarse en calor que quizá
        has percibido al encender una licuadora o un ventilador: las aspas se mueven y después de un tiempo el aparato se calienta. Estas transformaciones
        pueden cuantificarse, y el número que resulta es siempre el mismo sin importar la cantidad de transformaciones que sucedan.
        \\[1em]
        Para más información, consulta este enlace: \href{https://youtu.be/NzxAy1k6D8U}{https://youtu.be/NzxAy1k6D8U}
    }
\end{opening}
\begin{questions}
    \include*{../questions/question001}
    \newpage
    \include*{../questions/question002}
    % \newpage
    % \include*{../questions/question003}
\end{questions}

%\vfill
%\puntuacion

\end{document}