\documentclass[12pt,addpoints,answers]{guia}
\grado{2$^\circ$ de Secundaria}
\cicloescolar{2022-2023}
\materia{Física 2}
\guia{5}
\unidad{3}
\title{Conservación de la energía mecánica}
%\unidad{3}
\title{El título de la guía}
\aprendizajes{
    \begin{itemize}[leftmargin=*,label=\small\color{colorrds}\faIcon{user-graduate}]
        \item Analiza la energía mecánica (cinética y potencial) y describe
              casos donde se conserva.
    \end{itemize}
}
\author{JC Melchor Pinto}
\begin{document}
\pagestyle{headandfoot}

\INFO
%\printanswers
\begin{opening}[Práctica interactiva]
    {
    \vspace{-0.7cm}
    \begin{figure}[H]
        \centering
        \includegraphics[width=.12\linewidth]{../images/phet}
    \end{figure}
    Para la realización de esta guía deberas contar con un dispositivo con
    conexión a Internet. Ingresa a la simulación PhET ``Energía en la Pista de
    Patinaje: conceptos básicos'', disponible en el siguiente enlace: \\
    {
    \small
    \url{https://phet.colorado.edu/sims/html/energy-skate-park-basics/latest/energy
        -skate-park-basics_es.html}
    }\\
    \begin{center}
        \begin{minipage}{0.8\textwidth}
            \begin{minipage}{.1\linewidth}
                \color{colorrds}\huge\faHome
            \end{minipage}%
            \begin{minipage}{.90\linewidth}
                \begin{tcolorbox}[width=\linewidth ,halign=left,colframe=rdsdark,arc=3mm, sharp corners=west]
                    Ingresa en la sección llamada ``Introducción''.\\
                \end{tcolorbox}
            \end{minipage}
        \end{minipage}

        \begin{minipage}{0.8\textwidth}
            \begin{minipage}{.1\linewidth}
                \color{colorrds}\huge\faSearch
            \end{minipage}%
            \begin{minipage}{.9\linewidth}
                \begin{tcolorbox}[width=\linewidth,halign=left,colframe=rdsdark,arc=3mm, sharp corners=west]
                    Explora el simulador.\\
                \end{tcolorbox}
            \end{minipage}
        \end{minipage}

        \begin{minipage}{0.8\textwidth}
            \begin{minipage}{.1\linewidth}
                \color{colorrds}\huge\faGamepad
            \end{minipage}%
            \begin{minipage}{.9\linewidth}
                \begin{tcolorbox}[width=\linewidth,halign=left,colframe=rdsdark,arc=3mm, sharp corners=west]
                    Familiarizate con los controles durante algunos minutos.\\
                \end{tcolorbox}
            \end{minipage}
        \end{minipage}

        \begin{minipage}{0.8\textwidth}
            \begin{minipage}{.1\linewidth}
                \color{colorrds}\huge\faCheckSquare[regular]
            \end{minipage}%
            \begin{minipage}{.9\linewidth}
                \begin{tcolorbox}[width=\linewidth,halign=left,colframe=rdsdark,arc=3mm, sharp corners=west]
                    Activa el Gráfico de barras.\\
                \end{tcolorbox}
            \end{minipage}
        \end{minipage}

        \begin{minipage}{0.8\textwidth}
            \begin{minipage}{.1\linewidth}
                \color{colorrds}\huge\faSnowboarding[regular]
            \end{minipage}%
            \begin{minipage}{.9\linewidth}
                \begin{tcolorbox}[width=\linewidth,halign=left,colframe=rdsdark,arc=3mm, sharp corners=west]
                    Toma a la patinadora y muévela por la simulación.\\
                \end{tcolorbox}
            \end{minipage}
        \end{minipage}

    \end{center}
    }
\end{opening}
\begin{questions}
    \include*{../questions/question023}
    %\newpage
    \include*{../questions/question024}
    \include*{../questions/question025}
    \newpage
    \include*{../questions/question026}
    \include*{../questions/question027}
    % \newpage
    \include*{../questions/question028}

    \fullwidth{\begin{closing}[Recapitulando]{
                Para finalizar esta práctica, sintetiza la información asimilada y contesta de forma breve a las siguientes preguntas:

                \begin{itemize}
                    \item ¿De qué depende la energía potencial?
                    \item ¿De qué depende la energía cientica?
                    \item Usando la información aprendida con la  simulación ¿qué dice la ley de la conservación de la energía?
                \end{itemize}
            }
        \end{closing}
    }
\end{questions}

% \vfill
% \puntuacion

\end{document}