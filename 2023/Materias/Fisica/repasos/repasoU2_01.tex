\documentclass[12pt]{repaso}
\grade{2$^\circ$ de Secundaria}
\cycle{2022-2023}
\subject{Física 2}
\guide{2}
\title{Repaso para el examen de la Unidad}
\aprendizajes{

    \begin{itemize}[leftmargin=*,label=\small\color{colorrds}\faIcon{user-graduate}
        ]
        \item Describe, representa y experimenta la fuerza como la interacción
              entre
              objetos y reconoce distintos tipos de fuerza.
        \item Identifica y describe la presencia de fuerzas en interacciones
              cotidianas
              (fricción, flotación, fuerzas en equilibrio).
        \item Analiza la gravitación y su papel en
              la explicación del movimiento de los
              planetas y en la caída de los cuerpos
              (atracción) en la superficie terrestre.
        \item Analiza la energía mecánica (cinética y potencial) y
              describe
              casos donde se conserva.
    \end{itemize}
}
\requisitos{
    \begin{itemize}
        \item Requisito 1
        \item Requisito 2
    \end{itemize}
}
\author{J. C. Melchor Pinto}

\begin{document}
\pagestyle{headandfoot}
%\thispagestyle{plain}
\addpoints
\INFO
%\printanswers
\begin{multicols}{2}
    \include*{../blocks/block001}
    \include*{../blocks/block003}
    \include*{../blocks/block000}
    \include*{../blocks/block002}
\end{multicols}
\newpage
\include*{../blocks/block004}
\begin{questions}

    \include*{../questions/question029}
    \include*{../questions/question030}
    \include*{../questions/question032}
    \newpage
    \include*{../questions/question031}
    \include*{../questions/question033}
    \begin{multicols}{2}
        \include*{../questions/question034}
    \end{multicols}
    \newpage
    \include*{../questions/question035}
    \newpage
    \include*{../questions/question009}
    \newpage
    \include*{../questions/question008}
    \question Observa las imagenes y responde a las preguntas:

    \begin{parts}
        \include*{../parts/question028a}
        \include*{../parts/question028b}
    \end{parts}
\end{questions}
\end{document}