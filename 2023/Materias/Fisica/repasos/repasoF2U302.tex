\documentclass[12pt,addpoints,answers]{repaso}
\grado{2}
\nivel{Secundaria}
\cicloescolar{2023-2024}
\materia{Ciencias y Tecnología: Física}
\unidad{3}
\title{Practica la Unidad}
\aprendizajes{
\item Experimenta e interpreta algunas manifestaciones y aplicaciones de la electricidad. Relaciona e interpreta fenómenos comunes del magnetismo y experimenta con la interacción entre imanes. 
\item Experimenta e interpreta el comportamiento de la luz como resultado de la interacción entre electricidad y magnetismo. Explica el funcionamiento de aparatos tecnológicos de comunicación, a partir de las ondas electromagnéticas.
\item Indaga algunos avances recientes en la comprensión sobre la evolución del Universo y su composición.  Relaciona e interpreta las características y dinámica del Sistema Solar con la gravitación y el movimiento de los planetas.
\item Indaga sobre fenómenos meteorológicos extremos. Propone medidas de mitigación y adaptación, encaminadas al cuidado del medio ambiente y el bienestar común, viables para su aplicación en su escuela y comunidad.
}
\author{Melchor Pinto, J.C.}
\begin{document}
\INFO%
\ejemplosboxed[{
               \begin{multicols}{2}
                    \begin{parts}
                         \part  \fillin[][0.5cm]
                         \begin{solutionbox}{2cm}
                         \end{solutionbox}
                         \part \fillin[][0.5cm]
                         \begin{solutionbox}{2cm}
                         \end{solutionbox}
                    \end{parts}
               \end{multicols}
          }]

\begin{questions}
     \questionboxed[6]{
          \begin{multicols}{3}
               \begin{parts}
                    \part
                    \begin{solutionbox}{2cm}
                    \end{solutionbox}
                    \part
                    \begin{solutionbox}{2cm}
                    \end{solutionbox}
                    \part
                    \begin{solutionbox}{2cm}
                    \end{solutionbox}
               \end{parts}
          \end{multicols}
     }

\end{questions}
\end{document}