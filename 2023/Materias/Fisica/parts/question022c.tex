\part Calcula la energía cinética, potencial y mecánica de la pelota cada segundo, desde que se suelta, es decir, desde t = 0 s, y para el valor del tiempo de
caída. Anota los resultados en una tabla como la Tabla (\ref{tab:pelota}) y grafícalos.
\renewcommand{\arraystretch}{1.6}

\begin{table}[H]
    \centering
    \begin{tabular}{|p{3.5cm}*{11}{|p{0.8cm}}|}
        \hline
        Tiempo [seg.]         & 0 & 1 & 2 & 3 & 4 & 5 & 6 & 7 & 8 & 9 & 10 \\ \hline
        Energía potencial [J] &   &   &   &   &   &   &   &   &   &   &    \\ \hline
        Energía cinética [J]  &   &   &   &   &   &   &   &   &   &   &    \\ \hline
        Energía mecánica [J]  &   &   &   &   &   &   &   &   &   &   &    \\ \hline
    \end{tabular}
    \caption{Cálculos de la energía de la pelota}
    \label{tab:pelota}
\end{table}