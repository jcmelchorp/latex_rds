\part El cometa Halley se desplaza en una órbita elíptica alrededor del Sol. A medida que se aleja del Sol, pierde rapidez.

\textbf{¿Cuál de las siguientes opciones explica mejor por qué el cometa pierde rapidez en este recorrido?}

\begin{choices}
    \choice El Sol ejerce un torque sobre el cometa Halley durante su desplazamiento alrededor del Sol, lo que disminuye el momento angular del sistema Sol-cometa.
    \CorrectChoice Cuando el cometa Halley se aleja del Sol, el sistema Sol-cometa gana energía potencial y el cometa Halley pierde energía cinética.
    \choice La fuerza centrípeta que se ejerce sobre el cometa Halley es menor que la fuerza gravitacional que se ejerce sobre él en esta parte de la órbita.
    \choice Un componente de la fuerza gravitacional sobre el cometa Halley es perpendicular a la dirección de su movimiento, lo que produce que el cometa pierda rapidez.
\end{choices}
