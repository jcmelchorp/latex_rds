¿Puede un objeto en una playa tener la misma energía potencial gravitacional
que otro de la misma masa que está en la Ciudad de México a una altitud de 2,240 m sobre el nivel del mar? Explica.


\begin{solutionbox}{1.6cm}
    No, mientras los dos objetos compartan el marco de referencia, es decir que en
    ambos midan su altura desde el mismo punto, el objeto que se encuentre a una
    altura mayor tendrá mayor energía potencial.
\end{solutionbox}
