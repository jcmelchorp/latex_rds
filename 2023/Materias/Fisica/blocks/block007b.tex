\begin{warncard}[adjusted title={Composición y descomposición de la luz blanca}]
    Como vimos, una de las manifestaciones del modelo atómico son los espectros luminosos que se forman a partir de los \comillas{saltos} de los electrones entre sus órbitas atómicas.
    Cada vez que un electrón \comillas{regresa} a
    su orbital atómico de origen después
    de ser excitado, emite luz (u otro tipo
    de radiación), la cual forma los espectros conocidos. Revisemos el espectro
    de emisión de la luz de mercurio (figura 3.30).
    Observa que los colores están contenidos dentro del arco iris y están ordenados de
    la misma manera. En general, cualquier tipo de luz puede descomponerse en colores;
    el arcoíris es una manifestación de la descomposición de la luz solar. Entre sus diversos experimentos de óptica, cuando Newton hizo pasar luz blanca a través de un pris-
    ma de vidrio observó la descomposición de la luz en colores y así comprobó que la
    luz blanca es en realidad una mezcla de los colores del arcoíris. Newton trató de explicar este fenómeno mediante el modelo corpuscular (modelo de partículas) al con-
    siderar que la luz, al igual que la materia, estaba constituida por partículas de distintos
    tamaños, y pensó que, según su tamaño, cada partícula producía en el ojo una sensación de color distinta. Esta idea despertó las críticas entre sus contemporáneos; más
    de 100 años después, James Clerk Maxwell (1831-1879) descubrió que la luz está
    constituida por ondas electromagnéticas.
\end{warncard}