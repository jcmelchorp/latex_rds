\begin{sectionbox}{Composición y descomposición de la luz blanca}
    En general, cualquier tipo de luz puede descomponerse en colores;
           el arcoíris es una manifestación de la descomposición de la luz solar. Entre sus diversos experimentos de óptica, cuando Newton hizo pasar luz blanca a través de un prisma de vidrio observó la descomposición de la luz en colores y así comprobó que la
           luz blanca es en realidad una mezcla de los colores del arcoíris. Newton trató de explicar este fenómeno mediante el modelo corpuscular (modelo de partículas) al considerar que la luz, al igual que la materia, estaba constituida por partículas de distintos
           tamaños, y pensó que, según su tamaño, cada partícula producía en el ojo una sensación de color distinta. Esta idea despertó las críticas entre sus contemporáneos; más
           de 100 años después, \textbf{James Clerk Maxwell (1831-1879)} descubrió que la luz está
           constituida por ondas electromagnéticas.
           
           Los fenómenos de reflexión y refracción de la luz son un buen punto de partida para
           estudiar las propiedades físicas de la luz como onda.
   
           Cuando la luz incide sobre un objeto pueden suceder las siguientes situaciones, aunque
           en distinto grado.
   
           \begin{itemize}
               \item El objeto absorbe casi toda la luz.
               \item El objeto sólo absorbe una parte de la luz (lo que da origen a los colores).
               \item Nada de la luz que llega la absorbe el objeto.
           \end{itemize}
   
           Si un objeto no permite que pase luz a través de él, pero tampoco la absorbe, como sucede en los espejos, toda esa luz rebota en su superficie, fenómeno que se conoce
           como reflexión. ¿Recuerdas que este fenómeno también lo observaste para una onda mecánica? Los objetos en los que puedes distinguir algún color sólo reflejan una fracción de
           la luz blanca que les llega (la que corresponde a su color) y absorben la otra parte. Otros materiales, como el aire, el agua y el vidrio, permiten que casi toda la luz pase a través de ellos, y por ello no tienen color:
           son objetos transparentes. Cuando vemos algo es porque refleja
           parte de la luz que recibe. El haz de luz que llega a una
           superficie recibe el nombre de rayo incidente, y el que rebota se conoce como rayo reflejado. En el experimento con
           el láser el rayo de luz incidente es el que sale del láser, y el rayo de luz que llega a la hoja de papel es el reflejado.
   
           La recta imaginaria, que es perpendicular a la superficie, se llama recta normal. El
           ángulo de incidencia y el ángulo de reflexión se forman entre la recta normal y los rayos de incidencia y de reflexión, respectivamente; siempre miden lo mismo (figura 3.39).
           En tu experimento tal vez determinaste que estos ángulos son parecidos, pero no iguales; esto se debe a que la actividad y las mediciones no son del todo precisas.
           Por otra parte, en el experimento del recipiente con la moneda notaste que al verter
           agua en el recipiente volvías a ver la moneda como si se hubiera movido de su posición
           original. Por supuesto, la moneda no se movió; lo que sucede es que cuando la luz pasa
           de un medio a otro (por ejemplo del aire al agua) se desvía, fenómeno que se conoce como
           refracción. En el experimento la luz que refleja la moneda se refracta
           al salir de la superficie del agua y eso hace que la puedas ver.
           Los fenómenos de reflexión y de refracción de la luz se explican
           con base en la hipótesis de que la luz es una onda. Imagina que
           sujetas firmemente el extremo de un resorte a una pared, y el otro
           extremo lo comprimes y estiras una vez; se formará así una onda
           longitudinal en el resorte y en el momento que esa onda choque
           con la pared regresará hacia ti; de esta manera la onda se está
           reflejando. En forma análoga, cuando un rayo de luz llega a la superficie de un espejo, las ondas de luz chocan y se reflejan.
           La luz blanca, con sus colores, se encuentra en una pequeña parte
           del espectro electromagnético que se conoce como espectro visible.
           Cada uno de los colores que componen la luz visible tiene asociada
           una longitud de onda, como se observa en la infografía de la lección
           anterior. Para el ojo humano es imposible ver más allá del espectro
           visible, tanto para mayores longitudes de onda (después del infrarrojo) como para menores (antes del ultravioleta).

       \end{sectionbox}