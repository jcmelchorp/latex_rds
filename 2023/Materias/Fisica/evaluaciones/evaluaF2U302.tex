\documentclass[12pt,addpoints]{evalua}
\grado{2$^\circ$ de Secundaria}
\cicloescolar{2023-2024}
\materia{Ciencias y Tecnología: Física}
\unidad{2}
\title{Examen de la Unidad}
\aprendizajes{\footnotesize%
    \item Describe la generación, diversidad y comportamiento de las ondas
    electromagnéticas como resultado de la interacción entre electricidad y
    magnetismo.
    \item Describe cómo se lleva a cabo la exploración de los cuerpos
    celestes por medio de la detección de las ondas electromagnéticas que emiten.
    \item Describe algunos avances en las características y composición del
    Universo (estrellas, galaxias y otros sistemas).
    \item Describe las características y dinámica del Sistema Solar.
    \item Identifica algunos aspectos sobre la evolución del Universo.
}
\author{Prof.: Julio César Melchor Pinto}
\begin{document}
\begin{multicols}{2}
    \include*{../blocks/block006b}
    \include*{../blocks/block006c}
\end{multicols}%
\begin{questions}    \large
    \question[5] \include*{../questions/question087a}
    \newpage
    %\question[4] \include*{../questions/question086a}
    %\question[6] \include*{../questions/question079b}
    \question[6]{
        Elige la respuesta correcta.

        \begin{multicols}{2}
            \begin{parts}
                \part La relación de proporcionalidad entre la velocidad con la que se
                alejan las galaxias y la distancia a la que se encuentran.
                %(pág. \pageref{086a_c})

                \begin{choices}
                    \choice Ley de Hook
                    \choice Ley de Faraday
                    \CorrectChoice Ley de Hubble
                    \choice Ley de Moore
                \end{choices}

                \part Pulso eléctrico que se propaga a través de la neurona.

                \begin{choices}
                    \CorrectChoice Potencial de acción
                    \choice Potencial eléctrico
                    \choice Potencial magnético
                    \choice Energía potencial
                \end{choices}

                \part Perturbación eléctrica que se genera cuando una neurona recibe un estímulo.
                \begin{choices}
                    \choice Impulso eléctrico
                    \CorrectChoice Impulso nervioso
                    \choice Impulso magnético
                    \choice Impulso atómico
                \end{choices}

                \part Pulso eléctrico que se propaga a través de la neurona.
                \begin{choices}
                    \CorrectChoice Potencial de acción
                    \choice Potencial eléctrico
                    \choice Potencial magnético
                    \choice Energía potencial
                \end{choices}
                \part Células receptoras de luz capaces de percibir colores, pero para que funcionen es necesario que haya suficiente luz.

                \begin{choices}
                    \choice Bastones
                    \choice Esferas
                    \CorrectChoice Conos
                    \choice Rizos
                \end{choices}

                \part Perturbación eléctrica que se genera cuando una neurona recibe un estímulo.

                \begin{choices}
                    \choice Impulso eléctrico
                    \CorrectChoice Impulso nervioso
                    \choice Impulso magnético
                    \choice Impulso atómico
                \end{choices}





                }
            \end{parts}
        \end{multicols}
    }


    \question[20] \include*{../questions/question076a}
    \newpage
    \question[10] \include*{../questions/question088b}
    \newpage
    \question[20] \include*{../questions/question085d}
    \newpage
    \question[10] \include*{../questions/question090a}
    \question[10] \include*{../questions/question089b}
    \newpage
    \question[10] \include*{../questions/question087c}
    % \question[15] \include*{../questions/question085c}
\end{questions}
\end{document}