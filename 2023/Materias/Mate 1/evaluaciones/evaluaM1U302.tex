\documentclass[12pt,addpoints]{evalua}
\grado{1$^\circ$ de Secundaria}
\cicloescolar{2023-2024}
\materia{Matemáticas 1}
\unidad{3}
\title{Examen de la Unidad}
\aprendizajes{
    \item Calcula valores faltantes en problemas de proporcionalidad directa, con constante natural, fracción o decimal (incluyendo tablas de variación).
}
     \author{Prof.: Julio César Melchor Pinto}
\begin{document}
\begin{questions}
    \section*{\ifprintanswers   {Sucesiones}\else{}\fi}
    \subsection*{\ifprintanswers{Completa la sucesión aritmética 1}\else{}\fi}
    \subsection*{\ifprintanswers{Completa la sucesión aritmética 2}\else{}\fi}
    \subsection*{\ifprintanswers{Completa la sucesión geométrica}\else{}\fi}
    \subsection*{\ifprintanswers{Diferencia de una sucesión}\else{}\fi}
    \subsection*{\ifprintanswers{Término de una sucesión}\else{}\fi}
    \section*{\ifprintanswers   {Proporcionalidad y estadística}\else{}\fi}
    \subsection*{\ifprintanswers{Razones y proporciones}\else{}\fi}
    \subsection*{\ifprintanswers{Variación directa e inversa}\else{}\fi}
    \subsection*{\ifprintanswers{Mediana y moda}\else{}\fi}
    \subsection*{\ifprintanswers{Promedio}\else{}\fi}
    \subsection*{\ifprintanswers{Interpretación de gráficas}\else{}\fi}
    \section*{\ifprintanswers   {Círculo}\else{}\fi}
    \subsection*{\ifprintanswers{Diámetro de un círculo}\else{}\fi}
    \subsection*{\ifprintanswers{Radio de un círculo}\else{}\fi}
    \subsection*{\ifprintanswers{Perímetro}\else{}\fi}
    \subsection*{\ifprintanswers{Área}\else{}\fi}
    \subsection*{\ifprintanswers{Resolución de problemas}\else{}\fi}
    \section*{\ifprintanswers   {Ecuaciones}\else{}\fi}
    \subsection*{\ifprintanswers{Lenguaje algebraico}\else{}\fi}
    \subsection*{\ifprintanswers{Ecuaciones x+a=b}\else{}\fi}
    \subsection*{\ifprintanswers{Ecuaciones ax=b}\else{}\fi}
    \subsection*{\ifprintanswers{Ecuaciones ax+b=c}\else{}\fi}
    \subsection*{\ifprintanswers{Resolución de problemas}\else{}\fi}
    \section*{\ifprintanswers   {Figuras y cuerpos geométricos}\else{}\fi}
    \subsection*{\ifprintanswers{Perímetro}\else{}\fi}
    \subsection*{\ifprintanswers{Área}\else{}\fi}
    \subsection*{\ifprintanswers{Área lateral y total}\else{}\fi}
    \subsection*{\ifprintanswers{Volumen}\else{}\fi}
    \subsection*{\ifprintanswers{Resolución de problemas}\else{}\fi}
    \question[10]
\end{questions}
\end{document}