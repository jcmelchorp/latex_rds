\documentclass[12pt,addpoints,answers]{evalua}
\grado{1$^\circ$ de Secundaria}
\cicloescolar{2023-2024}
\materia{Matemáticas 1}
\unidad{2}
\title{Examen de la Unidad}
\aprendizajes{
    \item Determina y usa la jerarquía de operaciones y los paréntesis en
    operaciones con números naturales, enteros y
    decimales (para multiplicación y división, sólo números
    positivos).
\item Resuelve problemas de cálculo de porcentajes, de tanto por
    ciento y de la cantidad base.
}
     \author{Prof.: Julio César Melchor Pinto}
\begin{document}
\begin{questions}
    \section*{\ifprintanswers{Operaciones con decimales}\else{}\fi}
    \subsection*{\ifprintanswers{Suma de decimales}\else{}\fi}


    \subsection*{\ifprintanswers{Resta de decimales}\else{}\fi}
    \question[10] Realiza las siguientes restas de decimales:
    \begin{multicols}{2}
        \begin{parts}
            \part \opsub[hfactor=decimal,resultstyle=\color{white},displayintermediary=None,carryadd=false,carrysub=false]{16.991}{15.981}
            \part \opsub[hfactor=decimal,resultstyle=\color{white},displayintermediary=None,carryadd=false,carrysub=false]{24.97}{19.34}
        \end{parts}
    \end{multicols}

    \subsection*{\ifprintanswers{Multiplicación de decimales}\else{}\fi}
    \question[10] Realiza las siguientes multiplicaciones de decimales:
    \begin{multicols}{2}

        \begin{parts}
            \part \opmul[hfactor=decimal,resultstyle=\color{white},displayintermediary=None,carryadd=false,carrysub=false]{4.5}{2.3}
            \part \opmul[hfactor=decimal,resultstyle=\color{white},displayintermediary=None,carryadd=false,carrysub=false]{17.31}{4.81}
        \end{parts}
    \end{multicols}

    \subsection*{\ifprintanswers{División de decimales}\else{}\fi}
    \question[10] Realiza las siguientes divisiones de decimales:
    \begin{multicols}{2}
        \begin{parts}
            % \part \longdivision[stage=0]{4.5}{2.3}
            % \part \longdivision[stage=0]{17.31}{4.81}
            \part $4.5 \divisionsymbol 2.3=$
            \begin{solutionbox}{1.5cm}
            \end{solutionbox}
            \part $17.31 \divisionsymbol 4.81=$
            \begin{solutionbox}{1.5cm}
            \end{solutionbox}
        \end{parts}
    \end{multicols}

    \subsection*{\ifprintanswers{Resolución de problemas}\else{}\fi}
    \question[10] Resuelve los siguientes problemas:
    \begin{parts}
        \part Una pintura tiene un costo de 33.24 pesos el litro, una persona compra 53 litros. ¿Cuánto debe pagar?
        \begin{solutionbox}{2cm}
            \opmul[hfactor=decimal,resultstyle=\color{white},displayintermediary=None,carryadd=false,carrysub=false]{33.24}{53}
        \end{solutionbox}

        \part La mamá de Susana compró 11 metros de franela y pagó 103.40 pesos. ¿Cuánto cuesta el metro de franela?
        \begin{solutionbox}{2cm}
            \opdiv[hfactor=decimal,resultstyle=\color{white},displayintermediary=None,carryadd=false,carrysub=false]{103.40}{11}
        \end{solutionbox}

        \part El precio de 385 artículos comerciales es de 1,232 pesos. ¿Cuál es el precio unitario de cada artículo?
        \begin{solutionbox}{2cm}
            \opdiv[hfactor=decimal,resultstyle=\color{white},displayintermediary=None,carryadd=false,carrysub=false]{1232}{385}
        \end{solutionbox}

    \end{parts}



    \section*{\ifprintanswers{Operaciones con fracciones}\else{}\fi}
    \subsection*{\ifprintanswers{Suma y resta con denominadores iguales}\else{}\fi}
    \question[10] Realiza las siguientes sumas de fracciones con denominadores iguales:
    \begin{parts}
        \part $\dfrac{3}{5}+\dfrac{2}{5}=$ \fillin[$\dfrac{5}{5}=1$][1in]
        \part $\dfrac{7}{8}+\dfrac{3}{8}=$ \fillin[$\dfrac{10}{8}=\dfrac{5}{4}$][1in]
    \end{parts}

    \subsection*{\ifprintanswers{Suma y resta denominadores diferentes}\else{}\fi}
    \question[10] Realiza las siguientes sumas de fracciones con denominadores diferentes:
    \begin{parts}
        \part $\dfrac{3}{5}+\dfrac{2}{3}=$ \fillin[$\dfrac{9}{15}+\dfrac{10}{15}=\dfrac{19}{15}$][1in]
        \part $\dfrac{7}{8}+\dfrac{3}{4}=$ \fillin[$\dfrac{7}{8}+\dfrac{6}{8}=\dfrac{13}{8}$][1in]
    \end{parts}

    \subsection*{\ifprintanswers{Multiplicación de fracciones}\else{}\fi}
    \question[10] Realiza las siguientes multiplicaciones de fracciones:
    \begin{parts}
        \part $\dfrac{3}{5}\times\dfrac{2}{3}=$ \fillin[$\dfrac{6}{15}$][1in]
        \part $\dfrac{7}{8}\times\dfrac{3}{4}=$ \fillin[$\dfrac{21}{32}$][1in]
    \end{parts}

    \subsection*{\ifprintanswers{División de fracciones}\else{}\fi}
    \question[10] Realiza las siguientes divisiones de fracciones:
    \begin{parts}
        \part $\dfrac{3}{5}\divisionsymbol\dfrac{2}{3}=$ \fillin[$\dfrac{9}{10}$][1in]
        \part $\dfrac{7}{8}\divisionsymbol\dfrac{3}{4}=$ \fillin[$\dfrac{28}{24}$][1in]
    \end{parts}

    \subsection*{\ifprintanswers{Resolución de problemas}\else{}\fi}
    \question[10] Resuelve los siguientes problemas:
    \begin{parts}
        \part Un granjero siembra 2/5 de su granja con maíz y 3/10 con soya, ¿qué cantidad de su granja queda por sembrar?
        \begin{solutionbox}{2.5cm}
            Para conocer la cantidad de su granja que queda por sembrar, se debe restar 2/5 y 3/10 a 1; entonces:
            \[1-\dfrac{2}{5}-\dfrac{3}{10}=\dfrac{10}{10}-\dfrac{4}{10}-\dfrac{3}{10}=\dfrac{3}{10}\]

        \end{solutionbox}

        \part Un reloj se adelanta 3/7 de minuto cada hora. ¿Cuánto se adelantará en 5 horas?
        \begin{solutionbox}{2.5cm}
            Para conocer cuánto se adelantará en 5 horas, se debe multiplicar 3/7 por 5; entonces:
            \[\dfrac{3}{7}\times 5=\dfrac{15}{7}\]
        \end{solutionbox}
    \end{parts}

    \section*{\ifprintanswers{Porcentajes}\else{}\fi}
    \subsection*{\ifprintanswers{Porcentajes a decimal}\else{}\fi}
    \question[10] Escribe como decimal los siguientes porcentajes:
    \begin{parts}
        \part 25\% = \fillin[0.25][1in]
        \part 75\% = \fillin[0.75][1in]
        \part 50\% = \fillin[0.5][1in]
        \part 10\% = \fillin[0.1][1in]
        \part 5\% = \fillin[0.5][1in]
        \part 0.5\% = \fillin[0.1][1in]
    \end{parts}
    \subsection*{\ifprintanswers{Decimal a porcentaje}\else{}\fi}
    \question[10] Escribe como porcentaje los siguientes decimales:
    \begin{parts}
        \part $0.52$  = \fillin[52\%][1in]
        \part $0.09$  = \fillin[9\%][1in]
        \part $6.5$   = \fillin[650\%][1in]
        \part $0.704$ = \fillin[70.4\%][1in]
        \part $0.1$   = \fillin[10\%][1in]
        \part $1$     = \fillin[100\%][1in]
    \end{parts}
    \subsection*{\ifprintanswers{Porcentaje de cantidades}\else{}\fi}
    \question[10] Calcula el porcentaje de las siguientes cantidades:
    \begin{parts}
        \part 80\% de 250 = \fillin[200][1in]
        \part 15\% de 900 = \fillin[135][1in]
        \part 50\% de 600 = \fillin[300][1in]
        \part 13\% de 1200 = \fillin[156][1in]
        \part 5\% de 715 = \fillin[35.75][1in]
        \part 35\% de 415 = \fillin[145.25][1in]
        \part Si se sabe que 210 es el 21\% de cierta cantidad, ¿cuál es esta cantidad?
        \begin{solutionbox}{2.5cm}
            Para conocer la cantidad, se debe dividir 210 entre 21; entonces:
            \[100\times\dfrac{210}{21}=10\]
        \end{solutionbox}

        \part Si se sabe que 200 es el 250\% de cierta cantidad, ¿cuál es esta cantidad?
        \begin{solutionbox}{2.5cm}
            Para conocer la cantidad, se debe dividir 200 entre 250; entonces:
            \[100\times\dfrac{200}{250}=80\]
        \end{solutionbox}

        \part Si se sabe que 120 es el 35\% de cierta cantidad, ¿cuál es esta cantidad?
        \begin{solutionbox}{2.5cm}
            Para conocer la cantidad, se debe dividir 120 entre 35; entonces:
            \[100\times\dfrac{120}{35}=342.86\]
        \end{solutionbox}

    \end{parts}
    \subsection*{\ifprintanswers{Resolución de problemas}\else{}\fi}
    \question[10] Resuelve los siguientes problemas:
    \begin{parts}
        \part El costo de una computadora es de 12220 pesos, si la tasa de impuesto es del 15\%. ¿Cuánto será el total a pagar por la computadora?
        \begin{solutionbox}{2.5cm}
            Para conocer el total a pagar por la computadora, se debe multiplicar 12220 por 15\%; entonces:
            \[12220\times 115\%= 14053\]
            Por lo tanto, el total a pagar por la computadora es de 14053 pesos.
        \end{solutionbox}

        \part El 24\% de los habitantes de un pueblo tienen menos de 30 años. ¿Cuántos habitantes tiene el pueblo si hay 120 jóvenes menores de 30 años?
        \begin{solutionbox}{2.5cm}
            Para conocer el total de habitantes del pueblo, se debe dividir 120 entre 24\%; entonces:
            \[100\times\dfrac{120}{24}=500\]
            Por lo tanto, el pueblo tiene 500 habitantes.
        \end{solutionbox}
    \end{parts}

    \section*{\ifprintanswers{Potencias y raíces}\else{}\fi}
    \subsection*{\ifprintanswers{Potenciación}\else{}\fi}
    \question[10] Realiza las siguientes potencias:
    \begin{parts}
        \part $2^3=$ \fillin[8][1in]
        \part $3^2=$ \fillin[9][1in]
        \part $5^2=$ \fillin[25][1in]
        \part $10^4=$ \fillin[10000][1in]
        \part $3^5=$ \fillin[243][1in]
        \part $\left(\dfrac{1}{3}\right)^3=$ \fillin[$\dfrac{1}{27}$][1in]
        \part $\left(\dfrac{2}{3}\right)^4=$ \fillin[$\dfrac{16}{81}$][1in]
        \part $\left(\dfrac{1}{9}\right)^2=$ \fillin[$\dfrac{1}{81}$][1in]
        \part $\left(\dfrac{4}{3}\right)^2=$ \fillin[$\dfrac{1}{1000}$][1in]
        \part $\left(\dfrac{3}{2}\right)^5=$ \fillin[$\dfrac{1}{8}$][1in]
    \end{parts}

    \subsection*{\ifprintanswers{Notación científica 1}\else{}\fi}

    \subsection*{\ifprintanswers{Notación científica 2}\else{}\fi}
    \subsection*{\ifprintanswers{Raíces}\else{}\fi}

    \section*{\ifprintanswers{Sistema de unidades}\else{}\fi}
    \subsection*{\ifprintanswers{Unidades de longitud}\else{}\fi}
    \subsection*{\ifprintanswers{Unidades de masa}\else{}\fi}
    \subsection*{\ifprintanswers{Unidades de capacidad}\else{}\fi}
    \subsection*{\ifprintanswers{Unidades de área y volumen}\else{}\fi}
    \subsection*{\ifprintanswers{Unidades de capacidad 2}\else{}\fi}
\end{questions}
\end{document}