\documentclass[12pt,addpoints]{evalua}
\grado{1$^\circ$ de Secundaria}
\cicloescolar{2023-2024}
\materia{Matemáticas 1}
\unidad{2}
\title{Examen de la Unidad}
\aprendizajes{
    \item Determina\footnotesize% y usa la jerarquía de operaciones y los paréntesis en
    operaciones con números naturales, enteros y
    decimales (para multiplicación y división, sólo números
    positivos).
\item Resuelve problemas de cálculo de porcentajes, de tanto por
    ciento y de la cantidad base.
}
\author{Prof.: Julio César Melchor Pinto}
\begin{document}
\begin{questions}
    \question[5]{Realiza las siguientes sumas de decimales:
        \begin{multicols}{3}
            \begin{parts}
                \part \opadd[hfactor=decimal,resultstyle=\color{white},displayintermediary=None,carryadd=false,carrysub=false]{16.981}{15.891}
                \part \opadd[hfactor=decimal,resultstyle=\color{white},displayintermediary=None,carryadd=false,carrysub=false]{620.64}{515.92}
                \part \opadd[hfactor=decimal,resultstyle=\color{white},displayintermediary=None,carryadd=false,carrysub=false]{24.97}{19.34}
                % \part \opadd[hfactor=decimal,resultstyle=\color{white},displayintermediary=None,carryadd=false,carrysub=false]{509.44}{338.79}
                % \part \opadd[hfactor=decimal,resultstyle=\color{white},displayintermediary=None,carryadd=false,carrysub=false]{33.31}{19.54}
            \end{parts}
        \end{multicols}
    }

    \question[5]{ Realiza las siguientes restas de decimales:
        \begin{multicols}{3}
            \begin{parts}
                \part \opsub[hfactor=decimal,resultstyle=\color{white},displayintermediary=None,carryadd=false,carrysub=false]{55.394}{49.093}
                \part \opsub[hfactor=decimal,resultstyle=\color{white},displayintermediary=None,carryadd=false,carrysub=false]{5.134}{2.347}
                \part \opsub[hfactor=decimal,resultstyle=\color{white},displayintermediary=None,carryadd=false,carrysub=false]{968.31}{134.67}
            \end{parts}
        \end{multicols}
    }

    \question[5]{ Realiza las siguientes multiplicaciones de decimales:
        \begin{multicols}{3}
            \begin{parts}
                \part \opmul[hfactor=decimal,resultstyle=\color{white},displayintermediary=None]{17.31}{4.81}
                \part \opmul[hfactor=decimal,resultstyle=\color{white},displayintermediary=None]{12.34}{7.21}
                \part \opmul[hfactor=decimal,resultstyle=\color{white},displayintermediary=None]{198.4}{12.2}
            \end{parts}
        \end{multicols}
    }

    \question[5]{ Realiza las siguientes divisiones con decimales:
        \begin{multicols}{3}
            \begin{parts}
                \part ${12.4}\divisionsymbol{5.1}=$
                \part ${8.32}\divisionsymbol{1.2}=$
                \part ${54}\divisionsymbol{2.5}=$
            \end{parts}
        \end{multicols}
    }

    \question[5]{Resuelve los siguientes problemas:
        \begin{parts}
            \part Una pintura tiene un costo de 33.24 pesos el litro, una persona compra 53 litros. ¿Cuánto debe pagar?
            \begin{solutionbox}{2cm}
                \opmul[hfactor=decimal,resultstyle=\color{white},displayintermediary=None,carryadd=false,carrysub=false]{33.24}{53}
            \end{solutionbox}

            \part La mamá de Susana compró 11 metros de franela y pagó 103.40 pesos. ¿Cuánto cuesta el metro de franela?
            \begin{solutionbox}{2cm}
                \opdiv[hfactor=decimal,resultstyle=\color{white},displayintermediary=None,carryadd=false,carrysub=false]{103.40}{11}
            \end{solutionbox}

            \part El precio de 385 artículos comerciales es de 1,232 pesos. ¿Cuál es el precio unitario de cada artículo?
            \begin{solutionbox}{2cm}
                \opdiv[hfactor=decimal,resultstyle=\color{white},displayintermediary=None,carryadd=false,carrysub=false]{1232}{385}
            \end{solutionbox}

        \end{parts}
    }

    \question[5]{ Realiza las siguientes sumas y restas de fracciones con denominadores iguales:
        \begin{multicols}{3}
            \begin{parts}
                \part $\dfrac{3}{5}+\dfrac{2}{5}=$ \fillin[$\dfrac{5}{5}=1$][1in]
                \part $\dfrac{7}{8}-\dfrac{3}{8}=$ \fillin[$\dfrac{5}{8}$][1in]
                \part $\dfrac{37}{12}-\dfrac{11}{12}=$ \fillin[$\dfrac{13}{6}=1$][1in]
                \part $\dfrac{7}{4}+\dfrac{3}{4}=$ \fillin[$\dfrac{}{}+\dfrac{}{}=\dfrac{}{}$][1in]
                \part $\dfrac{1}{3}+\dfrac{2}{3}=$ \fillin[$\dfrac{}{}+\dfrac{}{}=\dfrac{}{}$][1in]
                \part $\dfrac{5}{4}+\dfrac{8}{4}=$ \fillin[$\dfrac{}{}+\dfrac{}{}=\dfrac{}{}$][1in]
            \end{parts}
        \end{multicols}
    }

    \question[5]{ Realiza las siguientes sumas y restas de fracciones con denominadores diferentes:
        \begin{multicols}{3}
            \begin{parts}
                \part $\dfrac{3}{5}+\dfrac{2}{3}=$ \fillin[$\dfrac{9}{15}+\dfrac{10}{15}=\dfrac{19}{15}$][1in]
                \part $\dfrac{7}{8}+\dfrac{3}{4}=$ \fillin[$\dfrac{7}{8}+\dfrac{6}{8}=\dfrac{13}{8}$][1in]
                \part $\dfrac{3}{5}+\dfrac{2}{3}=$ \fillin[$\dfrac{9}{5}+\dfrac{0}{15}=\dfrac{19}{15}$][1in]
                \part $\dfrac{7}{8}+\dfrac{3}{4}=$ \fillin[$\dfrac{}{}+\dfrac{}{}=\dfrac{}{}$][1in]
                \part $\dfrac{3}{5}+\dfrac{2}{3}=$ \fillin[$\dfrac{}{}+\dfrac{}{}=\dfrac{}{}$][1in]
                \part $\dfrac{7}{8}+\dfrac{3}{4}=$ \fillin[$\dfrac{}{}+\dfrac{}{}=\dfrac{}{}$][1in]
            \end{parts}
        \end{multicols}
    }

    \question[5]{ Realiza las siguientes multiplicaciones de fracciones:
        \begin{multicols}{3}
            \begin{parts}
                \part $\dfrac{3}{5}\times\dfrac{2}{3}=$ \fillin[$\dfrac{6}{15}$][1in]
                \part $\dfrac{7}{8}\times\dfrac{3}{4}=$ \fillin[$\dfrac{21}{32}$][1in]
                \part $\dfrac{7}{8}\times\dfrac{9}{4}=$ \fillin[$\dfrac{}{}+\dfrac{}{}=\dfrac{}{}$][1in]
                \part $\dfrac{7}{8}\times\dfrac{7}{4}=$ \fillin[$\dfrac{}{}+\dfrac{}{}=\dfrac{}{}$][1in]
                \part $\dfrac{3}{5}\times\dfrac{2}{3}=$ \fillin[$\dfrac{}{}+\dfrac{}{}=\dfrac{}{}$][1in]
                \part $\dfrac{7}{8}\times\dfrac{3}{4}=$ \fillin[$\dfrac{}{}+\dfrac{}{}=\dfrac{}{}$][1in]
            \end{parts}
        \end{multicols}
    }

    \question[5]{ Realiza las siguientes divisiones de fracciones:
        \begin{multicols}{3}
            \begin{parts}
                \part $ \dfrac{3}{5} \divisionsymbol\dfrac{2}{3}=$ \fillin[$\dfrac{9}{10}$][1in]
                \part $ \dfrac{7}{8} \divisionsymbol\dfrac{3}{4}=$ \fillin[$\dfrac{}{}$][1in]
                \part $ \dfrac{7}{8} \divisionsymbol\dfrac{9}{4}=$ \fillin[$\dfrac{}{}$][1in]
                \part $ \dfrac{7}{8} \divisionsymbol\dfrac{7}{4}=$ \fillin[$\dfrac{}{}$][1in]
                \part $ \dfrac{3}{5} \divisionsymbol\dfrac{2}{3}=$ \fillin[$\dfrac{}{}$][1in]
                \part $ \dfrac{7}{8} \divisionsymbol\dfrac{3}{4}=$ \fillin[$\dfrac{}{}$][1in]
            \end{parts}
        \end{multicols}
    }

    \question[5]{ Resuelve los siguientes problemas:
        \begin{parts}
            \part Un granjero siembra 2/5 de su granja con maíz y 3/10 con soya, ¿qué cantidad de su granja queda por sembrar?
            \begin{solutionbox}{2.5cm}
                Para conocer la cantidad de su granja que queda por sembrar, se debe restar 2/5 y 3/10 a 1; entonces:
                \[1-\dfrac{2}{5}-\dfrac{3}{10}=\dfrac{10}{10}-\dfrac{4}{10}-\dfrac{3}{10}=\dfrac{3}{10}\]

            \end{solutionbox}

            \part Un reloj se adelanta 3/7 de minuto cada hora. ¿Cuánto se adelantará en 5 horas?
            \begin{solutionbox}{2.5cm}
                Para conocer cuánto se adelantará en 5 horas, se debe multiplicar 3/7 por 5; entonces:
                \[\dfrac{3}{7}\times 5=\dfrac{15}{7}\]
            \end{solutionbox}
        \end{parts}
    }

    \question[5]{ Escribe como decimal los siguientes porcentajes:
        \begin{multicols}{3}
            \begin{parts}
                \part 25\% = \fillin[0.25][1in]
                \part 75\% = \fillin[0.75][1in]
                \part 50\% = \fillin[0.5][1in]
                \part 10\% = \fillin[0.1][1in]
                \part 5\% = \fillin[0.5][1in]
                \part 0.5\% = \fillin[0.1][1in]
            \end{parts}
        \end{multicols}
    }

    \question[5]{ Escribe como porcentaje los siguientes decimales:
        \begin{multicols}{3}
            \begin{parts}
                \part $0.52$  = \fillin[52\%][1in]
                \part $0.09$  = \fillin[9\%][1in]
                \part $6.5$   = \fillin[650\%][1in]
                \part $0.704$ = \fillin[70.4\%][1in]
                \part $0.1$   = \fillin[10\%][1in]
                \part $1$     = \fillin[100\%][1in]
            \end{parts}
        \end{multicols}
    }

    \question[5]{ Calcula el porcentaje de las siguientes cantidades:
        \begin{multicols}{3}
            \begin{parts}
                \part 80\% de 250 = \fillin[200][0.5in]
                \part 15\% de 900 = \fillin[135][0.5in]
                \part 50\% de 600 = \fillin[300][0.5in]
                \part 13\% de 1200 = \fillin[156][0.5in]
                \part 5\% de 715 = \fillin[35.75][0.5in]
                \part 35\% de 415 = \fillin[145.25][0.5in]

                \part Si se sabe que 210 es el 21\% de cierta cantidad, ¿cuál es esta cantidad?
                \begin{solutionbox}{1cm}
                    Para conocer la cantidad, se debe dividir 210 entre 21; entonces:
                    \[100\times\dfrac{210}{21}=10\]
                \end{solutionbox}

                \part Si se sabe que 200 es el 250\% de cierta cantidad, ¿cuál es esta cantidad?
                \begin{solutionbox}{1cm}
                    Para conocer la cantidad, se debe dividir 200 entre 250; entonces:
                    \[100\times\dfrac{200}{250}=80\]
                \end{solutionbox}

                \part Si se sabe que 120 es el 35\% de cierta cantidad, ¿cuál es esta cantidad?
                \begin{solutionbox}{1cm}
                    Para conocer la cantidad, se debe dividir 120 entre 35; entonces:
                    \[100\times\dfrac{120}{35}=342.86\]
                \end{solutionbox}
            \end{parts}
        \end{multicols}
    }

    % \question[5]{Resuelve los siguientes problemas con porcentajes:
    %       \begin{parts}

    %       \end{parts}
    % }

    \question[5]{ Resuelve los siguientes problemas:
        \begin{parts}
            \part El costo de una computadora es de 12220 pesos, si la tasa de impuesto es del 15\%. ¿Cuánto será el total a pagar por la computadora?
            \begin{solutionbox}{2.5cm}
                Para conocer el total a pagar por la computadora, se debe multiplicar 12220 por 15\%; entonces:
                \[12220\times 115\%= 14053\]
                Por lo tanto, el total a pagar por la computadora es de 14053 pesos.
            \end{solutionbox}

            \part El 24\% de los habitantes de un pueblo tienen menos de 30 años. ¿Cuántos habitantes tiene el pueblo si hay 120 jóvenes menores de 30 años?
            \begin{solutionbox}{2.5cm}
                Para conocer el total de habitantes del pueblo, se debe dividir 120 entre 24\%; entonces:
                \[100\times\dfrac{120}{24}=500\]
                Por lo tanto, el pueblo tiene 500 habitantes.
            \end{solutionbox}
        \end{parts}
    }

    \question[5]{ Realiza las siguientes potencias:
        \begin{multicols}{3}
            \begin{parts}
                \part $2^3=$ \fillin[8][1in]
                \part $3^2=$ \fillin[9][1in]
                \part $5^2=$ \fillin[25][1in]
                \part $10^4=$ \fillin[10000][1in]
                \part $3^5=$ \fillin[243][1in]
                \part $\left(\dfrac{1}{3}\right)^3=$ \fillin[$\dfrac{1}{27}$][1in]
                \part $\left(\dfrac{2}{3}\right)^4=$ \fillin[$\dfrac{16}{81}$][1in]
                \part $\left(\dfrac{1}{9}\right)^2=$ \fillin[$\dfrac{1}{81}$][1in]
                \part $\left(\dfrac{4}{3}\right)^2=$ \fillin[$\dfrac{1}{1000}$][1in]
                \part $\left(\dfrac{3}{2}\right)^5=$ \fillin[$\dfrac{1}{8}$][1in]
            \end{parts}
        \end{multicols}
    }

    \question[5]{Escribe la forma desarrollada de los siguientes números:
        \begin{multicols}{3}
            \begin{parts}
                \part $1.0934\times10^{4}=$
                \part $3.39\times10^{3}=$
                \part $12\times10^{5}=$
                \part $4\times10^{2}=$
                \part $2.08\times10^{6}=$
                \part $0.5\times10^{3}=$
            \end{parts}
        \end{multicols}
    }

    \question[5]{Escribe con notación científica los siguientes números:
        \begin{multicols}{3}
            \begin{parts}
                \part $7600=$
                \part $0.04=$
                \part $5000000=$
                \part $0.1=$
                \part $25=$
                \part $1.01=$
            \end{parts}
        \end{multicols}
    }


    \question[5]{Calcula las siguientes raíces cuadradas:
        \begin{multicols}{3}
            \begin{parts}
                \part $\sqrt{169}=$
                \part $\sqrt{1.44}=$
                \part $\sqrt{0.09}=$
                \part $\sqrt{2.25}=$
                \part $\sqrt{196}=$
                \part $\sqrt{900}$
            \end{parts}
        \end{multicols}
    }

    \question[5]{Convierte las siguientes unidades de longitud y de masa como se te pide:
        \begin{parts}
            \part Convierte 4.9 kilómetros a metros.
            \part Convierte 34 metros a hectómetros
            \part Convierte 98 milímetros a centímetros
            \part Convierte 134 kilómetros a metros
            \part Convierte 134 centímetros a decámetros
            \part Convierte 342 gramos a hectogramos.
            \part Convierte 8334 centigramos a gramos.
            \part Convierte 93.4 miligramos a centigramos.
            \part Convierte 29 decagramos a miligramos.
            \part Convierte 9 gramos a miligramos.
        \end{parts}
    }

    \question[5]{Convierte las siguientes unidades de capacidad como se te pide:
        \begin{parts}
            \part Convierte 27 hectolitros a decilitros.
            \part Convierte 8 mililitros a centilitros.
            \part Convierte 1094 mililitros a decilitros.
            \part Convierte 702 mililitros a decilitros.
            \part Convierte 19 litros a mililitros.
            \part Convierte 8200 litros a metros cúbicos.
            \part Convierte 4.8 decímetros cúbicos a litros.
            \part Convierte 750 litros a metros cúbicos.
            \part Convierte 567 milímetros cúbicos a litros.
            \part Convierte 4100 litros a metros cúbicos.

        \end{parts}
    }

    \question[5]{Convierte las siguientes unidades de área y volumen como se te pide:
        \begin{parts}
            \part Convierte 8.03 metros cúbicos a milímetros cúbicos
            \part Convierte 8 kilómetros cuadrados a metros cuadrados
            \part Convierte 88 metros cuadrados a kilómetros cuadrados
            \part Convierte 18 decámetros cúbicos a milímetros cúbicos
            \part Convierte 801 milímetros cuadrados a decámetros cuadrados
        \end{parts}
    }
\end{questions}
\end{document}