\documentclass[12pt,addpoints]{evalua}
\grado{1$^\circ$ de Secundaria}
\cicloescolar{2023-2024}
\materia{Matemáticas 1}
\unidad{2}
\title{Examen de la Unidad}
\aprendizajes{
    \item Determina y usa la jerarquía de operaciones y los paréntesis en
    operaciones con números naturales, enteros y
    decimales (para multiplicación y división, sólo números
    positivos).
    \item Resuelve problemas de cálculo de porcentajes, de tanto por
    ciento y de la cantidad base.
}
\author{Prof.: Julio César Melchor Pinto}
\begin{document}
\begin{questions}
    % \section*{\ifprintanswers{Operaciones con decimales}\else{}\fi}
    \question[8]{ Realiza las siguientes operaciones de decimales:

        \begin{multicols}{2}
            \begin{parts}
                \ifprintanswers{\large\part \quad \opadd[hfactor=decimal,resultstyle=\color{red},carryadd=true,carrysub=false]{24.97}{19.34} \\[3em]}
                \else{          \large\part \quad \opadd[hfactor=decimal,resultstyle=\color{white},carryadd=false,carrysub=false]{24.97}{19.34} \\[3em]}
                \fi

                \ifprintanswers{\large\part  \quad  \opsub[hfactor=decimal,resultstyle=\color{red},carrysub=true]{968.31}{134.67} \\[1ex]}
                \else{          \large\part  \quad  \opsub[hfactor=decimal,resultstyle=\color{white},carrysub=false]{968.31}{134.67} \\[1ex]}
                \fi
                \columnbreak%

                \ifprintanswers{\part \quad \opmul[hfactor=decimal,resultstyle=\color{red},displayintermediary=all]{198.4}{12.2} \\[4em]}
                \else{          \large\part \quad \opmul[hfactor=decimal,resultstyle=\color{white},displayintermediary=None]{198.4}{12.2} \\[4em]}
                \fi

                \ifprintanswers{\large\part\opdiv[period,style=text,hrulewidth=0.2pt,vruleperiod=0.7,hfactor=decimal,resultstyle=\color{red}]{8.32}{1.2}}
                \else{          \large\part  \quad $1.2 \overline{) \ 8.32 \ }$}
                \fi
            \end{parts}
        \end{multicols}
    }

    % \subsection*{\ifprintanswers{Resolución de problemas}\else{}\fi}

    \question[10] Resuelve los siguientes problemas:
    \begin{parts}
        \part La mamá de Susana compró 11 m (metros) de franela y pagó 103.40 pesos. ¿Cuánto cuesta el metro de franela?

        \begin{solutionbox}{2cm}
            \opdiv[hfactor=decimal,resultstyle={\ifprintanswers{\color{red}}\else{\color{white}}\fi},displayintermediary=None,carryadd=false,carrysub=false]{103.40}{11}
        \end{solutionbox}

        % \part El precio de 385 artículos comerciales es de 1,232 pesos. ¿Cuál es el precio unitario de cada artículo?

        % \begin{solutionbox}{2cm}
        %     \opdiv[hfactor=decimal,resultstyle=\color{white},displayintermediary=None,carryadd=false,carrysub=false]{1232}{385}
        % \end{solutionbox}
    \end{parts}

    \newpage
    % \section*{\ifprintanswers{Operaciones con fracciones}\else{}\fi}

    % \subsection*{\ifprintanswers{Operaciones con fracciones}\else{}\fi}

    \question[6] Realiza las siguientes operaciones con fracciones:

    \begin{multicols}{3}

        \begin{parts}\large
            \part $\dfrac{3}{5}+\dfrac{2}{5}=$ \fillin[$\dfrac{5}{5}=1$][0in] \\[3em]

            \part $ \dfrac{3}{5} \divisionsymbol\dfrac{2}{3}=$ \fillin[$\dfrac{9}{10}$][0in]  \\[1em]

            \columnbreak%

            \part $\dfrac{7}{8}-\dfrac{3}{4}=$ \fillin[$\dfrac{7}{8}-\dfrac{6}{8}=\dfrac{1}{8}$][0in] \\[3em]

            \part $\dfrac{7}{8}\times\dfrac{3}{4}=$ \fillin[$\dfrac{21}{32}$][0in]  \\[1em]

            \columnbreak%

            \part $\dfrac{7}{8}+\dfrac{3}{4}=$ \fillin[$\dfrac{7}{8}+\dfrac{6}{8}=\dfrac{13}{8}$][0in]  \\[3em]

            \part $ \dfrac{7}{8} \divisionsymbol \dfrac{3}{4}=$ \fillin[$\dfrac{28}{24}=\dfrac{14}{12}$][0in]  \\[1em]

        \end{parts}
    \end{multicols}


    % \subsection*{\ifprintanswers{Resolución de problemas}\else{}\fi}

    \question[10] Resuelve los siguientes problemas:

    \begin{parts}
        % \part Un granjero siembra 2/5 de su granja con maíz y 3/10 con soya, ¿qué cantidad de su granja queda por sembrar?

        % \begin{solutionbox}{2.5cm}
        %     Para conocer la cantidad de su granja que queda por sembrar, se debe restar 2/5 y 3/10 a 1; entonces:
        %     \[1-\dfrac{2}{5}-\dfrac{3}{10}=\dfrac{10}{10}-\dfrac{4}{10}-\dfrac{3}{10}=\dfrac{3}{10}\]

        % \end{solutionbox}

        \part Un reloj se adelanta 3/7 de minuto cada hora. ¿Cuánto se adelantará en 5 horas?

        \begin{solutionbox}{2.5cm}
            Para conocer cuánto se adelantará en 5 horas, se debe multiplicar 3/7 por 5; entonces:
            \[\dfrac{3}{7}\times 5=\dfrac{15}{7}\]
        \end{solutionbox}
    \end{parts}

    % \section*{\ifprintanswers{Porcentajes}\else{}\fi}
    % \subsection*{\ifprintanswers{Porcentajes a decimal}\else{}\fi}

    \question[6] Escribe como decimal los siguientes porcentajes:

    \begin{multicols}{3}
        \begin{parts}\large
            \part $10.8\% =$ \fillin[0.108][0in]

            \part $5\% =$ \fillin[0.05][0in]

            \part $0.5\% =$ \fillin[0.005][0in]
        \end{parts}
    \end{multicols}

    % \newpage
    % \subsection*{\ifprintanswers{Decimal a porcentaje}\else{}\fi}

    \question[6] Escribe como porcentaje los siguientes decimales:

    \begin{multicols}{3}
        \begin{parts}\large
            \part $0.704=\quad$  \fillin[70.4][0in] \%

            \part $0.014=\quad$    \fillin[1.4][0in]   \%

            \part $1=\quad$      \fillin[100][0in]  \%
        \end{parts}
    \end{multicols}

    % \subsection*{\ifprintanswers{Porcentaje de cantidades}\else{}\fi}

    \question[10] Calcula el porcentaje de las siguientes cantidades:

    \begin{parts}
        \setlength{\columnsep}{1.2cm}
        \begin{multicols}{3}
            \part  {\large 15\% de 900 es: \fillin[135][0in] }

            \begin{solutionbox}{2cm}
                \[900\times\dfrac{15}{100}=135\]
            \end{solutionbox}

            \part  {\large 0.5\% de 1200 es: \fillin[6][0in] }

            \begin{solutionbox}{2cm}
                \[1200\times\dfrac{0.5}{100}=6\]
            \end{solutionbox}

            \part  {\large 3.5\% de 415 es: \fillin[14.525][0in] }

            \begin{solutionbox}{2cm}
                \[415\times\dfrac{3.5}{100}=14.525\]
            \end{solutionbox}
        \end{multicols}

        \begin{multicols}{2}

            \part Si se sabe que 210 es el 21\% de cierta cantidad, ¿cuál es esta cantidad?

            \begin{solutionbox}{2.8cm}
                Para conocer la cantidad, se debe dividir 210 entre 21; entonces:
                \[100\times\dfrac{210}{21}=1000\]
            \end{solutionbox}

            \part Si se sabe que 120 es el 35\% de cierta cantidad, ¿cuál es esta cantidad?

            \begin{solutionbox}{2.8cm}
                Para conocer la cantidad, se debe dividir 120 entre 35; entonces:
                \[100\times\dfrac{120}{35}=342.86\]
            \end{solutionbox}

        \end{multicols}

    \end{parts}

    % \subsection*{\ifprintanswers{Resolución de problemas}\else{}\fi}

    \question[8] Resuelve los siguientes problemas:

    \begin{multicols}{2}
        \begin{parts}
            \part El costo de una computadora es de \$12,220 pesos, si la tasa de impuesto es del 15\%. ¿Cuánto será el total a pagar por la computadora?

            \begin{solutionbox}{4cm}
                Para conocer el total a pagar por la computadora, se debe multiplicar 12220 por 15\%; entonces:
                \[12220\times 15\%= 14,050\]
                Por lo tanto, el total a pagar por la computadora es de 14053 pesos.
            \end{solutionbox}

            \part El 24\% de los habitantes de un pueblo tienen menos de 30 años. ¿Cuántos habitantes tiene el pueblo si hay 120 jóvenes menores de 30 años?

            \begin{solutionbox}{4cm}
                Para conocer el total de habitantes del pueblo, se debe dividir 120 entre 24\%; entonces:
                \[100\times\dfrac{120}{24}=500\]
                Por lo tanto, el pueblo tiene 500 habitantes.
            \end{solutionbox}
        \end{parts}
    \end{multicols}


    % \section*{\ifprintanswers{Potencias y raíces}\else{}\fi}
    % \subsection*{\ifprintanswers{Potenciación}\else{}\fi}

    \question[6] Realiza las siguientes potencias:

    \begin{multicols}{3}
        \begin{parts}\large
            \part $2^3=$ \fillin[8][0in] \\[1em]

            % \part $3^2=$ \fillin[9][0in]

            \columnbreak%

            \part $10^4=$ \fillin[10,000][0in] \\[1em]

            % \part $\left(\dfrac{1}{3}\right)^3=$ \fillin[$\dfrac{1}{27}$][0in]

            \columnbreak%

            \part $\left(\dfrac{2}{3}\right)^4=$ \fillin[$\dfrac{16}{81}$][0in] \\[1em]

            % \part $\left(\dfrac{4}{3}\right)^2=$ \fillin[$\dfrac{1}{1000}$][0in]

        \end{parts}
    \end{multicols}

    % \subsection*{\ifprintanswers{Notación científica}\else{}\fi}

    \question[6]{Escribe la forma desarrollada de los siguientes números:

        \begin{multicols}{3}
            \begin{parts}\large
                \part $1.0934\times10^{4}=$ \fillin[10,934][0in] \\[3em]

                % \part $3.39\times10^{3}=$ \\[1em]

                \columnbreak%

                \part $1.2\times10^{6}=$ \fillin[1,200,000][0in] \\[3em]

                % \part $4\times10^{2}=$ \\[1em]

                \columnbreak%

                \part $2.08\times10^{5}=$ \fillin[208,000][0in] \\[3em]

                % \part $0.5\times10^{3}=$ \\[1em]

            \end{parts}
        \end{multicols}
    }

    \question[6]{Escribe con notación científica los siguientes números:

        \begin{multicols}{3}
            \begin{parts}\large
                % \part $0.04=$ \\[3em]

                \part $760000000=$ \fillin[$7.6\times10^{8}$][0in] \\[3em]

                \columnbreak%

                \part $5000000=$ \fillin[$5\times10^{6}$][0in] \\[3em]

                % \part $0.1=$ \\[1em]

                \columnbreak%

                % \part $25=$  \\[3em]

                \part $1.01=$ \fillin[$1.01\times10^{0}$][0in] \\[3em]
            \end{parts}
        \end{multicols}
    }


    % \subsection*{\ifprintanswers{Raíces}\else{}\fi}
    \question[6]{Calcula las siguientes raíces cuadradas:
        \begin{multicols}{3}
            \begin{parts}\large
                \part $\sqrt{169}=$ \fillin[13][0in]\\[3em]
                % \part $\sqrt{1.44}=$\\[1em]

                \columnbreak%

                \part $\sqrt{0.09}=$ \fillin[0.3][0in] \\[3em]
                % \part $\sqrt{2.25}=$\\[1em]

                \columnbreak%

                \part $\sqrt{196}=$ \fillin[14][0in]\\[3em]
                % \part $\sqrt{900}=$\\[1em]
            \end{parts}
        \end{multicols}
    }

    \newpage
    % \section*{\ifprintanswers{Sistema de unidades}\else{}\fi}
    % \subsection*{\ifprintanswers{Unidades de longitud y masa}\else{}\fi}

    \question[4]{Convierte las siguientes unidades de longitud y de masa como se te pide:

        \begin{multicols}{2}
            \begin{parts}
                \part Convierte 34 m (metros) a Hm (hectómetros)

                \begin{solutionbox}{2cm}
                    \[ 34 \text{ m} = 34 \times \dfrac{1}{100} \text{ Hm}=0.34 \text{ Hm}\]
                \end{solutionbox}

                % \part Convierte 98 mm (milímetros) a  cm (centímetros)

                % \begin{solutionbox}{2.5cm}
                % \end{solutionbox}

                \part Convierte 93.4 mg (miligramos) a gr. (gramos).

                \begin{solutionbox}{2cm}
                    \[ 93.4 \text{ mg} = 93.4 \times \dfrac{1}{1000} \text{ g}=0.0934 \text{ g}\]
                \end{solutionbox}

                % \part Convierte 29 Dg (decagramos) a mg (miligramos).

                % \begin{solutionbox}{2.5cm}
                % \end{solutionbox}
            \end{parts}
        \end{multicols}
    }

    % \subsection*{\ifprintanswers{Unidades de capacidad}\else{}\fi}
    \question[4]{Convierte las siguientes unidades de capacidad como se te pide:

        \begin{multicols}{2}
            \begin{parts}
                % \part Convierte 27 hL (hectolitros) a dL (decilitros).

                % \begin{solutionbox}{2cm}
                % \end{solutionbox}

                \part Convierte 19 L (litros) a mL (mililitros).

                \begin{solutionbox}{2cm}
                    \[ 19 \text{ L} = 19 \times 1000 \text{ mL}=19,000 \text{ mL}\]

                \end{solutionbox}

                \columnbreak%

                % \part Convierte 4.8 dm$^3$ (decímetros cúbicos) a L (litros).

                % \begin{solutionbox}{2cm}
                % \end{solutionbox}

                \part Convierte 567 mm$^3$ (milímetros cúbicos) a L (litros).

                \begin{solutionbox}{2cm}
                    \[ 567 \text{ mm$^3$} = 567 \times \dfrac{1}{1000^3} \text{ L}=0.000000567 \text{ L}\]
                \end{solutionbox}

            \end{parts}
        \end{multicols}

    }
    % \subsection*{\ifprintanswers{Unidades de área y volumen}\else{}\fi}
    \question[4]{Convierte las siguientes unidades de área y volumen como se te pide:

        \begin{multicols}{2}
            \begin{parts}
                % \part Convierte 8.03 m$^3$ (metros cúbicos)  a mm$^3$ (milímetros cúbicos).

                % \begin{solutionbox}{2cm}
                % \end{solutionbox}

                \part Convierte 8 km$^2$ (kilómetros cuadrados) a m$^2$ (metros cuadrados).

                \begin{solutionbox}{2cm}
                    \[ 8 \text{ km$^2$} = 8 \times 100^3 \text{ m$^2$}=8,000,000 \text{ m$^2$}\]
                \end{solutionbox}

                \part Convierte 18 Dm$^3$ (decámetros cúbicos) a mm$^3$ (milímetros cúbicos).

                \begin{solutionbox}{2cm}
                    \[ 18 \text{ Dm$^3$} = 18 \times 1000^4 \text{ mm$^3$}=18 \times 10^{12} \text{ mm$^3$}\]
                \end{solutionbox}

                % \part Convierte 801 mm$^2$ (milímetros cuadrados) a Dm$^2$ (decámetros cuadrados).

                % \begin{solutionbox}{2cm}
                % \end{solutionbox}

            \end{parts}
        \end{multicols}
    }
\end{questions}
\end{document}