\documentclass[12pt]{evalua}
\grado{1$^\circ$ de Secundaria}
\cicloescolar{2022-2023}
\materia{Matemáticas 1}
\guide{2}
\title{Examen de la Unidad}
\aprendizajes{
    \begin{itemize}[leftmargin=*,label=\small\color{colorrds}\faIcon{user-graduate}
        ]
        \item Determina y usa la jerarquía de operaciones y los paréntesis en
              operaciones con números naturales, enteros y
              decimales (para multiplicación y división, sólo números
              positivos).
        \item Calcula valores faltantes en problemas de proporcionalidad
              directa,
              con constante natural, fracción o decimal (incluyendo tablas de
              variación).
    \end{itemize}
}

\author{Prof.: Julio César Melchor Pinto}
\begin{document}
%\printanswers
% {\small
% \begin{multicols}{2}
%     \include*{../blocks/block003}
%     \include*{../blocks/block001}
%     \include*{../blocks/block000}
%     \include*{../blocks/block002}
%     \include*{../blocks/block004}
% \end{multicols}
% % \include*{../blocks/block005}
% % \include*{../blocks/block006}
% }
\begin{questions}

    \question Señala si las relaciones son directamente proporcionales o inversamente proporcionales.

    \begin{multicols}{2}
        \begin{parts}
            \part[5] La población mundial y el consumo de agua.

            \begin{checkboxes}%[label=$\square$,font=\color{colorrds},itemjoin={\qquad}]
                \choice Directamente proporcional
                \choice Inversamente proporcional
            \end{checkboxes}

            \part[5]  La población mundial y la cantidad de agua disponible por persona.

            \begin{checkboxes}%[label=$\square$,font=\color{colorrds},itemjoin={\qquad}]
                \choice Directamente proporcional
                \choice Inversamente proporcional
            \end{checkboxes}

            % \part[5] La distancia al sol y la temperatura.

            % \begin{checkboxes}%[label=$\square$,font=\color{colorrds},itemjoin={\qquad}]
            %     \choice Directamente proporcional
            %     \choice Inversamente proporcional
            % \end{checkboxes}

            % \part[5]  El tamaño de un planeta y su fuerza de gravedad.

            % \begin{checkboxes}%[label=$\square$,font=\color{colorrds},itemjoin={\qquad}]
            %     \choice Directamente proporcional
            %     \choice Inversamente proporcional
            % \end{checkboxes}


            \part[5]  La velocidad de un móvil y la distancia recorrida.

            \begin{checkboxes}%[label=$\square$,font=\color{colorrds},itemjoin={\qquad}]
                \CorrectChoice Directamente proporcional
                \choice Inversamente proporcional
            \end{checkboxes}

            \columnbreak

            \part[5] La cantidad de imágenes guardadas en el celular y la cantidad de espacio libre.

            \begin{checkboxes}%[label=$\square$,font=\color{colorrds},itemjoin={\qquad}]
                \choice Directamente proporcional
                \choice Inversamente proporcional
            \end{checkboxes}

            \part[5]  El tamaño de un archivo y el tiempo de descarga.

            \begin{checkboxes}%[label=$\square$,font=\color{colorrds},itemjoin={\qquad}]
                \choice Directamente proporcional
                \choice Inversamente proporcional
            \end{checkboxes}

            \part[5] La velocidad de conexión a Internet y el tiempo de descarga de archivos.

            \begin{checkboxes}%[label=$\square$,font=\color{colorrds},itemjoin={\qquad}]
                \choice Directamente proporcional
                \CorrectChoice Inversamente proporcional
            \end{checkboxes}
        \end{parts}
    \end{multicols}

    \newpage
    \question Resuelve los siguientes problemas de proporcionalidad directa e inversa.
    \begin{parts}

        % \part[10] Para construir una obra en 45 días, se requieren 12 albañiles.
        % Si se necesita realizar la obra en 9 días, ¿Cuántos albañiles necesito?

        % \begin{solutionbox}{3cm}

        % \end{solutionbox}

        \part[10] Un grupo de 20 excursionistas lleva provisiones para 15 días. Si al momento de partir, el grupo aumenta a 24, ¿Cuántos días les durarán las proviciones?

        \begin{solutionbox}{4cm}

        \end{solutionbox}

        \part[10] Si por el consumo de 40 m$^3$ de agua se pagan \$780,¿Cuanto se pagará por un consumo de 47 m$^3$?

        \begin{solutionbox}{4cm}

        \end{solutionbox}

        % \part[10] Para hacer una obra en 42 días, se emplean 23 obreros, ¿Cuántos obreros se necesitan para hacerlo en 7 días?

        % \begin{solutionbox}{3cm}

        % \end{solutionbox}

        \part[10] Un banco cobra \$90 dolares al año por utilizar una tarjeta de crédito, ¿Cuánto cobrará en 9 años?

        \begin{solutionbox}{4cm}

        \end{solutionbox}

        \part[10] Si 12 personas pintan un edificio en 5 días, ¿Cuántos días tardarían 20 personas?

        \begin{solutionbox}{4cm}

        \end{solutionbox}
    \end{parts}

    \question Resuelve las siguientes operaciones.
    \begin{multicols}{2}
        \begin{parts}
            \part[5] $50 \times 100 \div 2=$
            \begin{solutionbox}{3cm}
            \end{solutionbox}
            \part[5] $\left(700 \times 7\right) - 100=$
            \begin{solutionbox}{3cm}
            \end{solutionbox}
            % \part[5] $\left(30 \times 30\right) + 100=$
            % \begin{solutionbox}{3cm}
            % \end{solutionbox}
            \part[5] $\left(20 \times 9\right) + 20=$
            \begin{solutionbox}{3cm}
            \end{solutionbox}
            % \part[5] $\left(8 \times 8\right) - 4=$
            % \begin{solutionbox}{3cm}
            % \end{solutionbox}
            % \part[5] $\left(4 \times 5\right) +200=$
            % \begin{solutionbox}{3cm}
            % \end{solutionbox}
            \part[5] $\left(.5 \times 10\right) \times 2=$
            \begin{solutionbox}{3cm}
            \end{solutionbox}
            \part[5] $\left(33 \div 3\right) +4=$
            \begin{solutionbox}{3cm}
            \end{solutionbox}
            \part[5] $.18 \times 100 +2=$
            \begin{solutionbox}{3cm}
            \end{solutionbox}
            % \part[5] $1000 \times 16 =$
            % \begin{solutionbox}{3cm}
            % \end{solutionbox}
            % \part[5] $2500 \times 100=$
            % \begin{solutionbox}{3cm}
            % \end{solutionbox}
            % \part[5] $\left(60 \times 10\right) -80=$
            % \begin{solutionbox}{3cm}
            % \end{solutionbox}
        \end{parts}
    \end{multicols}
\end{questions}
\end{document}