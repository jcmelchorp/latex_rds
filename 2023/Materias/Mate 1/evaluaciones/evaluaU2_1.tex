\documentclass[12pt]{evalua}
\grado{1$^\circ$ de Secundaria}
\cicloescolar{2022-2023}
\materia{Matemáticas 1}
\guide{2}
\title{Examen de la Unidad}
\aprendizajes{
    \begin{itemize}[leftmargin=*,label=\small\color{colorrds}\faIcon{user-graduate}
        ]
        \item Determina y usa la jerarquía de operaciones y los paréntesis en
              operaciones con números naturales, enteros y
              decimales (para multiplicación y división, sólo números
              positivos).
        \item Calcula valores faltantes en problemas de proporcionalidad
              directa,
              con constante natural, fracción o decimal (incluyendo tablas de
              variación).
        \item Resuelve problemas de cálculo de porcentajes, de tanto por
              ciento y de la cantidad base.
        \item Calcula el perímetro de polígonos y del círculo, y áreas de
              triángulos y cuadriláteros desarrollando
              y aplicando fórmulas.
    \end{itemize}
}
\reglas{
    \item[\color{red!70!black}\faIcon{toilet-paper-slash}] No se
    permite salir del sal\'on
    de
    clases.
    \item[\color{red!70!black}\faIcon{handshake-slash}] No se permite
    intercambiar
    ningun tipo de material.
    \item[\color{red!70!black}\faIcon{mobile-alt}] No se permite el uso de
    celular
    o cualquier otro dispositivo.
    \item[\color{red!70!black}\faIcon{paper-plane}] No se permite el uso
    de apuntes,
    libros, notas o formularios.
    \item[\color{red!70!black}\faIcon{eye-slash}] No se permite mirar el examen
    de
    otros alumnos.
    \item[\color{red!70!black}\faIcon{people-arrows}] No se permite la
    comunicaci\'on oral o escrita con otros alumnos.
}
\author{Prof.: Julio César Melchor Pinto}
\begin{document}
%\printanswers
{\small
\begin{multicols}{2}
    \include*{../blocks/block003}
    \include*{../blocks/block001}
    \include*{../blocks/block000}
    \include*{../blocks/block002}
    \include*{../blocks/block004}
\end{multicols}
% \include*{../blocks/block005}
% \include*{../blocks/block006}
}
\begin{questions}
    \begin{multicols}{2}
        \include*{../questions/question001}
    \end{multicols}
    \begin{multicols}{2}
        \include*{../questions/question011}
    \end{multicols}
    \newpage
    \include*{../questions/question003}
    \include*{../questions/question004}
    % \newpage
    % \include*{../questions/question002}
    \newpage
    % \newpage
    % \include*{../questions/question005}
    \include*{../questions/question006}
    % \include*{../questions/question008}
    % \include*{../questions/question009}
    \newpage
    \include*{../questions/question010}
\end{questions}
\end{document}