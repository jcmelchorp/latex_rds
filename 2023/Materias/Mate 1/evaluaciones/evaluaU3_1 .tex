\documentclass[12pt,addpoints,answers]{evalua}
\grado{1$^\circ$ de Secundaria}
\cicloescolar{2022-2023}
\materia{Matemáticas 1}
\unidad{3}
\title{Examen de la Unidad}
\aprendizajes{
   \item Resuelve problemas mediante la formulación y solución algebraica de ecuaciones lineales.
   \item Analiza y compara situaciones de variación lineal a partir de sus representaciones tabular, gráfica y algebraica. Interpreta y resuelve problemas que se modelan con estos tipos de variación.
   \item Calcula valores faltantes en problemas de proporcionalidad directa, con constante natural, fracción o decimal (incluyendo tablas de variación).
}
\author{Prof.: Julio César Melchor Pinto}
\begin{document}
\begin{multicols}{2}%
\end{multicols}%
\begin{questions}
    \question[24] \include*{../questions/question067a}
    \newpage
    \question[14] \include*{../questions/question074a}
    \newpage
    \question[10] \include*{../questions/question077a}
    % \question[10]
    % \question[10]
    % \question[10]
    % \question[10]
    % \question[10]
    % \question[10]
    % \question[10]
\end{questions}
\end{document}