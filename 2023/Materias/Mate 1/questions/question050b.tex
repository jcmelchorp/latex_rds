\textbf{Escribe la ecuación} y \textbf{encuentra la soluci\'on} que representa a cada uno de los siguientes incisos.

\begin{parts}
    \part Un n\'umero tal que, al multiplicarlo por 2 y al resultado restarle 10, resulta -2, ¿cu\'al es ese n\'umero?

    \begin{solutionbox}{5cm}
        \begin{multicols}{2}
            \begin{tabular}{rll}
                $x    $ & $   $ & \quad un n\'umero              \\
                $2x   $ & $   $ & \quad multiplicado por 2       \\
                $2x-10$ & $   $ & \quad al resultado restarle 10 \\
                $2x-10$ & $=-2$ & \quad resulta -2
            \end{tabular}

            La solución a la ecuación es:
            \begin{align*}
                2x-10 & =-2          \\
                2x    & =-2+10       \\
                2x    & =8           \\
                x     & =\frac{8}{2} \\[0.5em]
                x     & =4
            \end{align*}
        \end{multicols}
    \end{solutionbox}

    \part Un n\'umero tal que, al sumarle 4, dividir la suma entre 3 y sumar 2 al cociente da como resultado 5, ¿cu\'al es ese n\'umero?

    \begin{solutionbox}{7cm}
        \begin{multicols}{2}
            \begin{tabular}{rll}
                $x    $           & $   $ & \quad un n\'umero             \\
                $x+4  $           & $   $ & \quad al sumarle 4            \\
                $\frac{x+4}{3}$   & $   $ & \quad dividir la suma entre 3 \\
                $\frac{x+4}{3}+2$ & $   $ & \quad sumar 2 al cociente     \\
                $\frac{x+4}{3}+2$ & $=5$  & \quad da como resultado 5
            \end{tabular}

            La solución a la ecuación es:
            \begin{align*}
                \frac{x+4}{3}+2 & =5       \\
                \frac{x+4}{3}   & =5-2     \\
                \frac{x+4}{3}   & =3       \\
                x+4             & =3\cdot3 \\
                x+4             & =9       \\
                x               & =9-4     \\
                x               & =5
            \end{align*}
        \end{multicols}
    \end{solutionbox}

    \part Un n\'umero tal que, al multiplicarlo por 10, al producto sumarle 4, a la suma dividirla entre 2 y al cociente restarle 6 resulta otra vez uno,  ¿cu\'al es ese n\'umero?

    \begin{solutionbox}{8cm}
        \begin{multicols}{2}
            \begin{tabular}{rll}
                $x    $             & $   $ & \quad un n\'umero                 \\
                $10x  $             & $   $ & \quad multiplicarlo por 10        \\
                $10x+4$             & $   $ & \quad al producto sumarle 4       \\
                $\frac{10x+4}{2}$   & $   $ & \quad a la suma dividirla entre 2 \\
                $\frac{10x+4}{2}-6$ & $   $ & \quad al cociente restarle 6      \\
                $\frac{10x+4}{2}-6$ & $=1$  & \quad resulta otra vez uno
            \end{tabular}

            La solución a la ecuación es:
            \begin{align*}
                \frac{10x+4}{2}-6 & =1             \\
                \frac{10x+4}{2}   & =1+6           \\
                \frac{10x+4}{2}   & =7             \\
                10x+4             & =7\cdot2       \\
                10x+4             & =14            \\
                10x               & =14-4          \\
                10x               & =10            \\
                x                 & =\frac{10}{10} \\[0.5em]
                x                 & =1
            \end{align*}
        \end{multicols}
    \end{solutionbox}

    \part Un n\'umero tal que, su doble multiplicado por 284 es 428, ¿cu\'al es ese n\'umero?

    \begin{solutionbox}{7cm}
        \begin{multicols}{2}
            \begin{tabular}{rll}
                $x    $      & $   $  & \quad un n\'umero          \\
                $2x   $      & $   $  & \quad su doble             \\
                $2x\cdot284$ & $   $  & \quad multiplicado por 284 \\
                $2x\cdot284$ & $=428$ & \quad es 428
            \end{tabular}

            La solución a la ecuación es:
            \begin{align*}
                2x\cdot284 & =428                   \\
                2x         & =\frac{428}{284}       \\
                x          & =\frac{428}{284\cdot2} \\[0.5em]
                x          & =\frac{428}{568}       \\[0.5em]
                x          & =\frac{107}{142}
            \end{align*}
        \end{multicols}
    \end{solutionbox}

    \part Un n\'umero tal que, multiplicado por el doble de 10 da como resultado 20, ¿cu\'al es ese n\'umero?

    \begin{solutionbox}{7cm}
        \begin{multicols}{2}
            \begin{tabular}{rll}
                $x    $          & $   $ & \quad un n\'umero                     \\
                $2\cdot10$       & $   $ & \quad el doble de 10                  \\
                $x\cdot2\cdot10$ & $   $ & \quad multiplicado por el doble de 10 \\
                $x\cdot2\cdot10$ & $=20$ & \quad da como resultado 20
            \end{tabular}

            La solución a la ecuación es:
            \begin{align*}
                x\cdot2\cdot10 & =20                  \\
                x\cdot2        & =\frac{20}{10}       \\
                x              & =\frac{20}{10\cdot2} \\[0.5em]
                x              & =\frac{20}{20}       \\[0.5em]
                x              & =1
            \end{align*}
        \end{multicols}
    \end{solutionbox}
\end{parts}
