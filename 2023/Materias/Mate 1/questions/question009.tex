Observa en la figura \ref{fig:hexagon} que los lados del hexágono regular grande miden el triple
que los lados del hexágono regular pequeño.

\begin{minipage}{0.25\textwidth}
    \begin{figure}[H]
        \centering
        \includegraphics[width=\linewidth]{hexagon.png}
        \captionof{figure}{DIagrama de los hexágonos del problema}
        \label{fig:hexagon}
    \end{figure}
\end{minipage}\hfill
\begin{minipage}{0.65\textwidth}
    \begin{parts}
        \part Escribe una expresión algebraica para el perímetro del hexágono pequeño a partir de la longitud de uno de sus lados.

        \begin{solutionbox}{1.2cm}

        \end{solutionbox}

        \part Expresa en términos de la longitud de los lados del hexágono pequeño la longitud de un lado del hexágono grande.

        \begin{solutionbox}{1.2cm}

        \end{solutionbox}

        \part Expresa algebraicamente el perímetro del polígono grande en términos de la longitud del hexágono pequeño.

        \begin{solutionbox}{1.2cm}

        \end{solutionbox}

        \part ¿Cuántas veces es más grande el perímetro del hexágono mayor respecto al del hexágono pequeño?

        \begin{solutionbox}{1.2cm}

        \end{solutionbox}
    \end{parts}
\end{minipage}
