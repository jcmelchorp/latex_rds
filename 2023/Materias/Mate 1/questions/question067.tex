Coloca el valor de la razón entre el precio y el peso de los siguientes productos de reciclaje.

\renewcommand{\arraystretch}{1}
\begin{table}[H]
    \rowcolors{2}{colorrds!10}{lightgray!10}
    \centering
    \begin{tabular}{|r|c|c|c|}
        \toprule
        \rowcolor{colorrds!80}
        \textbf{\color{white}Producto} & \textbf{\color{white}Peso} & \textbf{\color{white}Precio} & \textbf{\color{white}Razón $\left(\dfrac{\text{precio}}{\text{peso}}\right)$} \\\midrule
        Periódico                      & 600                        & 480                          &
        \ifprintanswers
            \textbf{$\frac{480}{600}=0.8$}
        \else
        \textbf{$\frac{480}{600}=0.8$}            \fi                                                                                                                              \\\hline
        Cartón                         & 1250                       & 750                          &
        \ifprintanswers
            \textbf{$\frac{750}{1250}=0.6$}
        \else
            \quad
        \fi                                                                                                                                                                        \\\hline
        PET                            & 600                        & 264                          &
        \ifprintanswers
            \textbf{$\frac{264}{600}=0.44$}
        \else
            \quad
        \fi                                                                                                                                                                        \\\hline
        Vidrio                         & 200                        & 1250                         &
        \ifprintanswers
            \textbf{$\frac{1250}{200}=6.25$}
        \else
            \quad
        \fi                                                                                                                                                                        \\\hline
        Papel                          & 400                        & 2000                         &
        \ifprintanswers
            \textbf{$\frac{2000}{400}=5$}
        \else
            \quad
        \fi                                                                                                                                                                        \\\cline{2-4}
        \bottomrule
    \end{tabular}
\end{table}

\begin{parts}
    \part Por vender 20 kg de cartón se obtubo \$\fillin[$12$][1cm].

    \begin{solutionbox}{2.21cm}\scriptsize
        \begin{tabular}{r>{\centering}p{0.2cm}l}
            \textbf{Peso} &               & \textbf{Precio}                                               \\
            \hline
            $1250$ kg     & $\Rightarrow$ & $\$750$                                                       \\
            $20$ kg       & $\Rightarrow$ & $x=\dfrac{20 \text{ kg}  \times \$750}{1250 \text{ kg}}=\$12$
        \end{tabular}
    \end{solutionbox}

    \part Al llevar \fillin[$45$][0.8cm] kg de periódico, recibió \$36.

    \begin{solutionbox}{2.1cm}\scriptsize
        \begin{tabular}{r>{\centering}p{0.2cm}l}
            \textbf{Precio} &               & \textbf{Peso}                                                \\
            \hline
            $\$480$         & $\Rightarrow$ & $600$ kg                                                     \\
            $\$36$          & $\Rightarrow$ & $x=\dfrac{\$36  \times 600 \text{ kg}}{\$480}=45 \text{ kg}$ \\
        \end{tabular}
    \end{solutionbox}

    \part Por los 14 kg de PET que llevó, recibió \$\fillin[$\$6.16$][1.2cm]

    \begin{solutionbox}{2.1cm}\scriptsize
        \begin{tabular}{r>{\centering}p{0.2cm}l}
            \textbf{Peso} &               & \textbf{Precio}                                                \\
            \hline
            $600$ kg      & $\Rightarrow$ & $\$264$                                                        \\
            $14$ kg       & $\Rightarrow$ & $x=\dfrac{14 \text{ kg}  \times \$264}{600 \text{ kg}}=\$6.16$ \\
        \end{tabular}
    \end{solutionbox}


    \part Al vender \fillin[$333.86$][1cm] kg de PET, recibió $\$146.9$.

    \begin{solutionbox}{2.1cm}\scriptsize
        \begin{tabular}{r>{\centering}p{0.2cm}l}
            \textbf{Precio} &               & \textbf{Peso}                                                       \\
            \hline
            $\$264$         & $\Rightarrow$ & $600$ kg                                                            \\
            $\$146.9$       & $\Rightarrow$ & $x=\dfrac{\$146.9  \times 600 \text{ kg}}{\$264}=333.86 \text{ kg}$ \\
        \end{tabular}
    \end{solutionbox}


    \part Al vender \fillin[$40$][1cm] kg de vidrio, recibió \$250.

    \begin{solutionbox}{2.1cm}\scriptsize
        \begin{tabular}{r>{\centering}p{0.2cm}l}
            \textbf{Precio} &               & \textbf{Peso}                                                  \\
            \hline
            $\$1250$        & $\Rightarrow$ & $200$ kg                                                       \\
            $\$250$         & $\Rightarrow$ & $x=\dfrac{\$250  \times 200 \text{ kg}}{\$1250}=40 \text{ kg}$ \\
        \end{tabular}
    \end{solutionbox}



\end{parts}