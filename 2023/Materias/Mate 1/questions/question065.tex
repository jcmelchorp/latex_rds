Mario corre todas las mañanas un cuarto de kilómetro cada minuto.

\begin{parts}
    \part Completa la Tabla \ref{tab:mario} para obtener la distancia que Mario recorre en diferentes tiempos. Consideren que siempre lo hace con la misma rapidez.

    \begin{table}[H]
        \rowcolors{2}{colorrds!10}{lightgray!10}
        \centering
        \caption{Distancia que recorre Mario en diferentes tiempos.}
        \label{tab:mario}
        \begin{tabular}{|*{2}{c|}*{1}{>{\centering}p{2.5cm}|}*{1}{p{3cm}|}}
            \toprule
            \rowcolor{colorrds!80}
            \textbf{\color{white}Tiempo (min)} & \textbf{\color{white}Distancia (km)} & \textbf{\color{white}Razón entre distancia y tiempo} & \textbf{\color{white}Constante de proporcionalidad} \\ \midrule
            1                                  & \ifprintanswers0.25\fi               & \ifprintanswers0.25\fi                               & \ifprintanswers0.25\fi                              \\ \hline
            4                                  & 1                                    & \ifprintanswers0.25\fi                               & \ifprintanswers0.25\fi                              \\ \hline
            16                                 & \ifprintanswers4\fi                  & \ifprintanswers0.25\fi                               & \ifprintanswers0.25\fi                              \\ \hline
            32                                 & \ifprintanswers8\fi                  & \ifprintanswers0.25\fi                               & \ifprintanswers0.25\fi                              \\ \hline
            64                                 & \ifprintanswers16\fi                 & \ifprintanswers0.25\fi                               & \ifprintanswers0.25\fi                              \\ \hline
            128                                & \ifprintanswers32\fi                 & \ifprintanswers0.25\fi                               & \ifprintanswers0.25\fi                              \\ \hline
            \bottomrule
        \end{tabular}
    \end{table}

    \part ¿La distancia que Mario recorre es proporcional al tiempo?

    \begin{solutionbox}{1.5cm}
        Sí, pues la razón entre distancia y tiempo es constante.
    \end{solutionbox}

    \part Analiza los valores de la Tabla \ref{tab:mario}. ¿Qué cantidades cambian y cuáles no?

    \begin{solutionbox}{1.5cm}
        El tiempo y la distancia cambian, pero la razón entre distancia y tiempo
        permanece constante.
    \end{solutionbox}

\end{parts}