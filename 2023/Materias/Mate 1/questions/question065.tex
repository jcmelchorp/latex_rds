\question[10] Mario corre todas las mañanas un cuarto de kilómetro cada minuto.

\begin{parts}
    \part completa la Tabla \ref{tab:mario} para obtener la distancia que Mario recorre en dife-
    rentes tiempos. Consideren que siempre lo hace con la misma rapidez.

    \begin{table}[H]
        \centering
        \caption{Distancia que recorre Mario en diferentes tiempos.}
        \label{tab:mario}
        \begin{tabular}{c|c|c|c}
            %\toprule
            Tiempo (min) & Distancia (km)         & Razón entre distancia y tiempo & Constante de proporcionalidad \\
            \midrule
            1            & \ifprintanswers0.25\fi & \ifprintanswers0.25\fi         & \ifprintanswers0.25\fi        \\
            \hline
            4            & 1                      & \ifprintanswers0.25\fi         & \ifprintanswers0.25\fi        \\
            \hline
            16           & \ifprintanswers4\fi    & \ifprintanswers0.25\fi         & \ifprintanswers0.25\fi        \\
            \hline
            32           & \ifprintanswers8\fi    & \ifprintanswers0.25\fi         & \ifprintanswers0.25\fi        \\
            \hline
            64           & \ifprintanswers16\fi   & \ifprintanswers0.25\fi         & \ifprintanswers0.25\fi        \\
            \hline
            128          & \ifprintanswers32\fi   & \ifprintanswers0.25\fi         & \ifprintanswers0.25\fi        \\
            \bottomrule
        \end{tabular}
    \end{table}

    \part ¿La distancia que Mario recorre es proporcional al tiempo?

    \begin{solutionbox}{1.5cm}

    \end{solutionbox}
    \part Analiza los valores de la Tabla \ref{tab:mario}. ¿Qué cantidades cambian y cuáles no?

    \begin{solutionbox}{1.5cm}

    \end{solutionbox}
\end{parts}