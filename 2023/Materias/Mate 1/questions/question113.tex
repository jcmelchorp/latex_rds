Un padre repartirá 3000 dólares entre sus cinco hijos, de tal forma que la diferencia entre los montos que reciba cada hijo sea la misma.
Si le entrega 1000 dólares a su hijo mayor,
\textbf{¿cuánto dinero recibirán los otros cuatro hijos?}

\begin{solutionbox}{8cm}
    \begin{multicols}{2}
        Sabemos que la suma de la serie que representa la reparticion del dinero es:
        \[s_{5}=\dfrac{5(1000+a_{5})}{2}=3,000\]
        despejando $a_5$:
        \begin{align*}
            \dfrac{5(1000+a_{5})}{2} & =3,000     \\
            5(360+a_{5})             & =6,000     \\
            360+a_{5}                & = 1,200    \\
            a_{5}                    & =1,200-360 \\
            a_{5}                    & =840
        \end{align*}
        El hijo menor recibió 840 dólares; calculando la regla de recurrencia:
        \[a_{5}=d(5-1)+1000=840\]
        despejando $d$:
        \begin{align*}
            d(5-1)+1000 & =840      \\
            d(5-1)      & =840-1000 \\
            d           & =-40
        \end{align*}
        Por lo tanto, el hijo mayor (1) recibe 1000 dólares, el segundo recibe 960 dólares, el tercero recibe 920 dólares, el cuarto recibe 880 dólares, y el menor (5) recibe 840 dólares.
    \end{multicols}
\end{solutionbox}