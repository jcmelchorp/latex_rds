\question[10] Completa la Tabla \ref{tab:edades} que muestra las edades de un grupo de 30 alumnos en el que
hay la misma cantidad de niños cuyas edades son de 11 y 14 años.

\begin{minipage}{0.5\textwidth}
    \begin{table}[H]
        \rowcolors{2}{colorrds!10}{lightgray!10}
        \centering
        \caption{Edades de un grupo de alumnos.}
        \label{tab:edades}
        \begin{tabular}{|*{1}{c|}*{1}{>{\centering}p{2cm}|}*{1}{p{4cm}|}}
            \toprule
            \rowcolor{colorrds!80}
            \textbf{\color{white}Edad (años)} & \textbf{\color{white}Cantidad de alumnos} & \textbf{\color{white}Razón entre cantidad de alumnos y edad} \\\midrule
            11                                & \ifprintanswers $6$\fi                    & \ifprintanswers $0.\overline{54}$\fi                         \\\hline
            12                                & 8                                         & \ifprintanswers $0.\overline{6}$\fi                          \\\hline
            13                                & 10                                        & \ifprintanswers$0.\overline{769230}$\fi                      \\\hline
            14                                & \ifprintanswers$6$\fi                     & \ifprintanswers$0.\overline{428571}$\fi                      \\\cline{2-3}
            \bottomrule
        \end{tabular}
    \end{table}
\end{minipage}\hfill
\begin{minipage}{0.45\textwidth}
    \begin{parts}
        \part ¿El número de alumnos en el grupo es proporcional a su edad? \emph{Explica por qué.}

        \begin{solutionbox}{1.6cm}
            No. Porque la razón entre el número de alumnos y su edad no es constante.
        \end{solutionbox}

        \part ¿Se puede calcular la constante de proporcionalidad para esta situación?

        \begin{solutionbox}{1.6cm}
            No, pues la situación no es de variación proporcional.
        \end{solutionbox}
    \end{parts}
\end{minipage}


