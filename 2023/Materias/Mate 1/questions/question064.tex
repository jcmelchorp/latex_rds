\question[10] Completa la Tabla \ref{tab:edades} que muestra las edades de un grupo de 30 alumnos en el que
hay la misma cantidad de niños cuyas edades son de 11 y 14 años.

\begin{parts}
    \part completa la Tabla \ref{tab:mario} para obtener la distancia que Mario recorre en dife-
    rentes tiempos. Consideren que siempre lo hace con la misma rapidez.

    \begin{table}[H]
        \centering
        \caption{Edades de un grupo de alumnos.}
        \label{tab:edades}
        \begin{tabular}{c|c|c}
            %\toprule
            Edad (años) & Cantidad de alumnos & Razón entre cantidad de alumnos y edad \\
            \midrule
            11          & \ifprintanswers6\fi & \ifprintanswers0.\overline{54}\fi      \\
            \hline
            12          & 8                   & \ifprintanswers0.\overline{6}\fi       \\
            \hline
            13          & 10                  & \ifprintanswers0.\overline{769230}\fi  \\
            \hline
            14          & \ifprintanswers6\fi & \ifprintanswers0.\overline{428571}\fi  \\
            \bottomrule
        \end{tabular}
    \end{table}

    \part ¿El número de alumnos en el grupo es proporcional a su edad?

    \begin{solutionbox}{1.5cm}

    \end{solutionbox}

    \part ¿Se puede calcular la constante de proporcionalidad para esta situación?

    \begin{solutionbox}{1.5cm}

    \end{solutionbox}
\end{parts}