Un padre repartirá 1000 dólares entre sus cinco hijos y decidió que cada hijo reciba un monto de dinero distinto de tal forma que la diferencia de montos entre los hijos sea la misma.
Si le dará 300 dólares al hijo mayor,
\textbf{¿cuánto dinero recibirá el hijo menor?}

\begin{solutionbox}{7cm}
    Sabemos que la suma de la serie que representa la reparticion del dinero es:
    \[s_{5}=\dfrac{5(300+a_{5})}{2}=1,000\]
    despejando $a_5$:
    \begin{align*}
        \dfrac{5(300+a_{5})}{2} & =1,000   \\
        5(300+a_{5})            & =2,000   \\
        360+a_{5}               & =480     \\
        a_{5}                   & =480-360 \\
        a_{5}                   & =120
    \end{align*}
    El hijo menor recibirá 120 dolares.
\end{solutionbox}