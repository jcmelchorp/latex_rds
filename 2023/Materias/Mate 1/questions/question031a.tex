Completa la Tabla \ref{tab:resorte} considerando el alargamiento del resorte y el peso que se coloca.

\begin{multicols}{2}
\begin{table}[H]
    \centering
    \caption{Datos sobre el alargamiento de un resorte debido al peso sostenido.}
    \label{tab:resorte}
    \begin{tabular}{|>{\columncolor{colorrds!80}\color{white}\bfseries}c|c|c|c|c|c|c|}
        \toprule
        Peso (kg)         & 0                    & $\frac{1}{2}$        & 1 & 2                    & 3                    & 5                     \\\cline{2-7}\midrule
        Alargamiento (cm) & \ifprintanswers 0\fi & \ifprintanswers 1\fi & 2 & \ifprintanswers 4\fi & \ifprintanswers 6\fi & \ifprintanswers 10\fi \\\cline{2-7}
        \bottomrule
    \end{tabular}
\end{table}
\begin{parts}
\part Dibuja en el plano cartesiano de la Figura \ref{fig:20230320205645_a} los puntos que corresponden
al alargamiento del resorte y el peso que se le coloca, y únelos con una
línea.

\part ¿Qué tipo de relación funcional existe entre el alargamiento del resorte y el peso que se coloca?

\begin{solutionbox}{1.2cm}
    Es una relación de variación proporcional.
\end{solutionbox}

\part ¿En qué punto de la gráfica la línea interseca al eje vertical?

\begin{solutionbox}{1.2cm}
    En el punto (0, 0).
\end{solutionbox}


\begin{minipage}[t][][t]{0.35\textwidth}\begin{figure}[H]
\begin{center}
\centering
\ifprintanswers
    \includegraphics[width=.7\linewidth]{../images/20230320215503}
\else
    \includegraphics[width=.7\linewidth]{../images/20230320205645}
\fi
\caption{Plano cartesiano}
\label{fig:20230320205645_a}
\end{figure}
% \columnbreak
\part ¿En qué se parecen y en qué difieren las dos gráficas de esta actividad?

\begin{solutionbox}{1.6cm}
    Ambas gráficas son líneas rectas; pero sólo la segunda interseca al
    eje vertical en el origen.
\end{solutionbox}
\end{parts}
\end{multicols}