Escribe la \textbf{expresion algebraica} que representa a cada uno de los siguientes enunciados:

\begin{parts}
    \part El doble de la suma de un n\'umero con cinco es 32.                                        \fillin[$2(x+5)=32$][3cm]
    \part La suma del doble de un n\'umero con cinco es igual a la suma del mismo n\'umero con dos.  \fillin[$(2x+5)=x+2$][3cm]
    \part El doble de un n\'umero es igual a la suma del mismo número con dos.                       \fillin[$2x=x+2$][3cm]
    \part La mitad de la suma de un n\'umero con dos, es uno.                                        \fillin[$\frac{1}{2}(x+2)=1$][3cm]
    \part La suma de la mitad de un n\'umero con dos, es dos.                                        \fillin[$\frac{1}{2}x+2=2$][3cm]
\end{parts}