Completa la Tabla \ref{tab:3.1}; luego, responde lo que se pide.

%\cellwidth{2em}
% \renewcommand{\arraystretch}{2}

\begin{table}[H]
    \rowcolors{3}{colorrds!10}{lightgray!10}
    \centering
    % \large
    \caption{}
    \label{tab:3.1}
    \begin{tabular}{|r|*{8}{p{1cm}|}}
        \toprule
        \rowcolor{colorrds!80}
        \textbf{\color{white}Posición del término} & \textbf{\color{white}1} & \textbf{\color{white}2} & \textbf{\color{white}3} & \textbf{\color{white}4} & \textbf{\color{white}\ifprintanswers5 \fi} & \textbf{\color{white}\ifprintanswers 6\fi} & \textbf{\color{white}\ifprintanswers 7 \fi} & \textbf{\color{white}\ifprintanswers 8 \fi} \\ \midrule
        Término de la sucesión                     & \ifprintanswers12\fi    & \ifprintanswers3\fi     & \ifprintanswers-6\fi    & \ifprintanswers-15\fi   & \ifprintanswers-24\fi                      & -33                                        & -42                                         & \ifprintanswers-50\fi                       \\ \hline
        Diferencias                                & \ifprintanswers-9\fi    & \ifprintanswers-9\fi    & \ifprintanswers-9\fi    & \ifprintanswers-9\fi    & \ifprintanswers-9\fi                       & \ifprintanswers-9\fi                       & \ifprintanswers-9\fi                        & \ifprintanswers-9\fi                        \\ \cline{2-9}
        \bottomrule
    \end{tabular}
\end{table}

\begin{multicols}{2}
    \begin{parts}
        ¿Cuál es el primer término de la sucesión?

        \begin{solutionbox}{4em}
            12
        \end{solutionbox}

        A partir del primer término, ¿cómo se obtiene el segundo?

        \begin{solutionbox}{4em}
            Restando 9 unidades.
        \end{solutionbox}

        ¿Cómo se obtiene el tercer término de la sucesión a partir del primero?

        \begin{solutionbox}{5em}
            Restando 2 por 9 = 18 unidades. El término es $12 - 18 = - 6$
        \end{solutionbox}

        Analiza los resultados del renglón de las diferencias. ¿Qué observas?

        \begin{solutionbox}{4em}
            Que se obtiene el mismo valor de -9.
        \end{solutionbox}

        \columnbreak
        Escribe la regla general de la diferencia entre dos términos consecutivos de la
        sucesión.

        \begin{solutionbox}{5em}
            $n - m = -9$. Donde $n$ es el enésimo elemento y $m$ el consecutivo.
        \end{solutionbox}

        Escribe el término que ocupa la posición 60.

        \begin{solutionbox}{6em}
            La regla general de la sucesión es $-9n + 21$. El término de la posición
            60 es $-9(60) + 21 = -540 + 21 = -519$.
        \end{solutionbox}

        ¿Qué posición tiene el término -78?

        \begin{solutionbox}{4em}
            Se resuelve $-9n + 21 = -78$. Se obtiene $n = 11$.
        \end{solutionbox}

        ¿Hay alguna posición en la que aparezca el número -138? ¿Cuál?

        \begin{solutionbox}{6em}
            Se resuelve $-9n + 21 = -138$. Se obtiene $n = 17.66$\dots Como no es entero, el
            número -138 no es elemento de la sucesión.
        \end{solutionbox}

        % Compara la regla que obtuviste con la de tus compañeros. Si hay diferencias, argumenta tu respuesta y corrige si es necesario.

        % \begin{solutionbox}{2cm}
        % \end{solutionbox}
    \end{parts}

\end{multicols}
