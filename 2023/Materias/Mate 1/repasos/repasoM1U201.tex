\documentclass[12pt]{repaso}
\grado{1$^\circ$ de Secundaria}
\cicloescolar{2022-2023}
\materia{Matemáticas 1}
\unidad{3}
\title{Repaso para el examen de la Unidad}
\aprendizajes{
        \item Determina y usa la jerarquía de operaciones y los paréntesis en
              operaciones con números naturales, enteros y
              decimales (para multiplicación y división, sólo números
              positivos).
        \item Calcula valores faltantes en problemas de proporcionalidad
              directa,
              con constante natural, fracción o decimal (incluyendo tablas de
              variación).
        \item Resuelve problemas de cálculo de porcentajes, de tanto por
              ciento y de la cantidad base.
        \item Calcula el perímetro de polígonos y del círculo, y áreas de
              triángulos y cuadriláteros desarrollando
              y aplicando fórmulas.
}
\author{J. C. Melchor Pinto}
\begin{document}
\INFO%
\begin{multicols}{2}
    \include*{../blocks/block001}
    \include*{../blocks/block003}
    \include*{../blocks/block000}
    \include*{../blocks/block002}
\end{multicols}

\begin{questions}
    {\printanswers
        \include*{../questions/question001}
        \include*{../questions/question004}
        \newpage
        \include*{../questions/question002}
    }
    \newpage
    \include*{../questions/question003}
    \newpage
    \include*{../questions/question005}
    \include*{../questions/question006}
    \newpage
    \include*{../questions/question007}
    \include*{../questions/question007}
    \include*{../questions/question008}
    \include*{../questions/question009}
    \newpage
    \include*{../questions/question011}
    \include*{../questions/question010}
    % \newpage
\end{questions}

%\vfill
%\puntuacion

\end{document}