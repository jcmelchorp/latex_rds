\documentclass[12pt,addpoints]{repaso}
\grado{1}
\nivel{Secundaria}
\cicloescolar{2022-2023}
\materia{Matemáticas}
\unidad{3}
\title{Preparación para el examen de la Unidad}
\aprendizajes{
    \item Resuelve problemas mediante la formulación y solución algebraica de ecuaciones lineales.
    \item Analiza y compara situaciones de variación lineal a partir de sus representaciones tabular, gráfica y algebraica. Interpreta y resuelve problemas que se modelan con estos tipos de variación.
    \item Calcula valores faltantes en problemas de proporcionalidad directa, con constante natural, fracción o decimal (incluyendo tablas de variación).
    }
\author{JC Melchor Pinto}
\begin{document}
\INFO%
\begin{questions}
    \questionboxed[10]{\include*{../questions/question077a}}
    %\ejemplosboxed[\include*{../questions/question067a}]
    \questionboxed[20]{\include*{../questions/question067}}
    \ejemplosboxed[\include*{../questions/question074a}]
    \questionboxed[15]{\include*{../questions/question073a}}
    \ejemplosboxed[\include*{../questions/question050c}]
    \questionboxed[40]{\include*{../questions/question050d}}
    \ejemplosboxed[\include*{../questions/question050a}]
    \questionboxed[15]{\include*{../questions/question050e}}
    % \questionboxed[25]{\include*{../questions/question050b}}
\end{questions}
\end{document}