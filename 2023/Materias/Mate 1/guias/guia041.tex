\documentclass[12pt,addpoints,answers]{guia}
\grado{1$^\circ$ de Secundaria}
\cicloescolar{2022-2023}
\materia{Matemáticas 1}
\guia{41}
\unidad{3}
\title{Problemas verbales sobre series y sucesiones aritméticas}
\aprendizajes{\item Formula expresiones algebraicas de primer grado a partir de sucesiones y las utiliza para
analizar propiedades de la sucesión que representan.}
\author{JC Melchor Pinto}
\begin{document}
\INFO%
\begin{importantbox}
    La \textbf{regla de recurrencia} de una sucesión es una expresión algebraica que permite calcular el valor de cada término con sólo saber su posición en la serie ($n$).
\end{importantbox}
\ejemplosboxed[\include*{../questions/question110}]
\begin{questions}
    \questionboxed[15]{\include*{../questions/question112}}
    \questionboxed[20]{\include*{../questions/question115}}
    \questionboxed[20]{\include*{../questions/question116}}
    \ejemplosboxed[\include*{../questions/question111}]
    \questionboxed[10]{\include*{../questions/question117}}
    \ejemplosboxed[\include*{../questions/question113a}]
    \questionboxed[15]{\include*{../questions/question114}}
    \ejemplosboxed[\include*{../questions/question113}]
    \questionboxed[20]{\include*{../questions/question118}}
\end{questions}
\end{document}