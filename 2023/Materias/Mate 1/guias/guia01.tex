\documentclass[12pt,addpoints,answers]{guia}
\grado{1$^\circ$ de Secundaria}
\cicloescolar{2022-2023}
\materia{Matemáticas 1}
\guia{1}
\unidad{3}
\title{Relaciones proporcionales}
\aprendizajes{\item Calcula valores faltantes en problemas de proporcionalidad directa,
        con constante natural, fracción o decimal (incluyendo tablas de variación). 
    }
\author{JC Melchor Pinto}
\begin{document}
\pagestyle{headandfoot}
%\thispagestyle{plain}

\INFO
\printanswers
%\pagestyle{headandfoot}

% \begin{startInfo}[]
%     {
%     }
% \end{startInfo}
\begin{questions}
    \questionboxed[50] \include*{../questions/question001}
    % \newpage
    \questionboxed[50] \include*{../questions/question002}
\end{questions}

%\vfill
%\puntuacion

\end{document}