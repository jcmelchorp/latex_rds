\documentclass[12pt,addpoints,answers]{guia}
\grado{1$^\circ$ de Secundaria}
\cicloescolar{2022-2023}
\materia{Matemáticas 1}
\guia{40}
\unidad{3}
\title{Series y sucesiones aritméticas}
\aprendizajes{\item Formula expresiones algebraicas de primer grado a partir de sucesiones y las utiliza para
analizar propiedades de la sucesión que representan.}
\author{JC Melchor Pinto}
\begin{document}
\INFO%
\begin{multicols}{2}
    \include*{../blocks/block040a}
    \include*{../blocks/block040d}
    \include*{../blocks/block040b}
    \include*{../blocks/block040c}
\end{multicols}
\begin{importantbox}
    La \textbf{regla de recurrencia} de una sucesión es una expresión algebraica que permite calcular el valor de cada término con sólo saber su posición en la serie ($n$).
\end{importantbox}
\begin{questions}
    \questionboxed[15]{\include*{../questions/question020}}
    \questionboxed[10]{\include*{../questions/question021}}
    \ejemplosboxed[\include*{../questions/question026}]
    \questionboxed[10]{\include*{../questions/question024}}
    \questionboxed[15]{\include*{../questions/question027}}
    \ejemplosboxed[\include*{../questions/question028}]
    \questionboxed[10]{\include*{../questions/question023}}
    \ejemplosboxed[\include*{../questions/question029}]
    \questionboxed[10]{\include*{../questions/question041}}%
    \questionboxed[15]{\include*{../questions/question025}}
    \ejemplosboxed[\include*{../questions/question040}]
    \questionboxed[15]{\include*{../questions/question022}}
\end{questions}
\end{document}