\documentclass[12pt,addpoints]{guia}
\grado{1$^\circ$ de Secundaria}
\cicloescolar{2022-2023}
\materia{Matemáticas 1}
\guia{30}
\unidad{3}
\title{Variación proporcional con tablas}
\aprendizajes{\item Analiza y compara situaciones de variación lineal a partir de sus representaciones tabular, gráfica y algebraica. 
\item Interpreta y resuelve problemas que se modelan con estos tipos de variación.
    }
\author{JC Melchor Pinto}
\begin{document}
\INFO%
\begin{multicols}{2}
    \include*{../blocks/block030a}
    \include*{../blocks/block030b}
\end{multicols}%
\begin{questions}
    \questionboxed[20]{\include*{../questions/question064}}
    \ejemplosboxed[\include*{../questions/question066}]
    \questionboxed[20]{\include*{../questions/question068a}}
    \ejemplosboxed[\include*{../questions/question067a}]
    \questionboxed[20]{\include*{../questions/question065}}
    \questionboxed[20]{\include*{../questions/question067}}
    \questionboxed[20]{\include*{../questions/question068}}
\end{questions}
\end{document}
