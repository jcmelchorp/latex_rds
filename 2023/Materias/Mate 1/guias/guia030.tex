\documentclass[12pt,addpoints,answers]{guia}
\grado{1$^\circ$ de Secundaria}
\cicloescolar{2022-2023}
\materia{Matemáticas 1}
\guia{30}
\unidad{3}
\title{Variación proporcional con tablas}
\aprendizajes{
    \begin{itemize}[leftmargin=*,label=\small\color{colorrds}\faIcon{user-graduate}]
        \item Analiza y compara situaciones de variación lineal a partir de sus representaciones tabular, gráfica y algebraica. Interpreta y resuelve problemas que se modelan con estos tipos de variación.
    \end{itemize}
}
\author{J. C. Melchor Pinto}
\begin{document}
\pagestyle{headandfoot}

\INFO
%\printanswers
\vspace{-0.8cm}
\begin{multicols}{2}
    \include*{../blocks/block030a}
    \include*{../blocks/block030b}
\end{multicols}
\begin{questions}
    \questionboxed[15]{\include*{../questions/question064}}
    \questionboxed[15]{\include*{../questions/question066}}
    \questionboxed[15]{\include*{../questions/question068a}}
    \questionboxed[15]{\include*{../questions/question067a}}
    \questionboxed[15]{\include*{../questions/question065}}
    \questionboxed[10]{\include*{../questions/question067}}
    \questionboxed[15]{\include*{../questions/question068}}
\end{questions}
\end{document}
