\documentclass[12pt,addpoints,answers]{guia}
\grado{1$^\circ$ de Secundaria}
\cicloescolar{2022-2023}
\materia{Matemáticas 1}
\guia{36}
\unidad{3}
\title{Pendiente de una recta y razón de cambio}
\aprendizajes{
    \begin{itemize}[leftmargin=*,label=\small\color{colorrds}\faIcon{user-graduate}]
        \item Analiza y compara situaciones de variación lineal a partir de sus representaciones tabular, gráfica y algebraica. Interpreta y resuelve problemas que se modelan con estos tipos de variación.
    \end{itemize}
}
\author{J. C. Melchor Pinto}
\begin{document}
\pagestyle{headandfoot}
%\thispagestyle{plain}

\INFO
%\printanswers
%\pagestyle{headandfoot}

% \begin{startInfo}[]
%     {
%     }
% \end{startInfo}
\begin{questions}
    \questionboxed[15] \include*{../questions/question076a}
    % \newpage
    \questionboxed[15] \include*{../questions/question076b}
    % \newpage
    \questionboxed[15] \include*{../questions/question076c}
\end{questions}
\end{document}