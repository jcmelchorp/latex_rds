\documentclass[12pt]{guia}
\grade{1$^\circ$ de Secundaria}
\cycle{2022-2023}
\subject{Matemáticas 1}
\guide{31}
\title{Variación proporcional con gráficas}
\aprendizajes{
    \begin{itemize}[leftmargin=*,label=\small\color{colorrds}\faIcon{user-graduate}]
        \item Analiza y compara situaciones de variación lineal a partir de sus representaciones tabular, gráfica y algebraica. Interpreta y resuelve problemas que se modelan con estos tipos de variación.
    \end{itemize}
}
\requisitos{
    \begin{itemize}
        \item Requisito 1
        \item Requisito 2
    \end{itemize}
}
\author{J. C. Melchor Pinto}

\begin{document}
\pagestyle{headandfoot}
\addpoints
\INFO
%\printanswers
% \begin{multicols}{2}
\include*{../blocks/block030a}
\include*{../blocks/block031b}
% \end{multicols}
% \newpage
\begin{questions}
    % \begin{multicols}{2}
    \include*{../questions/question031b}
    \include*{../questions/question069}
    % \end{multicols}
    \newpage
    \begin{importantbox}
        Una situación en la que la relación entre las variables involucradas es una variación
        proporcional tiene asociada la gráfica de una \textbf{línea recta}.
    \end{importantbox}
    \include*{../questions/question070}
    \newpage
    \include*{../questions/question071}
    \newpage
    \include*{../questions/question031}
    \include*{../questions/question031a}
\end{questions}
\end{document}