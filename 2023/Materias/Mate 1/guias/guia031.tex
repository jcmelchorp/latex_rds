\documentclass[12pt,addpoints,answers]{guia}
\grado{1$^\circ$ de Secundaria}
\cicloescolar{2022-2023}
\materia{Matemáticas 1}
\guia{31}
\title{Variación proporcional con gráficas}
\aprendizajes{\item Analiza y compara situaciones de variación lineal a partir de sus representaciones tabular, gráfica y algebraica.
\item Interpreta y resuelve problemas que se modelan con estos tipos de variación.}
\author{JC Melchor Pinto}
\begin{document}
\INFO%
\include*{../blocks/block030a}%
\ejemplosboxed[\include*{../questions/question031c}]
\begin{questions}
    \questionboxed[20]{\include*{../questions/question031b}}
    \ejemplosboxed[\include*{../questions/question031}]
    \questionboxed[20]{\include*{../questions/question031a}}
    \begin{importantbox}
        Una situación en la que la relación entre las variables involucradas es una variación
        proporcional tiene asociada la gráfica de una \textbf{línea recta}.
    \end{importantbox}
    \questionboxed[20]{\include*{../questions/question069}}
    \questionboxed[20]{\include*{../questions/question070}}
    \questionboxed[20]{\include*{../questions/question071}}
\end{questions}
\end{document}