\documentclass[12pt,addpoints,answers]{guia}
\grado{1$^\circ$ de Secundaria}
\cicloescolar{2022-2023}
\materia{Matemáticas 1}
\guia{31}
\title{Variación proporcional con gráficas}
\aprendizajes{\item Analiza y compara situaciones de variación lineal a partir de sus representaciones tabular, gráfica y algebraica. Interpreta y resuelve problemas que se modelan con estos tipos de variación.
    }
\author{JC Melchor Pinto}
\begin{document}
\pagestyle{headandfoot}

\INFO
\printanswers
\vspace{-0.5cm}
% \begin{multicols}{2}
\include*{../blocks/block030a}
\include*{../blocks/block031b}
% \end{multicols}
% \newpage
\begin{questions}
    % \begin{multicols}{2}
    \questionboxed[10] \include*{../questions/question031b}
    \questionboxed[10] \include*{../questions/question069}
    % \end{multicols}

    \begin{importantbox}
        Una situación en la que la relación entre las variables involucradas es una variación
        proporcional tiene asociada la gráfica de una \textbf{línea recta}.
    \end{importantbox}

    \questionboxed[10] \include*{../questions/question070}
    \newpage
    \questionboxed[10] \include*{../questions/question071}
    \newpage
    \questionboxed[10] \include*{../questions/question031}
    \questionboxed[10] \include*{../questions/question031a}
\end{questions}
\end{document}