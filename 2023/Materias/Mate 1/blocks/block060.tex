\begin{infocard}{Relación funcional}
    Cuando una cantidad depende o se relaciona con otra de manera proporcional, se
    dice que entre ellas hay una \textbf{relación funcional}.

    En las relaciones funcionales, las
    cantidades que cambian se llaman \textbf{variables}, y las cantidades que no cambian se
    denominan \textbf{constantes}.

    En estos casos, cuando conocemos el valor de una variable,
    es posible determinar el de la otra. Por ello, la primera se conoce como \textbf{variable independiente} y la segunda, \textbf{variable dependiente}.
\end{infocard}