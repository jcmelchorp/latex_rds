\begin{infocard}{Proporcionalidad directa e inversa}
    \textbf{Caso de proporcionalidad directa}
    Colocaremos en una tabla los 3 datos (a los que llamamos $a$, $b$ y $c$) y la incógnita, es decir, el dato que queremos averiguar (que llamaremos $x$). Después, aplicaremos la siguiente fórmula:
    \begin{figure}[H]
        \centering
        \includegraphics[width=.9\linewidth]{../images/formula-regla-de-3-img1}
        \caption{Solución de una relación proporcional \textbf{directa} por medio de la regla de 3}
        \label{fig:}
    \end{figure}
    \tcblower
    \textbf{Caso de proporcionalidad inversa}
    Colocaremos los 3 datos y la incógnita en la tabla igual que los hemos colocado en el caso anterior. Pero aplicaremos una fórmula distinta:
    \begin{figure}[H]
        \centering
        \includegraphics[width=.9\linewidth]{../images/formula-regla-de-3-img3}
        \caption{Solución de una relación proporcional \textbf{inversa} por medio de la regla de 3}
        \label{fig:}
    \end{figure}
\end{infocard}
