\documentclass[12pt]{book}
\usepackage[undotted]{minitoc}
\usepackage{amsmath}
\usepackage{amssymb}
\usepackage{graphicx}
\usepackage[spanish]{babel}
\decimalpoint
\usepackage[utf8]{inputenc}
\usepackage{tikz}
\usepackage{longdivision}
\usetikzlibrary{shadows.blur}
\usepackage{titlesec}
\usepackage{pgfkeys}
\usepackage{multicol}
%% equivalencias al entero
\newcounter{CountOfSections}
\newcommand{\fracgraph}[3][1]{%
    % #1 = optional height
\begin{tikzpicture}
    \draw (0,0) rectangle (#2,#1) node [midway] {1};

    \setcounter{CountOfSections}{0}%
    \foreach \Size/\Options in {#3} {%
        \stepcounter{CountOfSections}%
        \pgfmathsetmacro{\YCoord}{#1*\arabic{CountOfSections}}%
        \draw  (0,-\YCoord) rectangle (#2,-\YCoord+#1);
        \pgfmathsetmacro{\Xincrement}{#2/\Size}%
         \foreach \x in {1,...,\Size} {%
            \pgfmathsetmacro{\Xcoord}{\x*\Xincrement}%
            \pgfmathsetmacro{\XcoordLabel}{(\x-0.5)*\Xincrement}%
            \draw [fill=\Options]  ($(\Xcoord-\Xincrement,-\YCoord)$)  rectangle ($(\Xcoord,-\YCoord+#1)$);
            \node at ($(\XcoordLabel,-\YCoord+0.5*#1)$) {$\frac{1}{\Size}$};
        }%
    }%
\end{tikzpicture}
}
%%%%
%% Recta numérica (intervalo)
\newcommand{\rectnuminterval}[5][]{
        \begin{tikzpicture}[#1]
            \draw[very thick] (#2,0.5) -- (#3,0.5);
            \path [draw=black, fill=#4] (#2,0.5) circle (2pt);
            \path [draw=black, fill=#5, thick] (#3,0.5) circle (2pt);
            \draw[latex-latex] (-3.5,0) -- (3.5,0) ;
            \foreach \x in  {-3,-2,-1,0,1,2,3}
            \draw[shift={(\x,0)},color=black] (0pt,3pt) -- (0pt,-3pt);
            \foreach \x in {-3,-2,-1,0,1,2,3}
            \draw[shift={(\x,0)},color=black] (0pt,0pt) -- (0pt,-3pt) node[below]
            {$\x$};
        \end{tikzpicture}
    }
%%%%
\addtocontents{toc}{\setcounter{tocdepth}{4}}
\setcounter{minitocdepth}{3}
\nomtcrule         % removes rules = horizontal lines from minitocs
\nomtcpagenumbers  % remove page numbers from minitocs

\newcommand{\mychapter}[1]{
    \thispagestyle{empty}
    \chapter*{#1}
    \addcontentsline{toc}{chapter}{#1}
    \mtcaddchapter{}
    \mtcsettitle{minitoc}{}
    {\Large \scshape En esta unidad estudiaremos \dots}
    \minitoc
    \newpage
    \thispagestyle{empty} \mbox{}
    \newpage
}
\addto\captionsspanish{%
    \renewcommand{\chaptername}{Unidad}
    \renewcommand{\contentsname}{Contenido} 
}
%\usepackage{remreset}
% \makeatletter
% \@removefromreset{section}{chapter}
% \makeatother
% \renewcommand{\thechapter}{\arabic{chapter}}
% \newcommand*\chapterlabel{}
% \titleformat{\chapter}

\begin{document}
\dominitoc[n]%% Crea minitoc
\title{Matemáticas 1}
\author{Melchor Pinto, JC}
\date{\today}
\maketitle
\tableofcontents

\begin{mainmatter}
    \mychapter{Unidad 1}

    \section{Fracciones y decimales}
    En esta sección, estudiaremos las fracciones y los decimales, que son dos formas de representar números racionales. Los números racionales son aquellos que se pueden expresar como el cociente de dos números enteros. Por ejemplo, $\frac{1}{2}$, $5$ o $7.6$. En esta sección, aprenderemos a representar números racionales en forma de fracción, en forma decimal y de forma entera. También aprenderemos a comparar fracciones, decimales y enteros, y a convertir de una forma a otra.

    \subsection{Equivalencias de fracciones y decimales}

    \subsubsection{Fracciones equivalentes}
    Cuando dos fracciones representan el mismo número, decimos que son equivalentes. Por ejemplo, $\frac{1}{2}$ y $\frac{2}{4}$ son equivalentes, porque representan el mismo número. Para verlo, podemos dividir una pizza en dos partes iguales, y luego dividir una de esas partes en dos partes iguales.

    Aquí podemos notar que $\frac{1}{2} = \frac{2}{4}$, y que $\frac{2}{4} = \frac{4}{8}$.

    \begin{tikzpicture}
        % Crust
        \draw[fill=brown!80!black] (0, 0) circle (3cm);
        % Tomato Sauce
        \fill[red!80!black] (0, 0) circle (2.75cm);
        % Cheese
        \fill[yellow!80!white] (0, 0) circle (2.6cm);
        % Pepperoni
        \foreach \x in {5,55,...,365}
        \fill[red!60!black] (\x:1.8cm) circle (0.25cm);
        \foreach \x in {-10,80,...,305}
        \fill[red!60!black] (\x:0.9cm) circle (0.18cm);
        \foreach \x in {-25,20,...,290}
        \fill[red!70!black] (\x:1.2cm) circle (0.16cm);
        % Division lines for pizza slices
        \foreach \x in {0,90,180}
        \draw[line width=0.15cm, color=white] (0, 0) -- (\x:3.1cm);
        % Fractions labels
        \foreach \angle/\label in {45/4, 135/4
                , 270/2}
        \draw (\angle:3.5cm) node {\Large \bfseries $\frac{1}{\label}$};
    \end{tikzpicture}
    \begin{tikzpicture}
        % Crust
        \draw[fill=brown!80!black] (0, 0) circle (3cm);
        % Tomato Sauce
        \fill[red!80!black] (0, 0) circle (2.75cm);
        % Cheese
        \fill[yellow!80!white] (0, 0) circle (2.6cm);
        % Pepperoni
        \foreach \x in {5,55,...,365}
        \fill[red!60!black] (\x:1.8cm) circle (0.25cm);
        \foreach \x in {-10,80,...,305}
        \fill[red!60!black] (\x:0.9cm) circle (0.18cm);
        \foreach \x in {-25,20,...,290}
        \fill[red!70!black] (\x:1.2cm) circle (0.16cm);
        % Division lines for pizza slices
        \foreach \x in {0,45,...,180,270}
        \draw[line width=0.15cm, color=white] (0, 0) -- (\x:3.1cm);
        % Fractions labels
        \foreach \angle/\label in {20/8, 65/8,110/8, 155/8
                , 225/4,315/4}
        \draw (\angle:3.5cm) node {\Large \bfseries $\frac{1}{\label}$};
    \end{tikzpicture}


    % Representacion detallada en Tikz de una pizza de peperoni.
    \begin{tikzpicture}
        % Crust
        \draw[fill=brown!80!black] (0, 0) circle (3cm);
        % Tomato Sauce
        \fill[red!80!black] (0, 0) circle (2.75cm);
        % Cheese
        \fill[yellow!80!white] (0, 0) circle (2.6cm);
        % Pepperoni
        \foreach \x in {5,55,...,365}
        \fill[red!60!black] (\x:1.8cm) circle (0.25cm);
        \foreach \x in {-10,80,...,305}
        \fill[red!60!black] (\x:0.9cm) circle (0.18cm);
        \foreach \x in {-25,20,...,290}
        \fill[red!70!black] (\x:1.2cm) circle (0.16cm);
        % Division lines for pizza slices
        \foreach \x in {0,45,90,180}
        \draw[line width=0.25cm, color=white] (0, 0) -- (\x:3.1cm);
        % Fractions labels
        \foreach \angle/\label in {20/8, 65/8, 135/4
                , 270/2}
        \draw (\angle:3.5cm) node {\Large \bfseries $\frac{1}{\label}$};
    \end{tikzpicture}

    % explicación de la equivalencia al entero de fracciones
    \subsubsection{Equivalencias al entero}
    Una fracción es equivalente al entero cuando el numerador es igual al denominador. Por ejemplo, $\frac{3}{3}$ es equivalente a $1$, porque
    \begin{multicols}{2}
        \fracgraph{5}{2/cyan!50,3/red!40,4/brown!50}

        \begin{align*}
            \color{cyan!80!black}\dfrac{1}{2}+\dfrac{1}{2}                            & =\dfrac{2}{2}=1 \\[0.5em]
            \color{red!80!black}\dfrac{1}{3}+\dfrac{1}{3}+\dfrac{1}{3}                & =\dfrac{3}{3}=1 \\[0.5em]
            \color{brown!80!black}\dfrac{1}{4}+\dfrac{1}{4}+\dfrac{1}{4}+\dfrac{1}{4} & =\dfrac{4}{4}=1
        \end{align*}
    \end{multicols}

    \subsubsection{Convierte fracciones a decimales}
    Para convertir una fracción a decimal, dividimos el numerador entre el denominador. Por ejemplo, para convertir $\frac{3}{4}$ a decimal, dividimos $3$ entre $4$.
    \begin{center}
        \longdivision{3}{4} \qquad entonces, $\dfrac{3}{4}=0.75$
    \end{center}

    \subsubsection{Convierte decimales a fracciones}
    Para convertir un decimal a fracción, debemos escribir el número decimal como una fracción decimal, y luego simplificarla. Por ejemplo, para convertir $0.75$ a fracción, escribimos $0.75$ como $\frac{75}{100}$, y luego simplificamos la fracción.
    \[0.75=\frac{75}{100}\]
    En una \textbf{fracción decimal} el denominador debe ser una potencia de $10$. En este caso, el denominador es $100$, que es una potencia de $10$. Si el decimal tiene un solo dígito después del punto decimal, el denominador debe ser $10$. Por ejemplo, para convertir $0.5$ a fracción, escribimos $0.5$ como $\frac{5}{10}$, y luego simplificamos la fracción.
    \[0.5=\frac{5}{10}\]
    Para \textbf{simplificar una fracción}, dividimos el numerador y el denominador entre el máximo común divisor de ambos. En este caso, el máximo común divisor de $75$ y $100$ es $25$, por lo que la fracción simplificada es
    \[\frac{75}{100}=\frac{75\div25}{100\div25}=\frac{3}{4}\]

    \subsection{Decimales peri\'odicos}
    Un decimal periódico es un decimal que tiene un patrón de dígitos que se repite infinitamente. Por ejemplo, al convertir la fracción $\frac{1}{3}$ a un número decimal, se obtiene:
    \begin{center}
        \longdivision[stage=1,repeating decimal style = none]{1}{3}
    \end{center}
    $0.3333\dots$ es un decimal periódico, porque el patrón $3$ se repite infinitamente. El patrón de un decimal periódico se llama \textbf{período}. El período de $0.3333\dots$ es $3$, y se escribe como:
    \[0.3333\dots=0.\overline{3} \qquad \text{(Se pronuncia ``\emph{cero punto tres periódico}'')}\]

    \subsubsection{Redondeo y truncamiento}
    Cuando convertimos una fracción a decimal, a veces obtenemos un decimal que no termina nunca de escribirse por completo. Por ejemplo, al convertir $\frac{1}{3}$ a decimal, obtenemos $0.3333\dots$. En este caso, podemos \textbf{redondear} el decimal a un número finito de dígitos, o podemos \textbf{truncar} el decimal a un número finito de dígitos.

    \paragraph{Redondeo} Redondear un número quiere decir reducir el número de cifras manteniendo un valor parecido. El resultado es menos exacto, pero más fácil de usar. Para redondear un decimal, debemos decidir cuántos dígitos queremos en el resultado. Luego, miramos el siguiente dígito después del último dígito que queremos en el resultado. Si el siguiente dígito es $5$ o más, sumamos $1$ al último dígito que queremos en el resultado. Si el siguiente dígito es menor que $5$, no sumamos nada al último dígito que queremos en el resultado.

    Por ejemplo, para redondear $0.6666\dots$ a $2$ dígitos, miramos el tercer dígito, que es $6$. Como $6$ es mayor que $5$, sumamos $1$ al segundo dígito, y obtenemos $0.67$.

    https://www.expii.com/t/rounding-decimals-definition-examples-9071

    Redondea 6,121.87856 a la milésima más cercana.
    Paso uno: Identifique el dígito en el valor posicional dado. Desde nuestro decimal
    es 6,121.87856 , el número en el lugar de las milésimas es 8:
    \[6,121.87\boxed{8}56\]


    Paso dos: identifica el dígito al lado. El dígito al lado del 8 en el
    lugar de las milésimas es 5:
    \[6,121.87 \boxed{8} \boxed{5} 6\]

    Paso tres: redondea el nuevo número (el compuesto por los dígitos del Paso
    Uno y Paso Dos) a la decena más cercana. Este nuevo número es 85, que obtuvimos
    del 8 en el paso uno y del 5 en el paso dos:
    \[6,121.87 \boxed{85} 6\]
    Cuando redondeamos a la decena más cercana, obtenemos 90:
    \[6,121.87 \boxed{90} 6\]
    Paso cuatro: ¡Elimine todos los dígitos después del valor posicional dado! Recuerda, nuestro
    el valor posicional dado era el lugar de las milésimas. Cuando eliminamos todos los dígitos después
    el 9 en el lugar de las milésimas, obtenemos:
    \[6,121.879\]

    Redondear un número es una forma de hacer que un número sea menos exacto al comvertirlo
    en una estimación.

    ¿Por qué querrías que un número fuera menos preciso?, podrías preguntar. Para principiantes,
    Los números redondeados son fáciles de hacer con los cálculos. Es mucho más fácil agregar,
    \[300 + 500\],
    que sumar,
    \[312 + 498\]
    Otras veces simplemente no necesitas precisión. Si fuiste a un concierto, no necesitas decir que había 399,342 personas
    en el estadio. Simplemente diría que había alrededor de 400,000.

    Para redondear decimales, primero debe asegurarse de conocer su lugar valores. En su mayoría, se le pedirá que redondee un número a un valor posicional específico. Para hacer esto, mira el número a la derecha del valor posicional. Si es un 5 o más, eleva el número en el valor posicional en uno. Si se
    es un 4 o menos, deja el número solo en el valor posicional. Una vez que hagas esto,
    convierte todos los dígitos a la derecha del valor posicional en 0.

    Supongamos que queremos aproximar 42.49275 a la milésima más cercana.

    Primero se localiza el valor posicional dado. En este caso, milésimas.
    \[42.49\boxed{2} 75\]

    A continuación, miramos el dígito a la derecha.
    \[42.49\boxed{2} \boxed{7}5\]

    Aquí, es 7. Como esto es mayor que 5, aumentamos el número del valor posicional en uno.
    \[42.49\boxed{3} \boxed{0}5\]
    Finalmente, eliminamos todos los dígitos a la derecha del valor posicional.
    \[42.49\boxed{3} \]


    Ahora supongamos que queremos redondear 42.09998 a la diezmilésima más cercana.

    Ubicando el valor posicional encontramos al digito 9.
    \[42.099\boxed{9} \boxed{8}\]
    El número a su derecha, 8, que es mayor que 5 por lo que aumentamos nuestro valor posicional en uno. Pero
    esto convierte nuestro 9 en un 10. Convierte el 9 en un 0 y aumenta
    el siguiente dígito a la izquierda en uno. Esto convertirá cualquier 9 seguido en 0
    finalmente aumentando décimas en uno.
    \[42. \boxed{100}\boxed{0}\boxed{8}\]
    Finalmente, eliminamos cualquier dígito a la derecha de nuestro valor posicional.
    \[42.100\]
    Podemos omitir cualquier exceso de ceros.
    \[42.1\]

    En el caso especial en el que queremos redondear al número entero más cercano,
    no habrá nada después del punto decimal una vez que hayamos terminado.

    Redondear decimales funciona igual que redondear números enteros.

    Para redondear un decimal a un valor posicional dado, mire el dígito en el lugar
    A la derecha.

    Si el dígito es menor que 5, se redondea hacia abajo. Si el dígito es mayor que
    o igual a 5, se redondea hacia arriba. Por ejemplo, redondeemos 1.41 al más cercano
    décimos Seleccionemos el dígito de las décimas, 4, con azul.
    \[1. {\color{blue!80!black} 4 } 1\]

    A continuación, seleccione el dígito a la derecha.
    \[1. {\color{blue!80!black} 4 } {\color{green!80!black} 1 }\]

    Como 1 es menor que 5, vamos a redondear hacia abajo. Dejamos 4 tal cual y
    haga que todos los dígitos a la derecha sean 0.
    \[1. {\color{blue!80!black} 4 } {\color{green!80!black} 0 }\]

    \paragraph{Truncamiento} Para truncar un decimal, debemos decidir cuántos dígitos queremos en el resultado. Luego, eliminamos todos los dígitos después del último dígito que queremos en el resultado.
    Por ejemplo, para truncar $0.6666\dots$ a $2$ dígitos, eliminamos todos los dígitos después del segundo dígito, y obtenemos $0.66$.

    \section{Recta Num\'erica, Densidad y Orden}
    La recta numérica es una línea recta en la que se representan los números reales. La recta numérica es una forma de representar los números reales en una línea recta. En ella, cada punto representa un número real. Los números se pueden representar y se pueden comparar los números reales en la recta numérica.

    \subsection{Orden de fracciones y decimales}

    \subsubsection{Orden en los n\'umeros fraccionarios}
    Para comparar fracciones, debemos convertirlas a un mismo denominador. Por ejemplo, para comparar $\frac{1}{2}$ y $\frac{3}{4}$, debemos convertirlas a un mismo denominador. El mínimo común múltiplo de $2$ y $4$ es $4$, por lo que debemos convertir $\frac{1}{2}$ a $\frac{2}{4}$, y $\frac{3}{4}$ a $\frac{3}{4}$. Ahora podemos compararlas, y vemos que:

    \[\frac{2}{4}<\frac{3}{4}\]

    \begin{center}
        \begin{tikzpicture}[scale=1]
            % Recta numérica
            \draw[latex-latex] (-0.8,0) -- (10.2,0);

            % Números racionales entre 0 y 1
            \foreach \x in {0,2.5,5,7.5,10}
            \draw (\x,0.1) -- (\x,-0.1);

            % Etiquetas de números racionales
            \foreach \x/\label in { 0/{0}, 2.5/{\dfrac{1}{4}}, 5/{{\color{gray}\dfrac{1}{2}=}\dfrac{2}{4}}, 7.5/{\dfrac{3}{4}},10/{1}}
            \node[below] at (\x,-0.1) {$\label$};

            % Puntos rojo y verde
            \fill[color=red!50!black] (5,0) circle (0.16);
            \fill[color=red!50!black] (7.5,0) circle (0.16);
        \end{tikzpicture}
    \end{center}

    \subsubsection{Orden en los n\'umeros decimales}

    Para comparar decimales, debemos convertirlos a un mismo número de decimales. Por ejemplo, para comparar $0.5$ y $0.35$, debemos convertirlos a un mismo número de decimales. El número de decimales de $0.5$ es $1$, y el número de decimales de $0.35$ es $2$. Para convertir $0.5$ a $0.50$, debemos agregar un $0$ al final. Ahora podemos compararlos, y vemos que:

    \[0.50>0.35\]

    \begin{center}
        \begin{tikzpicture}[scale=1]
            % Recta numérica
            \draw[latex-latex] (-0.8,0) -- (10.2,0);

            % Números racionales entre 0 y 1
            \foreach \x in {0,2.5,5,7.5,10}
            \draw (\x,0.1) -- (\x,-0.1);

            % Etiquetas de números racionales
            \foreach \x/\label in { 0/{0}, 2.5/{0.25}, 3.5/{0.35}, 5/{0.50}, 7.5/{0.75},10/{1}}
            \node[below] at (\x,-0.1) {$\label$};

            % Puntos rojo y verde
            \fill[color=red!50!black] (5,0) circle (0.16);
            \fill[color=red!50!black] (3.5,0) circle (0.16);
        \end{tikzpicture}
    \end{center}

    \subsection{Densidad de fracciones y decimales}

    En teoría de conjuntos, se dice que un conjunto numérico es \textbf{denso}, si entre dos elementos cualesquiera del conjunto, existe otro elemento del conjunto. Por ejemplo, el conjunto de los números racionales es denso, porque entre dos números racionales cualesquiera, existe otro número racional.

    Por ejemplo, entre $\frac{1}{2}$ y $\frac{3}{4}$, existe $\frac{5}{8}$.\\

    % Representacion detallada en Tikz de una recta numérica que muestra la densidad de los números racionales. 
    % Se muestra la recta numérica con los números racionales entre 0 y 1, y se muestra con etiquetas que entre dos números racionales como el 1/2 y  el 3/4,
    % existe otro número racional como el 5/8.
    \begin{tikzpicture}[scale=2]
        % Recta numérica
        \draw[latex-latex] (4,0) -- (8,0);

        % Números racionales entre 0 y 1
        \foreach \x in { 5, 6, 7  }
        \draw (\x,0.1) -- (\x,-0.1);

        % Etiquetas de números racionales
        \foreach \x/\label in { 5/{\dfrac{1}{2}}, 6.25/{\dfrac{5}{8}}, 7.5/{\dfrac{3}{4}}}
        \node[below] at (\x,-0.1) {$\label$};

        % Puntos rojo y verde
        \fill[color=red!50!black] (5,0) circle (0.08);
        \fill[color=red!50!black] (7.5,0) circle (0.08);
        \fill[color=green!50!black] (6.25,0) circle (0.08);
    \end{tikzpicture}

    Si observamos ahora en medio del $\frac{1}{2}$ y el $\frac{5}{8}$, está el $\frac{9}{16}$.\\

    % Representacion detallada en Tikz de una recta numérica que muestra la densidad de los números racionales. 
    % Se muestra la recta numérica con los números racionales entre 0 y 1, y se muestra con etiquetas que entre dos números racionales como el 1/2 y  el 5/8,
    % existe otro número racional como el 9/16
    \begin{tikzpicture}[scale=2]
        % Recta numérica
        \draw[latex-latex] (4,0) -- (8,0);

        % Números racionales entre 0 y 1
        \foreach \x in {  5, 6, 7}
        \draw (\x,0.1) -- (\x,-0.1);

        % Etiquetas de números racionales
        \foreach \x/\label in {5/{\dfrac{1}{2}}, 5.625/{\dfrac{9}{16}}, 6.25/{\dfrac{5}{8}}, 7.5/{\dfrac{3}{4}}}
        \node[below] at (\x,-0.1) {$\label$};

        % Puntos rojo y verde
        \fill[color=red!50!black] (5,0) circle (0.08);
        \fill[color=green!50!black] (6.25,0) circle (0.08);
        \fill[color=red!50!black] (7.5,0) circle (0.08);
        \fill[color=blue!50!black] (5.625,0) circle (0.08);
    \end{tikzpicture}

    Y si observamos ahora en medio del $\frac{1}{2}$ y el $\frac{9}{16}$, esta el $\frac{19}{32}$.\\

    % Representacion detallada en Tikz de una recta numérica que muestra la densidad de los números racionales.
    % Se muestra la recta numérica con los números racionales entre 0 y 1, y se muestra con etiquetas que entre dos números racionales como el 1/2 y  el 9/16,
    % existe otro número racional como el 19/32

    \begin{tikzpicture}[scale=2]
        % Recta numérica
        \draw[latex-latex] (4,0) -- (8,0);

        % Números racionales entre 0 y 1
        \foreach \x in {  5, 6, 7}
        \draw (\x,0.1) -- (\x,-0.1);

        % Etiquetas de números racionales
        \foreach \x/\label in {5/{\dfrac{1}{2}}, 5.3125/{\dfrac{19}{32}}, 5.625/{\dfrac{9}{16}}, 6.25/{\dfrac{5}{8}}, 7.5/{\dfrac{3}{4}}}
        \node[below] at (\x,-0.1) {$\label$};
        % Puntos rojo y verde
        \fill[color=red!50!black] (5,0) circle (0.08);
        \fill[color=green!50!black] (6.25,0) circle (0.08);
        \fill[color=red!50!black] (7.5,0) circle (0.08);
        \fill[color=blue!50!black] (5.625,0) circle (0.08);
        \fill[color=yellow!50!black] (5.3125,0) circle (0.08);
    \end{tikzpicture}

    \subsubsection{Densidad de los decimales}
    Los decimales también son densos. Por ejemplo, entre $0.5$ y $0.6$, existe $0.55$.\\

    % Representacion detallada en Tikz de una recta numérica que muestra la densidad de los números decimales.
    % Se muestra la recta numérica con los números decimales entre 0 y 1, y se muestra con etiquetas que entre dos números decimales como el 0.5 y  el 0.6,
    % existe otro número decimal como el 0.55
    \begin{tikzpicture}[scale=3]
        % Recta numérica
        \draw[latex-latex] (4.8,0) -- (6.2,0);

        % Números decimales entre 0 y 1
        \foreach \x in {  5, 6}
        \draw (\x,0.08) -- (\x,-0.08);

        % Etiquetas de números decimales
        \foreach \x/\label in {5/{0.5}, 5.5/{0.55}, 6/{0.6}}
        \node[below] at (\x,-0.05) {$\label$};

        % Puntos rojo y verde
        \fill[color=red!50!black] (5,0) circle (0.048);
        \fill[color=red!50!black] (6,0) circle (0.048);
        \fill[color=blue!50!black] (5.5,0) circle (0.048);
    \end{tikzpicture}

    Si observamos de cerca ahora en medio del $0.5$ y el $0.55$, esta el $0.525$.\\

    % Representacion detallada en Tikz de una recta numérica que muestra la densidad de los números decimales.
    % Se muestra la recta numérica con los números decimales entre 0 y 1, y se muestra con etiquetas que entre dos números decimales como el 0.5 y  el 0.55,
    % existe otro número decimal como el 0.525
    \begin{tikzpicture}[scale=5]
        % Recta numérica
        \draw[latex-latex] (4.8,0) -- (6.2,0);
        % Números decimales entre 0 y 1
        \foreach \x in {  5, 6}
        \draw (\x,0.06) -- (\x,-0.06);
        % Etiquetas de números decimales
        \foreach \x/\label in {5/{0.5}, 5.25/{0.525}, 5.5/{0.55}, 6/{0.6}}
        \node[below] at (\x,-0.05) {$\label$};
        % Puntos rojo y verde
        \fill[color=red!50!black] (5,0) circle (0.032);
        \fill[color=red!50!black] (6,0) circle (0.032);
        \fill[color=blue!50!black] (5.5,0) circle (0.032);
        \fill[color=green!50!black] (5.25,0) circle (0.032);
    \end{tikzpicture}

    Y si observamos de cerca ahora en medio del $0.5$ y el $0.525$, esta el $0.5125$.\\

    % Representacion detallada en Tikz de una recta numérica que muestra la densidad de los números decimales.
    % Se muestra la recta numérica con los números decimales entre 0 y 1, y se muestra con etiquetas que entre dos números decimales como el 0.5 y  el 0.525,
    % existe otro número decimal como el 0.5125
    \begin{tikzpicture}[scale=10]
        % Recta numérica
        \draw[latex-latex] (4.8,0) -- (6.2,0);
        % Números decimales entre 0 y 1
        \foreach \x in {  5, 6}
        \draw (\x,0.04) -- (\x,-0.04);
        % Etiquetas de números decimales
        \foreach \x/\label in {5/{0.5}, 5.125/{0.5125}, 5.25/{0.525}, 5.5/{0.55}, 6/{0.6}}
        \node[below] at (\x,-0.05) {$\label$};
        % Puntos rojo y verde
        \fill[color=red!50!black] (5,0) circle (0.02);
        \fill[color=red!50!black] (6,0) circle (0.02);
        \fill[color=blue!50!black] (5.5,0) circle (0.02);
        \fill[color=green!50!black] (5.25,0) circle (0.02);
        \fill[color=yellow!50!black] (5.125,0) circle (0.02);
    \end{tikzpicture}

    A esta propiedad de los números racionales (fracciones, decimales y enteros), se le llama \textbf{densidad}.


    \section{Aritmética de números enteros (positivos y negativos)}

    \subsection{Adición de n\'umeros enteros}
    \subsubsection{Suma de numeros enteros}
    \subsubsection{Conmutatividad aditiva}
    \subsubsection{Resta de n\'umeros enteros}
    \subsection{Producto de n\'umeros enteros}
    \subsubsection{Multiplicación de números enteros}
    \subsubsection{Conmutatividad multiplicativa}
    \subsubsection{División de números enteros}

    \section{Aritmética de números racionales (fraccionarios y decimales)}

    \subsection{Adición de n\'umeros fraccionarios y decimales}
    \subsubsection{Suma de numeros fraccionarios y decimales}
    \subsubsection{Resta de n\'umeros fraccionarios y decimales}
    \subsection{Producto de n\'umeros fraccionarios y decimales}
    \subsubsection{Multiplicación de números fraccionarios y decimales}
    \subsubsection{División de números fraccionarios y decimales}

    \section{Jerarqu\'ia de operaciones y signos de agrupaci\'on}
    \subsection{Jerarqu\'ia de operaciones}
    \subsection{Signos de agrupaci\'on}

    \mychapter{Unidad 2}

    \section{Perímetros y áreas de figuras geométricas}
    \subsection{Perímetro de triángulos y cuadriláteros}
    \subsection{Áreas de triángulos y cuadriláteros}

    \section{\'Angulos, tri\'angulos y cuadril\'ateros}
    \subsection{\'Angulos y rectas paralelas}
    \subsection{Suma de los \'angulos interiores de un tri\'angulo y de un cuadril\'atero}
    \subsubsection{\'Angulos de un tri\'angulo}
    \subsubsection{\'Angulos de un cuadril\'atero}

    \section{Medidas de tendencia central}
    \subsection{Media aritm\'etica o promedio}
    \subsubsection{El rango}
    \subsection{La mediana}
    \subsection{La moda}

    \section{El azar y la probabilidad frecuencial}
    \subsection{Tipos, recolección y organización de datos}
    \subsection{Experimentos aleatorios y deterministas}
    \subsection{Espacio muestral de un experimento aleatorio}
    \subsection{Cálculo de la probabilidad frecuencial}

    \mychapter{Unidad 3}

    \section{Proporcionalidad}
    \subsection{Valor faltante}
    \subsection{Razón unitaria}
    \subsection{Porcentajes}
    \subsubsection{Cálculo del porcentaje}
    \subsubsection{Problemas con porcentajes}
    \subsection{Gráficas circulares}
    \subsubsection{Recolecta y registra datos}
    \subsubsection{Registra datos en gráficas circulares}
    \subsubsection{Leer e interpretar datos en gráficas circulares}

    \section{Situaciones de variación proporcional}
    \subsection{Comparación de situaciones de variación proporcional con tablas}
    \subsection{Comparación de situaciones de variación proporcional con gráficas}
    \subsection{Comparación de situaciones de variación proporcional con expresiones algebraicas}

    \section{Pendiente de una recta y razón de cambio}
    \subsection{Variación proporcional y pendiente}
    \subsection{Razón de cambio y variación}
    \subsection{Efectos en la recta al cambiar la pendiente}
    \subsection{Efectos en la recta al cambiar la ordenada al origen}

    \section{Análisis y comparación de situaciones de variación lineal}
    \subsection{Efectos de la recta al cambiar la ordenada al origen}
    \subsection{Situaciones de variación lineal asociadas a la física, la biología y la economía}


    \section{Fundamentos de álgebra}
    \subsection{Lenguaje algebraico y expresiones algebraicas}
    \subsection{Aritmética de expresiones algebraicas}
    \subsubsection{Adición de expresiones algebraicas}
    \subsubsection{Producto y cociente de números racionales con expresiones algebraicas}
    \subsection{Ecuaciones}
    \subsubsection{Solución de ecuaciones}





\end{mainmatter}
\end{document}
