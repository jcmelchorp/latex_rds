\documentclass[12pt]{book}
\usepackage[undotted]{minitoc}
\usepackage{amsmath}
\usepackage{amssymb}
\usepackage{graphicx}
\usepackage[spanish]{babel}
\usepackage[utf8]{inputenc}
\usepackage{tikz}
\usetikzlibrary{shadows.blur}
\usepackage{titlesec}

\addtocontents{toc}{\setcounter{tocdepth}{4}}
\setcounter{minitocdepth}{3}
\nomtcrule         % removes rules = horizontal lines from minitocs
\nomtcpagenumbers  % remove page numbers from minitocs

\newcommand{\mychapter}[1]{
    \thispagestyle{empty}
    \chapter*{#1}
    \addcontentsline{toc}{chapter}{#1}
    \mtcaddchapter{}
    \mtcsettitle{minitoc}{}
    {\Large \scshape En esta unidad estudiaremos \dots}
    \minitoc
    \newpage
    \thispagestyle{empty} \mbox{}
    \newpage
}
\addto\captionsspanish{%
    \renewcommand{\chaptername}{Unidad}
    \renewcommand{\contentsname}{Contenido} 
}
%\usepackage{remreset}
% \makeatletter
% \@removefromreset{section}{chapter}
% \makeatother
% \renewcommand{\thechapter}{\arabic{chapter}}
% \newcommand*\chapterlabel{}
% \titleformat{\chapter}

\begin{document}
\dominitoc[n]%% Crea minitoc
\title{Matemáticas 1}
\author{Melchor Pinto, JC}
\date{\today}
\maketitle
\tableofcontents

\begin{mainmatter}
    \mychapter{Unidad 1}

    \section{Fracciones y decimales}
    \subsection{Equivalencias de fracciones y decimales}
    \subsection{Decimales peri\'odicos}
    \subsubsection{Redondeo y truncamiento}

    \section{Recta Num\'erica, Densidad y Orden}
    \subsection{Fracciones y decimales en la Recta Num\'erica}
    \subsection{Densidad de fracciones y decimales}
    \subsection{Orden de fracciones y decimales}
    \subsubsection{Orden en los n\'umeros fraccionarios}
    \subsubsection{Orden en los n\'umeros decimales}

    \section{Aritmética de números enteros (positivos y negativos)}

    \subsection{Adición de n\'umeros enteros}
    \subsubsection{Suma de numeros enteros}
    \subsubsection{Conmutatividad aditiva}
    \subsubsection{Resta de n\'umeros enteros}
    \subsection{Producto de n\'umeros enteros}
    \subsubsection{Multiplicación de números enteros}
    \subsubsection{Conmutatividad multiplicativa}
    \subsubsection{División de números enteros}

    \section{Aritmética de números racionales (fraccionarios y decimales)}

    \subsection{Adición de n\'umeros fraccionarios y decimales}
    \subsubsection{Suma de numeros fraccionarios y decimales}
    \subsubsection{Resta de n\'umeros fraccionarios y decimales}
    \subsection{Producto de n\'umeros fraccionarios y decimales}
    \subsubsection{Multiplicación de números fraccionarios y decimales}
    \subsubsection{División de números fraccionarios y decimales}

    \section{Jerarqu\'ia de operaciones y signos de agrupaci\'on}
    \subsection{Jerarqu\'ia de operaciones}
    \subsection{Signos de agrupaci\'on}

    \mychapter{Unidad 2}

    \section{Perímetros y áreas de figuras geométricas}
    \subsection{Perímetro de triángulos y cuadriláteros}
    \subsection{Áreas de triángulos y cuadriláteros}

    \section{\'Angulos, tri\'angulos y cuadril\'ateros}
    \subsection{\'Angulos y rectas paralelas}
    \subsection{Suma de los \'angulos interiores de un tri\'angulo y de un cuadril\'atero}
    \subsubsection{\'Angulos de un tri\'angulo}
    \subsubsection{\'Angulos de un cuadril\'atero}

    \section{Medidas de tendencia central}
    \subsection{Media aritm\'etica o promedio}
    \subsubsection{El rango}
    \subsection{La mediana}
    \subsection{La moda}

    \section{El azar y la probabilidad frecuencial}
    \subsection{Tipos, recolección y organización de datos}
    \subsection{Experimentos aleatorios y deterministas}
    \subsection{Espacio muestral de un experimento aleatorio}
    \subsection{Cálculo de la probabilidad frecuencial}

    \mychapter{Unidad 3}

    \section{Proporcionalidad}
    \subsection{Valor faltante}
    \subsection{Razón unitaria}
    \subsection{Porcentajes}
    \subsubsection{Cálculo del porcentaje}
    \subsubsection{Problemas con porcentajes}
    \subsection{Gráficas circulares}
    \subsubsection{Recolecta y registra datos}
    \subsubsection{Registra datos en gráficas circulares}
    \subsubsection{Leer e interpretar datos en gráficas circulares}

    \section{Situaciones de variación proporcional}
    \subsection{Comparación de situaciones de variación proporcional con tablas}
    \subsection{Comparación de situaciones de variación proporcional con gráficas}
    \subsection{Comparación de situaciones de variación proporcional con expresiones algebraicas}

    \section{Pendiente de una recta y razón de cambio}
    \subsection{Variación proporcional y pendiente}
    \subsection{Razón de cambio y variación}
    \subsection{Efectos en la recta al cambiar la pendiente}
    \subsection{Efectos en la recta al cambiar la ordenada al origen}

    \section{Análisis y comparación de situaciones de variación lineal}
    \subsection{Efectos de la recta al cambiar la ordenada al origen}
    \subsection{Situaciones de variación lineal asociadas a la física, la biología y la economía}


    \section{Fundamentos de álgebra}
    \subsection{Lenguaje algebraico y expresiones algebraicas}
    \subsection{Aritmética de expresiones algebraicas}
    \subsubsection{Adición de expresiones algebraicas}
    \subsubsection{Producto y cociente de números racionales con expresiones algebraicas}
    \subsection{Ecuaciones}
    \subsubsection{Solución de ecuaciones}





\end{mainmatter}
\end{document}
