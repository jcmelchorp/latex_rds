\documentclass[12pt]{book}
\usepackage[undotted]{minitoc}
\usepackage{amsmath}
\usepackage{amssymb}
\usepackage{graphicx}
\usepackage[spanish]{babel}
\decimalpoint
\usepackage[utf8]{inputenc}
\usepackage{tikz}
\usepackage{longdivision}
\usetikzlibrary{shadows.blur}
\usepackage{titlesec}
\usepackage{pgfkeys}
\usepackage{multicol}
%% equivalencias al entero
\newcounter{CountOfSections}
\newcommand{\fracgraph}[3][1]{%
    % #1 = optional height
\begin{tikzpicture}
    \draw (0,0) rectangle (#2,#1) node [midway] {1};

    \setcounter{CountOfSections}{0}%
    \foreach \Size/\Options in {#3} {%
        \stepcounter{CountOfSections}%
        \pgfmathsetmacro{\YCoord}{#1*\arabic{CountOfSections}}%
        \draw  (0,-\YCoord) rectangle (#2,-\YCoord+#1);
        \pgfmathsetmacro{\Xincrement}{#2/\Size}%
         \foreach \x in {1,...,\Size} {%
            \pgfmathsetmacro{\Xcoord}{\x*\Xincrement}%
            \pgfmathsetmacro{\XcoordLabel}{(\x-0.5)*\Xincrement}%
            \draw [fill=\Options]  ($(\Xcoord-\Xincrement,-\YCoord)$)  rectangle ($(\Xcoord,-\YCoord+#1)$);
            \node at ($(\XcoordLabel,-\YCoord+0.5*#1)$) {$\frac{1}{\Size}$};
        }%
    }%
\end{tikzpicture}
}
%%%%
%% Recta numérica (intervalo)
\newcommand{\rectnuminterval}[5][]{
        \begin{tikzpicture}[#1]
            \draw[very thick] (#2,0.5) -- (#3,0.5);
            \path [draw=black, fill=#4] (#2,0.5) circle (2pt);
            \path [draw=black, fill=#5, thick] (#3,0.5) circle (2pt);
            \draw[latex-latex] (-3.5,0) -- (3.5,0) ;
            \foreach \x in  {-3,-2,-1,0,1,2,3}
            \draw[shift={(\x,0)},color=black] (0pt,3pt) -- (0pt,-3pt);
            \foreach \x in {-3,-2,-1,0,1,2,3}
            \draw[shift={(\x,0)},color=black] (0pt,0pt) -- (0pt,-3pt) node[below]
            {$\x$};
        \end{tikzpicture}
    }
%%%%
\addtocontents{toc}{\setcounter{tocdepth}{4}}
\setcounter{minitocdepth}{3}
\nomtcrule         % removes rules = horizontal lines from minitocs
\nomtcpagenumbers  % remove page numbers from minitocs

\newcommand{\mychapter}[1]{
    \thispagestyle{empty}
    \chapter*{#1}
    \addcontentsline{toc}{chapter}{#1}
    \mtcaddchapter{}
    \mtcsettitle{minitoc}{}
    {\Large \scshape En esta unidad estudiaremos \dots}
    \minitoc
    \newpage
    \thispagestyle{empty} \mbox{}
    \newpage
}
\addto\captionsspanish{%
    \renewcommand{\chaptername}{Unidad}
    \renewcommand{\contentsname}{Contenido} 
}
%\usepackage{remreset}
% \makeatletter
% \@removefromreset{section}{chapter}
% \makeatother
% \renewcommand{\thechapter}{\arabic{chapter}}
% \newcommand*\chapterlabel{}
% \titleformat{\chapter}

\begin{document}
\dominitoc[n]%% Crea minitoc
\title{Matemáticas 1}
\author{Melchor Pinto, JC}
\date{\today}
\maketitle
\tableofcontents

\begin{mainmatter}
    \mychapter{Unidad 1}


    \subsection{Producto de n\'umeros enteros}
    \subsubsection{Multiplicación de números enteros}
    \subsubsection{Conmutatividad multiplicativa}
    \subsubsection{División de números enteros}

    \section{Aritmética de números racionales (fraccionarios y decimales)}

    \subsection{Adición de n\'umeros fraccionarios y decimales}
    \subsubsection{Suma de numeros fraccionarios y decimales}
    \subsubsection{Resta de n\'umeros fraccionarios y decimales}
    \subsection{Producto de n\'umeros fraccionarios y decimales}
    \subsubsection{Multiplicación de números fraccionarios y decimales}
    \subsubsection{División de números fraccionarios y decimales}

    \section{Jerarqu\'ia de operaciones y signos de agrupaci\'on}
    \subsection{Jerarqu\'ia de operaciones}
    \subsection{Signos de agrupaci\'on}

    \mychapter{Unidad 2}

    \section{Perímetros y áreas de figuras geométricas}
    \subsection{Perímetro de triángulos y cuadriláteros}
    \subsection{Áreas de triángulos y cuadriláteros}

    \section{\'Angulos, tri\'angulos y cuadril\'ateros}
    \subsection{\'Angulos y rectas paralelas}
    \subsection{Suma de los \'angulos interiores de un tri\'angulo y de un cuadril\'atero}
    \subsubsection{\'Angulos de un tri\'angulo}
    \subsubsection{\'Angulos de un cuadril\'atero}

    \section{Medidas de tendencia central}
    \subsection{Media aritm\'etica o promedio}
    \subsubsection{El rango}
    \subsection{La mediana}
    \subsection{La moda}

    \section{El azar y la probabilidad frecuencial}
    \subsection{Tipos, recolección y organización de datos}
    \subsection{Experimentos aleatorios y deterministas}
    \subsection{Espacio muestral de un experimento aleatorio}
    \subsection{Cálculo de la probabilidad frecuencial}

    \mychapter{Unidad 3}

    \section{Proporcionalidad}
    \subsection{Valor faltante}
    \subsection{Razón unitaria}
    \subsection{Porcentajes}
    \subsubsection{Cálculo del porcentaje}
    \subsubsection{Problemas con porcentajes}
    \subsection{Gráficas circulares}
    \subsubsection{Recolecta y registra datos}
    \subsubsection{Registra datos en gráficas circulares}
    \subsubsection{Leer e interpretar datos en gráficas circulares}

    \section{Situaciones de variación proporcional}
    \subsection{Comparación de situaciones de variación proporcional con tablas}
    \subsection{Comparación de situaciones de variación proporcional con gráficas}
    \subsection{Comparación de situaciones de variación proporcional con expresiones algebraicas}

    \section{Pendiente de una recta y razón de cambio}
    \subsection{Variación proporcional y pendiente}
    \subsection{Razón de cambio y variación}
    \subsection{Efectos en la recta al cambiar la pendiente}
    \subsection{Efectos en la recta al cambiar la ordenada al origen}

    \section{Análisis y comparación de situaciones de variación lineal}
    \subsection{Efectos de la recta al cambiar la ordenada al origen}
    \subsection{Situaciones de variación lineal asociadas a la física, la biología y la economía}


    \section{Fundamentos de álgebra}
    \subsection{Lenguaje algebraico y expresiones algebraicas}
    \subsection{Aritmética de expresiones algebraicas}
    \subsubsection{Adición de expresiones algebraicas}
    \subsubsection{Producto y cociente de números racionales con expresiones algebraicas}
    \subsection{Ecuaciones}
    \subsubsection{Solución de ecuaciones}





\end{mainmatter}
\end{document}
