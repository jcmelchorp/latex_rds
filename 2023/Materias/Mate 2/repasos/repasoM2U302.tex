\documentclass[12pt,addpoints]{repaso}
\grado{2}
\nivel{Secundaria}
\cicloescolar{2023-2024}
\materia{Matemáticas}
\unidad{3}
\title{Practica la Unidad}
\aprendizajes{
    \item Usa e interpreta las medidas de tendencia central (moda, media aritmética y mediana), y decide cuál de ellas conviene más en el análisis de los datos en cuestión.
    \item Resuelve problemas de proporcionalidad directa e inversa y de reparto proporcional.
    \item Resuelve problemas mediante la formulación y solución algebraica de ecuaciones lineales.
    \item Verifica algebraicamente la equivalencia de expresiones de primer grado, formuladas a partir de sucesiones.      
}
\author{Melchor Pinto, J.C.}
\begin{document}
\INFO%
\begin{questions}
    \questionboxed[10]{}
    \section*{\ifprintanswers   {Probabilidad y estadística}\else{}\fi}
    \subsection*{\ifprintanswers{Mediana y moda}\else{}\fi}
    \subsection*{\ifprintanswers{Promedio}\else{}\fi}
    \subsection*{\ifprintanswers{Interpretación de gráficas}\else{}\fi}
    \subsection*{\ifprintanswers{Eventos mutuamente excluyentes}\else{}\fi}
    \subsection*{\ifprintanswers{Eventos dependientes e independientes}\else{}\fi}
    \section*{\ifprintanswers   {Razones y proporciones}\else{}\fi}
    \subsection*{\ifprintanswers{Relaciones proporcionales}\else{}\fi}
    \subsection*{\ifprintanswers{Constante de proporcionalidad}\else{}\fi}
    \subsection*{\ifprintanswers{Regla de correspondencia}\else{}\fi}
    \subsection*{\ifprintanswers{Proporción directa e inversa}\else{}\fi}
    \subsection*{\ifprintanswers{Proporciones compuestas}\else{}\fi}
    \section*{\ifprintanswers   {Sucesiones aritméticas}\else{}\fi}
    \subsection*{\ifprintanswers{Completando la sucesión}\else{}\fi}
    \subsection*{\ifprintanswers{Diferencia de una sucesión}\else{}\fi}
    \subsection*{\ifprintanswers{Término enésimo 1}\else{}\fi}
    \subsection*{\ifprintanswers{Término general}\else{}\fi}
    \subsection*{\ifprintanswers{Término enésimo 2}\else{}\fi}
    \section*{\ifprintanswers   {Ecuaciones lineales}\else{}\fi}
    \subsection*{\ifprintanswers{Lenguaje algebraico}\else{}\fi}
    \subsection*{\ifprintanswers{Sustitución de valores}\else{}\fi}
    \subsection*{\ifprintanswers{Ecuaciones de primer grado 1}\else{}\fi}
    \subsection*{\ifprintanswers{Ecuaciones de primer grado 2}\else{}\fi}
    \subsection*{\ifprintanswers{Resolución de problemas}\else{}\fi}
    \section*{\ifprintanswers   {Sistemas de ecuaciones}\else{}\fi}
    \subsection*{\ifprintanswers{Método de eliminación}\else{}\fi}
    \subsection*{\ifprintanswers{Método de sustitución}\else{}\fi}
    \subsection*{\ifprintanswers{Método de igualación}\else{}\fi}
    \subsection*{\ifprintanswers{Sistema de ecuaciones 2x2 1}\else{}\fi}
    \subsection*{\ifprintanswers{Sistema de ecuaciones 2x2}\else{}\fi}
\end{questions}
\end{document}