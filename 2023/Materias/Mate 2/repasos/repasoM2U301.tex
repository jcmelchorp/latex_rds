\documentclass[12pt,addpoints]{repaso}
\grado{2$^\circ$ de Secundaria}
\cicloescolar{2022-2023}
\materia{Matemáticas 2}
\guia{3}
\unidad{3}
\title{Preparación para el Examen de la Unidad}
\aprendizajes{
    \item Verifica algebraicamente la equivalencia de expresiones de primer
    grado, formuladas a partir de sucesiones.
    \item Formula expresiones de primer grado para representar propiedades
    (perímetros y áreas)
    de figuras geométricas y verifica equivalencia de expresiones, tanto
    algebraica como geométricamente (análisis de las figuras).
    \item Calcula el volumen de prismas y cilindros rectos.
    }
\author{JC Melchor Pinto}
\begin{document}
\INFO%
\begin{multicols}{2}%
    \include*{../blocks/block002}
    \include*{../blocks/block035c}
    \include*{../blocks/block030e}
    \include*{../blocks/block035b}
\end{multicols}%
\begin{questions}    
    \ejemplosboxed[\include*{../questions/question106j}]
    \questionboxed[10]{\include*{../questions/question106h}}
    \ejemplosboxed[\include*{../questions/question110} ]
    \ejemplosboxed[\include*{../questions/question103i}]
    \ejemplosboxed[\include*{../questions/question100} ]
    \ejemplosboxed[\include*{../questions/question104c}]
    \ejemplosboxed[\include*{../questions/question102d}]
    \ejemplosboxed[\include*{../questions/question071} ]
\end{questions}
\end{document}