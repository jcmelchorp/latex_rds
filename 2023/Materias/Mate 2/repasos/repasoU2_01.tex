\documentclass[12pt]{repaso}
\grade{2$^\circ$ de Secundaria}
\cycle{2022-2023}
\subject{Matemáticas 2}
\guide{2}
\title{Repaso para el examen de la Unidad}
\aprendizajes{
    \begin{itemize}[leftmargin=*,label=\small\color{colorrds}\faIcon{user-graduate}]
        \item Resuelve problemas de proporcionalidad directa e inversa y de reparto proporcional.
        \item Obtiene la expresión algebraica y construye gráficas de una situación de
        proporcionalidad directa e inversa.
        \item Construye polígonos regulares a partir de algunas
        medidas (lados, apotema, diagonales, etcétera).
        \item Descompone figuras en otras para calcular su área.
        \item Calcula el perímetro y el área de polígonos regulares y del círculo a partir de diferentes datos.
    \end{itemize}
}
\requisitos{
    \begin{itemize}
        \item Requisito 1
        \item Requisito 2
    \end{itemize}
}
\author{J. C. Melchor Pinto}

\begin{document}
\pagestyle{headandfoot}
\addpoints
\INFO
%\printanswers
\vspace{-1cm}
\begin{multicols}{2}
    \include*{../blocks/block001}
    \include*{../blocks/block003}
    \include*{../blocks/block000}
\end{multicols}
\include*{../blocks/block002}
\begin{questions}
    {
        \printanswers
        \include*{../questions/question004}
        \include*{../questions/question001}
        \include*{../questions/question005}
        \include*{../questions/question002}
        \include*{../questions/question003}
        \newpage
        \include*{../questions/question009}
    }
    \include*{../questions/question011}
    \newpage
    \include*{../questions/question012}
    \include*{../questions/question010}
    \newpage
    {
        \printanswers
        %\include*{../questions/question007}
        \include*{../questions/question013}
        \include*{../questions/question014}
    }
    \newpage
    \include*{../questions/question006}
    \include*{../questions/question008}
    \include*{../questions/question015}
\end{questions}

%\vfill
%\puntuacion

\end{document}