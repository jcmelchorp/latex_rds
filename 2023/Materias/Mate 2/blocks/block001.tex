%----------------------------------------------------------------------------------------
%	Síntesis
%----------------------------------------------------------------------------------------
\begin{tcolorbox}[
        colback=colorrds!5!white,
        colframe=colorrds!35!white,
        coltitle=black,
        fonttitle=\bfseries,
        center title,
        title=Proporcionalidad]

    Una relación de \textbf{proporcionalidad directa} es aquella entre dos variables que aumentan o disminuyen en el mismo sentido.\\

    En una \textbf{variación proporcional directa}, la constante de proporcionalidad se obtiene
    calculando el cociente de dos cantidades que se corresponden.

    \DrawLine

    Una relación de \textbf{Proporcionalidad inversa}  es aquella entre dos variables en donde, al aumentar una variable, la otra disminuye.\\

    En una \textbf{variación proporcional inversa}, el producto de dos cantidades que se
    corresponden es la constante de proporcionalidad.


    \DrawLine

    Resolver un problema de \textbf{reparto proporcional} consiste en dividir una cantidad en
    partes que guarden entre sí ciertas razones. Para realizar el reparto, se encuentran
    los valores faltantes en una relación proporcional directa.\\

    En un problema de \textbf{reparto proporcional inverso}, se busca convertirlo en una proporción directa. Por ello, se utiliza el inverso multiplicativo o recíproco.

\end{tcolorbox}



