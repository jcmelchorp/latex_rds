Se fabricará una ventana de forma circular con un marco de acero inoxidable y
vidrio templado. El grosor del cancel es de 3 cm y el radio de la ventana de 50
cm. El precio del acero es de \$120.00 el metro y el del vidrio es de \$160.00
por metro cuadrado.

\begin{parts}
    \part ¿Cuántos metros de marco se ocuparán?

    \begin{solutionbox}{2.5cm}
        Si el radio en metros del cancel es de $r=0.3+0.1$, la longitud del
        metal utilizado podrá calcularse con base a la circunferencia $P$ del
        círculo
        con dicho radio, es decir:
        \[
            P=2\pi r=2\cdot 3.14 \cdot 0.4 \text{m} =2.512 \text{m}\]
    \end{solutionbox}

    \part ¿Cuántos metros cuadrados de vidrio se ocuparán?

    \begin{solutionbox}{1.5cm}
        \[A_v=\pi r^2=\pi\cdot0.30^2=0.283 \text{m}^2\]
    \end{solutionbox}

    \part ¿Cuál es el precio total de la ventana?

    \begin{solutionbox}{2cm}
        Multiplicando cada uno de los precios por la cantidad de material
        correspondiente, se tiene el precio total $T$ como sigue:
        \[ T= 2.512 \cdot \$1200 + 0.283 \cdot \$1600 = \$ 3467.2 \]
    \end{solutionbox}

\end{parts}