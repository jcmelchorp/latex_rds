Completa la Tabla \ref{tab:3.6}.

% \begin{multicols}{2}

\begin{table}[H]
    \rowcolors{4}{colorrds!10}{lightgray!10}
    \centering
    \caption{}
    \label{tab:3.6}
    \begin{tabular}{|r|*{5}{p{1.8em}|}}
        \toprule
        \rowcolor{colorrds!80}
                               & \multicolumn{5}{c|}{\bfseries\color{white}\minitab[c]{Posición en la sucesión}}                     \\ \cline{2-6}
        \multirow{-2}*{\cellcolor{colorrds!80}\bfseries\color{white}\minitab[c]{Regla de                                             \\ recurrencia}}                       & 1                                                                       & 2 & 3 & 4 & 5 \\ \hline
        $4\left(n+1\right)$    & 8                                                                               & 12 & 16 & 20 & 24 \\ \hline
        $4n+4$                 & 8                                                                               & 12 & 16 & 20 & 24 \\ \hline
        $2n+2\left(n-1\right)$ & 2                                                                               & 6  & 10 & 14 & 18 \\ \hline
        $4\left(n-1\right)+8$  & 8                                                                               & 12 & 16 & 20 & 24 \\ \cline{2-6}
        \bottomrule
    \end{tabular}
\end{table}

\begin{parts}
    \part ¿Hay reglas con las que obtienes los mismos términos? ¿Cuáles?

    \begin{solutionbox}{1.2cm}
        Sí, con $4(n + 1), 4n +4 y 4(n - 1)+8$.
    \end{solutionbox}

    \part Si sustituyes el mismo valor en dos o más reglas y obtienes el mismo término, ¿qué puedes decir acerca de las reglas?

    \begin{solutionbox}{1.2cm}
        Que son las mismas reglas escritas de manera diferente.
    \end{solutionbox}

    \part Simplifica las reglas de la Tabla \ref{tab:3.6}. ¿A qué expresión llegaste en cada caso?

    \begin{solutionbox}{2.4cm}
        $4(n + 1) = 4n + 4$\\
        $4n + 4 \text{ ya está simplificada}$\\
        $2n + 2(n - 1) = 2n + 2n - 2 = 4n - 2$\\
        $4(n - 1) + 8 = 4n - 4 + 8 = 4n + 4$\\
    \end{solutionbox}

    \part Con base en lo anterior, ¿qué puedes concluir acerca de las reglas?

    \begin{solutionbox}{2cm}
        Tres son la misma regla:\\
        $4(n + 1) = 4n + 4 = 4(n - 1) + 8$.\\
        La regla $2n + 2(n - 1)$ es diferente a las anteriores.
    \end{solutionbox}

    \part Reúnete con tus compañeros. Respondan con argumentos: dadas dos o más reglas, ¿qué
    significa que sean equivalentes?, ¿cómo pueden saber si lo son?

    \begin{solutionbox}{2cm}
        Se busca que los alumnos expliquen que dos o más reglas son equiva-
        lentes si para cualquier posición se obtienen los mismos términos o bien, que
        expliquen que mediante operaciones algebraicas, se puede transformar una
        expresión en la otra.
    \end{solutionbox}
\end{parts}
% \end{multicols}
