\question[10] Completa la Tabla \ref{tab:3.6}.

\begin{table}[H]
    \centering
    \caption{}
    \label{tab:3.6}
    \begin{tabular}{c|c|c|c|c|c|}
        \multirow{2}*{\minitab[c]{Regla de                \\[-0.5em] recurrencia}} &
        \multicolumn{5}{c|}{\minitab[c]{Número de término \\[-0.5em] en la sucesión}}
        \\ \cline{2-6}
                               & 1 & 2 & 3 & 4 & 5        \\ \hline
        $4\left(n+1\right)$    &   &   &   &   &          \\ \hline
        $4n+4$                 &   &   &   &   &          \\ \hline
        $2n+2\left(n-1\right)$ &   &   &   &   &          \\ \hline
        $4\left(n-1\right)+8$  &   &   &   &   &          \\ \cline{2-6}
    \end{tabular}
\end{table}

\begin{parts}
    \part ¿Hay reglas con las que obtienes los mismos términos? ¿Cuáles?
    \part Si sustituyes el mismo valor en dos o más reglas y obtienes el mismo término, ¿qué puedes decir acerca de las reglas?
    \part Simplifica las reglas de la Tabla \ref{tab:3.6}. ¿A qué expresión llegaste en cada caso?
    \part Con base en lo anterior, ¿qué puedes concluir acerca de las reglas?
    \part Reúnete con tus compañeros. Respondan con argumentos: dadas dos o más reglas, ¿qué
    significa que sean equivalentes?, ¿cómo pueden saber si lo son?
\end{parts}