\question[10] Un empaque para pelotas de tenis es un cilindro recto al que le caben tres pelotas,
cada una mide 6.8 cm de diámetro.

\begin{parts}
    \part ¿Cuánto miden el radio y la altura del empaque si se fabrica justo con las medidas de las pelotas de tenis?

    \begin{solutionbox}{1.5cm}
        El radio es el mismo de una pelota: 3.4 cm. La altura es la del diámetro de
        3 pelotas: 20.4 cm
    \end{solutionbox}

    \part ¿Cuánto es su volumen?

    \begin{solutionbox}{1.5cm}
        \[V=\pi(3.4)^2 (20.4) = 740.86 cm^3\]
    \end{solutionbox}

    \part Si el empaque se fabrica con 3 mm de holgura en la parte superior y lateral, ¿cuáles son sus dimensiones?
    \begin{solutionbox}{1.5cm}
        Al radio se le añade la mitad de la holgura: $3.4 + 0.15 = 3.55$ cm. A la altura se le suma la holgura: $20.4 + 0.3 = 20.7 \text{ cm}$.
    \end{solutionbox}

    \part ¿Cuál es su volumen?

    \begin{solutionbox}{1.5cm}
        Volumen del empaque con holgura: $V = \pi(3.55)^2 (20.7) = 819.55 cm^3$.
    \end{solutionbox}
\end{parts}


