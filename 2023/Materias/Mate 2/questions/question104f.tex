Una lata de verduras mide lo mismo de radio que de altura, que es de 16 cm.

\begin{parts}
    \part ¿Cuál es su volumen?

    \begin{solutionbox}{1.5cm}
        \[ V = \pi(16)^2(3) = 12,869 \text{ cm}^3\]
    \end{solutionbox}

    \part ¿De cuánto será el volumen de otra lata de verduras si mide 16 cm de
    diámetro y de altura?

    \begin{solutionbox}{1.5cm}
        \[V = \pi(8)^2(2) (16) = 3,217 \text{ cm}^3\]
    \end{solutionbox}

    \part  ¿Cuál lata tiene mayor volumen? ¿Cuántas veces es mayor ese volumen? \textbf{¿Cuál es su volumen?}

    \begin{solutionbox}{1.5cm}
        La primera, tiene 4 veces más volumen que la segunda.
    \end{solutionbox}
\end{parts}
