\question[10] Completa la tabla \ref{tab:3.1} Luego responde lo que se pide.

\begin{table}[H]
    \centering
    \caption{}
    \label{tab:3.1}
    \begin{tabular}{r|c|c|c|c|c|c|c|c|}
        \cline{2-9}
        Posición del término   & 1                    & 2                    & 3                    & 4                     & 5                     & \ifprintanswers6\fi  & \ifprintanswers7\fi  & 8                     \\ \hline
        Término de la sucesión & \ifprintanswers12\fi & \ifprintanswers3\fi  & \ifprintanswers-6\fi & \ifprintanswers-15\fi & \ifprintanswers-24\fi & -33                  & -42                  & \ifprintanswers-50\fi \\ \hline
        Diferencias            & \ifprintanswers-9\fi & \ifprintanswers-9\fi & \ifprintanswers-9\fi & \ifprintanswers-9\fi  & \ifprintanswers-9\fi  & \ifprintanswers-9\fi & \ifprintanswers-9\fi & \ifprintanswers-9\fi  \\ \cline{2-9}
    \end{tabular}
\end{table}

\begin{multicols}{2}
    \begin{parts}
        \part ¿Cuál es el primer término de la sucesión?

        \begin{solutionbox}{1.2cm}
            12
        \end{solutionbox}

        \part A partir del primer término, ¿cómo se obtiene el segundo?

        \begin{solutionbox}{1.2cm}
            Restando 9 unidades.
        \end{solutionbox}

        \part ¿Cómo se obtiene el tercer término de la sucesión a partir del primero?

        \begin{solutionbox}{1.8cm}
            Restando 2 por 9 = 18 unidades. El término es 12 - 9 = - 6
        \end{solutionbox}

        \part Analiza los resultados del renglón de las diferencias. ¿Qué observas?

        \begin{solutionbox}{1.2cm}
            Que se obtiene el mismo valor de -9.
        \end{solutionbox}

        \part Escribe la regla general de la diferencia entre dos términos consecutivos de la
        sucesión.

        \begin{solutionbox}{1.8cm}
            $n - m = -9$. Donde $n$ es el enésimo elemento y $m$ el consecutivo.
        \end{solutionbox}

        \part Escribe el término que ocupa la posición 60.

        \begin{solutionbox}{4cm}
            La regla general de la sucesión es $-9n + 21$. El término de la posición
            60 es $-9(60) + 21 = -540 + 21 = -519$.\\
            Se resuelve -9n + 21 = -78. Se obtiene n = 11.\\
            Se resuelve -9n + 21 = -138. Se obtiene n = 17.66\dots Como no es entero, el
            número -138 no es elemento de la sucesión.
        \end{solutionbox}

        \part ¿Qué posición tiene el término -78?

        \begin{solutionbox}{1.2cm}
        \end{solutionbox}

        \part ¿Hay alguna posición en la que aparezca el número -138? ¿Cuál?

        \begin{solutionbox}{2cm}
        \end{solutionbox}

        \part Compara la regla que obtuviste con la de tus compañeros. Si hay diferencias, argu-
        menta tu respuesta y corrige si es necesario.

        \begin{solutionbox}{2cm}
        \end{solutionbox}
    \end{parts}

\end{multicols}

\begin{importantbox}
    La \textbf{regla de recurrencia} de una sucesión es una expresión algebraica que permite calcular el valor de cada término.
\end{importantbox}