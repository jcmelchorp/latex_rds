\question[10] Completa la tabla \ref{tab:3.1} Luego responde lo que se pide.

\begin{table}[H]
    \centering
    \caption{}
    \label{tab:3.1}
    \begin{tabular}{r|c|c|c|c|c|c|c|c|}
        \cline{2-9}
        Posición del término   & 1 & 2 & 3 & 4 & 5 &     &     & 8 \\ \hline
        Término de la sucesión &   &   &   &   &   & -33 & -42 &   \\ \hline
        Diferencias            &   &   &   &   &   &     &     &   \\ \cline{2-9}
    \end{tabular}
\end{table}

\begin{parts}
    \part ¿Cuál es el primer término de la sucesión?

    \begin{solutionbox}{1.5cm}
    \end{solutionbox}

    \part A partir del primer término, ¿cómo se obtiene el segundo?

    \begin{solutionbox}{1.5cm}
    \end{solutionbox}

    \part ¿Cómo se obtiene el tercer término de la sucesión a partir del primero?

    \begin{solutionbox}{1.5cm}
    \end{solutionbox}

    \part Analiza los resultados del renglón de las diferencias. ¿Qué observas?

    \begin{solutionbox}{1.5cm}
    \end{solutionbox}

    \part Escribe la regla general de la diferencia entre dos términos consecutivos de la
    sucesión.

    \begin{solutionbox}{1.5cm}
    \end{solutionbox}

    \part Escribe el término que ocupa la posición 60.

    \begin{solutionbox}{1.5cm}
    \end{solutionbox}

    \part ¿Qué posición tiene el término -78?

    \begin{solutionbox}{1.5cm}
    \end{solutionbox}

    \part ¿Hay alguna posición en la que aparezca el número -138? ¿Cuál?

    \begin{solutionbox}{1.5cm}
    \end{solutionbox}

    \part Compara la regla que obtuviste con la de tus compañeros. Si hay diferencias, argu-
    menta tu respuesta y corrige si es necesario.

    \begin{solutionbox}{1.5cm}
    \end{solutionbox}
\end{parts}

\begin{importantbox}
    La \textbf{regla de recurrencia} de una sucesión es una expresión algebraica que permite calcular el valor de cada término.
\end{importantbox}