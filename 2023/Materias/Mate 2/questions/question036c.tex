Completen la Tabla \ref{tab:3.19} con el cálculo del
volumen de cada sólido. Luego contesta a las preguntas.

\renewcommand{\arraystretch}{1.4}

\begin{table}[H]
    \rowcolors{1}{}{lightgray!20}
    \centering
    \caption{Prisma recto a partir de un polígono regular}
    \label{tab:3.19}
    \begin{tabular}{>{\centering}p{3cm}|>{\centering}p{1.2cm}|>{\centering}p{1.5cm}|>{\centering}p{1.9cm}|>{\centering}p{2.1cm}|>{\centering}p{2.6cm}|p{2.6cm}|}
        \toprule                 \rowcolor{colorrds!80}
        \textbf{\color{white}Figura de la base} & \textbf{\color{white}Lado} & \textbf{\color{white}Apotema} & \textbf{\color{white}Radio del círculo circunscrita} & \textbf{\color{white}Área de la base} & \textbf{\color{white}Altura} & \textbf{\color{white}Volumen [cm$^3$]} \\ \midrule
        \rowcolor{colorrds!50}
        Cuadrado                                & 4.246                      & 1.06                          &  3                              & 18                 &   5        &  90                  \\ \hline \rowcolor{colorrds!20}
        Hexágono                                & 2.6                        & 3                             & \ifprintanswers 3\fi                                 & \ifprintanswers 23.4\fi               & \ifprintanswers  5 \fi       & \ifprintanswers 117\fi                 \\ \hline \rowcolor{colorrds!50}
        Octágono                                & 2.28                       & 2.75                          & \ifprintanswers 3\fi                                 & \ifprintanswers 25.08\fi              & \ifprintanswers 5 \fi        & \ifprintanswers 125.4\fi               \\ \hline \rowcolor{colorrds!20}
        Círculo                                 & \dots                      & \dots                         & \ifprintanswers 3\fi                                 & \ifprintanswers 28.27\fi              & \ifprintanswers  5 \fi       & \ifprintanswers 141.37\fi              \\ \hline
        \bottomrule
    \end{tabular}
\end{table}