En cada caso, indica si las expresiones son equivalentes y argumenta.


\begin{parts}
    \part \dashedbox{$5n - 5$} y \dashedbox{$5(n - 1)$}

    \begin{minipage}{0.3\textwidth}
        \begin{checkboxes}
            \CorrectChoice Son equivalentes
            \choice No son equivalentes
        \end{checkboxes}
    \end{minipage}\hfill
    \begin{minipage}{0.6\textwidth}
        \begin{solutionbox}{2cm}
            $5(n - 1) = 5n - 5$.
        \end{solutionbox}
    \end{minipage}

    \part \dashedbox{$4 - 2n$} y \dashedbox{$2 - 2 (n - 1)$}

    \begin{minipage}{0.3\textwidth}
        \begin{checkboxes}
            \CorrectChoice Son equivalentes
            \choice No son equivalentes
        \end{checkboxes}
    \end{minipage}\hfill
    \begin{minipage}{0.60\textwidth}
        \begin{solutionbox}{2cm}
            $2 - 2(n - 1) = 2 - 2n + 2 = 4 - 2n$.
        \end{solutionbox}
    \end{minipage}

    \part \dashedbox{$35+4n$} y \dashedbox{$28+4\left(n+2\right)$}

    \begin{minipage}{0.3\textwidth}
        \begin{checkboxes}
            \choice Son equivalentes
            \CorrectChoice No son equivalentes
        \end{checkboxes}
    \end{minipage}\hfill
    \begin{minipage}{0.6\textwidth}
        \begin{solutionbox}{2cm}
            $28 + 4(n + 2) = 28 + 4n + 8 = 36 + 4n$.
        \end{solutionbox}
    \end{minipage}

    \part \dashedbox{$3n - 9$} y \dashedbox{$3 (n - 2) - 3$}

    \begin{minipage}{0.3\textwidth}
        \begin{checkboxes}
            \CorrectChoice Son equivalentes
            \choice No son equivalentes
        \end{checkboxes}
    \end{minipage}\hfill
    \begin{minipage}{0.6\textwidth}
        \begin{solutionbox}{2cm}
            $3(n - 2) - 3 = 3n - 6 - 3 = 3n - 9$.
        \end{solutionbox}
    \end{minipage}



    \part \dashedbox{$n+\dfrac{3}{2}$} y \dashedbox{$\dfrac{3}{2}n+\left(-\dfrac{3}{2}-\dfrac{n}{2}\right)$}

    \begin{minipage}{0.3\textwidth}
        \begin{checkboxes}
            \choice Son equivalentes
            \CorrectChoice No son equivalentes
        \end{checkboxes}
    \end{minipage}\hfill
    \begin{minipage}{0.6\textwidth}
        \begin{solutionbox}{2cm}
            $\dfrac{3}{2}n+\left(-\dfrac{3}{2}-\dfrac{n}{2}\right)=\dfrac{3}{2}n-\dfrac{3}{2}-\dfrac{1}{2}n=n-\dfrac{3}{2}$.
        \end{solutionbox}
    \end{minipage}
\end{parts}