\question Elige en cada menú, las palabras que completan las afirmaciones.
\begin{center}
    \dashedbox{poliedro} \quad \dashedbox{pol\'igono} \quad \dashedbox{polinomio} \quad
    \dashedbox{pent\'agono} \quad \dashedbox{dodecaedro} \quad \dashedbox{cilindro} \quad
    \dashedbox{prismas} \quad \dashedbox{las pir\'amides} \quad \dashedbox{los cil\'indros} \quad
    \dashedbox{paralelas} \quad \dashedbox{perpendiculares} \quad \dashedbox{iguales} \quad
\end{center}

Un \fillin[poliedro][2cm] es un cuerpo geométrico, formado por varias caras planas,
que encierra un volumen finito. Un ejemplo, es el \fillin[dodecaedro][3cm]
que está formado por doce caras. Dentro del conjunto de los poliedros están los \fillin[prismas][2cm]
que están formados por dos caras poligonales opuestas e iguales entre sí.
Finalmente, los prismas rectos son aquellos cuyas caras laterales son
\fillin[perpendiculares][3.5cm] a la base.