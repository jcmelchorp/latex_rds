\begin{multicols}{2}
    \begin{figure}[H]
        \centering
        \includegraphics[width=0.55\linewidth]{../images/20230316211429}
        \caption{}
        \label{fig:20230316211429}
    \end{figure}
    \columnbreak
    La Figura \ref{fig:20230316211429} está formada por 2 prismas rectangulares.
    \textbf{¿Cuál es el volumen de esta figura?}
\end{multicols}
% \begin{minipage}{0.3\linewidth}

% \end{minipage}\hfill
% \begin{minipage}{0.7\linewidth}
\begin{solutionbox}{16cm}
    \begin{minipage}[t]{.3\textwidth}
        \begin{figure}[H]
            \centering
            \includegraphics[width=0.9\linewidth]{../images/20230316211644}
            \caption{Descomposicion de la Figura \ref{fig:20230316211429} en dos.}
            \label{fig:20230316211644}
        \end{figure}
    \end{minipage}\hfill
    \begin{minipage}[t]{.55\textwidth}
        Podemos pensar en esta figura como 2 prismas rectangulares pegados (ver Figura \ref{fig:20230316201857}). Encontremos el volumen de cada prisma por separado.\\
        El volumen de un prisma rectangular es igual al largo $x$, por el ancho $y$, por la altura $z$:
        \[ V = xyz \]

        Para uno de los prismas, como el que aparece en la Figura \ref{fig:20230316202402}, se sabe que:\\
        \[ V = 5\times 4\times 3=60\]
        Volumen del prisma color turquesa es 90 pulgadas cúbicas.\\

        Para la segunda sección del prisma, como en la Figura \ref{fig:20230316205119}, se sabe que:\\
        \[ V = 2\times 4\times 8=64\]
        Volumen del prisma color púrpura es 72 pulgadas cúbicas.\\
        Ahora sumamos para obtener el volumen de toda la figura.
        \[ V_T = 60+64=124\]
        Volumen de toda la figura $V_T$ es $162$ pulgadas cúbicas
    \end{minipage}
    \begin{multicols}{2}
        \begin{figure}[H]
            \centering
            \includegraphics[width=0.6\linewidth]{../images/20230316211742}
            \caption{Primera sección del prisma de la Figura \ref{fig:20230316211429}}
            \label{fig:20230316211742}
        \end{figure}
        \columnbreak
        \begin{figure}[H]
            \centering
            \includegraphics[width=0.3\linewidth]{../images/20230316211827}
            \caption{Segunda sección del prisma de la Figura \ref{fig:20230316211429}}
            \label{fig:20230316211827}
        \end{figure}
    \end{multicols}
\end{solutionbox}
% \end{minipage}




