A partir de la información dada sobre un polígono regular,
traza la figura descrita en los siguientes incisos y \textbf{calcula su
    perímetro}.

\begin{parts}
    \part Su lado mide 1.5 cm y se puede trazar únicamente una diagonal desde
    cualquier vértice.

    \begin{minipage}{0.45\textwidth}
        \begin{center}
            Figura:
        \end{center}
        \begin{solutionbox}{2cm}\end{solutionbox}
    \end{minipage}\hfill
    \begin{minipage}{0.45\textwidth}
        \begin{center}
            Perímetro:
        \end{center}
        \begin{solutionbox}{2cm}\end{solutionbox}
    \end{minipage}

    \part El valor de un ángulo central es de 120$^\circ$ y mide 1 cm de lado.

    \begin{minipage}{0.45\textwidth}
        \begin{center}
            Figura:
        \end{center}
        \begin{solutionbox}{2cm}\end{solutionbox}
    \end{minipage}\hfill
    \begin{minipage}{0.45\textwidth}
        \begin{center}
            Perímetro:
        \end{center}
        \begin{solutionbox}{2cm}\end{solutionbox}
    \end{minipage}

    \part Cada lado mide 1 cm y se puede descomponer en 8 triángulos
    equiláteros congruentes.

    \begin{minipage}{0.45\textwidth}
        \begin{center}
            Figura:
        \end{center}
        \begin{solutionbox}{2cm}\end{solutionbox}
    \end{minipage}\hfill
    \begin{minipage}{0.45\textwidth}
        \begin{center}
            Perímetro:
        \end{center}
        \begin{solutionbox}{2cm}\end{solutionbox}
    \end{minipage}

\end{parts}