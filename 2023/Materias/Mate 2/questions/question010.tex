\question A partir de la información dada sobre un polígono regular, traza la figura descrita en los siguientes incisos y \textbf{calcula su perímetro}.

\begin{parts}
    \part[5] Su lado mide 1.6 cm y se puede trazar únicamente una diagonal desde cualquier vértice.

    \begin{minipage}{0.45\textwidth}
        \begin{center}
            Figura:
        \end{center}
        \begin{solutionbox}{3cm}\end{solutionbox}
    \end{minipage}\hfill
    \begin{minipage}{0.45\textwidth}
        \begin{center}
            Perímetro:
        \end{center}
        \begin{solutionbox}{3cm}\end{solutionbox}
    \end{minipage}


    \part[5] El valor de un ángulo central es de 72º y mide 2 cm de lado.

    \begin{minipage}{0.45\textwidth}
        \begin{center}
            Figura:
        \end{center}
        \begin{solutionbox}{3cm}\end{solutionbox}
    \end{minipage}\hfill
    \begin{minipage}{0.45\textwidth}
        \begin{center}
            Perímetro:
        \end{center}
        \begin{solutionbox}{3cm}\end{solutionbox}
    \end{minipage}

    \part[5] Cada lado mide 1 cm y se puede descomponer en 6 triángulos equiláteros congruentes.

    \begin{minipage}{0.45\textwidth}
        \begin{center}
            Figura:
        \end{center}
        \begin{solutionbox}{3cm}\end{solutionbox}
    \end{minipage}\hfill
    \begin{minipage}{0.45\textwidth}
        \begin{center}
            Perímetro:
        \end{center}
        \begin{solutionbox}{3cm}\end{solutionbox}
    \end{minipage}

\end{parts}
