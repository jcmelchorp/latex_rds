\question Seis albañiles construyen una casa en 90 días.

\begin{multicols}{2}
    \begin{parts}
        \part[5] ¿Qué tipo de variación proporcional es? Argumenta tu respuesta.

        \begin{solutionbox}{1.5cm}
            Es una variación proporcional inversa.
        \end{solutionbox}

        % \part Describe un procedimiento para obtener la constante de proporcionalidad del
        % problema.

        % \begin{solutionbox}{2cm}
        %     Constante de proporcionalidad inversa es 540: \[6 \times 90 = 540\]
        % \end{solutionbox}

        \part[5] ¿Cuántos dias tardarán nueve albañiles, trabajando al mismo ritmo, en construir una casa del mismo tamaño?

        \begin{solutionbox}{1.5cm}
            Si trabajan 9 albañiles tardarán 60 días.
        \end{solutionbox}

        \columnbreak

        \part[10] Completa la tabla \ref{tab:albaniles_tabla}.

        \renewcommand{\arraystretch}{1.6}
        \begin{table}[H]
            \centering
            \caption{}
            \label{tab:albaniles_tabla}
            \begin{tabular}{>{\centering}p{1.5cm}>{\centering}p{1.5cm}p{3cm}}
                \rowcolor{YellowGreen!80}
                \textbf{Alba\~niles} & \textbf{Días de trabajo} & \textbf{Constante de proporcionalidad} \\ \hline
                \rowcolor{YellowGreen!50}
                1                    &                          &                                        \\ \hline
                \rowcolor{YellowGreen!20}
                2                    &                          &                                        \\ \hline
                \rowcolor{YellowGreen!50}
                                     & 108                      &                                        \\ \hline
                \rowcolor{YellowGreen!20}
                6                    & 90                       & $6 \times 90 =540$                     \\ \hline
                \rowcolor{YellowGreen!50}
                                     & 67.5                     &                                        \\ \hline
                \rowcolor{YellowGreen!20}
                15                   &                          &
            \end{tabular}
        \end{table}
    \end{parts}
\end{multicols}