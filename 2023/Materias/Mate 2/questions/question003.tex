Un grupo de 25 personas puede levantar una cosecha en 30 días. Al cabo de 12 días de trabajo,
se les unen personas de otro grupo, de modo que en 6 día
más terminan la cosecha.

\textbf{¿Cuántas personas había en el segundo grupo?}
%\emph{Escoge 1 respuesta:}

\begin{oneparchoices}
    \choice 25 personas
    \choice 45 personas
    \choice 30 personas
    \CorrectChoice 15 personas
\end{oneparchoices}

% \begin{oneparchoices}
%     \choice 25 personas
%     \choice 45 personas
%     \choice 30 personas
%     \choice 15 personas
% \end{oneparchoices}

\begin{solutionbox}{8cm}
    Sabemos que 25 personas levantarían la cosecha en 30 días. Como luego de los primeros 12 días de trabajo llegaron más personas, observamos que, en esta situación, a mayor cantidad de personas, menos días se necesitarán para terminar la cosecha.
    \begin{table}[H]
        \centering
        \begin{tabular}{|l|c|l|}
            \hline
            Cantidad de personas        & 25 & 25+x \\
            \hline
            Cantidad de días de trabajo & 18 & 6    \\
            \hline
        \end{tabular}
    \end{table}
    Como es una relación inversamente proporcional, planteamos la siguiente relación:
    \begin{align*}
        18 \times 25 & = 6 \times (25+x) \\
        450          & = 150 +6x         \\
        6x           & = 300             \\
        x            & = 50
    \end{align*}
    % \end{minipage}
    En el segundo grupo, había 50 personas más.
\end{solutionbox}
