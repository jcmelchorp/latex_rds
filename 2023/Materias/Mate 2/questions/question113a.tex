Un padre repartirá 1200 dólares entre sus cinco hijos, de tal forma que la diferencia entre los montos que reciba cada hijo sea la misma.
Si le entrega 360 dólares a su hijo mayor,
\textbf{¿cuánto dinero recibirá el hijo menor?}

\begin{solutionbox}{7cm}
    Sabemos que la suma de la serie que representa la reparticion del dinero es:
    \[s_{5}=\dfrac{5(360+a_{5})}{2}=1,200\]
    despejando $a_5$:
    \begin{align*}
        \dfrac{5(360+a_{5})}{2} & =1,200   \\
        5(360+a_{5})            & =2,400   \\
        360+a_{5}               & =480     \\
        a_{5}                   & =480-360 \\
        a_{5}                   & =120
    \end{align*}
    El hijo menor recibirá 120 dolares.
\end{solutionbox}