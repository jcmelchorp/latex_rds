Coloca el número que completa la equivalencia.

\begin{center}
    \dashedbox{$1$} \qquad \dashedbox{$2$} \qquad \dashedbox{$6$} \qquad \dashedbox{$3$} \qquad \dashedbox{$4$}
\end{center}

\begin{multicols}{2}
    \begin{parts}
        \part La expresión $6(b+8) - 4(b+4)$ es equivalente a \dashedbox{\rule{0em}{0.8em}\rule{0.8em}{0em}\fillin[4][1cm]}$(b+9) - 2(b+2)$.

        \begin{solutionbox}{3.5cm}
            \begin{align*}
                6(b+8) - 4(b+4) & = 6b+48-4b-16      \\
                                & = 6b+48-2b-2b-4-12 \\
                                & = 4b+36-2(b+2)     \\
                                & = 4(b+9)-2(b+2)
            \end{align*}
        \end{solutionbox}

        \part La expresión $5(b-4) + 3(b+2)$ es equivalente a \dashedbox{\rule{0em}{0.8em}\rule{0.8em}{0em}\fillin[6][1cm]}$(b-4) + 2(b+5)$.

        \begin{solutionbox}{3.5cm}
            \begin{align*}
                5(b-4) + 3(b+2) & = 5b-20+3b+6      \\
                                & = 6(b-4)+2b+6+4-4 \\
                                & = 6(b-4)+2b+10    \\
                                & = 6(b-4)+2(b+5)
            \end{align*}
        \end{solutionbox}

        \part La expresión $-3(b+4) - 2(b-5)$ es equivalente a $-4(b+3) - $\dashedbox{\rule{0em}{0.8em}\rule{0.8em}{0em}\fillin[1][1cm]}$(b-10)$.

        \begin{solutionbox}{3.5cm}
            \begin{align*}
                -3(b+4) - 2(b-5) & = -3b-12-2b+10   \\
                                 & = -3b-12-b-b+10  \\
                                 & = -4b-12-(b-10)  \\
                                 & = -4(b+3)-(b-10)
            \end{align*}
        \end{solutionbox}

        \columnbreak

        \part La expresión $(b+7) - 8(b+1)$ es equivalente a $-9(b+1) + $ \dashedbox{\rule{0em}{0.8em}\rule{0.8em}{0em}\fillin[2][1cm]} $(b+4)$.

        \begin{solutionbox}{3.5cm}
            \begin{align*}
                (b+7) - 8(b+1) & = b+7-8b-8          \\
                               & =  b+b-b-1+8-8(b+1) \\
                               & =  2b+8-9(b+1)      \\
                               & =2(b+4)-9(b+1)
            \end{align*}
        \end{solutionbox}

        \part La expresión $11(b-3) - 6(b+5)$ es equivalente a  \dashedbox{\rule{0em}{0.8em}\rule{0.8em}{0em}\fillin[3][1cm]}$(b-1) + 2(b-30)$.

        \begin{solutionbox}{4cm}
            \begin{align*}
                11(b-3) - 6(b+5) & = 11b-33-6b-30    \\
                                 & =  11b-30-3-6b-30 \\
                                 & =   9b+2b-3-6b-60 \\
                                 & =9b-3-6b+2b-60    \\
                                 & =3(b-1)+2(b-30)
            \end{align*}
        \end{solutionbox}
    \end{parts}
\end{multicols}