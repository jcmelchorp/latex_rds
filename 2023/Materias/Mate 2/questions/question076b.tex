\question[10] Coloca el número que completa la equivalencia.
\begin{center}
    \dashedbox{$1$} \qquad \dashedbox{$2$} \qquad \dashedbox{$6$} \qquad \dashedbox{$3$} \qquad \dashedbox{$4$}
\end{center}

\begin{parts}
    \part La expresión $6(b+8) - 4(b+4)$ es equivalente a  \dashedbox{\rule{0em}{0.8em}\rule{0.8em}{0em}}$(b+9) - 2(b+2)$.
    \part La expresión $5(b-4) + 3(b+2)$ es equivalente a \dashedbox{\rule{0em}{0.8em}\rule{0.8em}{0em}}$(b-4) + 2(b+5)$.
    \part La expresión $-3(b+4) - 2(b-5)$ es equivalente a $-4(b+3) - $\dashedbox{\rule{0em}{0.8em}\rule{0.8em}{0em}}$(b-10)$.
    \part La expresión $(b+7) - 8(b+1)$ es equivalente a $-9(b+1) + $ \dashedbox{\rule{0em}{0.8em}\rule{0.8em}{0em}} $(b+4)$.
    \part La expresión $11(b-3) - 6(b+5)$ es equivalente a  \dashedbox{\rule{0em}{0.8em}\rule{0.8em}{0em}}$(b-1) + 2(b-30)$.
\end{parts}