Completa la Tabla \ref{tab:3.4}. Luego responde lo que se pide.

\vspace{-0.2cm}
\begin{table}[H]
    \centering
    \rowcolors{3}{colorrds!10}{lightgray!10}
    \caption{}
    \label{tab:3.4}
    \begin{tabular}{|r|*{8}{c|}}
        \toprule
        \rowcolor{colorrds!80}
                               & \multicolumn{8}{c|}{\bfseries\color{white} Número de término en la sucesión}                                           \\
        \multirow{-2}*{\cellcolor{colorrds!80}\bfseries\color{white}\minitab[c]{Regla de                                                                \\[-0.5em] recurrencia}}                   & 1 & 2 & 3  & 5 & 12 & 25 & 50 & 100 \\ \hline
        $21-9n$                & 12                                                                           & 3 & -6 & -24 & -87 & -204 & -429 & -879 \\ \hline
        $-3\left(3n-7\right)$  & 12                                                                           & 3 & -6 & -24 & -87 & -204 & -429 & -879 \\ \hline
        $12-9\left(n-1\right)$ & 12                                                                           & 3 & -6 & -24 & -87 & -204 & -429 & -879 \\ \cline{2-9}
        \bottomrule
    \end{tabular}
\end{table}

\begin{parts}
    \part Compara los términos. ¿Qué observan?

    \begin{solutionbox}{4em}
        Que son iguales para las tres expresiones.
    \end{solutionbox}

    \part ¿Por qué piensas que ocurre lo que observaste? Hagan una conjetura

    \begin{solutionbox}{4em}
        Las tres expresiones algebraicas son equivalentes.
    \end{solutionbox}


    \part Reflexionen: en la segunda regla, ¿qué operación deben realizar con el -3 y los
    términos dentro del paréntesis? Y en la tercera regla ¿qué operación deben realizar con el término numérico y los términos dentro del paréntesis? No se olviden
    de la regla de los signos. En cada regla, ¿cuáles términos son semejantes? Hagan
    las operaciones necesarias y simplifiquen las dos últimas reglas.

    \begin{solutionbox}{6em}
        Se busca que los alumnos mencionen que el - 3 de la segunda regla y el - 9
        de la tercera se multiplican por los términos dentro de los paréntesis respectivos y que todos los términos son semejantes. Además, que obtengan mediante
        la siguiente operación el resultado: $12 - 9n + 9 = 21 - 9n$.
    \end{solutionbox}


\end{parts}

