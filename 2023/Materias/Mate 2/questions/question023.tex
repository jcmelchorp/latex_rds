\question[10] Completa la tabla \ref{tab:3.4} Luego responde lo que se pide.

\begin{table}[H]
    \centering
    \caption{}
    \label{tab:3.4}
    \begin{tabular}{c|c|c|c|c|c|c|c|c|c|}
        \multirow{2}*{\minitab[c]{Regla de                              \\[-0.5em] recurrencia}} & \multicolumn{9}{c|}{Número de término en la sucesión}                                      \\ \cline{2-10}
                               & 1 & 2 & 3 & 4 & 5 & 12 & 25 & 50 & 100 \\ \hline
        $21-9n$                &   &   &   &   &   &    &    &    &     \\ \hline
        $-3\left(3n-7\right)$  &   &   &   &   &   &    &    &    &     \\ \hline
        $12-9\left(n-1\right)$ &   &   &   &   &   &    &    &    &     \\ \cline{2-10}
    \end{tabular}
\end{table}

\begin{parts}
    \part Compara los términos. ¿Qué observan?

    \begin{solutionbox}{1.5cm}
    \end{solutionbox}

    \part ¿Por qué piensas que ocurre lo que observaste? Hagan una conjetura

    \begin{solutionbox}{1.5cm}
    \end{solutionbox}


    \part Reflexionen: en la segunda regla, ¿qué operación deben realizar con el –3 y los
    términos dentro del paréntesis? Y en la tercera regla ¿qué operación deben rea-
    lizar con el término numérico y los términos dentro del paréntesis? No se olviden
    de la regla de los signos. En cada regla, ¿cuáles términos son semejantes? Hagan
    las operaciones necesarias y simplifiquen las dos últimas reglas.

    \begin{solutionbox}{1.5cm}
    \end{solutionbox}


\end{parts}

\begin{importantbox}
    La \textbf{regla de recurrencia} de una sucesión es una expresión algebraica que permite calcular el valor de cada término.
\end{importantbox}