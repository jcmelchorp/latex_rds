Una expedición turística al desierto de Sonora consta de 60 personas
adultas y cuenta con víveres para 12 días. Al momento de partir, se integran 12 personas más.

\begin{multicols}{2}
    \begin{parts}
        \part ¿Con qué tipo de variación proporcional se puede modelar la situación?

        \begin{solutionbox}{1.8cm}La situación describe una variación de proporcionalidad inversa.\end{solutionbox}


        \part ¿para cuántos días alcanzarán los víveres para las personas de la
        excursión si todas comen las mismas porciones?

        \begin{solutionbox}{1.8cm}Las $60$ (es decir, $50+10$) personas tendrán comida para 10 días.\end{solutionbox}

        \columnbreak

        \part Completa la tabla \ref{tab:sonora_tabla}.

        \renewcommand{\arraystretch}{1.6}
        \begin{table}[H]
            \centering
            \caption{Tabla comparativa entre personas y víveres}
            \ifprintanswers
                \begin{tabular}{>{\centering}p{2cm}>{\centering}p{2.5cm}p{3cm}}
                    \rowcolor{YellowGreen!80}
                    \textbf{Personas} & \textbf{Días que duran los viveres} & \textbf{Constante de proporcionalidad} \\ \hline
                    \rowcolor{YellowGreen!50}
                    60                & 12                                  & $60 \times 12 =720$                    \\ \hline
                    \rowcolor{YellowGreen!20}
                    10                & 72                                  & $10 \times 72 =720$                    \\ \hline
                    \rowcolor{YellowGreen!50}
                    20                & 36                                  & $20 \times 36 =720$                    \\ \hline
                    \rowcolor{YellowGreen!20}
                    40                & 18                                  & $40 \times 18 =720$                    \\ \hline
                    \rowcolor{YellowGreen!50}
                    72                & 10                                  & $72 \times 10 =720$
                \end{tabular}
            \else
                \begin{tabular}{>{\centering}m{2cm}>{\centering}p{2.5cm}p{3cm}}
                    \rowcolor{OliveGreen!50}
                    \textbf{Personas} & \textbf{Días que duran los viveres} & \textbf{Constante de proporcionalidad} \\ \hline
                    \rowcolor{YellowGreen!50}
                    60                & 12                                  &                                        \\ \hline
                    \rowcolor{LimeGreen!50}
                    10                &                                     &                                        \\ \hline
                    \rowcolor{YellowGreen!50}
                    20                &                                     &                                        \\ \hline
                    \rowcolor{LimeGreen!50}
                    40                &                                     &                                        \\ \hline
                    \rowcolor{YellowGreen!50}
                    72                &                                     &
                \end{tabular}
            \fi
            \label{tab:sonora_tabla}
        \end{table}
    \end{parts}
\end{multicols}