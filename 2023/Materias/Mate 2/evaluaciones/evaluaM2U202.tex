\documentclass[12pt,addpoints]{evalua}
\grado{2$^\circ$ de Secundaria}
\cicloescolar{2023-2024}
\materia{Matemáticas 2}
\unidad{2}
\title{Examen de la Unidad}
\aprendizajes{
      \item Calcula el volumen de prismas y cilindros rectos.
      \item Calcula el perímetro de polígonos y del círculo, y áreas de triángulos y cuadriláteros desarrollando y aplicando fórmulas.
      \item Analiza y compara situaciones de variación lineal a partir de sus representaciones tabular, gráfica y algebraica. 
}
\author{Prof.: Julio César Melchor Pinto}
\begin{document}
\begin{questions}
      \section*{\ifprintanswers   {Círculo}\else{}\fi}
      \subsection*{\ifprintanswers{Diámetro de un círculo}\else{}\fi}
      \subsection*{\ifprintanswers{Radio de un círculo}\else{}\fi}
      \subsection*{\ifprintanswers{Perímetro}\else{}\fi}
      \subsection*{\ifprintanswers{Área}\else{}\fi}
      \subsection*{\ifprintanswers{Resolución de problemas}\else{}\fi}

      \section*{\ifprintanswers   {Polígonos y circunferencias}\else{}\fi}
      \subsection*{\ifprintanswers{Ángulos interiores}\else{}\fi}
      \subsection*{\ifprintanswers{Ángulos centrales y exteriores}\else{}\fi}
      \subsection*{\ifprintanswers{Ángulos centrales e inscritos}\else{}\fi}
      \subsection*{\ifprintanswers{Arco de una circunferencia}\else{}\fi}
      \subsection*{\ifprintanswers{Área de un sector circular}\else{}\fi}

      \section*{\ifprintanswers   {Figuras y cuerpos geométricos}\else{}\fi}
      \subsection*{\ifprintanswers{Perímetro}\else{}\fi}
      \subsection*{\ifprintanswers{Área}\else{}\fi}
      \subsection*{\ifprintanswers{Área lateral y total}\else{}\fi}
      \subsection*{\ifprintanswers{Volumen}\else{}\fi}
      \subsection*{\ifprintanswers{Resolución de problemas}\else{}\fi}

      \section*{\ifprintanswers   {Monomios y polinomios}\else{}\fi}
      \subsection*{\ifprintanswers{Lenguaje algebraico}\else{}\fi}
      \subsection*{\ifprintanswers{Suma de monomios y polinomios}\else{}\fi}
      \subsection*{\ifprintanswers{Resta de monomios y polinomios}\else{}\fi}
      \subsection*{\ifprintanswers{Operaciones combinadas}\else{}\fi}
      \subsection*{\ifprintanswers{Perímetro de figuras geométricas}\else{}\fi}

      \section*{\ifprintanswers   {Operaciones con monomios y polinomios}\else{}\fi}
      \subsection*{\ifprintanswers{Suma y resta de exponentes}\else{}\fi}
      \subsection*{\ifprintanswers{Multiplicación de exponentes}\else{}\fi}
      \subsection*{\ifprintanswers{Multiplicación y división de monomios}\else{}\fi}
      \subsection*{\ifprintanswers{Multiplicación de polinomios}\else{}\fi}
      \subsection*{\ifprintanswers{Áreas de figuras geométricas}\else{}\fi}

      \section*{\ifprintanswers   {Sistema de unidades}\else{}\fi}
      \subsection*{\ifprintanswers{Unidades de longitud}\else{}\fi}
      \subsection*{\ifprintanswers{Unidades de masa}\else{}\fi}
      \subsection*{\ifprintanswers{Unidades de capacidad}\else{}\fi}
      \subsection*{\ifprintanswers{Unidades de área y volumen}\else{}\fi}
      \subsection*{\ifprintanswers{Unidades de capacidad 2}\else{}\fi}


      \section*{\ifprintanswers{Leyes de los exponentes}\else{}\fi}
      \questionboxed[6]{Realiza las siguientes operaciones con exponentes:
            \begin{multicols}{3}
                  \begin{parts}
                        \subsection*{\ifprintanswers{Suma de exponentes}\else{}\fi}
                        \part $(-5a^4)(-3a^2)=$
                        \begin{solutionbox}{1.5cm}
                              $(-5a^4)(-3a^2) = 15a^6$
                        \end{solutionbox}

                        \part $(-3a^4)(8a^2)=$
                        \begin{solutionbox}{1.5cm}
                              $(-3a^4)(8a^2) = -24a^6$
                        \end{solutionbox}

                        \part $4x^2\cdot x^5\cdot 5x^8=$
                        \begin{solutionbox}{1.5cm}
                              $4x^2\cdot x^5\cdot 5x^8 = 20x^{15}$
                        \end{solutionbox}

                        \part $x^2y^3z^4 \cdot x^5z^4=$
                        \begin{solutionbox}{1.5cm}
                              $x^2y^3z^4 \cdot x^5z^4 = x^7y^3z^8$
                        \end{solutionbox}

                        \columnbreak

                        \part $x^3x^2x^3=$
                        \begin{solutionbox}{1.5cm}
                              $x^3x^2x^3 = x^8$
                        \end{solutionbox}

                        \part $7x^2\cdot 3x^4 \cdot 6x^2=$
                        \begin{solutionbox}{1.5cm}
                              $7x^2\cdot 3x^4 \cdot 6x^2 = 126x^8$
                        \end{solutionbox}

                        \subsection*{\ifprintanswers{Resta de exponentes}\else{}\fi}
                        \part $\dfrac{x^{13}y^{18}z^{4}}{x^{11}y^{9}z^{4}}=$
                        \begin{solutionbox}{2cm}
                              $\dfrac{x^{13}y^{18}z^{4}}{x^{11}y^{9}z^{4}} = x^2y^9$
                        \end{solutionbox}

                        \part $\dfrac{x^{4}y^{12}z^{13}}{x^{3}y^{12}z^{13}}=$
                        \begin{solutionbox}{2cm}
                              $\dfrac{x^{4}y^{12}z^{13}}{x^{3}y^{12}z^{13}} = x$
                        \end{solutionbox}

                        \part $\dfrac{81a^5b^{12}c^9}{9a^3b^{7}c^5}=$
                        \begin{solutionbox}{2cm}
                              $\dfrac{81a^5b^{12}c^9}{9a^3b^{7}c^5} = 9a^2b^5c^4$
                        \end{solutionbox}

                        \subsection*{\ifprintanswers{Multiplicación de exponentes}\else{}\fi}
                        \part $(a^3b^2c^4)^3=$
                        \begin{solutionbox}{1.5cm}
                              $(a^3b^2c^4)^3 = a^9b^6c^{12}$
                        \end{solutionbox}

                        \part $\left(x^4 y^5\right)^6=$
                        \begin{solutionbox}{1.5cm}
                              $\left(x^4 y^5\right)^6 = x^{24}y^{30}$
                        \end{solutionbox}

                        \part $\left(a^3 b^5 c^{11} \right)^7=$
                        \begin{solutionbox}{1.5cm}
                              $\left(a^3 b^5 c^{11} \right)^7 = a^{21}b^{35}c^{77}$
                        \end{solutionbox}
                  \end{parts}
            \end{multicols}
      }

      \subsection*{\ifprintanswers{División de exponentes}\else{}\fi}
      \questionboxed[4]{Simplifica las siguientes expresiones algebraicas con exponentes:
            \begin{multicols}{2}
                  \begin{parts}
                        \part $\sqrt{x^4}=$ \fillin[$x^2$][0in]
                        \part $\sqrt[6]{x^6y^{12}}=$ \fillin[$xy^2$][0in]
                        \part $\sqrt[3]{x^6y^{12}z^{18}}=$ \fillin[$xy^2z^6$][0in]
                        \part $\sqrt[4]{x^{12}y^{8}z^{16}}=$ \fillin[$x^3y^2z^4$][0in]
                        \part $\sqrt{x^{20}y^{12}z^{6}}=$ \fillin[$x^{10}y^6z^3$][0in]
                        \part $\sqrt[5]{a^{15}b^{20}}=$ \fillin[$a^3b^4$][0in]
                  \end{parts}
            \end{multicols}
      }

      \subsection*{\ifprintanswers{Exponentes negativos}\else{}\fi}
      \questionboxed[4]{Convierte las expresiones algebraicas usando exponentes positivos:
            \begin{multicols}{2}
                  \begin{parts}
                        \part $\dfrac{5}{x^{-8}}=$ \fillin[$5x^8$][0in]
                        \part $5x^{-7}=$ \fillin[$\dfrac{5}{x^7}$][0in]
                        \part $y^{-5}=$ \fillin[$\dfrac{1}{y^5}$][0in]
                        \part $3y^{-9}=$ \fillin[$\dfrac{3}{y^9}$][0in]
                        \part $\dfrac{1}{x^{-7}}=$ \fillin[$x^7$][0in]
                        \part $\dfrac{2}{y^{-2}}=$ \fillin[$2y^2$][0in]
                  \end{parts}
            \end{multicols}
      }\end{questions}
\end{document}