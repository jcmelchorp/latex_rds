\documentclass[12pt,addpoints,answers]{evalua}
\grado{2$^\circ$ de Secundaria}
\cicloescolar{2022-2023}
\materia{Matemáticas 2}
\unidad{3}
\title{Examen de la Unidad}
\aprendizajes{
    \item Verifica algebraicamente la equivalencia de expresiones de primer
    grado, formuladas a partir de sucesiones.
    \item Formula expresiones de primer grado para representar propiedades
    (perímetros y áreas)
    de figuras geométricas y verifica equivalencia de expresiones, tanto
    algebraica como geométricamente (análisis de las figuras).
    \item Calcula el volumen de prismas y cilindros rectos.
}
\author{Prof.: Julio César Melchor Pinto}
\begin{document}
\begin{multicols}{2}%
    \include*{../blocks/block002}
    \include*{../blocks/block030e}
    \include*{../blocks/block035c}
    \include*{../blocks/block035b}
\end{multicols}%
\begin{questions}
    \question[10] \include*{../questions/question106j}
    \question[10] \include*{../questions/question110}
    \question[10] \include*{../questions/question103i}
    \question[10] \include*{../questions/question102b}
    \question[10] \include*{../questions/question100}
    \question[10] \include*{../questions/question102d}
    \question[10] \include*{../questions/question071}
    \question[15] \include*{../questions/question104c}
\end{questions}
\end{document}