\documentclass[12pt]{evalua}
\grado{2$^\circ$ de Secundaria}
\cicloescolar{2022-2023}
\materia{Matemáticas 2}
\guide{2}
\title{Examen de la Unidad}
\aprendizajes{
    \begin{itemize}[leftmargin=*,label=\small\color{colorrds}\faIcon{user-graduate}]
        \item Resuelve problemas de proporcionalidad directa e inversa y de reparto proporcional.
        \item Obtiene la expresión algebraica y construye gráficas de una situación de
        proporcionalidad directa e inversa.
        \item Construye polígonos regulares a partir de algunas
        medidas (lados, apotema, diagonales, etcétera).
        \item Descompone figuras en otras para calcular su área.
        \item Calcula el perímetro y el área de polígonos regulares y del círculo a partir de diferentes datos.
    \end{itemize}
}

\author{Prof.: Julio César Melchor Pinto}
\begin{document}
%\printanswers
\vspace{-0.5cm}
{\small
    \begin{multicols}{2}
        \include*{../blocks/block001}
        \include*{../blocks/block000}
        \include*{../blocks/block003}
    \end{multicols}
    \include*{../blocks/block002}
}
\begin{questions}
    % \include*{../questions/question004}
    % \include*{../questions/question001}
    % \include*{../questions/question005}
    % \include*{../questions/question002}
    \include*{../questions/question003}
    \newpage
    \include*{../questions/question009}
    %\include*{../questions/question011}
    % \newpage
    %\include*{../questions/question012}
    \include*{../questions/question010}
    % \newpage
    % \include*{../questions/question013}
    %\include*{../questions/question014}
    \newpage
    \include*{../questions/question006}
    %\include*{../questions/question008}
    \include*{../questions/question015}
\end{questions}

\end{document}