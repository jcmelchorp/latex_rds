\documentclass[12pt,addpoints]{evalua}
\grado{2$^\circ$ de Secundaria}
\cicloescolar{2023-2024}
\materia{Matemáticas 2}
\unidad{3}
\title{Examen de la Unidad}
\aprendizajes{
            \item Resuelve problemas de proporcionalidad directa e inversa y de reparto proporcional.
            \item Resuelve problemas mediante la formulación y solución algebraica de ecuaciones lineales.
            \item Analiza y compara situaciones de variación lineal a partir de sus representaciones tabular, gráfica y algebraica. Interpreta y resuelve problemas que se modelan con estos tipos de variación.
            \item Verifica algebraicamente la equivalencia de expresiones de primer grado, formuladas a partir de sucesiones.      }
\author{Prof.: Julio César Melchor Pinto}
\begin{document}
\begin{questions}
      \question[5]{Contesta las siguientes preguntas:

            % \begin{multicols}{2}
            %       \begin{parts}
            %             \part 
            Las calificaciones de un salón de secundaria son las siguientes: 80, 82, 85, 88, 90, 88, 91, 85, 95, 88, 88, 97, 100. ¿Cuál es la mediana de las calificaciones?
            \fillin[88][2cm]

            \begin{solutionbox}{2cm}
                  Ordenando los datos se obtiene:\\[-0.2em]
                  $\left\{80, 82, 85, 85, 88, 88, 88, 88, 90, 91, 95, 97, 100 \right\} $\\[-0.2em]
                  $\therefore$ Mediana es 88
            \end{solutionbox}

            %       \part 
            %       Las edades de un grupo de personas son: 44, 41, 47, 48, 44, 39, 45, 49, 44 y 47 años. ¿Cuál es la mediana de las edades?
            %       \fillin[44.5][2cm]

            %       \begin{solutionbox}{2cm}
            %             Ordenando los datos se obtiene: \\[-0.2em]
            %             $\left\{39, 41, 44, 44, 44, 45, 47, 47, 48, 49 \right\} $ \\[-0.2em]
            %             $\therefore$ Mediana es 44.5
            %       \end{solutionbox}

            % \end{parts}
            % \end{multicols}
      }

      \question[5]{Contesta las siguientes preguntas:

            \begin{multicols}{2}
                  \begin{parts}
                        \part El número de goles en las últimas 3 temporadas de un delantero fueron: 22, 26 y 31, ¿cuál es el promedio de goles por temporada?
                        \fillin[26.33][2cm]

                        \begin{solutionbox}{2.5cm}
                              Para encontrar el promedio sumamos el total de goles en esas temporadas y luego dividimos esa suma por el número de temporadas. En este caso, el promedio es
                              $(22+26+31)/3=26.33$
                        \end{solutionbox}

                        \part En un grupo de 11 personas se registraron los siguientes pesos: 62, 64, 65, 59, 68, 72, 77, 71, 82, 69 y 76 kg. ¿Cuál es el promedio de los pesos?
                        \fillin[69.54][2cm]

                        \begin{solutionbox}{2.5cm}
                              Al sumar los pesos: 62 + 64 + 65 + 59 + 68 + 72 + 77 + 71 + 82 + 69 + 76 = 765 kg, y dividir por 11 personas, obtenemos un promedio de aproximadamente 69.55 kg.
                        \end{solutionbox}

                  \end{parts}
            \end{multicols}
      }

      \question[5]{Los resultados de una encuesta se muestran en la siguiente gráfica de barras:

            \begin{multicols}{2}
                  \begin{parts}
                        \part  ¿Cuántas personas participaron en la encuesta? \fillin[95][2cm]
                        \part  ¿Cuál es la fruta menos preferida por las personas? \fillin[Naranja][2cm]
                        \part  ¿Cuál es la fruta preferida por las personas? \fillin[Manzana][2cm]

                        \includegraphics[width=.85\linewidth]{mexmat00001.png}
                  \end{parts}
            \end{multicols}
      }

      \question[3]{Resuelve los siguientes problemas:

            \begin{multicols}{3}

                  \begin{parts}
                        \part En una urna hay 10 pelotas azules, 5 verdes, 15 blancas y 20 negras. Calcula la probabilidad de sacar una pelota negra.

                        \begin{solutionbox}{3cm}
                        \end{solutionbox}

                        \columnbreak%

                        \part Si se lanzan tres monedas al aire, calcula la probabilidad de que caiga puro sol.

                        \begin{solutionbox}{3cm}
                        \end{solutionbox}

                        \columnbreak%

                        \part En una urna hay 8 pelotas moradas, 12 naranjas, 7 rojas, 11 azules y 7 blancas. Calcula la probabilidad de sacar una pelota negra.

                        \begin{solutionbox}{3cm}
                        \end{solutionbox}
                  \end{parts}
            \end{multicols}
      }

      \question[5]{Determina si las siguientes tablas de datos son o no una relación proporcional:

            \begin{multicols}{2}
                  \begin{parts}
                        \part
                        \begin{minipage}[t]{0.35\linewidth}
                              \includegraphics[width=\textwidth]{mex_0071.png}
                        \end{minipage}\hfill%
                        \begin{minipage}[b]{0.6\linewidth}
                              \begin{choices}
                                    \choice Propocional
                                    \CorrectChoice No proporcional
                              \end{choices}
                        \end{minipage}

                        \begin{solutionbox}{3cm}
                              $7\div 1=7$\\[-0.2em]
                              $9\div 2=4.5$\\[-0.2em]
                              $11\div 3=3.\overline{6}$\\[-0.2em]
                              $13\div 4=3.25$\\[-0.2em]
                              $15\div 5=3$\\[-0.2em]
                              $\therefore$ es una relación no proporcional.
                        \end{solutionbox}


                        % \part \includegraphics[width=.7\linewidth]{mex_0071.png}

                        % \begin{choices}
                        %     \choice Propocional
                        %     \choice No proporcional
                        % \end{choices}

                        \part
                        \begin{minipage}[t]{0.35\linewidth}
                              \includegraphics[width=\textwidth]{mex_0072.png}
                        \end{minipage}\hfill%
                        \begin{minipage}[b]{0.6\linewidth}
                              \begin{choices}
                                    \CorrectChoice Propocional
                                    \choice No proporcional
                              \end{choices}
                        \end{minipage}

                        \begin{solutionbox}{3cm}
                              $43.2\div 18=2.4$\\[-0.2em]
                              $33.6\div 14=2.4$\\[-0.2em]
                              $24\div 10=2.4$\\[-0.2em]
                              $14.4\div 6=2.4$\\[-0.2em]
                              $4.8\div 2=2.4$\\[-0.2em]
                              $\therefore$ es una relación proporcional.
                        \end{solutionbox}


                  \end{parts}
            \end{multicols}
      }

      \question[5]{Determina el valor de la constante de proporcionalidad para cada una de las siguientes tablas:

            \begin{multicols}{2}
                  \begin{parts}
                        \part
                        \begin{minipage}[t]{0.35\linewidth}
                              \includegraphics[width=\textwidth]{mex_0078.png}
                        \end{minipage}\hfill%
                        \begin{minipage}[b]{0.6\linewidth}
                              \begin{solutionbox}{2.8cm}\small%

                                    \begin{multicols}{2}
                                          $ 2\div 1=2$\\[-0.2em]
                                          $ 4\div 2=2$\\[-0.2em]
                                          $ 6\div 3=2$\\[-0.2em]
                                          $ 8\div 4=2$\\[-0.2em]
                                          $10\div 5=2$

                                          \columnbreak% 

                                          $\therefore$ La constante de proporcionalidad es 2.
                                    \end{multicols}
                              \end{solutionbox}
                        \end{minipage}


                        % \part \includegraphics[width=.7\linewidth]{mex_0071.png}

                        % \begin{choices}
                        %     \choice Propocional
                        %     \choice No proporcional
                        % \end{choices}

                        \part
                        \begin{minipage}[t]{0.35\linewidth}
                              \includegraphics[width=\textwidth]{mex_0073.png}
                        \end{minipage}\hfill%
                        \begin{minipage}[b]{0.6\linewidth}
                              \begin{solutionbox}{3.5cm}\small%

                                    \begin{multicols}{2}
                                          $\frac{16}{5}\div  4=\frac{4}{5}$\\[0.3em]
                                          $\frac{32}{5}\div  8=\frac{4}{5}$\\[0.3em]
                                          $\frac{48}{5}\div 12=\frac{4}{5}$\\[0.3em]
                                          $\frac{64}{5}\div 16=\frac{4}{5}$\\[0.3em]
                                          $16\div 20=\frac{4}{5}$

                                          \columnbreak%  

                                          $\therefore$ La constante de proporcionalidad es $\dfrac{4}{5}$.
                                    \end{multicols}
                              \end{solutionbox}
                        \end{minipage}
                  \end{parts}
            \end{multicols}
      }

      \question[5]{Escribe la regla de correspondencia (ecuación) de las siguientes tablas:

            \begin{multicols}{2}
                  \begin{parts}
                        \part%
                        \begin{minipage}[t]{0.25\linewidth}
                              \includegraphics[width=2\linewidth]{mex_0082.png}
                        \end{minipage}\hfill%
                        \begin{minipage}[b]{0.6\linewidth}
                              \begin{solutionbox}{2cm}
                                    La const. de prop. es $\frac{4}{5}$,

                                    $\therefore$ la ecuación es $y=\frac{4}{5}x$.
                              \end{solutionbox}
                        \end{minipage}


                        % \part \includegraphics[width=.7\linewidth]{mex_0071.png}

                        % \begin{choices}
                        %     \choice Propocional
                        %     \choice No proporcional
                        % \end{choices}

                        \part%
                        \begin{minipage}[t]{0.25\linewidth}
                              \includegraphics[width=2\linewidth]{mex_0083.png}
                        \end{minipage}\hfill%
                        \begin{minipage}[b]{0.6\linewidth}
                              \begin{solutionbox}{2cm}
                                    La const. de prop. es $\frac{1}{3}$,

                                    $\therefore$ la ecuación es $y=\frac{1}{3}x$.
                              \end{solutionbox}
                        \end{minipage}

                  \end{parts}
            \end{multicols}
      }

      \question[5]{Resuelve los siguientes problemas:

            % \begin{multicols}{2}
            %       \begin{parts}
            %             \part 
            Si 8 trabajadores construyen un muro en 15 horas, ¿cuánto tardarán 5 trabajadores en construir el mismo muro?
            \fillin[24][0.5cm]

            \begin{solutionbox}{2cm}
            \end{solutionbox}

            %             \part Un grifo tiene un caudal de salida de 18 litros por minuto y tarda 14 horas en llenar un tanque. ¿Cuánto tardaría si el caudal fuera de 7 litros por minuto?
            %             \fillin[36][0.5cm]

            %             \begin{solutionbox}{2cm}
            %             \end{solutionbox}
            %       \end{parts}
            % \end{multicols}
      }

      \question[5]{Escribe los términos faltantes de las siguientes sucesiones aritméticas:

            \begin{multicols}{3}
                  \begin{parts}
                        \part 28, 39, 50, \fillin[61][0.5cm], \fillin[72][0.5cm], \fillin[84][0.5cm], \dots
                        \part 56, 50, 44, \fillin[38][0.5cm], \fillin[32][0.5cm], \fillin[26][0.5cm], \dots
                        \part 33, 41, 49, \fillin[57][0.5cm], \fillin[65][0.5cm], \fillin[73][0.5cm], \dots
                  \end{parts}
            \end{multicols}%
      }


      \question[5]{Determina la diferencia de las siguientes sucesiones aritméticas:

            \begin{multicols}{2}
                  \begin{parts}
                        \part $-23,-15,-7,1,9,17,\ldots$ \fillin[$d=8$][0in]
                        \part $7,9,11,13,15,17,\ldots$ \fillin[$d=2$][0in]
                  \end{parts}
            \end{multicols}
      }

      \question[5]{Encuentra el \textit{n-ésimo} término de la siguientes sucesiones aritméticas:

            \begin{multicols}{2}
                  \begin{parts}
                        \part Calcula  el término número 44 de la siguiente sucesión aritmética: $a_n=-3n-15$

                        \begin{solutionbox}{1.5cm}
                              \[a_{44}=-3(44)-15=-132-15=-147\]
                        \end{solutionbox}

                        \part Calcula el término número 25 de la siguiente sucesión aritmética: $a_n=2n-6$

                        \begin{solutionbox}{1.5cm}
                              \[a_{25}=2(25)-6=50-6=44\]
                        \end{solutionbox}

                  \end{parts}
            \end{multicols}
      }

      \question[5]{Determina el término general de las siguientes sucesiones aritméticas:
            \begin{multicols}{2}
                  \begin{parts}
                        \part $40,35,30,25,20,\ldots$ \fillin[$5-5n$][1in]
                        \part $-2,-6,-10,-14,-18,\ldots$ \fillin[$-4n+2$][1in]
                  \end{parts}
            \end{multicols}
      }

      \question[5]{Encuentra el \textit{n-ésimo} término de la siguientes sucesiones aritméticas:
            \begin{multicols}{2}
                  \begin{parts}
                        \part Calcula el término número 28 de la siguiente sucesión aritmética: $-69,-72,-75,-78,-81,\ldots$
                        \begin{solutionbox}{1.5cm}
                              \[-3(28)-66=-84-66=-150\]
                        \end{solutionbox}

                        \part Calcula el término número 47 de la siguiente sucesión aritmética: $-5,0,5,10,15,\ldots$
                        \begin{solutionbox}{1.5cm}
                              \[5(47)-5=235-5=225\]
                        \end{solutionbox}
                  \end{parts}
            \end{multicols}
      }

      \question[5]{Escribe la expresión algebraica correcta para los siguientes enunciados:
            \begin{multicols}{2}
                  \begin{parts}
                        \part El cuadrado de la diferencia de dos números cualquiera.

                        \begin{solutionbox}{1.3cm}
                              $(x-y)^2$
                        \end{solutionbox}

                        \part El cubo de un número cualquiera aumentado en 10.

                        \begin{solutionbox}{1.3cm}
                              $x^3+10$
                        \end{solutionbox}
                  \end{parts}
            \end{multicols}
      }

      \question[5]{Encuentra el valor numérico de Las siguientes expresiones:

            \begin{multicols}{2}
                  \begin{parts}
                        \part $\large \dfrac{m-p}{n}$ cuando $m=8$, $n=5$ y $p=-2$.

                        \begin{solutionbox}{1.6cm}\footnotesize%
                              $\dfrac{m-p}{n}=\dfrac{8-(-2)}{5}=\dfrac{8´2}{5}=\dfrac{10}{5}=$ \fillin[2][0cm]
                        \end{solutionbox}

                        \part $\large a^{2}-2ab+b^{2}$ cuando $a=-4$ y $b=-7$.

                        \begin{solutionbox}{1.6cm}\footnotesize%
                              $a^{2}-2ab+b^{2}=(-4)^{2}-2(-4)(-7)+(-7)^{2}=16-56+49=$ \fillin[9][0cm]
                        \end{solutionbox}
                  \end{parts}
            \end{multicols}
      }

      \question[5]{Resuelve las siguientes ecuaciones:

            \begin{multicols}{3}
                  \begin{parts}
                        \part $ -x-2=15 $

                        \begin{solutionbox}{3.5cm}\footnotesize%
                              \begin{align*}
                                    -x-2 & = 15                \\
                                    -x   & =15+2               \\
                                    -x   & =17                 \\
                                    x    & = \frac{17}{-1}=-17
                              \end{align*}
                        \end{solutionbox}

                        \part $ 11x-33=55 $

                        \begin{solutionbox}{3.5cm}\footnotesize%
                              \begin{align*}
                                    11x-33 & = 55            \\
                                    11x    & =55+33          \\
                                    11x    & =88             \\
                                    x      & = \frac{88}{11}
                              \end{align*}
                        \end{solutionbox}

                        \part $\large -5x+9=-8x+3$

                        \begin{solutionbox}{3.5cm}\footnotesize%
                              \begin{align*}
                                    -5x+9   & =-8x+3 \\
                                    -5x     & =-8x-6 \\
                                    -5x +8x & =-6    \\
                                    3x      & =-6    \\
                                    x       & =-2
                              \end{align*}
                        \end{solutionbox}
                  \end{parts}
            \end{multicols}
      }

      \question[8]{Utilizando el m\'etodo de tu preferencia, encuentra el valor de $x$ y $y$ para
            cada uno de los siguientes sistemas de ecuaciones lineales:
            \begin{multicols}{2}
                  \begin{parts}
                        \part
                        \begin{align*}\Large%
                              2x+y & =  -10 \\
                              x-3y & =  2
                        \end{align*}

                        \begin{solutionbox}{4cm}
                              $x=-4$, $y=-2$
                        \end{solutionbox}

                        \part
                        \begin{align*}\Large%
                              \frac{3}{5}x+\frac{1}{4}y & =  2   \\[1em]
                              x-5y                      & =   25
                        \end{align*}

                        \begin{solutionbox}{3cm}
                              $x=5$, $y=-4$
                        \end{solutionbox}
                        % \vspace{4cm}
                  \end{parts}
            \end{multicols}
      }

      \question[3]{Numera correctamente los pasos para resolver un sistema de dos ecuaciones con dos inc\'ognitas por los m'etodos a continuaci\'on:
            \begin{choices}
                  \choice M\'etodo de sustitución:
                  \begin{itemize}
                        \item[\rule{1cm}{0.2mm}] Despejar una inc\'ognita en una de las ecuaciones.
                        \item[\rule{1cm}{0.2mm}] Resolver la ecuaci\'on resultante.
                        \item[\rule{1cm}{0.2mm}] Sustituir el valor obtenido en la ecuaci\'on en la que aparec\'ia la inc\'ognita despejada.
                        \item[\rule{1cm}{0.2mm}] Sustituir la expresi\'on de esta inc\'ognita en la otra ecuaci\'on para obtener una ecuaci\'on con una sola inc\'ognita.
                        \item[\rule{1cm}{0.2mm}] Sustituir los valores en las ecuaciones originales para comprobar que son la soluci\'on.
                  \end{itemize}

                  \choice M\'etodo de suma-resta:
                  \begin{itemize}
                        \item[\rule{1cm}{0.2mm}] Resolver la ecuaci\'on resultante.
                        \item[\rule{1cm}{0.2mm}] Sumar o restar las ecuaciones para eliminar una de las inc\'ognitas.
                        \item[\rule{1cm}{0.2mm}] Sustituir los valores en las ecuaciones originales para comprobar que son la soluci\'on.
                        \item[\rule{1cm}{0.2mm}] Multiplicar una o ambas ecuaciones por los n\'umeros necesarios para realizar la eliminaci\'on bajo la suma o resta.
                        \item[\rule{1cm}{0.2mm}] Sustituir el valor obtenido en una de las ecuaciones iniciales y resolverla.
                  \end{itemize}


                  \choice M\'etodo de igualaci\'on:
                  \begin{itemize}
                        \item[\rule{1cm}{0.2mm}] Resolver la ecuaci\'on resultante.
                        \item[\rule{1cm}{0.2mm}] Despejar la misma inc\'ognita en ambas ecuaciones.
                        \item[\rule{1cm}{0.2mm}] Sustituir los valores en las ecuaciones originales para comprobar que son la soluci\'on.
                        \item[\rule{1cm}{0.2mm}] Igualar las expresiones para obtener una ecuaci\'on con una inc\'ognita
                        \item[\rule{1cm}{0.2mm}] Sustituir el valor obtenido en cualquiera de las dos expresiones en las que aparec\'ia despejada la otra inc\'ognita.
                  \end{itemize}
            \end{choices}
      }

      \question[4]{Resuelve el siguiente sistema de ecuaciones lineales con decimales:
            \begin{align*}\Large%
                  -0.2x+0.4y= & 0.6 \\
                  x+2y=       & -3
            \end{align*}

            \begin{solutionbox}{4cm}
                  $x=-3$, $y=0$
            \end{solutionbox}
      }
\end{questions}
\end{document}