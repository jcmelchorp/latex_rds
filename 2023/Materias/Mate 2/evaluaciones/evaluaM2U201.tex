\documentclass[12pt,addpoints]{evalua}
\grado{2$^\circ$ de Secundaria}
\cicloescolar{2022-2023}
\materia{Matemáticas 2}
\unidad{2}
\title{Examen de la Unidad}
\aprendizajes{
    \item Resuelve problemas de proporcionalidad directa e inversa y de
    reparto proporcional.
    \item Obtiene la expresión algebraica y construye gráficas de una
    situación de
    proporcionalidad directa e inversa.
    \item Construye polígonos regulares a partir de algunas
    medidas (lados, apotema, diagonales, etcétera).
    \item Descompone figuras en otras para calcular su área.
    \item Calcula el perímetro y el área de polígonos regulares y del
    círculo a partir de diferentes datos.
}
\author{Prof.: Julio César Melchor Pinto}
\begin{document}
\begin{multicols}{2}
    \include*{../blocks/block001}
    \include*{../blocks/block003}
    \include*{../blocks/block004}
    \include*{../blocks/block002}
\end{multicols}
\newpage
\begin{questions} %%% REVISAR RESPUESTAS
    % \include*{../questions/question004}
    % \include*{../questions/question001}
    % \include*{../questions/question005}
    % \include*{../questions/question002}
    \question[20] \include*{../questions/question003}
    \question[20] \include*{../questions/question009}
    %\include*{../questions/question011}
    \newpage
    %\include*{../questions/question012}
    \question[20] \include*{../questions/question010}
    \question[20]  \include*{../questions/question015}
    % \newpage
    % \include*{../questions/question013}
    %\include*{../questions/question014}
    \newpage
    \question[20] \include*{../questions/question006}
    %\include*{../questions/question008}
\end{questions}

\end{document}