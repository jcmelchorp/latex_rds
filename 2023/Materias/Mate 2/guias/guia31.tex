\documentclass[12pt]{guia}
\grade{2$^\circ$ de Secundaria}
\cycle{2022-2023}
\subject{Matemáticas 2}
\guide{31}
\title{Problemas vervales sobre series y sucesiones aritméticas}
\aprendizajes{
    \begin{itemize}[leftmargin=*,label=\small\color{colorrds}\faIcon{user-graduate}]
        \item Verifica algebraicamente la equivalencia de expresiones de primer grado, formuladas a partirde sucesiones.
    \end{itemize}
}
\requisitos{
    \begin{itemize}
        \item Requisito 1
        \item Requisito 2
    \end{itemize}
}
\author{J. C. Melchor Pinto}

\begin{document}
\pagestyle{headandfoot}
\addpoints
\INFO
\printanswers
\vspace{-2em}
\begin{multicols}{2}
    \include*{../blocks/block030a}
    \include*{../blocks/block030d}
    \include*{../blocks/block030b}
    \include*{../blocks/block030c}
\end{multicols}
\begin{importantbox}
    La \textbf{regla de recurrencia} de una sucesión es una expresión algebraica que permite calcular el valor de cada término con sólo saber su posición en la serie ($n$).
\end{importantbox}
\begin{questions}
    \questionboxed[5]{\include*{../questions/question110}}
    \questionboxed[5]{\include*{../questions/question111}}
    \questionboxed[5]{\include*{../questions/question112}}
    \questionboxed[5]{\include*{../questions/question113}}
    \questionboxed[5]{\include*{../questions/question114}}
    \questionboxed[5]{\include*{../questions/question118}}
    \questionboxed[5]{\include*{../questions/question115}}
    \questionboxed[5]{\include*{../questions/question116}}
    \questionboxed[5]{\include*{../questions/question117}}
\end{questions}
\end{document}