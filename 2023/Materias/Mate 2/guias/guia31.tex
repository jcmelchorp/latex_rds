\documentclass[12pt,addpoints]{guia}
\grado{2$^\circ$ de Secundaria}
\cicloescolar{2022-2023}
\materia{Matemáticas 2}
\guia{31}
\unidad{3}
\title{Problemas verbales sobre series y sucesiones aritméticas}
\aprendizajes{\item Verifica algebraicamente la equivalencia de expresiones de primer grado, formuladas a partir de sucesiones.}
\author{JC Melchor Pinto}
\begin{document}
\INFO%
\include*{../blocks/block040a}
\begin{multicols}{2}
    \include*{../blocks/block040d}
    \include*{../blocks/block040e}
    \columnbreak
    \include*{../blocks/block040c}
    \include*{../blocks/block040b}
\end{multicols}%
\begin{importantbox}
    La \textbf{regla de recurrencia} de una sucesión es una expresión algebraica que permite calcular el valor de cada término con sólo saber su posición en la serie ($n$).
\end{importantbox}
\ejemplosboxed[\include*{../questions/question110}]
\begin{questions}
    \questionboxed[15]{\include*{../questions/question112}}
    \questionboxed[20]{\include*{../questions/question115}}
    \questionboxed[20]{\include*{../questions/question116}}
    \ejemplosboxed[\include*{../questions/question111}]
    \questionboxed[10]{\include*{../questions/question117}}
    \ejemplosboxed[\include*{../questions/question113a}]
    \questionboxed[15]{\include*{../questions/question114}}
    \ejemplosboxed[\include*{../questions/question113}]
    \questionboxed[20]{\include*{../questions/question118}}
\end{questions}
\end{document}