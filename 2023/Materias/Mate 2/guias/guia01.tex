\documentclass[12pt,addpoints,answers]{guia}
\grado{2$^\circ$ de Secundaria}
\cicloescolar{2022-2023}
\materia{Matemáticas 2}
\guia{1}
\unidad{3}
\title{Reparto proporcional inverso}
\aprendizajes{\item Resuelve problemas de proporcionalidad directa e inversa y de reparto proporcional.
        \item Analiza y compara situaciones de variación línea y proporcionalidad inversa, a partir de sus representaciones tabular, gráfica y algebraica, interpreta y resuelve problema que se modelan con este tipo de variación, incluyendo fenómenos de la física y otros contextos.
    }
\author{JC Melchor Pinto}
\begin{document}
\pagestyle{headandfoot}
\INFO%
\begin{questions}
    \questionboxed[25]{\include*{../questions/question001}}
    \questionboxed[25]{\include*{../questions/question003}}
    \questionboxed[25]{\include*{../questions/question002}}
\end{questions}

%\vfill
%\puntuacion

\end{document}