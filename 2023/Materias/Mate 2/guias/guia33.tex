\documentclass[12pt]{guia}
\grade{2$^\circ$ de Secundaria}
\cycle{2022-2023}
\subject{Matemáticas 2}
\guide{33}
\title{Equivalencia de expresiones algebraicas}
\aprendizajes{
    \begin{itemize}[leftmargin=*,label=\small\color{colorrds}\faIcon{user-graduate}]
        \item Formula expresiones de primer grado para representar propiedades (perímetros y áreas)
        de figuras geométricas y verifica equivalencia de expresiones, tanto algebraica como
        geométricamente (análisis de las figuras).
    \end{itemize}
}
\requisitos{
    \begin{itemize}
        \item Requisito 1
        \item Requisito 2
    \end{itemize}
}
\author{J. C. Melchor Pinto}

\begin{document}
\pagestyle{headandfoot}
\addpoints
\INFO
\printanswers
\begin{questions}
    \include*{../questions/question073}
    \include*{../questions/question076a}
    \include*{../questions/question076b}
    \begin{multicols}{2}
        \include*{../questions/question075a}
    \end{multicols}
    \newpage
    \include*{../questions/question078b}
\end{questions}

%\vfill
%\puntuacion

\end{document}