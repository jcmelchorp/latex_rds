\documentclass[12pt,addpoints]{guia}
\grado{2$^\circ$ de Secundaria}
\cicloescolar{2022-2023}
\materia{Matemáticas 2}
\guia{34}
\unidad{3}
\title{Expresiones algebraicas sobre perímetro y área}
\aprendizajes{\item Formula expresiones de primer grado para representar propiedades (perímetros y áreas)
        de figuras geométricas y verifica equivalencia de expresiones, tanto algebraica como
        geométricamente (análisis de las figuras).
    }
\author{JC Melchor Pinto}
\begin{document}
\INFO%
\begin{questions}
    \questionboxed[5]{\include*{../questions/question077d}}
    \questionboxed[5]{\include*{../questions/question077c}}
    \questionboxed[20]{\include*{../questions/question077l}}
    \questionboxed[5]{\include*{../questions/question077}}
    %%%% Equivalencia de expresiones algebraicas
    \ejemplosboxed[\include*{../questions/question077g}]
    \questionboxed[10]{\include*{../questions/question077j}}
    % \questionboxed[10]{\include*{../questions/question077k}}
    \ejemplosboxed[\include*{../questions/question077m}]
    \questionboxed[5]{\include*{../questions/question077e}}
    \questionboxed[5]{\include*{../questions/question077i}}
    \questionboxed[5]{\include*{../questions/question077p}}
    \questionboxed[5]{\include*{../questions/question077h}}
    \questionboxed[5]{\include*{../questions/question077n}}
    \questionboxed[5]{\include*{../questions/question077o}}
    \questionboxed[5]{\include*{../questions/question077f}}
    \questionboxed[10]{\include*{../questions/question077a}}
    \questionboxed[10]{\include*{../questions/question077b}}
\end{questions}
\end{document}