\documentclass[12pt]{guia}
\grade{2$^\circ$ de Secundaria}
\cycle{2022-2023}
\subject{Matemáticas 2}
\guide{34}
\title{Expresiones algebraicas sobre perímetro y área}
\aprendizajes{
    \begin{itemize}[leftmargin=*,label=\small\color{colorrds}\faIcon{user-graduate}]
        \item Formula expresiones de primer grado para representar propiedades (perímetros y áreas)
        de figuras geométricas y verifica equivalencia de expresiones, tanto algebraica como
        geométricamente (análisis de las figuras).
    \end{itemize}
}
\requisitos{
    \begin{itemize}
        \item Requisito 1
        \item Requisito 2
    \end{itemize}
}
\author{J. C. Melchor Pinto}

\begin{document}
\pagestyle{headandfoot}
\addpoints
\INFO
\printanswers
Recordemos las expresiones generales para obtener el perímetro o el área de una
figura geométrica.
\begin{questions}
    \include*{../questions/question077}
    \newpage
    \include*{../questions/question077d}
    \include*{../questions/question077c}
    \newpage
    \include*{../questions/question077e}
    %%%% Equivalencia de expresiones algebraicas
    \include*{../questions/question077g}
    \include*{../questions/question077i}
    \include*{../questions/question077j}
    \include*{../questions/question077k}
    \include*{../questions/question077l}
    \include*{../questions/question077m}
    \include*{../questions/question077p}
    % \begin{multicols}{2}
    \include*{../questions/question077h}
    \include*{../questions/question077n}
    \include*{../questions/question077o}
    \include*{../questions/question077f}
    % \end{multicols}
    \newpage
    \include*{../questions/question077a}
    \include*{../questions/question077b}
    \include*{../questions/question078a}
\end{questions}

%\vfill
%\puntuacion

\end{document}