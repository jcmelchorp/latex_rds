\documentclass[12pt]{guia}
\grade{2$^\circ$ de Secundaria}
\cycle{2022-2023}
\subject{Matemáticas 2}
\guide{34}
\title{Expresiones algebraicas sobre perímetro y área}
\aprendizajes{
    \begin{itemize}[leftmargin=*,label=\small\color{colorrds}\faIcon{user-graduate}]
        \item Formula expresiones de primer grado para representar propiedades (perímetros y áreas)
        de figuras geométricas y verifica equivalencia de expresiones, tanto algebraica como
        geométricamente (análisis de las figuras).
    \end{itemize}
}
\requisitos{
    \begin{itemize}
        \item Requisito 1
        \item Requisito 2
    \end{itemize}
}
\author{J. C. Melchor Pinto}

\begin{document}
\pagestyle{headandfoot}
\addpoints
\INFO
\printanswers
\begin{questions}
    \questionboxed[5]{\include*{../questions/question077}}
    % \newpag
    \questionboxed[5]{\include*{../questions/question077d}}
    \questionboxed[5]{\include*{../questions/question077c}}
    % \newpage
    \questionboxed[5]{\include*{../questions/question077e}}
    %%%% Equivalencia de expresiones algebraicas
    \questionboxed[10]{\include*{../questions/question077g}}
    \questionboxed[5]{\include*{../questions/question077i}}
    \questionboxed[10]{\include*{../questions/question077j}}
    % \questionboxed[10]{\include*{../questions/question077k}}
    \questionboxed[5]{\include*{../questions/question077l}}
    \questionboxed[5]{\include*{../questions/question077m}}
    \questionboxed[5]{\include*{../questions/question077p}}
    % \begin{multicols}{2}
    \questionboxed[5]{\include*{../questions/question077h}}
    \questionboxed[5]{\include*{../questions/question077n}}
    \questionboxed[5]{\include*{../questions/question077o}}
    \questionboxed[5]{\include*{../questions/question077f}}
    % \end{multicols}
    % \newpage
    \questionboxed[10]{\include*{../questions/question077a}}
    \questionboxed[10]{\include*{../questions/question077b}}
\end{questions}
\end{document}