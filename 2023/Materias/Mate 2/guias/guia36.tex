\documentclass[12pt,addpoints,answers]{guia}
\grado{2$^\circ$ de Secundaria}
\cicloescolar{2022-2023}
\materia{Matemáticas 2}
\guia{36}
\unidad{3}
\title{Descomposición de figuras para calcular su volumen}
\aprendizajes{
    \begin{itemize}[leftmargin=*,label=\small\color{colorrds}\faIcon{user-graduate}]
        \item Calcula el volumen de prismas y cilindros rectos.
    \end{itemize}
}
\author{J. C. Melchor Pinto}
\begin{document}
\pagestyle{headandfoot}

\INFO
\printanswers
\begin{multicols}{2}
    \include*{../blocks/block035a}
    \include*{../blocks/block035b}
    \include*{../blocks/block035d}
\end{multicols}
\begin{questions}
    \questionboxed[10]{\include*{../questions/question102a}}
    \questionboxed[10]{\include*{../questions/question102b}}
    \questionboxed[10]{\include*{../questions/question102c}}
    \questionboxed[10]{\include*{../questions/question102d}}
    \questionboxed[10]{\include*{../questions/question102e}}
    \questionboxed[10]{\include*{../questions/question102f}}
    \questionboxed[10]{\include*{../questions/question102g}}
    \questionboxed[10]{\include*{../questions/question104a}}
    \questionboxed[10]{\include*{../questions/question104b}}
    \questionboxed[10]{\include*{../questions/question104c}}
    % \questionboxed[10]{\include*{../questions/question104d}}
    % \questionboxed[10]{\include*{../questions/question104e}}
    % \questionboxed[10]{\include*{../questions/question104f}}
    % \questionboxed[10]{\include*{../questions/question104g}}
\end{questions}
\end{document}