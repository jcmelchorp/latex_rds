\documentclass[12pt]{guia}
\grade{2$^\circ$ de Secundaria}
\cycle{2022-2023}
\subject{Matemáticas 2}
\guide{20}
\title{Series y sucesiones aritméticas}
\aprendizajes{
    \begin{itemize}[leftmargin=*,label=\small\color{colorrds}\faIcon{user-graduate}]
        \item Verifica algebraicamente la equivalencia de expresiones de primer grado, formuladas a partir de sucesiones.
    \end{itemize}
}
\requisitos{
    \begin{itemize}
        \item Requisito 1
        \item Requisito 2
    \end{itemize}
}
\author{J. C. Melchor Pinto}

\begin{document}
\pagestyle{headandfoot}
\addpoints
\INFO
\printanswers
\begin{questions}
    \include*{../questions/question020}
    \include*{../questions/question021}
    \newpage
    \begin{importantbox}
        La \textbf{regla de recurrencia} de una sucesión es una expresión algebraica que permite calcular el valor de cada término.
    \end{importantbox}
    \include*{../questions/question023}
    \newpage
    \include*{../questions/question024}
    \newpage
    \begin{multicols}{2}
        \include*{../questions/question025}
        \include*{../questions/question026}
    \end{multicols}
    \include*{../questions/question027}
    \begin{multicols}{2}
        \include*{../questions/question028}
        \include*{../questions/question029}
    \end{multicols}
    \include*{../questions/question030}
    \include*{../questions/question031}
    \include*{../questions/question032}
    \include*{../questions/question022}
\end{questions}
\end{document}