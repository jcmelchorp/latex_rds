\documentclass[12pt,addpoints]{guia}
\grado{2$^\circ$ de Secundaria}
\cicloescolar{2022-2023}
\materia{Matemáticas 2}
\guia{32}
\unidad{3}
\title{Ejercicios sobre series y sucesiones aritméticas}
\aprendizajes{\item Verifica algebraicamente la equivalencia de expresiones de primer grado, formuladas a partirde sucesiones.}
\author{JC Melchor Pinto}
\begin{document}
\INFO%
\begin{multicols}{2}
    \include*{../blocks/block030a}
    \include*{../blocks/block030d}\include*{../questions/question116}
    \columnbreak
    \include*{../blocks/block030b}
    \include*{../blocks/block030c}
\end{multicols}%
\begin{importantbox}
    La \textbf{regla de recurrencia} de una sucesión es una expresión algebraica que permite calcular el valor de cada término con sólo saber su posición en la serie ($n$).
\end{importantbox}%
\ejemplosboxed[\include*{../questions/question106a}]
\begin{questions}
    \questionboxed[10]{\include*{../questions/question106b}}
    \questionboxed[10]{\include*{../questions/question106c}}
    \ejemplosboxed[\include*{../questions/question106d}]
    \questionboxed[10]{\include*{../questions/question106e}}
    \questionboxed[10]{\include*{../questions/question106f}}
    \ejemplosboxed[\include*{../questions/question106g}]
    \questionboxed[10]{\include*{../questions/question106h}}
    \ejemplosboxed[\include*{../questions/question106i}]
    \questionboxed[10]{\include*{../questions/question106j}}
    \ejemplosboxed[\include*{../questions/question107a}]
    \questionboxed[10]{\include*{../questions/question107f}}
    \ejemplosboxed[\include*{../questions/question107d}]
    \questionboxed[10]{\include*{../questions/question107e}}
    \questionboxed[10]{\include*{../questions/question107b}}
    \questionboxed[10]{\include*{../questions/question107c}}
\end{questions}
\end{document}