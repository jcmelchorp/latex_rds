\documentclass[12pt]{guia}
\grade{2$^\circ$ de Secundaria}
\cycle{2022-2023}
\subject{Matemáticas 2}
\guide{32}
\title{Ejercicios sobre series y sucesiones aritméticas}
\aprendizajes{
    \begin{itemize}[leftmargin=*,label=\small\color{colorrds}\faIcon{user-graduate}]
        \item Verifica algebraicamente la equivalencia de expresiones de primer grado, formuladas a partirde sucesiones.
    \end{itemize}
}
\requisitos{
    \begin{itemize}
        \item Requisito 1
        \item Requisito 2
    \end{itemize}
}
\author{J. C. Melchor Pinto}

\begin{document}
\pagestyle{headandfoot}
\addpoints
\INFO
%\printanswers
\vspace{-2em}
\begin{multicols}{2}
    \include*{../blocks/block030a}
    \include*{../blocks/block030d}
    \columnbreak
    \include*{../blocks/block030b}
    \include*{../blocks/block030c}
\end{multicols}
\begin{importantbox}
    La \textbf{regla de recurrencia} de una sucesión es una expresión algebraica que permite calcular el valor de cada término con sólo saber su posición en la serie ($n$).
\end{importantbox}
\newpage
\begin{questions}
    \questionboxed[10]{\include*{../questions/question106a}}
    \questionboxed[10]{\include*{../questions/question106b}}
    \questionboxed[10]{\include*{../questions/question106c}}
    \questionboxed[10]{\include*{../questions/question106d}}
    \questionboxed[10]{\include*{../questions/question106e}}
    % \questionboxed[10]{\include*{../questions/question106f}}
    % \questionboxed[10]{\include*{../questions/question106g}}
    % \questionboxed[10]{\include*{../questions/question106h}}
    % \questionboxed[10]{\include*{../questions/question106i}}
    % \questionboxed[10]{\include*{../questions/question106j}}
    \questionboxed[10]{\include*{../questions/question107a}}
    \questionboxed[10]{\include*{../questions/question107b}}
    \questionboxed[10]{\include*{../questions/question107c}}
    \questionboxed[10]{\include*{../questions/question107d}}
    \questionboxed[10]{\include*{../questions/question107e}}
    % \questionboxed[10]{\include*{../questions/question107f}}
\end{questions}

%\vfill
%\puntuacion

\end{document}