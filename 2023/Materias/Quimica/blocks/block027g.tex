\begin{warncard}[adjusted title={Masa atómica y masa molecular}]
    La masa atómica $m_a$ de un elemento es la masa de un átomo de ese elemento, la cual se encuentra definida dentro de la \emph{Tabla Periódica de los Elementos} y se expresa en Unidades de Masa Atómica (UMA).
    Al ser esta cantidad muy pequeña, el kilogramo resulta una unidad demasiado grande como para expresar su valor. Por ello, en 1961, la IUPAC acordó utilizar un nuevo patrón, la unidad de másica atómica o Dalton (Da), que equivale a la masa de la doceava parte del átomo de carbono-12:

    \[ 1 \text{ u} = 1.66053886\cdot 10^{-27} \text{ kg}\]

    Así, la masa del átomo de carbono-12 será de
    \[12 \text{ u} = 19.92646632\cdot 10^{-27} \text{ kg}\]

    \emph{Ejemplo:} \\
    La masa atómica $m_a$ del carbono es (ver tabla periódica):

    \[m_a(\ce{C})=12.01 \text{ UMA}\]

    \tcblower

    La masa molecular $m_m$ de una sustancia es la suma de las masas atómicas de los átomos que la componen, y se expresa también en UMA.\\

    \emph{Ejemplo:} \\
    La masa molecular $m_m$ del $CO_2$ es $44.01$ UMA, ya que

    \begin{align*}
        m_m(\ce{CO2}) & = m_a(C) + 2 \times m_a(\ce{O}) \\
                      & = 12.01 + 2(16.00)              \\
                      & = 44.01 \text{ UMA}
    \end{align*}
\end{warncard}