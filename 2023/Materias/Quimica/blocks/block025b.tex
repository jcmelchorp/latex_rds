\begin{infocard}{Reacción química}
    Las partículas que constituyen las distintas sustancias de nuestro entorno están en constante
    movimiento y se atraen unas a otras mediante fuerzas intermoleculares. En ocasiones,
    cuando dos o más sustancias entran en contacto, las colisiones y atracciones
    entre sus partículas pueden hacer que algunas moléculas se separen en los átomos que
    las forman. Estos átomos, a su vez, pueden interaccionar con otros y unirse a ellos para
    formar \textbf{nuevas moléculas}, ocurriendo así una \textbf{reacción química}.
    %
    % Una reacción química toma lugar cuando los materiales que se usan al inicio (reactivos) se
    % cambian a nuevos materiales (productos).
    %
    % En la vida diaria ocurren muchas reacciones químicas, por ejemplo la fermentación, la
    % combustión, la corrosión de metales (clavos oxidados), la efervescencia de pastillas,
    % la oxidación de las frutas (oscurecimiento del plátano o de la manzana), entre muchas otras.
    %
    % Los tipos más comunes de reacciones químicas son: síntesis, descomposición, desplazamiento  y doble desplazamiento.
\end{infocard}