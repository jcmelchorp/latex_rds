\begin{warncard}[adjusted title={Reacción de descomposición}]
    Las \textbf{reacciones de descomposición} ocurren cuando se descompone un reactivo en dos o más productos (que pueden ser elementos o compuestos) por medio de un agente externo o un catalizador.

    \[\ce{AB -> A + B}\]

    Donde A y B son elementos y/o compuestos y AB es un compuesto.

    \begin{itemize}
        \item[\color{Red}\faIcon{gripfire}] Si la descomposición ocurre por calentamiento a altas temperaturas se le llama \textbf{\color{Red}pirólisis}.
        \item[\color{orange}\faIcon{sun}] Si la descomposición ocurre mediante la luz se le conoce como \textbf{\color{orange}fotólisis}.
        \item[\color{Brown}\faIcon{flask}] Si la descomposición es provocada por un catalizador se le llama \textbf{\color{Brown}catálisis}.
        \item[\color{Goldenrod}\faIcon{bolt}] Si la descomposición se realiza con la ayuda de la electricidad se le llama \textbf{\color{Goldenrod}electrólisis}.
    \end{itemize}
\end{warncard}