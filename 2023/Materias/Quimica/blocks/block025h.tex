\begin{warncard}[adjusted title={Reacciones de desplazamiento doble}]
    En las \textbf{reacciones de desplazamiento doble} participan dos compuestos, en donde el catión de un compuesto se intercambia con el catión de otro compuesto. También se puede decir que los dos cationes intercambian aniones o compañeros. Estás reacciones se conocen también como de metátesis (que significa un cambio en el estado, en la sustancia o en la forma). Su ecuación general es:
    \[\ce{AX + BZ -> AZ + BX}\]

    En las reacciones de doble sustitución hay cuatro partículas separadas: los cationes A y B y los aniones Z y X y se llevan a cabo si se cumple una de las siguientes condiciones:

    \begin{enumerate}
        \item Si se forma un sólido insoluble o casi insoluble conocido como precipitado.
        \item Si se obtiene un compuesto covalente estable, agua o los gases comunes.
        \item Si se obtiene como producto un gas.
        \item Si hay desprendimiento de calor.
    \end{enumerate}
\end{warncard}