\begin{opening}[¿Cómo determinamos la cantidad de las sustancias?]
    ¿Te has preguntado alguna vez cuántos granos de arena hay en las playas más espectaculares de la
    península de Yucatán? ¿Cuántas estrellas existen en el Universo? ¿Cuántas veces es más grande tu
    cuerpo que una célula? ¿Qué tan pequeño es un átomo comparado con ella?
    Los químicos no pueden abrir un recipiente para ver y contar los átomos de
    una muestra de sustancia, así que usan
    una medida especial para determinar la cantidad de partículas, ya sean átomos, moléculas o iones.
\end{opening}