\documentclass[12pt,addpoints]{guia}
\grado{3$^\circ$ de Secundaria}
\cicloescolar{2022-2023}
\materia{Ciencias y Tecnología: Química}
\guia{29}
\unidad{3}
\title{Cantidad de sustancia}
\aprendizajes{\item Argumenta acerca de posibles cambios químicos en un sistema con base en evidencias experimentales. 
        \item Explica, predice y representa cambios químicos con base en la separación y unión de átomos o iones, y se recombinan para formar nuevas sustancias.}
\author{JC Melchor Pinto}
\begin{document}
\INFO%
\include*{../blocks/block027i}%
\ejemplosboxed[\include*{../questions/question027m}]
\begin{questions}    
   \questionboxed[10]{\include*{../questions/question027n}}
     \questionboxed[10]{\include*{../questions/question027o}}
    \questionboxed[10]{\include*{../questions/question027v}}
    \questionboxed[10]{\include*{../questions/question027w}}
     \questionboxed[10]{\include*{../questions/question027x}}
     \ejemplosboxed[\include*{../questions/question027ag}]
     \questionboxed[10]{\include*{../questions/question027ah}}
     \questionboxed[10]{\include*{../questions/question027ai}}
     \ejemplosboxed[\include*{../questions/question027p}]
    \questionboxed[10]{\include*{../questions/question027q}}
    \questionboxed[10]{\include*{../questions/question027r}}
    \ejemplosboxed[\include*{../questions/question027s}]
    \questionboxed[10]{\include*{../questions/question027t}}
\end{questions}
\end{document}