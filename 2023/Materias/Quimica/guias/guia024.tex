\documentclass[12pt,addpoints,answers]{guia}
\grado{3$^\circ$ de Secundaria}
\cicloescolar{2022-2023}
\materia{Ciencias y Tecnología: Química}
\guia{24}
\unidad{3}
\title{Relaciones entre la estructura y las propiedades de las sustancias}
\aprendizajes{\item Representa y diferencia mediante esquemas, modelos y
              simbología química, elementos y compuestos, así como
              átomos y moléculas.
        \item Explica y predice propiedades físicas de los materiales
              con base en modelos submicroscópicos sobre la
              estructura de átomos, moléculas o iones, y sus
              interacciones electrostáticas.
    }
\author{JC Melchor Pinto}
\begin{document}
\INFO%
\begin{multicols}{2}
    \include*{../blocks/block000}
\end{multicols}
\begin{opening}[Atracción electrostática]
    Objetivo:\\ Observar el comportamiento de diferentes sustancias ante la presencia de un objeto
    con carga eléctrica.

    \begin{minipage}{0.3\textwidth}
        \begin{figure}[H]
            \includegraphics[width=0.9\linewidth]{../images/atomicsmodels}
            \caption{Imagen ejemplo del experimento}
            \label{fig:atomicsmodels}
        \end{figure}

    \end{minipage}\hfill
    \begin{minipage}{0.6\textwidth}
        Materiales:
        \begin{itemize}
            \item Globo o varilla de plástico o vidrio
            \item Franela
            \item Bureta o jeringa de 10 mL
            \item Vaso de precipitados o de plástico
            \item Soporte universal
            \item Pinzas de laboratorio
            \item Diferentes sustancias líquidas (agua, etanol, acetona, hexano).
        \end{itemize}

        Procedimiento:
        \begin{enumerate}
            \item Froten el globo o la varilla de plástico o vidrio con una franela para cargarlo eléctricamente.
            \item Predigan, antes de realizar el experimento, si el objeto cargado atraerá, repelerá o no afectará el chorro de cada líquido. Justifiquen sus ideas.
            \item Sujeten la bureta al soporte y abran la llave para que salga un chorro de agua delgado, pero continuo.
            \item Acerquen el objeto cargado al chorro de agua sin tocarlo. ¿Qué sucede?
            \item Repitan el experimento con otros líquidos (etanol, acetona, hexano).
        \end{enumerate}

        Análisis de resultados y conclusiones:
        \begin{enumerate}
            \item Contrasten los resultados con sus predicciones.
            \item Generen hipótesis iniciales que expliquen por qué los líquidos que utilizaron se comportan de diferente manera.
        \end{enumerate}
    \end{minipage}
\end{opening}
\begin{multicols}{2}
    \include*{../blocks/block030a}
    \include*{../blocks/block030b}
\end{multicols}
%\fullwidth{
\begin{tcolorbox}[enhanced,attach boxed title to top center={yshift=-3mm,yshifttext=-1mm},
        colback=blue!5!white,colframe=blue!75!black,colbacktitle=red!80!black,
        title=Geometría molecular,fonttitle=\bfseries,
        boxed title style={size=small,colframe=red!50!black} ]
    Los materiales que nos rodean están hechos de diferentes sustancias, y sus propie-
    dades dependen de su composición y de la estructura de las partículas que los
    componen. Como hemos visto, las moléculas que conforman las sustancias están
    compuestas por distintos tipos de átomos enlazados de diversas maneras, lo cual
    determina si el material que forman será o no soluble en agua, si a temperatura
    ambiente existirá como líquido o sólido o si lo atraerá o no un cuerpo cargado.
    El tipo de átomos que se combinan para formar una molécula y la manera en la
    que se enlazan determina la \emph{geometría} de la partícula, es decir, la forma que ad-
    quiere; por ejemplo, las moléculas de dióxido de carbono (CO$_2$ ) están compuestas por
    un átomo de carbono unido mediante enlaces dobles a dos átomos de oxígeno organiza-
    dos en una línea. Se dice entonces que la molécula tiene geometría lineal (figura 2.38).
    No todas las moléculas compuestas por tres átomos son lineales. Las moléculas
    de agua H 2 O, por ejemplo, poseen una geometría plana angular debido a repulsiones
    entre los electrones de valencia en el átomo de oxígeno central y en los átomos de
    hidrógeno que lo rodean (figura 2.39).
    \begin{figure}[H]
        \includegraphics[width=0.45\textwidth]{20230321060221}
        \caption{Diferentes
            representaciones de
            la geometría lineal
            de la molécula de
            dióxido de carbono
            (CO$_2$ ) .}
        \label{fig:20230321060221}
    \end{figure}
    \begin{figure}[H]
        \includegraphics[width=0.45\textwidth]{20230321060418}
        \caption{Distintas
            representaciones
            de la geometría
            plana de la molécula
            de H 2 O.}
        \label{fig:20230321060418}
    \end{figure}
\end{tcolorbox}
% }

% \begin{questions}
%     hola
%     \begin{parts}
%         \part adiios
%     \end{parts}

%\fullwidth{
\begin{tcolorbox}[enhanced,attach boxed title to top center={yshift=-3mm,yshifttext=-1mm},
        colback=blue!5!white,colframe=blue!75!black,colbacktitle=red!80!black,
        title=Distribución de carga,fonttitle=\bfseries,
        boxed title style={size=small,colframe=red!50!black} ]
    La composición y geometría de las moléculas determina la forma en que los electro-
    nes de valencia se distribuyen entre los átomos que las conforman, lo cual, a su vez,
    afecta la interacción con otras moléculas. No todos los átomos atraen con la misma
    fuerza a los electrones de valencia que los enlazan. Algunos átomos, como los del flúor
    (F) y el oxígeno (O), atraen a los electrones con más fuerza que los átomos de carbono
    (C) e hidrógeno (H). Cuando un átomo de cloro se enlaza con uno de hidrógeno y ori-
    ginan una molécula de HCl (cloruro de hidrógeno), los electrones que forman el enlace
    pasan más tiempo cerca del átomo de cloro porque los atrae con más fuerza. Esto
    causa que la carga eléctrica no se distribuya uniformemente entre los dos átomos.
    \begin{figure}[H]
        \includegraphics[width=0.45\textwidth]{20230321060047}
        \caption{En la
            molécula de HCl, la
            carga electrónica
            no se distribuye
            uniformemente entre
            los dos átomos.}
        \label{fig:20230321060047}
    \end{figure}
    Dado que los electrones tienen carga eléctrica negativa y pasan más tiempo cerca
    del átomo de cloro en la molécula de HCl, la región donde se ubica el átomo de
    cloro es parcialmente negativa mientras la región que rodea al átomo de hidró-
    geno es parcialmente positiva. La molécula es eléctricamente neutra (posee el
    mismo número de protones que de electrones), pero la distribución de electrones
    no es la misma en todas partes. Por lo común, este fenómeno se representa asig-
    nando coloraciones distintas a las regiones negativas y positivas en una molécula.
    Como ilustra la figura 2.40, las zonas parcialmente negativas se colorean con tonos
    más rojizos, mientras las zonas parcialmente positivas con tonos más azulados.
\end{tcolorbox}
% }

%\fullwidth{
\begin{tcolorbox}[enhanced,attach boxed title to top center={yshift=-3mm,yshifttext=-1mm},
        colback=blue!5!white,colframe=blue!75!black,colbacktitle=red!80!black,
        title=Electronegatividad,fonttitle=\bfseries,
        boxed title style={size=small,colframe=red!50!black} ]
    El químico Linus Pauling propuso una manera de medir la fuerza con la que diferentes
    átomos atraen a los electrones en un enlace, propiedad que llamó electronegatividad.
    Para los elementos en la
    tabla periódica, la electronegatividad se incrementa
    de abajo hacia arriba y de izquierda a derecha, como
    muestra la figura 2.41.
    Los átomos de los elementos no metálicos poseen una electronegatividad más
    alta que los átomos de los elementos metálicos, lo cual quiere decir que atraen a los electrones en un enlace con
    más fuerza. Dentro de los elementos no metálicos, los átomos de flúor (F) y de oxígeno
    (O) son los más electronegativos mientras que los átomos de hidrógeno (H) y fósforo
    (P) son los menos electronegativos. El análisis de la electronegatividad de los dife-
    rentes átomos que se enlazan en una molécula permite predecir qué regiones de la
    partícula serán parcialmente negativas o positivas. En cualquier enlace químico entre
    dos átomos distintos, el átomo más electronegativo será parcialmente negativo y el
    átomo menos electronegativo será parcialmente positivo.
\end{tcolorbox}
% }

%\fullwidth{
\begin{tcolorbox}[enhanced,attach boxed title to top center={yshift=-3mm,yshifttext=-1mm},
        colback=blue!5!white,colframe=blue!75!black,colbacktitle=red!80!black,
        title=Polaridad molecular,fonttitle=\bfseries,
        boxed title style={size=small,colframe=red!50!black} ]
    El análisis de cómo los electrones de valencia se distribuyen entre los áto-
    mos de las moléculas permite clasificarlas en dos grandes grupos. En algu-
    nas moléculas la carga eléctrica se distribuye de manera uniforme entre los
    átomos. Este tipo de moléculas se denominan \emph{no polares}, y las moléculas
    en las que la carga eléctrica no está distribuida de manera uniforme se co-
    nocen como moléculas \emph{polares}. En la figura 2.42 se presentan dos ejemplos
    típicos de estos tipos de moléculas. Esta clasificación es importante porque
    las moléculas polares y las no polares interaccionan de manera distinta en-
    tre ellas y con otras moléculas, lo que les da propiedades distintas a las
    sustancias que se forman con estas partículas.
    \begin{figure}[H]
        \includegraphics[width=0.45\textwidth]{../images/20230321055533}
        \caption{La molécula de Cl$_2$ es no polar y la de HCl, polar.}
        \label{fig:20230321055533}
    \end{figure}
    \begin{figure}[H]
        \includegraphics[width=0.45\textwidth]{../images/20230321055646}
        \caption{LLa molécula de H$_2$O es polar pero la de CO$_2$ es no polar.}
        \label{fig:20230321055646}
    \end{figure}
    La polaridad de una molécula (esto es, si es polar o no polar) depende
    de los átomos que la constituyen, de cómo se enlazan entre sí y de la geo-
    metría de la molécula. Cuando las moléculas están formadas sólo por dos
    átomos, la decisión de si son o no polares es más o menos sencilla. Si los áto-
    mos tienen la misma electronegatividad, la molécula no será polar, pero si
    su electronegatividad es distinta, la molécula es polar porque los electrones
    no se distribuyen de manera uniforme entre los dos átomos.
    Si las moléculas tienen más de dos átomos, saber si son o no polares re-
    quiere mayor análisis de la forma en que se distribuye la carga electrónica
    sobre toda la molécula. Considera, por ejemplo, la distribución de la carga
    en las moléculas de CO 2 y H 2 O que se ilustran en la figura 2.43. La molécula
    de H 2 O se considera polar porque en un extremo es parcialmente negativa
    (donde se localiza el átomo de oxígeno) y el lado opuesto (donde están los
    hidrógenos) es parcialmente positivo. Esto es diferente de lo que observamos
    en la molécula de CO 2 , cuya región central, donde se localiza el átomo de
    carbono, es más positiva que las regiones laterales, donde se ubican los áto-
    mos de oxígeno. La distribución de carga en la molécula no es la misma, pero
    ningún extremo de la molécula es más positivo o negativo que el otro; cuan-
    do esto sucede, la molécula es no polar.
\end{tcolorbox}
% }

%\fullwidth{
\begin{tcolorbox}[enhanced,attach boxed title to top center={yshift=-3mm,yshifttext=-1mm},
        colback=blue!5!white,colframe=blue!75!black,colbacktitle=red!80!black,
        title=Fuerzas intermoleculares,fonttitle=\bfseries,
        boxed title style={size=small,colframe=red!50!black} ]
    %\centering Conjunto de fuerzas que determinan la interacción entre moléculas.
    Fuerza intermolecular se refiere a las interacciones que existen entre las moléculas conforme a su naturaleza. Generalmente, la clasificación es hecha de acuerdo a la polaridad de las moléculas que están interaccionando, o sobre la base de la naturaleza de las moléculas, de los elementos que la conforman.\\
    Las moléculas que conforman las sustancias están compuestas
    por protones y electrones con carga eléctrica. Por tanto, cuando
    una molécula se acerca a otra, los electrones de una de estas par-
    tículas pueden ser atraídos por los protones de la otra molécula,
    y viceversa; esto causa interacciones atractivas entre las moléculas
    a las que se denomina fuerzas intermoleculares. Estas interaccio-
    nes atractivas con frecuencia se representan con líneas delgadas
    punteadas entre moléculas, como ilustra la figura 2.44.
    Las fuerzas intermoleculares son responsables de que en ma-
    teriales sólidos y líquidos las moléculas se mantengan juntas.
    \begin{figure}[H]
        \includegraphics[width=0.45\textwidth]{../images/20230321054752.png}
        \caption{m}
        \label{fig:20230321054752}
    \end{figure}
    Debido a su presencia se debe proporcionar energía para separar
    las moléculas y transformar las sustancias de sólido a líquido o de
    líquido a gas. Con datos experimentales es posible inferir en qué
    sustancias las fuerzas intermoleculares son menores o mayores.
\end{tcolorbox}
% }

%\fullwidth{
\begin{tcolorbox}[enhanced,fit to height=8cm,
        colback=colorrds!25!black!10!white,colframe=colorrds!75!black,title=Fit box (5cm),
        drop fuzzy shadow,watermark color=white,watermark text=Fit]
    \lipsum[1-4]
\end{tcolorbox}

\begin{tcolorbox}[enhanced,fit to height=5cm,
        colback=colorrds!25!black!10!white,colframe=colorrds!75!black,title=Fit box (5cm),
        drop fuzzy shadow,watermark color=white,watermark text=Fit]
    \lipsum[1]
\end{tcolorbox}
% }
%\questionboxed[25]{\include*{../questions/question067a}}
%\questionboxed[25]{\include*{../questions/question068a}}
% \end{questions}
\end{document}
