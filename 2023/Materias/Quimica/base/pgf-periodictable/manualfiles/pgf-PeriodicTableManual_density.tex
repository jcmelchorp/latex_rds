\subsubsection{\texorpdfstring{\ding{224} The density}{The density}}
%%%%%%%%%%%%%%%%%%%%%%%%%%%%%%%%%%%%%%%%%%%%%%%%%%%%%%%%%%%%
% d color
\pgfPTMoption{4}{d color}{black}{%
Sets the density value text color.}
\\ [5pt]\pgfPTMbuildcellstyle{myd}(5,3)[(1;1-2;Z),(1;3;radio),(2-3;1.5-3.5;CS),(4;1-3;name),(5;1-3;d)]%
\pgfPTbuildcellstyle{myd}(5,3)[(1;1-2;Z),(1;3;radio),(2-3;1.5-3.5;CS),(4;1-3;name),(5;1-3;d)]%
\\ [-4pt]\pgfPTMmacrobox[l]{pgfPT}[Z list={1,...,36},cell style=myd,show title=false]%
\\ [5pt]\makebox[\linewidth][c]{\scalebox{.6}{\pgfPT[Z list={1,...,36},Z list={1,...,36},cell style=myd,show title=false]}}%
\\ [10pt]\pgfPTMmacrobox{pgfPT}[Z list={1,...,36},cell style=myd,show title=false,d color=red]%
\\ [5pt]\makebox[\linewidth][c]{\scalebox{.6}{\pgfPT[Z list={1,...,36},Z list={1,...,36},cell style=myd,show title=false,d color=red]}}%
\\ [0pt]\pgfPTendoption%
% d font
\pgfPTMoption{4}{d font}{\string\tiny\string\bfseries}{%
Sets the density value text font.
}
\\ [5pt]\pgfPTMmacrobox{pgfPT}[Z list={1,...,36},cell style=myd,show title=false]%
\\ [5pt]\makebox[\linewidth][c]{\scalebox{.6}{\pgfPT[Z list={1,...,36},Z list={1,...,36},cell style=myd,show title=false]}}%
\\ [10pt]\pgfPTMmacrobox{pgfPT}[Z list={1,...,36},cell style=myd,show title=false,d font=\string\tiny\string\itshape]%
\\ [5pt]\makebox[\linewidth][c]{\scalebox{.6}{\pgfPT[Z list={1,...,36},Z list={1,...,36},cell style=myd,show title=false,d font=\tiny\itshape]}}%
\\ [0pt]\pgfPTendoption%
% d unit=<g/cm3|g/dm3|both> .default=g/dm3
\vfill%
\pgfPTMoption[\pgfPTnewinversion{1.0.1}]{4}{d unit}{both}{%
Sets the unit for the density of the elements. The two possible values to this  key are \red{g/dm3} ($\mathsf{g/dm^3}$), \red{g/cm3} ($\mathsf{g/cm^3}$) and \red{both} ($\mathsf{g/dm^3}$ for elements in the gaseous state and $\mathsf{g/cm^3}$ for all other elements).
}%
\\ [5pt]\pgfPTMmacrobox{pgfPT}[Z list={1,...,36},cell style=myd,show title=false]%
\\ [10pt]\makebox[\linewidth][c]{\scalebox{.6}{\pgfPT[Z list={1,...,36},cell style=myd,show title=false]}}%
\\ [5pt]\pgfPTMmacrobox{pgfPT}[Z list={1,...,36},cell style=myd,show title=false,d unit=g/cm3]%
\\ [10pt]\makebox[\linewidth][c]{\scalebox{.6}{\pgfPT[Z list={1,...,36},cell style=myd,show title=false,d unit=g/cm3]}}%
\\ [5pt]\pgfPTMmacrobox{pgfPT}[Z list={1,...,36},cell style=myd,show title=false,d unit=g/dm3]%
\\ [10pt]\makebox[\linewidth][c]{\scalebox{.6}{\pgfPT[Z list={1,...,36},cell style=myd,show title=false,d unit=g/dm3]}}%
\\ [0pt]\pgfPTendoption%
\vfill\newpage\ \\ [-32pt]%
% d precision
\pgfPTMoption{4}{d precision}{-1}{%
Sets the density precision, \ie, the decimal places displayed in their value, performing the respective rounding, without zero padding the value.
\\ [10pt]\tikz{\node[text width=\linewidth-.6666em,fill=orange!5!white,draw=orange,rounded corners=2pt] {\textbf{\red{NOTE}}:\\ Rounding is performed over density values witch actually have a maximum 5 or 8 decimal places, when the values are in $\mathsf{g/dm^3}$ or in $\mathsf{g/cm^3}$, respectively.. So giving this key a value of -1 (the value of the melting or boiling point as-is) or 5 or 8 has the same effect.
\\ \textit{Therefore the values provided to this key should be any integer between -1 and 4 ($\mathsf{g/dm^3}$) or 7 ($\mathsf{g/cm^3}$). Any other integer provided will be processed as -1.}};}
}
\vfill%\\ [10pt]
\pgfPTMmacrobox[l]{pgfPTstyle}[Z list={1,...,54},cell style=myd,show title=false]%
\pgfPTstyle[Z list={1,...,54},cell style=myd,show title=false]%
\\ [-4pt]\pgfPTMmacrobox{pgfPT}[]%
\\ [10pt]\makebox[\linewidth][c]{\scalebox{.6}{\pgfPT}}%
\\ [10pt]\pgfPTMmacrobox{pgfPT}[d precision=0]%
\\ [10pt]\makebox[\linewidth][c]{\scalebox{.6}{\pgfPT[d precision=0]}}%
\\ [10pt]\pgfPTMmacrobox{pgfPT}[d precision=1]%
\\ [10pt]\makebox[\linewidth][c]{\scalebox{.6}{\pgfPT[d precision=1]}}%
\vfill%
\newpage%\\ [5pt]
\pgfPTMmacrobox{pgfPT}[d precision=2]%
\\ [10pt]\makebox[\linewidth][c]{\scalebox{.6}{\pgfPT[d precision=2]}}%
\\ [10pt]\pgfPTMmacrobox{pgfPT}[d precision=3]%
\\ [10pt]\makebox[\linewidth][c]{\scalebox{.6}{\pgfPT[d precision=3]}}%
\\ [10pt]\pgfPTMmacrobox{pgfPT}[d precision=4]%
\\ [10pt]\makebox[\linewidth][c]{\scalebox{.6}{\pgfPT[d precision=4]}}%
\\ [10pt]\pgfPTMmacrobox{pgfPT}[d precision=5]%
\\ [10pt]\makebox[\linewidth][c]{\scalebox{.6}{\pgfPT[d precision=5]}}%
\\ [0pt]\pgfPTendoption%
\newpage\ \\ [-32pt]%
% pseudo style -> d={c=??,f=??,p=??,u=??} ; d/.default={c=black,f=\tiny\bfseries,p=-1,u=both}%
%       d/.default={c=black,f=\tiny\bfseries,p=-1,u=both},%
\pgfPTMstyle{4}{d}{\{c=black,f=\string\tiny\string\bfseries,p=-1,u=both\}}%
{\ \\ [-3pt]\textit{Pseudo style} to set the keys: d \textbf{c}olor, d \textbf{f}ont, d \textbf{p}recision and/or d \textbf{u}nit.
None of the \textit{keys} -- c, f, p and u -- are mandatory.\hfill\textit{\textcolor{blue}{(\pgfPTnewinversion{1.0.1})}}
\\ [10pt]\makebox[\linewidth][c]{\use{\tikz{\node[text width=9cm] {d=\{c=<color>,f=<font commands>,p=<integer value>,u=<pm|A>\}};}}}%
}%
\\ [5pt]\pgfPTMmacrobox{pgfPT}[Z list={1,...,36},cell style=myd,show title=false,d={c=blue,p=2}]%
\\ [10pt]\makebox[\linewidth][c]{\scalebox{.6}{\pgfPT[Z list={1,...,36},cell style=myd,show title=false,d={c=blue,p=2}]}}%
\\ [0pt]\pgfPTendstyle%
%\newpage\ \\ [-32pt]%
\endinput
