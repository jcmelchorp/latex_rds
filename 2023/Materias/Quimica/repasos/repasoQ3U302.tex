\documentclass[12pt,addpoints]{repaso}
\grado{3}
\nivel{Secundaria}
\cicloescolar{2023-2024}
\materia{Ciencias y Tecnología: Química}
\unidad{3}
\title{Practica la Unidad}
\aprendizajes{
        \item Analiza el aporte energético de los alimentos y lo relaciona con las actividades físicas personales, a fin de tomar decisiones vinculadas a una dieta saludable. 
        \item Distingue las propiedades de ácidos y bases en su entorno, a partir de indicadores e interpreta la escala de acidez y basicidad. 
        \item Explica los factores que influyen en la rapidez de las reacciones químicas, con base en la identificación y control de variables mediante actividades experimentales y modelos corpusculares.
        \item Identifica reacciones de óxido-reducción en su entorno y comprende su importancia en diferentes ámbitos.  
}
\author{Melchor Pinto, J.C.}
\begin{document}
\INFO%
\ejemplosboxed[{
            \begin{multicols}{2}
                \begin{parts}
                    \part  \fillin[][0.5cm]
                    \begin{solutionbox}{2cm}
                    \end{solutionbox}
                    \part \fillin[][0.5cm]
                    \begin{solutionbox}{2cm}
                    \end{solutionbox}
                \end{parts}
            \end{multicols}
        }]

\begin{questions}
    \questionboxed[6]{Elige la respuesta correcta para cada una de las preguntas:

        ¿Qué nutriente puede usar el cuerpo humano como fuente de energía en casos extremos de desnutrición?

        Carbohidratos

        CorrectChoice Proteínas

        Grasas

        Lípidos


        ¿Por qué no es recomendable consumir grasas sólidas como la mantequilla?

        Participan en la generación de energía.

        Regulan la digestión de los alimentos.

        Aceleran la descalcificación de los huesos.

        CorrectChoice Pueden causar problemas cardíacos.


        ¿Qué tipo de proteínas se encargan de transportar sustancias dentro de las células o a través de las membranas celulares?

        Aminoácidos

        Monosacáridos

        CorrectChoice Enzimas

        Insaturadas


        ¿Cuál es el promedio de calorías que producen los carbohidratos por cada gramo que consume una persona?

        4000 Cal

        9000 Cal

        CorrectChoice 4 Cal

        9 Cal


        ¿Qué nutrimento se encuentra en menor concentración en el cuerpo humano?

        CorrectChoice Carbohidratos

        Grasas saturadas

        Proteínas

        Grasas insaturadas




        ¿Qué sucede con las proteínas cuando dejan de funcionar en el cuerpo humano?

        Se convierten en grasas para el almacenamiento de energía.

        CorrectChoice Se descomponen para liberar aminoácidos y formar nuevas proteínas.

        Se utilizan como fuente de energía para actividades diarias.

        Se excretan del cuerpo a través de la orina.


        ¿Cuál es la función de los ácidos grasos cuando se encuentran en exceso en el cuerpo humano?

        CorrectChoice Se utilizan para formar nuevas grasas que se almacenan y causan la obesidad.

        Se convierten en proteínas para el desarrollo muscular.

        Se excretan a través de la orina para eliminar el exceso de grasa.

        Se convierten en glucosa para mantener el nivel de azúcar en la sangre.


        ¿Qué carbohidrato cuesta ser digerido por nuestro organismo debido a su estructura de largas cadenas de glucosa?

        Fructosa

        Sacarosa

        Lactosa

        CorrectChoice Almidón


        ¿Qué nutrimento se encuentra en mayor concentración en el cuerpo humano?

        Carbohidratos

        Grasas saturadas

        CorrectChoice Proteínas

        Grasas insaturadas


        ¿Qué función cumplen las proteínas en el organismo de una persona?

        Son la principal fuente de energía.

        Almacenan información genética.

        Funcionan como aislantes térmicos.

        CorrectChoice Catalizan procesos metabólicos.


        ¿Qué biomoléculas sirven de soporte estructural en tejidos como el cabello?

        Carbohidratos

        Lípidos

        CorrectChoice Proteínas

        Ácidos nucleicos


        ¿Qué función cumplen los carbohidratos en el organismo de una persona?


        Almacenan información genética.

        Funcionan como aislantes térmicos.

        CorrectChoice Son la principal fuente de energía.

        Catalizan procesos metabólicos.


        ¿Qué biomoléculas funcionan como estructura de la membrana celular?

        Carbohidratos

        CorrectChoice Lípidos

        Proteínas

        Ácidos nucleicos
    }
\end{questions}
\end{document}
