\documentclass[12pt,addpoints]{repaso}
\grado{3}
\nivel{Secundaria}
\cicloescolar{2023-2024}
\materia{Ciencias y Tecnología: Química}
\unidad{3}
\title{Practica la Unidad}
\aprendizajes{
        \item Analiza el aporte energético de los alimentos y lo relaciona con las actividades físicas personales, a fin de tomar decisiones vinculadas a una dieta saludable. 
        \item Distingue las propiedades de ácidos y bases en su entorno, a partir de indicadores e interpreta la escala de acidez y basicidad. 
        \item Explica los factores que influyen en la rapidez de las reacciones químicas, con base en la identificación y control de variables mediante actividades experimentales y modelos corpusculares.
        \item Identifica reacciones de óxido-reducción en su entorno y comprende su importancia en diferentes ámbitos.  
}
\author{Melchor Pinto, J.C.}
\begin{document}
\INFO
\begin{questions}
    \questionboxed[6]{Señala si son verdaderas o falsas las siguientes afirmaciones.

        \begin{multicols}{2}
            \begin{parts}
                \part La mayoría de las medicinas se absorben en el estómago o el intestino y se distribuyen por la sangre.

                \begin{oneparchoices}
                    \CorrectChoice Verdadero
                    \choice Falso
                \end{oneparchoices}

                \part La velocidad de las reacciones metabólicas de las medicinas siempre es constante.
                \part La vida media de un medicamento corresponde con el tiempo necesario para que su concentración en el cuerpo se reduzca a la mitad.
                \part La eliminación de medicamentos en el medio ambiente solo ocurre a través de la orina y las heces.
                \part Los medicamentos que se desechan en el medio ambiente pueden alterar el ciclo de reproducción de los peces.
                \part Es recomendable evitar el sobreconsumo de medicamentos para reducir la liberación de desechos en el medio ambiente.
                \part En el diseño de fármacos se estudia la rapidez con que tarda en hacer efecto un nuevo medicamento.
                \part La forma en que el organismo absorbe, metaboliza y elimina un fármaco depende de la rapidez del proceso.
                \part La fecha de caducidad que aparece en un medicamento es más lejana que la determinada en los ensayos.
                \part Los sitios donde se almacenan diversos tipos de fármacos no intervienen en sus procesos de degradación.
                \part Las altas temperaturas aceleran la rapidez con que se descomponen los medicamentos.
                \part La energía cinética de una partícula debe ser mayor que la energía de activación para reaccionar tras el choque.
                \part La energía de activación se describe como una barrera que las partículas deben saltar para reaccionar.
                \part Los procesos con una energía de activación muy alta a temperatura ambiente son muy rápidos.
                \part Los procesos con energías de activación muy bajas no requieren de una fuente de calor para llevarse a cabo.
                \part La energía de activación es la energía necesaria para concluir un proceso químico.
                \part Para que una reacción química disminuya el tiempo en que se lleva a cabo es necesario mantener su energía inicial.
                \part La rapidez de reacción cambia al modificar ciertos factores como la concentración de los reactivos.
                \part Disminuir la temperatura de una reacción permite que el proceso ocurra miles de veces más rápido.
                \part La rapidez de reacción es menor cuando las sustancias en estado sólido se encuentran pulverizadas.
                \part El uso de combustibles alternativos ayudará a reducir el impacto ambiental de los vehículos eléctricos.
                \part La expansión del uso de vehículos eléctricos permitirá alcanzar las metas mundiales para la reducción de emisiones de sustancias contaminantes.
                \part Todas las baterías que se usan en vehículos eléctricos funcionan gracias a las reacciones de óxido-reducción en su interior.
                \part Las baterías plomo-ácido se utilizan únicamente en autos eléctricos para proporcionar energía suplementaria.
                \part Las baterías ion-litio son exclusivas para vehículos eléctricos y no se encuentran en otros productos electrónicos.
                \part Todas las partes de las baterías ion-litio son reciclables, lo que hace que el reciclaje sea económico.
                \part Las baterías níquel-hidruro metálico sólo se utilizan en autos híbridos y no en otros dispositivos electrónicos.
                % \part Las baterías plomo-ácido funcionan por medio de la oxidación de plomo metálico y la reducción de óxido de plomo en medio ácido.
                % \part Las baterías ion-litio funcionan a través de la oxidación y la reducción de átomos de litio.
                % \part Durante las reacciones de óxido-reducción, los números de oxidación de los elementos participantes permanecen constantes.
                % \part El sodio se oxida cuando su número de oxidación aumenta.
                % \part En la reacción de combinación para obtener cloruro de sodio, a partir de sodio y cloro, el cloro se reduce.
                % \part Las reacciones de síntesis no se consideran reacciones de óxido-reducción.
                % \part Cuando los átomos metálicos ganan electrones, se reducen en la reacción redox.
                % \part La síntesis de cobre metálico se logra a partir de aluminio metálico y cloruro de cobre (II).
                % \part Cada ion de cobre (II) requiere tres electrones para reducirse en el proceso.
                % \part Las sales tienen múltiples aplicaciones en diferentes industrias.
                % \part El carbonato de calcio se utiliza en la producción de vidrio y cemento, pero no como complemento alimentario.
                % \part Los pigmentos utilizados en la fabricación de pinturas son principalmente sales iónicas.
                % \part Los pigmentos se combinan con sustancias químicas como aceites y aglutinantes para que el color se adhiera a las superficies.
                % \part Las sales no se utilizan como espesantes, desecantes o desinfectantes.
                % \part Antes, la mayoría de las sustancias utilizadas en la producción de fertilizantes se extraían de depósitos naturales.
                % \part Actualmente, las sustancias utilizadas en la producción de fertilizantes se producen solo por reacciones ácido-base.
                % \part Según datos recientes, nuestro país ocupa el primer lugar en obesidad infantil a nivel mundial.
                % \part Si una persona posee un metabolismo basal bajo requiere mucha energía para sobrevivir y tiende a perder peso con facilidad.
                % \part Algunas actividades físicas de nivel bajo son: jugar basquetbol, futbol, correr, nadar y andar en bicicleta.
                % \part El metabolismo basal es la rapidez con la que el cuerpo consume energía para realizar sus funciones vitales.
                % \part Los hábitos alimentarios alrededor del mundo no dependen de la historia, la cultura y geografía de cada lugar.
                % \part La cantidad de energía que una persona necesita para sobrevivir y realizar sus actividades diarias es independiente de su edad, genero y actividad física.
                % \part Mariana realiza actividades físicas de nivel alto, ya que diariamente recibe entrenamiento de atletismo, y además practica voleibol y basquetbol.
                % \part Óscar requiere diariamente de un aporte calórico alto, porque trabaja en su oficina ocho horas diarias y en sus ratos de ocio acostumbra ver la televisión.
                % \part Una persona que tiene un metabolismo basal alto, requiere mayor energía para sobrevivir.
                % \part Las personas que habitan en climas fríos necesitan más energía para mantener la temperatura corporal que quienes habitan en climas templados.
                % \part Una dieta correcta contendrá todos los nutrimentos en proporciones apropiadas, no será un riesgo para la salud, cubrirá las necesidades nutrimentales de la persona y estará acorde con la cultura de quienes la consumen.
                % \part Germán es un estudiante de 17 años que realiza actividades físicas de nivel bajo, ya que no practica ningún deporte y sus pasatiempos consisten en ver televisión y dormir, por lo que su aporte energético es bajo.
                % \part María tiene 14 años y pesa 40 kg, requiere un aporte energético bajo ya que diariamente realiza actividades como nadar, jugar tenis y asistir a sus clases de baile.
                % \part El sobrepeso y la obesidad son padecimientos que pueden generarse cuando un individuo ingiere más calorías de las que gasta en sus actividades físicas y ésta se acumula en el cuerpo en forma de lípidos.
                % \part La cantidad de energía que una persona requiere sólo depende de factores hereditarios y no de sus características partículares.
                % \part La cantidad de energía que tu cuerpo necesita depende únicamente de tu edad y género.
                % \part El metabolismo basal es responsable del consumo de 70\% de las calorías que requiere tu cuerpo.
                % \part La energía requerida por el cuerpo se obtiene a través de reacciones químicas que forman parte del sistema digestivo.
                % \part El metabolismo basal es la cantidad de energía que se consume mientras el cuerpo está en reposo.
                % \part Si una persona no consume suficiente energía, se generan sustancias que aceleran el metabolismo basal.
                % \part Las personas con mayor masa muscular suelen tener un metabolismo basal más lento.
                % \part Las personas con mayor cantidad de grasa corporal suelen tener un metabolismo basal más alto.
                % \part Las actividades físicas con mayor intensidad requieren menos energía que las de menor intensidad.
                % \part Durante la época novohispana, la comida prehispánica mezcló técnicas culinarias con la comida española.
                % \part Muchos alimentos como el maíz, los frijoles, el chile, el jitomate y la cebolla son aportes de la diversidad alimentaria europea.
                % \part En los estados que se ubican en el sur del país la dieta se basa en la flora y fauna comestible de las zonas desérticas.
            \end{parts}
        \end{multicols}
    }

    \questionboxed[6]{Señala si son verdaderas o falsas las siguientes afirmaciones.

        \begin{multicols}{2}
            \begin{parts}
                \part Las baterías plomo-ácido funcionan por medio de la oxidación de plomo metálico y la reducción de óxido de plomo en medio ácido.

                \begin{oneparchoices}
                    \CorrectChoice Verdadero
                    \choice Falso
                \end{oneparchoices} \part Las baterías ion-litio funcionan a través de la oxidación y la reducción de átomos de litio.
                \part Durante las reacciones de óxido-reducción, los números de oxidación de los elementos participantes permanecen constantes.
                \part El sodio se oxida cuando su número de oxidación aumenta.
                \part En la reacción de combinación para obtener cloruro de sodio, a partir de sodio y cloro, el cloro se reduce.
                \part Las reacciones de síntesis no se consideran reacciones de óxido-reducción.
                \part Cuando los átomos metálicos ganan electrones, se reducen en la reacción redox.
                \part La síntesis de cobre metálico se logra a partir de aluminio metálico y cloruro de cobre (II).
                \part Cada ion de cobre (II) requiere tres electrones para reducirse en el proceso.
                \part Las sales tienen múltiples aplicaciones en diferentes industrias.
                \part El carbonato de calcio se utiliza en la producción de vidrio y cemento, pero no como complemento alimentario.
                \part Los pigmentos utilizados en la fabricación de pinturas son principalmente sales iónicas.
                \part Los pigmentos se combinan con sustancias químicas como aceites y aglutinantes para que el color se adhiera a las superficies.
                \part Las sales no se utilizan como espesantes, desecantes o desinfectantes.
                \part Antes, la mayoría de las sustancias utilizadas en la producción de fertilizantes se extraían de depósitos naturales.
                \part Actualmente, las sustancias utilizadas en la producción de fertilizantes se producen solo por reacciones ácido-base.
                \part Según datos recientes, nuestro país ocupa el primer lugar en obesidad infantil a nivel mundial.
                \part Si una persona posee un metabolismo basal bajo requiere mucha energía para sobrevivir y tiende a perder peso con facilidad.
                \part Algunas actividades físicas de nivel bajo son: jugar basquetbol, futbol, correr, nadar y andar en bicicleta.
                \part El metabolismo basal es la rapidez con la que el cuerpo consume energía para realizar sus funciones vitales.
                \part Los hábitos alimentarios alrededor del mundo no dependen de la historia, la cultura y geografía de cada lugar.
                \part La cantidad de energía que una persona necesita para sobrevivir y realizar sus actividades diarias es independiente de su edad, genero y actividad física.
                \part Mariana realiza actividades físicas de nivel alto, ya que diariamente recibe entrenamiento de atletismo, y además practica voleibol y basquetbol.
                \part Óscar requiere diariamente de un aporte calórico alto, porque trabaja en su oficina ocho horas diarias y en sus ratos de ocio acostumbra ver la televisión.
                \part Una persona que tiene un metabolismo basal alto, requiere mayor energía para sobrevivir.
                \part Las personas que habitan en climas fríos necesitan más energía para mantener la temperatura corporal que quienes habitan en climas templados.
                \part Una dieta correcta contendrá todos los nutrimentos en proporciones apropiadas, no será un riesgo para la salud, cubrirá las necesidades nutrimentales de la persona y estará acorde con la cultura de quienes la consumen.
                % \part Germán es un estudiante de 17 años que realiza actividades físicas de nivel bajo, ya que no practica ningún deporte y sus pasatiempos consisten en ver televisión y dormir, por lo que su aporte energético es bajo.
                % \part María tiene 14 años y pesa 40 kg, requiere un aporte energético bajo ya que diariamente realiza actividades como nadar, jugar tenis y asistir a sus clases de baile.
                % \part El sobrepeso y la obesidad son padecimientos que pueden generarse cuando un individuo ingiere más calorías de las que gasta en sus actividades físicas y ésta se acumula en el cuerpo en forma de lípidos.
                % \part La cantidad de energía que una persona requiere sólo depende de factores hereditarios y no de sus características partículares.
                % \part La cantidad de energía que tu cuerpo necesita depende únicamente de tu edad y género.
                % \part El metabolismo basal es responsable del consumo de 70\% de las calorías que requiere tu cuerpo.
                % \part La energía requerida por el cuerpo se obtiene a través de reacciones químicas que forman parte del sistema digestivo.
                % \part El metabolismo basal es la cantidad de energía que se consume mientras el cuerpo está en reposo.
                % \part Si una persona no consume suficiente energía, se generan sustancias que aceleran el metabolismo basal.
                % \part Las personas con mayor masa muscular suelen tener un metabolismo basal más lento.
                % \part Las personas con mayor cantidad de grasa corporal suelen tener un metabolismo basal más alto.
                % \part Las actividades físicas con mayor intensidad requieren menos energía que las de menor intensidad.
                % \part Durante la época novohispana, la comida prehispánica mezcló técnicas culinarias con la comida española.
                % \part Muchos alimentos como el maíz, los frijoles, el chile, el jitomate y la cebolla son aportes de la diversidad alimentaria europea.
                % \part En los estados que se ubican en el sur del país la dieta se basa en la flora y fauna comestible de las zonas desérticas.
            \end{parts}
        \end{multicols}
    }
    \questionboxed[6]{Señala si son verdaderas o falsas las siguientes afirmaciones.

        \begin{multicols}{2}
            \begin{parts}
                \part Germán es un estudiante de 17 años que realiza actividades físicas de nivel bajo, ya que no practica ningún deporte y sus pasatiempos consisten en ver televisión y dormir, por lo que su aporte energético es bajo.
                \part María tiene 14 años y pesa 40 kg, requiere un aporte energético bajo ya que diariamente realiza actividades como nadar, jugar tenis y asistir a sus clases de baile.
                \part El sobrepeso y la obesidad son padecimientos que pueden generarse cuando un individuo ingiere más calorías de las que gasta en sus actividades físicas y ésta se acumula en el cuerpo en forma de lípidos.
                \part La cantidad de energía que una persona requiere sólo depende de factores hereditarios y no de sus características partículares.
                \part La cantidad de energía que tu cuerpo necesita depende únicamente de tu edad y género.
                \part El metabolismo basal es responsable del consumo de 70\% de las calorías que requiere tu cuerpo.
                \part La energía requerida por el cuerpo se obtiene a través de reacciones químicas que forman parte del sistema digestivo.
                \part El metabolismo basal es la cantidad de energía que se consume mientras el cuerpo está en reposo.
                \part Si una persona no consume suficiente energía, se generan sustancias que aceleran el metabolismo basal.
                \part Las personas con mayor masa muscular suelen tener un metabolismo basal más lento.
                \part Las personas con mayor cantidad de grasa corporal suelen tener un metabolismo basal más alto.
                \part Las actividades físicas con mayor intensidad requieren menos energía que las de menor intensidad.
                \part Durante la época novohispana, la comida prehispánica mezcló técnicas culinarias con la comida española.
                \part Muchos alimentos como el maíz, los frijoles, el chile, el jitomate y la cebolla son aportes de la diversidad alimentaria europea.
                \part En los estados que se ubican en el sur del país la dieta se basa en la flora y fauna comestible de las zonas desérticas.
            \end{parts}
        \end{multicols}
    }

    \questionboxed[6]{Observa la imagen a continuación y elige la respuesta correcta:

        \includegraphics[width=\textwidth]{SINQU_U2_AC72_IMG1.png}

        \begin{multicols}{2}
            \begin{parts}
                \part El bicarbonato de sodio es una sustancia:

                \begin{oneparchoices}
                    \choice Básica
                    \choice Neutra
                    \choice Ácida
                    \choice Concentrada
                \end{oneparchoices}

                \part Ejemplos de sustancias ácidas.

                \begin{oneparchoices}
                    \choice Agua de mar, bicarbonato de sodio y depilador Ácido estomacal, amoniaco y depilador Agua pura, leche y sangre Ácido de batería, uvas y café negro
                    \choice
                    \choice
                    \choice
                \end{oneparchoices}


                \part ¿Qué pH tiene la sustancia que ayuda a contrarrestar la acidez estomacal?

                \begin{oneparchoices}
                    \choice pH = 10 pH = 14 pH = 2 pH = 7
                    \choice
                    \choice
                    \choice
                \end{oneparchoices}


                \part Producto de mayor acidez que el agua de lluvia normal.

                \begin{oneparchoices}
                    \choice Leche Agua pura Agua de mar Tomates
                    \choice
                    \choice
                    \choice
                \end{oneparchoices}


                \part Producto con menor carácter ácido que las uvas.

                \begin{oneparchoices}
                    \choice Refrescos Ácido estomacal Jugo de limón Café negro
                    \choice
                    \choice
                    \choice
                \end{oneparchoices}


                \part ¿Cuál de las siguientes sustancias tiene propiedades básicas?

                \begin{oneparchoices}
                    \choice Depilador Leche Agua de lluvia Café negro
                    \choice
                    \choice
                    \choice
                \end{oneparchoices}


                \part ¿Qué sustancia es más básica que la sangre?

                \begin{oneparchoices}
                    \choice Bicarbonato de sodio Agua pura Leche Tomates
                    \choice
                    \choice
                    \choice
                \end{oneparchoices}


                \part ¿Cuál de los siguientes es un ejemplo de una sustancia con un pH neutro?

                \begin{oneparchoices}
                    \choice Agua pura Amoniaco de casa Disoluciones antiácidas Limpiador de tuberías
                    \choice
                    \choice
                    \choice
                \end{oneparchoices}

                \part ¿Qué sustancia es más ácida que el jugo de limón?

                \begin{oneparchoices}
                    \choice Bicarbonato de sodio Ácido estomacal Refrescos Amoniaco de casa
                    \choice
                    \choice
                    \choice
                \end{oneparchoices}


                \part Es una sustancia ligeramente básica.

                \begin{oneparchoices}
                    \choice Limpiador de tuberías Agua pura Sangre Leche
                    \choice
                    \choice
                    \choice
                \end{oneparchoices}


                \part Producto de mayor basicidad en la escala.

                \begin{oneparchoices}
                    \choice Amoniaco de casa Depilador Limpiador de tuberías Ácido de batería
                    \choice
                    \choice
                    \choice
                \end{oneparchoices}


                \part Ejemplo de sustancia ligeramente ácida.

                \begin{oneparchoices}
                    \choice Agua pura Leche Sangre Ácido de batería
                    \choice
                    \choice
                    \choice
                \end{oneparchoices}


                \part Producto con menor carácter básico que las disoluciones antiácidas.

                \begin{oneparchoices}
                    \choice Amoniaco de casa Limpiador de tuberías Depilador Agua de Mar
                    \choice
                    \choice
                    \choice
                \end{oneparchoices}


                \part El agua pura es una sustancia:

                \begin{oneparchoices}
                    \choice neutra ligeramente ácida básica ácida.
                    \choice
                    \choice
                    \choice
                \end{oneparchoices}


                \part ¿Qué valor de pH se considera neutro?

                \begin{oneparchoices}
                    \choice pH = 7 pH = 0 pH = 14 pH = 8
                    \choice
                    \choice
                    \choice
                \end{oneparchoices}

            \end{parts}
        \end{multicols}
    }

    \questionboxed[6]{Completa la tabla colocando el nombre y la fórmula para cada sustancia o producto que usamos en la vida cotidiana.

        \centering
        \ifprintanswers{
            \includegraphics[width=0.9\textwidth]{SINQU_U2_AC70_IMG1_sol.png}
        }\else{
            \includegraphics[width=0.9\textwidth]{SINQU_U2_AC70_IMG1.png}
        }\fi
    }

    \questionboxed[6]{Completa la tabla colocando los datos de cada columna.

        \centering
        \ifprintanswers{
            \includegraphics[width=0.9\textwidth]{SINQU_U2_AC72_IMG02_sol.png}
        }\else{
            \includegraphics[width=0.9\textwidth]{SINQU_U2_AC72_IMG02.png}
        }\fi
    }

    \questionboxed[6]{Completa la tabla colocando los datos de cada columna.

        \centering
        \ifprintanswers{
            \includegraphics[width=0.9\textwidth]{SINQU_U2_AC72_IMG01_sol.png}
        }\else{
            \includegraphics[width=0.9\textwidth]{SINQU_U2_AC72_IMG01.png}
        }\fi
    }

    \questionboxed[6]{Analiza la ecuación química y elige la respuesta en cada pregunta.

        \ce{CO2 + H2O → C6H12O6 + O2}


        \begin{multicols}{2}
            \begin{parts}
                \part ¿Cuáles son los reactivos de la ecuación anterior?

                \begin{oneparchoices}
                    \choice \ce{CO2} y \ce{H2O}
                    \choice \ce{H2O} y \ce{O2}
                    \choice \ce{C6H12O6} y \ce{O2}
                    \choice \ce{CO2} y \ce{O2}
                \end{oneparchoices}

                \part El coeficiente asociado a los reactivos que balancea correctamente la reacción de la fotosíntesis es...

                \begin{oneparchoices}
                    \choice 12
                    \choice 3
                    \choice 2
                    \choice 6
                \end{oneparchoices}

                \part La reacción de fotosíntesis es un proceso de óxido-reducción. ¿Qué especie se reduce?

                \begin{oneparchoices}
                    \choice El H2O para formar parte del O2
                    \choice La molécula de C6H12O6
                    \choice La molécula de O
                    \choice El CO2 para formar C6H12O6
                \end{oneparchoices}

                \part El número de oxidación del átomo de oxígeno en la molécula de agua (\ce{H_2O} es 2 y en la molécula de oxígeno (\ce{O_2}) es cero. ¿Qué proceso se llevó a cabo?

                \begin{oneparchoices}
                    \choice Neutralización
                    \choice Oxidación
                    \choice Precipitación
                    \choice Reducción
                \end{oneparchoices}

                \part ¿Cuál es el número de oxidación del átomo de oxígeno en la molécula de dióxido de carbono (\ce{CO_2})

                \begin{oneparchoices}
                    \choice -1
                    \choice 0
                    \choice 1
                    \choice 2
                \end{oneparchoices}

                \part ¿Cuáles son los productos de la ecuación anterior?

                \begin{oneparchoices}
                    \choice \ce{CO2} y \ce{H2O}
                    \choice \ce{C6H12O6} y \ce{O2}
                    \choice \ce{C6H12} y \ce{O2}
                    \choice \ce{H2O} y \ce{O2}
                \end{oneparchoices}

                \part El coeficiente asociado a la molécula dióxido de carbono (\ce{CO_2}) que balancea correctamente la reacción de fotosíntesis es…

                \begin{oneparchoices}
                    \choice 2
                    \choice 3
                    \choice 6
                    \choice 9
                \end{oneparchoices}

                \part ¿Cuál es el número de oxidación del átomo de hidrógeno en la molécula de agua (\ce{H_2O})

                \begin{oneparchoices}
                    \choice -1
                    \choice 0
                    \choice 1
                    \choice 2
                \end{oneparchoices}

                \part ¿Cuál es el número de oxidación del átomo de carbono en la molécula de dióxido de carbono (\ce{CO_2})

                \begin{oneparchoices}
                    \choice 0
                    \choice +1
                    \choice +2
                    \choice +4
                \end{oneparchoices}

                \part La reacción de fotosíntesis es un proceso de óxido-reducción. ¿Qué especie se oxida?

                \begin{oneparchoices}
                    \choice La molécula de \ce{C6H12O6}
                    \choice El átomo de oxígeno de \ce{H2O} para formar parte del \ce{O2}
                    \choice El átomo de oxígeno de \ce{O2} para formar parte del \ce{CO2}
                    \choice La molécula de \ce{O2}
                \end{oneparchoices}


            \end{parts}
        \end{multicols}
    }

    \questionboxed[6]{Analiza la ecuación química y elige la respuesta en cada pregunta.

        \ce{C6H12O6 + O2 → CO2 + H2O}


        \begin{multicols}{2}
            \begin{parts}
                \part ¿Cuáles son los reactivos y cuáles los productos?

                \begin{oneparchoices}
                    \choice Reactivos: \ce{CO2} y \ce{H2O}; productos: \ce{C6H12O6} y \ce{O2}
                    \choice Reactivos: \ce{C6H12O6} y \ce{CO2}; productos: \ce{O2} y \ce{H2O}
                    \choice Reactivos: \ce{C6H12O6} y \ce{O2}; productos: \ce{CO2} y \ce{H2O}
                    \choice Reactivos: \ce{CO2} y \ce{O2}; productos: \ce{C6H12O6} y \ce{H2O}
                \end{oneparchoices}

                \part Son los coeficientes que balancean correctamente la reacción de respiración.

                \begin{oneparchoices}
                    \choice 2 y 2
                    \choice 4 y 2
                    \choice 3 y 2
                    \choice 6 y 6
                \end{oneparchoices}

                \part ¿Cuál es el tipo de enlace que describe a la molécula de CO2?

                \begin{oneparchoices}
                    \choice Iónico
                    \choice Covalente puro
                    \choice Metálico
                    \choice Covalente polar
                \end{oneparchoices}

                \part La reacción de respiración es un proceso de óxido-reducción. ¿Qué especie se reduce?

                \begin{oneparchoices}
                    \choice Los átomos de la molécula de O2 para formar parte del H2O
                    \choice La molécula de H2O
                    \choice El átomo de oxígeno de la molécula de H2O para formar parte del O2
                    \choice La molécula de CO2
                \end{oneparchoices}

                \part La molécula de glucosa (C6H12O6) se oxida para conformar la molécula de dióxido de carbono CO2; por lo tanto, éste se considera el agente:

                \begin{oneparchoices}
                    \choice Reductor
                    \choice Electrolito
                    \choice Oxidante
                    \choice Básica
                \end{oneparchoices}

            \end{parts}
        \end{multicols}
    }

    % \questionboxed[6]{Elige la respuesta correcta para cada una de las preguntas:

    %     ¿Qué nutriente puede usar el cuerpo humano como fuente de energía en casos extremos de desnutrición?

    %     Carbohidratos

    %     CorrectChoice Proteínas

    %     Grasas

    %     Lípidos


    %     ¿Por qué no es recomendable consumir grasas sólidas como la mantequilla?

    %     Participan en la generación de energía.

    %     Regulan la digestión de los alimentos.

    %     Aceleran la descalcificación de los huesos.

    %     CorrectChoice Pueden causar problemas cardíacos.


    %     ¿Qué tipo de proteínas se encargan de transportar sustancias dentro de las células o a través de las membranas celulares?

    %     Aminoácidos

    %     Monosacáridos

    %     CorrectChoice Enzimas

    %     Insaturadas


    %     ¿Cuál es el promedio de calorías que producen los carbohidratos por cada gramo que consume una persona?

    %     4000 Cal

    %     9000 Cal

    %     CorrectChoice 4 Cal

    %     9 Cal


    %     ¿Qué nutrimento se encuentra en menor concentración en el cuerpo humano?

    %     CorrectChoice Carbohidratos

    %     Grasas saturadas

    %     Proteínas

    %     Grasas insaturadas




    %     ¿Qué sucede con las proteínas cuando dejan de funcionar en el cuerpo humano?

    %     Se convierten en grasas para el almacenamiento de energía.

    %     CorrectChoice Se descomponen para liberar aminoácidos y formar nuevas proteínas.

    %     Se utilizan como fuente de energía para actividades diarias.

    %     Se excretan del cuerpo a través de la orina.


    %     ¿Cuál es la función de los ácidos grasos cuando se encuentran en exceso en el cuerpo humano?

    %     CorrectChoice Se utilizan para formar nuevas grasas que se almacenan y causan la obesidad.

    %     Se convierten en proteínas para el desarrollo muscular.

    %     Se excretan a través de la orina para eliminar el exceso de grasa.

    %     Se convierten en glucosa para mantener el nivel de azúcar en la sangre.


    %     ¿Qué carbohidrato cuesta ser digerido por nuestro organismo debido a su estructura de largas cadenas de glucosa?

    %     Fructosa

    %     Sacarosa

    %     Lactosa

    %     CorrectChoice Almidón


    %     ¿Qué nutrimento se encuentra en mayor concentración en el cuerpo humano?

    %     Carbohidratos

    %     Grasas saturadas

    %     CorrectChoice Proteínas

    %     Grasas insaturadas


    %     ¿Qué función cumplen las proteínas en el organismo de una persona?

    %     Son la principal fuente de energía.

    %     Almacenan información genética.

    %     Funcionan como aislantes térmicos.

    %     CorrectChoice Catalizan procesos metabólicos.


    %     ¿Qué biomoléculas sirven de soporte estructural en tejidos como el cabello?

    %     Carbohidratos

    %     Lípidos

    %     CorrectChoice Proteínas

    %     Ácidos nucleicos


    %     ¿Qué función cumplen los carbohidratos en el organismo de una persona?


    %     Almacenan información genética.

    %     Funcionan como aislantes térmicos.

    %     CorrectChoice Son la principal fuente de energía.

    %     Catalizan procesos metabólicos.


    %     ¿Qué biomoléculas funcionan como estructura de la membrana celular?

    %     Carbohidratos

    %     CorrectChoice Lípidos

    %     Proteínas

    %     Ácidos nucleicos
    % }
\end{questions}
\end{document}
