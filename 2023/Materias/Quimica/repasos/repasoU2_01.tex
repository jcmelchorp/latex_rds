\documentclass[12pt]{repaso}
\grade{3$^\circ$ de Secundaria}
\cycle{2022-2023}
\subject{Química 3}
\guide{2}
\title{Repaso para el examen de la Unidad}
\aprendizajes{

    \begin{itemize}[leftmargin=*,label=\small\color{colorrds}\faIcon{user-graduate}
        ]
        \item Deduce información acerca de la estructura atómica
        a partir de datos experimentales sobre propiedades
        atómicas periódicas.
  \item Representa y diferencia mediante esquemas, modelos y
        simbología química, elementos y compuestos, así como
        átomos y moléculas.
  \item Explica y predice propiedades físicas de los materiales
        con base en modelos submicroscópicos sobre la
        estructura de átomos, moléculas o iones, y sus
        interacciones electrostáticas.
    \end{itemize}
}
\requisitos{
    \begin{itemize}
        \item Requisito 1
        \item Requisito 2
    \end{itemize}
}
\author{J. C. Melchor Pinto}

\begin{document}
\pagestyle{headandfoot}
%\thispagestyle{plain}
\addpoints
\INFO
%\printanswers
\begin{multicols}{2}
    \include*{../blocks/block001}
    \include*{../blocks/block003}
    \include*{../blocks/block000}
    \include*{../blocks/block002}
\end{multicols}
\newpage
\include*{../blocks/block004}
\begin{questions}
    % {
    %     \printanswers
    % \include*{../questions/question001}
    % \include*{../questions/question002}
    % \include*{../questions/question003}
    \include*{../questions/question004}
    \include*{../questions/question005}
    \include*{../questions/question006}
    \include*{../questions/question007}
    \include*{../questions/question008}
    \include*{../questions/question009}
    \include*{../questions/question010}
    \include*{../questions/question011}
    \include*{../questions/question012}
    \include*{../questions/question013}
    \include*{../questions/question014}
    \include*{../questions/question015}

    \setbohr{
        shell-options-add = dashed,
        shell-dist = .5em,
        insert-missing
    }

    \bohr{0}{H+} \bohr{10}{F-}

    \newcommand{\Bond}[6]%
    % start, end, thickness, incolor, outcolor, iterations
    { \begin{pgfonlayer}{background}
            \colorlet{InColor}{#4}
            \colorlet{OutColor}{#5}
            \foreach \I in {#6,...,1}
                {   \pgfmathsetlengthmacro{\r}{#3/#6*\I}
                    \pgfmathsetmacro{\C}{sqrt(1-\r*\r/#3/#3)*100}
                    \draw[InColor!\C!OutColor, line width=\r] (#1.center) -- (#2.center);
                }
        \end{pgfonlayer}
    }

    \newcommand{\SingleBond}[2]%
    % start, end
    {   \Bond{#1}{#2}{1mm}{white}{gray!50}{10}
    }

    \newcommand{\RedBond}[2]%
    % start, end
    {   \Bond{#1}{#2}{1mm}{orange!70}{red!75!black}{10}
    }

    \newcommand{\BlueBond}[2]%
    % start, end
    {   \Bond{#1}{#2}{2mm}{cyan}{blue!50!black}{10}
    }

    \newcommand{\GreenBond}[2]%
    % start, end
    {   \Bond{#1}{#2}{0.7071mm}{green!25}{green!25!black}{10}
    }

    % \begin{tikzpicture}
    %     [   oxygen/.style={circle, ball color=red, minimum size=6mm, inner sep=0},
    %         hydrogen/.style={circle, ball color=white, minimum size=2.5mm, inner sep=0},
    %         carbon/.style={circle, ball color=black!75, minimum size=7mm, inner sep=0}
    %     ]
    %     \node[oxygen] (O1) at (0,0) {};
    %     \node[hydrogen] (H1) at (1,-0.5) {};
    %     \node[hydrogen] (H2) at (-1,-0.5) {};
    %     \SingleBond{O1}{H1}
    %     \SingleBond{O1}{H2}
    %     % \foreach \c in {1,...,6}
    %     %     {   \node[carbon] (C-\c) at ($(2,3)+(\c*60:1.5)$) {};
    %     %         \node[hydrogen] (H-\c-a) at ($(2,3)+(\c*60-15:2.5)$) {};
    %     %         \node[hydrogen] (H-\c-b) at ($(2,3)+(\c*60+15:2.5)$) {};
    %     %         \RedBond{C-\c}{H-\c-a}
    %     %         \BlueBond{C-\c}{H-\c-b}
    %     %     }
    %     % \foreach \c [evaluate={\c as \n using int(mod(\c,6)+1)}] in {1,...,6}
    %     %     {   \GreenBond{C-\c}{C-\n}
    %     %     }
    % \end{tikzpicture}
    % Methane: \chemfig{C(-[:0]H)(-[:90]H)(-[:180]H)(-[:270]H)}
    % Water: \chemfig{H-[:30]O-[:-30]H}
    % Hydrogen:  \chemfig{H-H} \\
    % Oxygen:    \chemfig{O=O} \\
    % Ethyne:    \chemfig{H-C~C-H}


    \pgfPT[Z exercise list={1,2,3,4,9,12,17,18,19,20,25,27,32,34,35,49,54,74,86,87},
        cell size=3em,Z list={1,...,36},exColor=red!50!black]
    % }
    % \end{parts}
\end{questions}
\end{document}