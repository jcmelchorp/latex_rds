\documentclass[12pt]{evalua}
\grado{3$^\circ$ de Secundaria}
\cicloescolar{2022-2023}
\materia{Química 3}
\guide{2}
\title{Examen de la Unidad}
\aprendizajes{
    \begin{itemize}[leftmargin=*,label=\small\color{colorrds}\faIcon{user-graduate}
        ]
        \item Deduce información acerca de la estructura atómica
              a partir de datos experimentales sobre propiedades
              atómicas periódicas.
        \item Representa y diferencia mediante esquemas, modelos y
              simbología química, elementos y compuestos, así como
              átomos y moléculas.
        \item Explica y predice propiedades físicas de los materiales
              con base en modelos submicroscópicos sobre la
              estructura de átomos, moléculas o iones, y sus
              interacciones electrostáticas.
    \end{itemize}
}

\author{Prof.: Julio César Melchor Pinto}
% \documentclass[tikz,border=40mm]{standalone}
% \usetikzlibrary{shapes,calc}
\usepackage{anyfontsize}
%\usepackage{mathptmx}
%\begin{document}
\newcommand{\ElemLabel}[4]{
    \begin{minipage}{2.2cm}
        \centering
        {#1 \hfill #2}\\[1ex]
        {\textbf{#3}}%
        \linebreak
        {{#4}}
    \end{minipage}
}

\newcommand{\NaturalElem}[4]{\ElemLabel{#1}{#2}{{\fontsize{36}{45}\selectfont #3}}{{\fontsize{12}{15}\selectfont \sffamily#4}}}

\newcommand{\SyntheticElem}[4]{\ElemLabel{#1}{#2}{\color{darkgray}{\fontsize{36}{45}\selectfont #3}}{{\fontsize{12}{15}\selectfont \sffamily#4}}}

\newcommand{\TablaPeriodica}[1][0.55]{
    \begin{tikzpicture}[font=\normalfont,scale=#1, transform shape]

        % Fill Color Styles
        \tikzstyle{ElementFill} = [fill=gray!20]
        \tikzstyle{AlkaliMetalFill} = [fill=RoyalBlue!55]
        \tikzstyle{AlkalineEarthMetalFill} = [fill=Yellow!40]
        \tikzstyle{MetalFill} = [fill=RoyalPurple!25]
        \tikzstyle{MetalloidFill} = [fill=SkyBlue!55]
        \tikzstyle{NonmetalFill} = [fill=BrickRed!30]
        \tikzstyle{HalogenFill} = [fill=Dandelion!55]
        \tikzstyle{NobleGasFill} = [fill=LimeGreen!55]
        \tikzstyle{LanthanideActinideFill} = [fill=Melon!30]

        % Element Styles
        \tikzstyle{Element} = [ElementFill,
        minimum width=2.5cm, minimum height=2.5cm, node distance=2.75cm]
        \tikzstyle{AlkaliMetal} = [Element, AlkaliMetalFill]
        \tikzstyle{AlkalineEarthMetal} = [Element, AlkalineEarthMetalFill]
        \tikzstyle{Metal} = [Element, MetalFill]
        \tikzstyle{Metalloid} = [Element, MetalloidFill]
        \tikzstyle{Nonmetal} = [Element, NonmetalFill]
        \tikzstyle{Halogen} = [Element, HalogenFill]
        \tikzstyle{NobleGas} = [Element, NobleGasFill]
        \tikzstyle{LanthanideActinide} = [Element, LanthanideActinideFill]
        \tikzstyle{PeriodLabel} = [font={\sffamily\LARGE}, node distance=2.0cm]
        \tikzstyle{GroupLabel} = [font={\sffamily\LARGE}, minimum width=2.75cm, node distance=2.0cm]

        % Group 1 - IA
        \node[Element] (H) {\NaturalElem{1} {1.0079}{H}{Hidr\'ogeno}};
        \node[below of=H, AlkaliMetal] (Li) {\NaturalElem{3}{6.941}{Li}{Litio}};
        \node[below of=Li, AlkaliMetal] (Na) {\NaturalElem{11}{22.990}{Na}{Sodio}};
        \node[below of=Na, AlkaliMetal] (K) {\NaturalElem{19}{39.098}{K}{Potasio}};
        \node[below of=K, AlkaliMetal] (Rb) {\NaturalElem{37}{85.468}{Rb}{Rubidio}};
        \node[below of=Rb, AlkaliMetal] (Cs) {\NaturalElem{55}{132.91}{Cs}{Cesio}};
        \node[below of=Cs, AlkaliMetal] (Fr) {\NaturalElem{87}{223}{Fr}{Francio}};

        % Group 2 - IIA
        \node[right of=Li, AlkalineEarthMetal] (Be) {\NaturalElem{4}{9.0122}{Be}{Berilio}};
        \node[below of=Be, AlkalineEarthMetal] (Mg) {\NaturalElem{12}{24.305}{Mg}{Magnesio}};
        \node[below of=Mg, AlkalineEarthMetal] (Ca) {\NaturalElem{20}{40.078}{Ca}{Calcio}};
        \node[below of=Ca, AlkalineEarthMetal] (Sr) {\NaturalElem{38}{87.62}{Sr}{Stroncio}};
        \node[below of=Sr, AlkalineEarthMetal] (Ba) {\NaturalElem{56}{137.33}{Ba}{Bario}};
        \node[below of=Ba, AlkalineEarthMetal] (Ra) {\NaturalElem{88}{226}{Ra}{Radio}};

        % Group 3 - IIIB
        \node[right of=Ca, Metal] (Sc) {\NaturalElem{21}{44.956}{Sc}{Escandio}};
        \node[below of=Sc, Metal] (Y) {\NaturalElem{39}{88.906}{Y}{Itrio}};
        \node[below of=Y, LanthanideActinide] (LaLu) {\NaturalElem{57-71}{}{*}{Lant\'anido}};
        \node[below of=LaLu, LanthanideActinide] (AcLr) {\NaturalElem{89-103}{}{**}{Act\'inido}};

        % Group 4 - IVB
        \node[right of=Sc, Metal] (Ti) {\NaturalElem{22}{47.867}{Ti}{Titanio}};
        \node[below of=Ti, Metal] (Zr) {\NaturalElem{40}{91.224}{Zr}{Circonio}};
        \node[below of=Zr, Metal] (Hf) {\NaturalElem{72}{178.49}{Hf}{Hafnio}};
        \node[below of=Hf, Metal] (Rf) {\SyntheticElem{104}{261}{Rf}{Rutherfordio}};

        % Group 5 - VB
        \node[right of=Ti, Metal] (V) {\NaturalElem{23}{50.942}{V}{Vanadio}};
        \node[below of=V, Metal] (Nb) {\NaturalElem{41}{92.906}{Nb}{Niobio}};
        \node[below of=Nb, Metal] (Ta) {\NaturalElem{73}{180.95}{Ta}{Tantalo}};
        \node[below of=Ta, Metal] (Db) {\SyntheticElem{105}{262}{Db}{Dubnio}};

        % Group 6 - VIB
        \node[right of=V, Metal] (Cr) {\NaturalElem{24}{51.996}{Cr}{Cromo}};
        \node[below of=Cr, Metal] (Mo) {\NaturalElem{42}{95.94}{Mo}{Molybdeno}};
        \node[below of=Mo, Metal] (W) {\NaturalElem{74}{183.84}{W}{Tungstenio}};
        \node[below of=W, Metal] (Sg) {\SyntheticElem{106}{266}{Sg}{Seaborgio}};

        % Group 7 - VIIB
        \node[right of=Cr, Metal] (Mn) {\NaturalElem{25}{54.938}{Mn}{Manganeso}};
        \node[below of=Mn, Metal] (Tc) {\NaturalElem{43}{96}{Tc}{Tecnecio}};
        \node[below of=Tc, Metal] (Re) {\NaturalElem{75}{186.21}{Re}{Renio}};
        \node[below of=Re, Metal] (Bh) {\SyntheticElem{107}{264}{Bh}{Bohrio}};

        % Group 8 - VIIIB
        \node[right of=Mn, Metal] (Fe) {\NaturalElem{26}{55.845}{Fe}{Hierro}};
        \node[below of=Fe, Metal] (Ru) {\NaturalElem{44}{101.07}{Ru}{Ruthenio}};
        \node[below of=Ru, Metal] (Os) {\NaturalElem{76}{190.23}{Os}{Osmio}};
        \node[below of=Os, Metal] (Hs) {\SyntheticElem{108}{277}{Hs}{Hassio}};

        % Group 9 - VIIIB
        \node[right of=Fe, Metal] (Co) {\NaturalElem{27}{58.933}{Co}{Cobalto}};
        \node[below of=Co, Metal] (Rh) {\NaturalElem{45}{102.91}{Rh}{Rodio}};
        \node[below of=Rh, Metal] (Ir) {\NaturalElem{77}{192.22}{Ir}{Iridio}};
        \node[below of=Ir, Metal] (Mt) {\SyntheticElem{109}{268}{Mt}{Meitnerio}};

        % Group 10 - VIIIB
        \node[right of=Co, Metal] (Ni) {\NaturalElem{28}{58.693}{Ni}{Niquel}};
        \node[below of=Ni, Metal] (Pd) {\NaturalElem{46}{106.42}{Pd}{Paladio}};
        \node[below of=Pd, Metal] (Pt) {\NaturalElem{78}{195.08}{Pt}{Platino}};
        \node[below of=Pt, Metal] (Ds) {\SyntheticElem{110}{281}{Ds}{Darmstadtio}};

        % Group 11 - IB
        \node[right of=Ni, Metal] (Cu) {\NaturalElem{29}{63.546}{Cu}{Cobre}};
        \node[below of=Cu, Metal] (Ag) {\NaturalElem{47}{107.87}{Ag}{Plata}};
        \node[below of=Ag, Metal] (Au) {\NaturalElem{79}{196.97}{Au}{Oro}};
        \node[below of=Au, Metal] (Rg) {\SyntheticElem{111}{280}{Rg}{Roentgenio}};

        % Group 12 - IIB
        \node[right of=Cu, Metal] (Zn) {\NaturalElem{30}{65.39}{Zn}{Zinc}};
        \node[below of=Zn, Metal] (Cd) {\NaturalElem{48}{112.41}{Cd}{Cadmio}};
        \node[below of=Cd, Metal] (Hg) {\NaturalElem{80}{200.59}{Hg}{Mercurio}};
        \node[below of=Hg, Metal] (Cn) {\SyntheticElem{112}{285}{Cn}{Copernicio}};

        % Group 13 - IIIA
        \node[right of=Zn, Metal] (Ga) {\NaturalElem{31}{69.723}{Ga}{Galio}};
        \node[above of=Ga, Metal] (Al) {\NaturalElem{13}{26.982}{Al}{Aluminio}};
        \node[above of=Al, Metalloid] (B) {\NaturalElem{5}{10.811}{B}{Boro}};
        \node[below of=Ga, Metal] (In) {\NaturalElem{49}{114.82}{In}{Indo}};
        \node[below of=In, Metal] (Tl) {\NaturalElem{81}{204.38}{Tl}{Talio}};
        \node[below of=Tl, Metal] (Nh) {\SyntheticElem{113}{284}{Nh}{Nihonio}};

        % Group 14 - IVA
        \node[right of=B, Nonmetal] (C) {\NaturalElem{6}{12.011}{C}{Carbono}};
        \node[below of=C, Metalloid] (Si) {\NaturalElem{14}{28.086}{Si}{Silicio}};
        \node[below of=Si, Metalloid] (Ge) {\NaturalElem{32}{72.64}{Ge}{Germanio}};
        \node[below of=Ge, Metal] (Sn) {\NaturalElem{50}{118.71}{Sn}{Esta\~no}};
        \node[below of=Sn, Metal] (Pb) {\NaturalElem{82}{207.2}{Pb}{Plomo}};
        \node[below of=Pb, Metal] (Fl) {\SyntheticElem{114}{289}{Fl}{Flerovio}};

        % Group 15 - VA
        \node[right of=C, Nonmetal] (N) {\NaturalElem{7}{14.007}{N}{Nitr\'ogeno}};
        \node[below of=N, Nonmetal] (P) {\NaturalElem{15}{30.974}{P}{F\'osforo}};
        \node[below of=P, Metalloid] (As) {\NaturalElem{33}{74.922}{As}{Ars\'enico}};
        \node[below of=As, Metalloid] (Sb) {\NaturalElem{51}{121.76}{Sb}{Antimonio}};
        \node[below of=Sb, Metal] (Bi) {\NaturalElem{83}{208.98}{Bi}{Bismuto}};
        \node[below of=Bi, Metal] (Mc) {\SyntheticElem{115}{288}{Mc}{Moscovio}};

        % Group 16 - VIA
        \node[right of=N, Nonmetal] (O) {\NaturalElem{8}{15.999}{O}{Ox\'igeno}};
        \node[below of=O, Nonmetal] (S) {\NaturalElem{16}{32.065}{S}{Az\'ufre}};
        \node[below of=S, Nonmetal] (Se) {\NaturalElem{34}{78.96}{Se}{Selenio}};
        \node[below of=Se, Metalloid] (Te) {\NaturalElem{52}{127.6}{Te}{Tellurio}};
        \node[below of=Te, Metalloid] (Po) {\NaturalElem{84}{209}{Po}{Polonio}};
        \node[below of=Po, Metal] (Lv) {\SyntheticElem{116}{293}{Lv}{Libermonio}};

        % Group 17 - VIIA
        \node[right of=O, Halogen] (F) {\NaturalElem{9}{18.998}{F}{Fluor}};
        \node[below of=F, Halogen] (Cl) {\NaturalElem{17}{35.453}{Cl}{Cloro}};
        \node[below of=Cl, Halogen] (Br) {\NaturalElem{35}{79.904}{Br}{Bromo}};
        \node[below of=Br, Halogen] (I) {\NaturalElem{53}{126.9}{I}{Yodo}};
        \node[below of=I, Halogen] (At) {\NaturalElem{85}{210}{At}{\'Astato}};
        \node[below of=At, Element] (Ts) {\SyntheticElem{117}{292}{Ts}{Teneso}};

        % Group 18 - VIIIA
        \node[right of=F, NobleGas] (Ne) {\NaturalElem{10}{20.180}{Ne}{Ne\'on}};
        \node[above of=Ne, NobleGas] (He) {\NaturalElem{2}{4.0025}{He}{Helio}};
        \node[below of=Ne, NobleGas] (Ar) {\NaturalElem{18}{39.948}{Ar}{Arg\'on}};
        \node[below of=Ar, NobleGas] (Kr) {\NaturalElem{36}{83.8}{Kr}{Kript\'on}};
        \node[below of=Kr, NobleGas] (Xe) {\NaturalElem{54}{131.29}{Xe}{Xen\'on}};
        \node[below of=Xe, NobleGas] (Rn) {\NaturalElem{86}{222}{Rn}{Rad\'on}};
        \node[below of=Rn, Nonmetal] (Og) {\SyntheticElem{118}{294}{Og}{Oganesón}};

        % Period
        \node[left of=H, PeriodLabel] (Period1) {1};
        \node[left of=Li, PeriodLabel] (Period2) {2};
        \node[left of=Na, PeriodLabel] (Period3) {3};
        \node[left of=K, PeriodLabel] (Period4) {4};
        \node[left of=Rb, PeriodLabel] (Period5) {5};
        \node[left of=Cs, PeriodLabel] (Period6) {6};
        \node[left of=Fr, PeriodLabel] (Period7) {7};

        % Group
        \node[above of=H, GroupLabel] (Group1) {1 \hfill IA};
        \node[above of=Be, GroupLabel] (Group2) {2 \hfill IIA};
        \node[above of=Sc, GroupLabel] (Group3) {3 \hfill IIIA};
        \node[above of=Ti, GroupLabel] (Group4) {4 \hfill IVB};
        \node[above of=V, GroupLabel] (Group5) {5 \hfill VB};
        \node[above of=Cr, GroupLabel] (Group6) {6 \hfill VIB};
        \node[above of=Mn, GroupLabel] (Group7) {7 \hfill VIIB};
        \node[above of=Fe, GroupLabel] (Group8) {8 \hfill VIIIB};
        \node[above of=Co, GroupLabel] (Group9) {9 \hfill VIIIB};
        \node[above of=Ni, GroupLabel] (Group10) {10 \hfill VIIIB};
        \node[above of=Cu, GroupLabel] (Group11) {11 \hfill IB};
        \node[above of=Zn, GroupLabel] (Group12) {12 \hfill IIB};
        \node[above of=B, GroupLabel] (Group13) {13 \hfill IIIA};
        \node[above of=C, GroupLabel] (Group14) {14 \hfill IVA};
        \node[above of=N, GroupLabel] (Group15) {15 \hfill VA};
        \node[above of=O, GroupLabel] (Group16) {16 \hfill VIA};
        \node[above of=F, GroupLabel] (Group17) {17 \hfill VIIA};
        \node[above of=He, GroupLabel] (Group18) {18 \hfill VIIIA};


        % Lanthanide
        \node[below of=Rf, LanthanideActinide, yshift=-1cm] (La) {\NaturalElem{57}{138.91}{La}{Lant\'anido}};
        \node[right of=La, LanthanideActinide] (Ce) {\NaturalElem{58}{140.12}{Ce}{Cerio}};
        \node[right of=Ce, LanthanideActinide] (Pr) {\NaturalElem{59}{140.91}{Pr}{Praseodymio}};
        \node[right of=Pr, LanthanideActinide] (Nd) {\NaturalElem{60}{144.24}{Nd}{Neodimio}};
        \node[right of=Nd, LanthanideActinide] (Pm) {\NaturalElem{61}{145}{Pm}{Prometio}};
        \node[right of=Pm, LanthanideActinide] (Sm) {\NaturalElem{62}{150.36}{Sm}{Samario}};
        \node[right of=Sm, LanthanideActinide] (Eu) {\NaturalElem{63}{151.96}{Eu}{Europio}};
        \node[right of=Eu, LanthanideActinide] (Gd) {\NaturalElem{64}{157.25}{Gd}{Gadolinio}};
        \node[right of=Gd, LanthanideActinide] (Tb) {\NaturalElem{65}{158.93}{Tb}{Terbio}};
        \node[right of=Tb, LanthanideActinide] (Dy) {\NaturalElem{66}{162.50}{Dy}{Disprosio}};
        \node[right of=Dy, LanthanideActinide] (Ho) {\NaturalElem{67}{164.93}{Ho}{Holmio}};
        \node[right of=Ho, LanthanideActinide] (Er) {\NaturalElem{68}{167.26}{Er}{Erbio}};
        \node[right of=Er, LanthanideActinide] (Tm) {\NaturalElem{69}{168.93}{Tm}{Tulio}};
        \node[right of=Tm, LanthanideActinide] (Yb) {\NaturalElem{70}{173.04}{Yb}{Yterbio}};
        \node[right of=Yb, LanthanideActinide] (Lu) {\NaturalElem{71}{174.97}{Lu}{Luterio}};

        % Actinide
        \node[below of=La, LanthanideActinide, yshift=-1cm] (Ac) {\NaturalElem{89}{227}{Ac}{Actinio}};
        \node[right of=Ac, LanthanideActinide]              (Th) {\NaturalElem{90}{232.04}{Th}{Torio}};
        \node[right of=Th, LanthanideActinide]              (Pa) {\NaturalElem{91}{231.04}{Pa}{Protactinio}};
        \node[right of=Pa, LanthanideActinide]              (U)  {\NaturalElem{92}{238.03}{U}{Uranio}};
        \node[right of=U, LanthanideActinide]               (Np) {\SyntheticElem{93}{237}{Np}{Neptunio}};
        \node[right of=Np, LanthanideActinide]              (Pu) {\SyntheticElem{94}{244}{Pu}{Plutonio}};
        \node[right of=Pu, LanthanideActinide]              (Am) {\SyntheticElem{95}{243}{Am}{Americio}};
        \node[right of=Am, LanthanideActinide]              (Cm) {\SyntheticElem{96}{247}{Cm}{Curio}};
        \node[right of=Cm, LanthanideActinide]              (Bk) {\SyntheticElem{97}{247}{Bk}{Berkelio}};
        \node[right of=Bk, LanthanideActinide]              (Cf) {\SyntheticElem{98}{251}{Cf}{Californio}};
        \node[right of=Cf, LanthanideActinide]              (Es) {\SyntheticElem{99}{252}{Es}{Einsteinio}};
        \node[right of=Es, LanthanideActinide]              (Fm) {\SyntheticElem{100}{257}{Fm}{Fermio}};
        \node[right of=Fm, LanthanideActinide]              (Md) {\SyntheticElem{101}{258}{Md}{Mendelevio}};
        \node[right of=Md, LanthanideActinide]              (No) {\SyntheticElem{102}{259}{No}{Nobelio}};
        \node[right of=No, LanthanideActinide]              (Lr) {\SyntheticElem{103}{262}{Lr}{Lawrencio}};

        % Draw dotted lines connecting Lanthanide breakout to main table
        \draw[thick,dotted] (LaLu.north west) -- (La.north west)
        (LaLu.south west) -- (La.south west);
        % Draw dotted lines connecting Actinide breakout to main table
        \draw[thick,dotted] (AcLr.north west) -- (Ac.north west)
        (AcLr.south west) -- (Ac.south west);

        % Legend
        \fill[AlkaliMetalFill] ($(La.north -| Fr.west) + (0,1em)$)
        rectangle +(1em, 1em) node[right, yshift=-1.2ex]  (AlkaliMetal) {Metales Alcalinos};
        \fill[AlkalineEarthMetalFill] ($(AlkaliMetal.west) - (1em,2em)$)
        rectangle +(1em, 1em) node[right, yshift=-1.2ex] (AlkalineEarthMetal) {Metales Alcalino-terreos};
        \fill[MetalFill] ($(AlkalineEarthMetal.west) - (1em,2em)$)
        rectangle +(1em, 1em) node[right, yshift=-1.2ex] (Metal) {Metal};
        \fill[MetalloidFill] ($(Metal.west) - (1em,2em)$)
        rectangle +(1em, 1em) node[right, yshift=-1.2ex] (Metalloid) {Metaloide};
        \fill[NonmetalFill] ($(Metalloid.west) - (1em,2em)$)
        rectangle +(1em, 1em) node[right, yshift=-1.2ex] (Non-metal) {No metal};
        \fill[HalogenFill] ($(Non-metal.west) - (1em,2em)$)
        rectangle +(1em, 1em) node[right, yshift=-1.2ex] (Halogen) {Hal\'ogeno};
        \fill[NobleGasFill] ($(Halogen.west) - (1em,2em)$)
        rectangle +(1em, 1em) node[right, yshift=-1.2ex] (NobleGas) {Gases Nobles};
        \fill[LanthanideActinideFill] ($(NobleGas.west) - (1em,2em)$)
        rectangle +(1em, 1em) node[right, yshift=-1.2ex] (Lanthanide/Actinide) {Lant\'anidos/Act\'inidos};

        \node at (Ac -| Fr) [draw, Element, fill=white] (legend) {\NaturalElem{Z}{Masa}{\LARGE S\'imbolo}{Nombre}};
        \node[align=left] at (Ac -| Ra) {Negro: Naturales\\\color{darkgray}Gris: Sint\'eticos};

        % Diagram Title
        % \node at (H.west -| Fe.north) [scale=2, font={\sffamily\Huge\bfseries}]
        % {Tabla Peri\'odica de los Elementos};

    \end{tikzpicture}
}
%\end{document}
\begin{document}
%\pagestyle{headandfoot}
%\thispagestyle{plain}
% \INFO
% \printanswers
\begin{multicols}{2}
    \include*{../blocks/vocabulario}
    \include*{../blocks/block002}
\end{multicols}
\newpage
\begin{questions}
    \include*{../questions/question005}
    \newpage
    \include*{../questions/question009}
    %\include*{../questions/question007}
    \newpage
    \include*{../questions/question004}
    \newpage
    \include*{../questions/question006}
    \include*{../questions/question008}
    \newpage
    %\newpage
    % \include*{../questions/question010}
    \include*{../questions/question011}
    \newpage
    \include*{../questions/question013}
    \newpage
    \include*{../questions/question014}
    \newpage
    \include*{../questions/question015}
    \include*{../questions/question012}
    \newpage
    \include*{../questions/question016}
    \newpage
    \include*{../questions/question017}
\end{questions}
\pagestyle{empty}
\newgeometry{left=30mm,top=0mm,bottom=15mm}
\begin{landscape}
    \vspace{30mm}\hspace{-20mm}
    \begin{table}[H]
        \caption{Tabla Peri\'odica de los Elementos.}
        \TablaPeriodica[0.50]
        \label{tab:periodic_table}
    \end{table}
\end{landscape}
% \restoregeometry

\end{document}