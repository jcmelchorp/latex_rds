\documentclass[12pt,addpoints]{evalua}
\grado{3$^\circ$ de Secundaria}
\cicloescolar{2023-2024}
\materia{Ciencias y Tecnología: Química}
\unidad{1}
\title{Examen de la Unidad}
\aprendizajes{\footnotesize%
\item Formula hipótesis para diferenciar propiedades extensivas e intensivas, mediante actividades experimentales y, con base en el análisis de resultados, elabora conclusiones.\\[-1.5em]
\item Reconoce la importancia del uso de instrumentos de medición, para identificar y diferenciar propiedades de sustancias y materiales cotidianos.\\[-1.5em]
\item Describe los componentes de una mezcla (soluto-disolvente; fase dispersa y fase dispersante) mediante actvidades experimentales y las clasifica en homogéneas y heterogéneas en materiales de uso cotidiano.\\[-1.5em]
\item Deduce métodos para separar mezclas (evaporación, decantación, filtración, extracción y cromatografía ) mediante actividades experimentales con base en las propiedades físicas de las sustancias.\\[-1.5em]
}
\author{Prof.: Julio César Melchor Pinto}
\begin{document}
\begin{questions}
      \question[10]Selecciona la opción que  resuelve correctamente cada uno de los siguientes problemas:

      % \begin{multicols}{2}
      \begin{parts}
            \part La máxima masa de glucosa que se disuelve en 0.1L de agua es 90.9 g a 25°C. ¿Cuál es la solubilidad en g/L?

            \begin{oneparchoices}
                  \choice 9090 g/L \CorrectChoice 909 g/L
                  \choice 9.09 g/L
                  \choice 0.909 g/L
            \end{oneparchoices}

            \part La máxima masa de fructosa que se disuelve en 1L de agua es 3750 g a 20°C. ¿Cuál es la solubilidad en g/dL?

            \begin{oneparchoices}
                  \CorrectChoice 375 g/dL \choice 37.5 g/dL \choice 20 g/dL
                  \choice 37500 g/dL
            \end{oneparchoices}

            \part  La máxima masa de dióxido de carbono que se disuelve en 1L de agua es 1.45g a 25°C. ¿Cuál es la solubilidad en g/dL?

            \begin{oneparchoices}\raggedright
                  \choice 1.45 g/dL     \choice 14.5 g/dL
                  \choice 145 g/dL     \CorrectChoice 0.145 g/dL
            \end{oneparchoices}

            \part ¿Cuál de los siguientes materiales es una mezcla heterogénea?

            \begin{oneparchoices}\raggedright
                  \choice Acero \choice Plata  \CorrectChoice Tierra \choice Metano
            \end{oneparchoices}

            \part ¿Qué método de separación de mezclas usarías para separar una muestra de arena que está suspendida en un líquido?

            \begin{oneparchoices}\raggedright
                  \choice Destilación \choice Cromatografía  \choice Magnetismo \CorrectChoice Decantación
            \end{oneparchoices}
      \end{parts}
      % \end{multicols}

      \newpage
      \question[20]Elige si son verdaderas o falsas las siguientes afirmaciones.
      \begin{multicols}{2}
            \begin{parts}
                  \part Solamente las sociedades modernas han aportado conocimientos que ayudan a la satisfacción de las necesidades humanas.

                  \begin{oneparcheckboxes}
                        \choice Verdadero \CorrectChoice Falso
                  \end{oneparcheckboxes}

                  % \part El Homo sapiens ``domesticó'' el fuego hace aproximadamente 1.6 millones de años.

                  % \begin{oneparcheckboxes}
                  %       \choice Verdadero \CorrectChoice Falso
                  % \end{oneparcheckboxes}

                  \part El cambio de estado gaseoso a líquido de un material es un proceso de sublimación.

                  \begin{oneparcheckboxes}
                        \choice Verdadero \CorrectChoice Falso
                  \end{oneparcheckboxes}

                  % \part El jabón es un invento moderno del siglo XIX que ayuda a mejorar nuestra calidad de vida.

                  % \begin{oneparcheckboxes}
                  %       \choice Verdadero \CorrectChoice Falso
                  % \end{oneparcheckboxes}

                  \part Todas las culturas de los cinco continentes han aportado conocimientos y avances tecnológicos en beneficio de la humanidad.

                  \begin{oneparcheckboxes}
                        \CorrectChoice Verdadero \choice Falso
                  \end{oneparcheckboxes}

                  \part El conocimiento empírico es igual al conocimiento científico.

                  \begin{oneparcheckboxes}
                        \choice Verdadero \CorrectChoice Falso
                  \end{oneparcheckboxes}

                  % \part La saponificación es el proceso químico que nos permite obtener jabones.

                  % \begin{oneparcheckboxes}
                  %       \CorrectChoice Verdadero \choice Falso
                  % \end{oneparcheckboxes}

                  % \part Existe evidencia de que el jabón se producía en Babilonia hace 6000 años.

                  % \begin{oneparcheckboxes}
                  %       \choice Verdadero \CorrectChoice Falso
                  % \end{oneparcheckboxes}

                  \part La expectativa de vida ha incrementado en los últimos 150 años gracias al descubrimiento de medicamentos y al desarrollo de los procesos de sanidad.

                  \begin{oneparcheckboxes}
                        \CorrectChoice Verdadero \choice Falso
                  \end{oneparcheckboxes}

                  % \part Las aportaciones de las culturas originarias en la satisfacción de necesidades también se ven reflejadas en la arquitectura y en la construcción.

                  % \begin{oneparcheckboxes}
                  %       \CorrectChoice Verdadero \choice Falso
                  % \end{oneparcheckboxes}

                  \part Las propiedades químicas del PVC no se pueden determinar debido a que es un material que presenta demasiada dureza.

                  \begin{oneparcheckboxes}
                        \choice Verdadero \CorrectChoice Falso
                  \end{oneparcheckboxes}


                  % \part El lustre y el brillo son propiedades físicas mecánicas que predominan con mayor frecuencia en los metales.

                  % \begin{oneparcheckboxes}
                  %       \choice Verdadero \CorrectChoice Falso
                  % \end{oneparcheckboxes}

                  \part El aroma, o incluso el sabor, de un material orgánico se clasifican como propiedades físicas cualitativas.

                  \begin{oneparcheckboxes}
                        \CorrectChoice Verdadero \choice Falso
                  \end{oneparcheckboxes}


                  \part Los conocimientos empíricos de los pueblos prehispánicos sobre plantas medicinales y hongos pueden ayudarnos a resolver problemas y necesidades actuales.

                  \begin{oneparcheckboxes}
                        \CorrectChoice Verdadero \choice Falso
                  \end{oneparcheckboxes}

                  % \part Debido al consumismo se acumulan bienes y servicios no esenciales.

                  % \begin{oneparcheckboxes}
                  %       \CorrectChoice Verdadero \choice Falso
                  % \end{oneparcheckboxes}

                  % \part A pesar de que se consume un exceso de recursos naturales el impacto del consumismo en la generación de residuos es mínimo.

                  % \begin{oneparcheckboxes}
                  %       \choice Verdadero \CorrectChoice Falso
                  % \end{oneparcheckboxes}

                  % \part Los consumidores responsables saben de las consecuencias del consumo a nivel ambiental, social y económico.

                  % \begin{oneparcheckboxes}
                  %       \CorrectChoice Verdadero \choice Falso
                  % \end{oneparcheckboxes}

                  % \part Ser un consumidor responsable implica respetar a la naturaleza.

                  % \begin{oneparcheckboxes}
                  %       \CorrectChoice Verdadero \choice Falso
                  % \end{oneparcheckboxes}

                  % \part Se estima que 2/3 de la comida en el mundo se pudre por no ser consumida.

                  % \begin{oneparcheckboxes}
                  %       \choice Verdadero \CorrectChoice Falso
                  % \end{oneparcheckboxes}

                  % \part Según la ONU, con 25\% de la comida que se desperdicia se podría alimentar a 870 millones de personas con hambre.

                  % \begin{oneparcheckboxes}
                  %       \CorrectChoice Verdadero \choice Falso
                  % \end{oneparcheckboxes}

                  % \part La mayoría de la energía mundial la consumen las personas en sus hogares.

                  % \begin{oneparcheckboxes}
                  %       \choice Verdadero \CorrectChoice Falso
                  % \end{oneparcheckboxes}

                  % \part Cada año se destruye 1 millón de hectáreas de bosques por el consumo excesivo de los recursos naturales.

                  % \begin{oneparcheckboxes}
                  %       \choice Verdadero \CorrectChoice Falso
                  % \end{oneparcheckboxes}

                  % \part El consumo responsable solamente implica el realizar grandes acciones como protestas.

                  % \begin{oneparcheckboxes}
                  %       \choice Verdadero \CorrectChoice Falso
                  % \end{oneparcheckboxes}

                  \part Para reducir el impacto del consumo de productos es importante el informarse para poder optar por opciones sostenibles.

                  \begin{oneparcheckboxes}
                        \CorrectChoice Verdadero \choice Falso
                  \end{oneparcheckboxes}

                  \part La condensación de un material ocurre cuando pasa de estado líquido a gaseoso.

                  \begin{oneparcheckboxes}
                        \choice Verdadero \CorrectChoice Falso
                  \end{oneparcheckboxes}

                  % \part Seguir el punto 12 de los ODS de la ONU nos ayuda a garantizar formas de consumo y producción sostenibles.

                  % \begin{oneparcheckboxes}
                  %       \CorrectChoice Verdadero \choice Falso
                  % \end{oneparcheckboxes}

                  % \part Reducir la cantidad de desechos que producimos es parte del consumo responsable.

                  % \begin{oneparcheckboxes}
                  %       \CorrectChoice Verdadero \choice Falso
                  % \end{oneparcheckboxes}

            \end{parts}
      \end{multicols}


      \question[16]Selecciona las respuestas correctas a cada pregunta.
      \begin{multicols}{2}
            \begin{parts}
                  \part ¿Cuáles son los principales contaminantes del aire?

                  \begin{checkboxes}
                        \choice Residuos de cobre
                        \CorrectChoice Monóxido de carbono
                        \choice Vapor de agua
                        \CorrectChoice Dióxido de azufre
                        \choice Residuos de plomo
                  \end{checkboxes}


                  \part ¿Qué factores disminuyen la cantidad de oxígeno en el agua?
                  \begin{checkboxes}
                        \CorrectChoice La actividad humana
                        \choice El consumo doméstico
                        \CorrectChoice La presión atmosférica
                        \choice El exceso de nutrientes
                        \choice El sobrepastoreo
                  \end{checkboxes}


                  \part  ¿Qué sustancias son capaces de modificar la toxicidad del agua y suelos?

                  \begin{checkboxes}
                        \CorrectChoice Residuos de cobre
                        \choice Dióxido de carbono
                        \choice Vapor de agua
                        \choice Dióxido de azufre
                        \CorrectChoice Residuos de plomo
                  \end{checkboxes}


                  \part ¿Cuáles son las principales causas de degradación del suelo en México?

                  \begin{checkboxes}
                        \choice El consumo doméstico
                        \choice La presión atmosférica
                        \choice El exceso de nutrientes
                        \CorrectChoice La actividad humana
                        \CorrectChoice El sobrepastoreo
                  \end{checkboxes}
            \end{parts}
      \end{multicols}

      \newpage
      \question[8]Señala si los siguientes procesos son \textit{físicos} o \textit{químicos}.
      \begin{multicols}{2}
            \begin{parts}
                  \part Romper un tazón de cerámica.

                  \begin{oneparcheckboxes}
                        \CorrectChoice Físico \choice Químico
                  \end{oneparcheckboxes}

                  \part Digerir y absorber los alimentos.

                  \begin{oneparcheckboxes}
                        \choice Físico \CorrectChoice Químico
                  \end{oneparcheckboxes}

                  \part Disolver azucar en una taza de té.

                  \begin{oneparcheckboxes}
                        \CorrectChoice Físico \choice Químico
                  \end{oneparcheckboxes}

                  \part Encender fuegos artificiales.

                  \begin{oneparcheckboxes}
                        \choice Físico \CorrectChoice Químico
                  \end{oneparcheckboxes}

                  \part Hornear un pastel de vainilla.

                  \begin{oneparcheckboxes}
                        \choice Físico \CorrectChoice Químico
                  \end{oneparcheckboxes}

                  \part Apretar una lata de aluminio.

                  \begin{oneparcheckboxes}
                        \CorrectChoice Físico \choice Químico
                  \end{oneparcheckboxes}

                  \part Mezclar pigmentos de colores.

                  \begin{oneparcheckboxes}
                        \CorrectChoice Físico \choice Químico
                  \end{oneparcheckboxes}

                  \part Cocinar un huevo estrellado.

                  \begin{oneparcheckboxes}
                        \choice Físico \CorrectChoice Químico
                  \end{oneparcheckboxes}
            \end{parts}
      \end{multicols}

      \question[16] A partir de la información que se presenta, coloca los datos que faltan en la tabla.

      \begin{table}[H]
            \centering
            \begin{tabular}{p{4cm}>{\centering\columncolor{DarkOliveGreen!10}}p{1.2cm}p{1.5cm}>{\columncolor{Sepia!10}}p{1.5cm}p{0.1cm}}
                  \bfseries Sustancia  & \bfseries ppm & \bfseries \%             & \bfseries mg/l       \\    \hline
                  Dióxido de azufre    & 0.13          & \fillin[0.000013][0.6in] & \fillin[0.13][0.6in] \\    \hline
                  Dióxido de nitrógeno & 0.21          & \fillin[0.000021][0.6in] & \fillin[0.21][0.6in] \\    \hline
                  Monóxido de carbono  & 11            & \fillin[0.0011][0.6in]   & \fillin[11][0.6in]   \\    \hline
                  Ozono                & 0.11          & \fillin[0.000011][0.6in] & \fillin[0.11][0.6in] \\    \hline
            \end{tabular}
      \end{table}

      \begin{minipage}[t][][t]{.4\textwidth}

            \question[5]Elige el método de separación que debe de usarse en cada mezcla.\\

            \begin{parts}
                  % \part Una mezcla de agua y arena.

                  % \begin{oneparcheckboxes}
                  %       \CorrectChoice Filtración  \choice Cromatografía  \choice Extracción  \choice Decantación
                  % \end{oneparcheckboxes}

                  \part Una mezcla de aire.

                  \begin{oneparchoices}
                        \choice Decantación  \choice Destilación \\ \choice Filtración  \CorrectChoice Cromatografía
                  \end{oneparchoices}

                  % \part Una mezcla de azufre y agua.

                  % \begin{oneparchoices}
                  %       \CorrectChoice Filtración  \choice Destilación  \choice Cromatografía  \choice Extracción
                  % \end{oneparchoices}

                  \part Una muestra de gasolina.

                  \begin{oneparchoices}
                        \choice Decantación  \CorrectChoice Destilación \\ \choice Filtración  \choice Cromatografía
                  \end{oneparchoices}

                  \part Una mezcla homogénea de líquidos.

                  \begin{oneparchoices}
                        \choice Decantación  \CorrectChoice Destilación \\ \choice Filtración  \choice Cromatografía
                  \end{oneparchoices}

                  \part Una mezcla de tinta negra.

                  \begin{oneparchoices}
                        \choice Decantación  \choice Destilación \\ \choice Filtración  \CorrectChoice Cromatografía
                  \end{oneparchoices}

                  % \part Una mezcla de agua y sal.

                  % \begin{oneparchoices}
                  %       \CorrectChoice Evaporación  \choice Cromatografía  \choice Filtración  \choice Destilación
                  % \end{oneparchoices}

                  % \part Una mezcla de sal, azufre y agua (recuerda que la sal se disuelve en agua pero el azufre no).

                  % \begin{oneparchoices}
                  %       \CorrectChoice Filtración y evaporación  \choice Cromatografía y evaporación  \choice Extracción y tamizado  \choice Destilación y filtración
                  % \end{oneparchoices}

                  \part Una mezcla de vinagre y aceite de olivo.

                  \begin{oneparchoices}
                        \CorrectChoice Decantación  \choice Destilación \\ \choice Filtración  \choice Cromatografía
                  \end{oneparchoices}

                  % \part Una mezcla de pan molido y clips.

                  % \begin{oneparchoices}
                  %       \choice Extracción  \CorrectChoice Filtración  \choice Decantación  \choice Cromatografía
                  % \end{oneparchoices}
            \end{parts}

      \end{minipage}\hfill%
      \begin{minipage}[t][][t]{.5\textwidth}

            \question[5]Indica si se trata de una mezcla homogénea o heterogénea.\\

            % \begin{multicols}{2}
            \begin{parts}
                  \part Perfume

                  \begin{oneparchoices}
                        \CorrectChoice Mezcla homogénea \choice Mezcla heterogénea
                  \end{oneparchoices}

                  \part Café

                  \begin{oneparchoices}
                        \CorrectChoice Mezcla homogénea \choice Mezcla heterogénea
                  \end{oneparchoices}

                  % \part Aceite trifásico

                  % \begin{oneparchoices}
                  %       \choice Mezcla homogénea \CorrectChoice Mezcla heterogénea
                  % \end{oneparchoices}

                  \part Acero

                  \begin{oneparchoices}
                        \CorrectChoice Mezcla homogénea \choice Mezcla heterogénea
                  \end{oneparchoices}

                  \part Vinagre y aceite

                  \begin{oneparchoices}
                        \choice Mezcla homogénea \CorrectChoice Mezcla heterogénea
                  \end{oneparchoices}

                  \part Granito

                  \begin{oneparchoices}
                        \choice  Mezcla homogénea \CorrectChoice Mezcla heterogénea
                  \end{oneparchoices}
            \end{parts}
      \end{minipage}

      \newpage
      \question[10] Relaciona cada enunciado con la propiedad física que representa.

      \begin{multicols}{2}
            \begin{parts}\raggedleft
                  \part Espacio que ocupa un material.               \fillin[D][0.2in]
                  \part Cantidad de materia de un material.          \fillin[B][0.2in]
                  \part Masa por unidad de volumen.                  \fillin[E][0.2in]
                  \part Depende de la cantidad total del sistema.    \fillin[A][0.2in]
                  \part Es independiente a la cantidad de sustancia. \fillin[C][0.2in]

            \end{parts}
            \columnbreak

            \begin{choices}
                  \choice Extensiva
                  \choice Masa
                  \choice Intensiva
                  \choice Volumen
                  \choice Densidad
            \end{choices}
      \end{multicols}

      \question[10] Elige la respuesta correcta.
      % \begin{multicols}{2}
      \begin{parts}
            \part ¿De qué manera es posible cambiar las propiedades de una mezcla?
            \begin{choices}
                  \choice Manteniendo las proporciones de sus solutos.
                  \CorrectChoice Modificando las proporciones de sus componentes.
                  \choice Modificando todos sus componentes.
                  \choice Manteniendo todos sus componentes.
            \end{choices}

            \part  ¿Qué es una disolución?

            \begin{choices}
                  \choice Una mezcla heterogénea de dos o más sustancias distintas.
                  \choice Una mezcla heterogénea de dos o más sustancias idénticas.
                  \CorrectChoice Una mezcla homogénea de dos o más sustancias distintas.
                  \choice Una mezcla homogénea de dos o más sustancias idénticas.
            \end{choices}



            \part ¿Qué concentración tiene una disolución de hidróxido de sodio preparada con 0.20 g de esta sustancia y 25 mL de disolvente?
            \begin{choices}
                  \CorrectChoice 0.008 g/mL
                  \choice 0.080 g/mL
                  \choice 1.250 g/mL
                  \choice 125.0 g/mL
            \end{choices}


            \part  ¿Con qué masa se prepararon 1 000 mL de una disolución de ácido acético a una concentración de 0.75 g/mL?
            \begin{choices}
                  \choice 133.3 g
                  \CorrectChoice 750.0 g
                  \choice 7.500 g
                  \choice  13.33 g
            \end{choices}

            \part  ¿Cómo se determina la concentración de una disolución?

            \begin{choices}
                  \choice $\text{Concentración} = \dfrac{\text{Masa de disolvente}}{\text{Volumen de soluto}}$        \\
                  \choice $\text{Concentración} = \dfrac{\text{Volumen de soluto}}{\text{Masa de disolvente}}$        \\
                  \CorrectChoice $\text{Concentración} = \dfrac{\text{Masa de soluto}}{\text{Volumen de disolvente}}$ \\
                  \choice $\text{Concentración} = \dfrac{\text{Volumen de disolvente}}{\text{Masa de soluto}}$        \\
            \end{choices}
      \end{parts}
      % \end{multicols}
\end{questions}
\end{document}