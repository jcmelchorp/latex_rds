Se encuentra que una tableta de vitamina B3 de 1.90 g contiene 0.0122 mol de nicotinamida (\ce{C6H6N2O}). (La masa molar de \ce{C6H6N2O} es 122.13 g/mol.)\\
\textbf{¿Cuál es el porcentaje de masa de \ce{C6H6N2O} en la tableta?}\\
\emph{Escribe tu respuesta usando tres cifras significativas.}

\begin{solutionbox}{6cm}
    El porcentaje de masa de una sustancia en una mezcla se puede determinar por la comparación de la masa de la sustancia en la mezcla contra la masa total de la mezcla.
    Primero, calculemos la masa de \ce{C6H6N2O} en la tableta. Utilizando la masa molar de \ce{C6H6N2O}, podemos convertir moles de \ce{C6H6N2O} a gramos de \ce{C6H6N2O}:
    \[ 0.0122 \text{mol \ce{C6H6N2O}} \times \frac{122.13 \text{g \ce{C6H6N2O}}}{1\ \text{mol \ce{C6H6N2O}}} = 1.49\ \text{g \ce{C6H6N2O}} \]
    Posteriormente, utilizando la masa calculada de \ce{C6H6N2O} y la masa total de la tableta de vitamina B3, podemos calcular el porcentaje de masa de \ce{C6H6N2O} en la tableta:
\[ \frac{1.49 \text{g \ce{C6H6N2O}}}{1.90 \text{g tableta}} \times 100\% = 78.4\% \]
    Por lo tanto, el porcentaje de masa de \ce{C6H6N2O} en la tableta es 78.4\%.
\end{solutionbox}