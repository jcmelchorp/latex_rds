Balancea la siguiente ecuación química:

\[
    \ce{CH4 + O2 -> CO2 + H2O}
\]

\begin{solutionbox}{7.5cm}
    \begin{multicols}{2}
        Si representamos la ecuación química con átomos de distintos colores para cada elemento, tenemos:
        \begin{table}[H]
            \centering
            \begin{tabular}{ccccc}
                \ce{CH4}                                                 & + \ce{O2}                                                & \ce{->} & \ce{CO2 }                                                & \ce{H2O}                                                 \\
                \includegraphics[height=0.5cm]{../images/20230415003537} & \includegraphics[height=0.5cm]{../images/20230415003542} &         & \includegraphics[height=0.5cm]{../images/20230415003547} & \includegraphics[height=0.5cm]{../images/20230415003551} \\
            \end{tabular}
        \end{table}
        Hay 4 H en los reactivos y 2 en los productos, por lo que hay que multiplicar por 2 al \ce{H2O}.
        \begin{table}[H]
            \centering
            \begin{tabular}{ccccc}
                \ce{CH4}                                                 & + \ce{O2}                                                & \ce{->} & \ce{CO2 }                                                & \ce{2H2O}                                                \\
                \includegraphics[height=0.5cm]{../images/20230415003537} & \includegraphics[height=0.5cm]{../images/20230415003542} &         & \includegraphics[height=0.5cm]{../images/20230415003547} & \includegraphics[height=0.5cm]{../images/20230415003551} \\[-0.5em]
                                                                         &                                                          &         &                                                          & \includegraphics[height=0.5cm]{../images/20230415003551}
            \end{tabular}
        \end{table}

        Ahora hay 4 O en los productos y 2 en los reactivos, por lo que hay que multiplicar por 2 al \ce{O2}. Y la ecuación balanceada es:
        \begin{table}[H]
            \centering
            \begin{tabular}{ccccc}
                \ce{CH4}                                                 & + \ce{2O2}                                               & \ce{->} & \ce{CO2 }                                                & \ce{2H2O}                                                \\
                \includegraphics[height=0.5cm]{../images/20230415003537} & \includegraphics[height=0.5cm]{../images/20230415003542} &         & \includegraphics[height=0.5cm]{../images/20230415003547} & \includegraphics[height=0.5cm]{../images/20230415003551} \\[-0.5em]
                                                                         & \includegraphics[height=0.5cm]{../images/20230415003542} &         &                                                          & \includegraphics[height=0.5cm]{../images/20230415003551}
            \end{tabular}
        \end{table}
        Por lo tanto, la ecuación química balanceada es:
        \[
            \ce{CH4 + 2O2 -> CO2 + 2H2O}
        \]
    \end{multicols}
\end{solutionbox}