Un suplemento de magnesio de 1.02 g contiene 25.0\% Mg por masa. El magnesio está presente en el suplemento en forma de \ce{MgO}(s) (masa molar 40.3 g/mol).
\textbf{¿Cuántos gramos de \ce{MgO}(s) hay en el suplemento de magnesio?}
\emph{Escribe tu respuesta usando tres cifras significativas.}


\begin{solutionbox}{8cm}
    El porcentaje de masa de un elemento en una mezcla puede decirnos cuántos gramos hay y por ende cuántos moles hay del elemento en la mezcla. Sí el elemento es parte de un compuesto, también podemos determinar cuántos gramos y moles del compuesto hay en la mezcla.
    Primero, utilicemos el porcentaje de masa de Mg en el suplemento:

    \[1.02 \text{g suplemento} \times \frac{25.0\text{\% Mg}}{100} = 0.255\text{g Mg}\]

    Posteriormente, usando la masa calculada y la masa molar del Mg podemos calcular cuántos moles de Mg hay y, entonces, cuántos moles de \ce{MgO}(s) hay en el suplemento:

    \[0.255\text{g Mg} \times \frac{1\text{mol Mg}}{24.3\text{g Mg}} \times \frac{1\text{mol \ce{MgO}(s)}}{1\text{mol Mg}} = 0.0105\text{mol \ce{MgO}(s)}\]

    Finalmente, usaremos la masa molar de \ce{MgO} para convertir moles de \ce{MgO} a gramos de \ce{MgO}:

    \[0.0105\text{mol \ce{MgO}(s)} \times \frac{40.3\text{g \ce{MgO}(s)}}{1\text{mol \ce{MgO}(s)}} = 0.423\text{g \ce{MgO}(s)}\]

    El suplemento de magnesio contiene 0.423 g de \ce{MgO}(s).
\end{solutionbox}