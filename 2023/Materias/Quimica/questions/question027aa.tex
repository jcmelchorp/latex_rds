Una tableta de vitamina C de 2.70 g contiene 0.0109 mol de ácido ascórbico (\ce{C6H8O6}). La masa molar de \ce{C6H8O6} es 176.12 g/mol.
\textbf{¿Cuál es el porcentaje de masa de \ce{C6H8O6} en la tableta?}\\

\begin{solutionbox}{7cm}
    El porcentaje de masa de una sustancia en una mezcla se puede determinar por la comparación de la masa de la sustancia en la mezcla contra la masa total de la mezcla.
    Primero, calculemos la masa de \ce{C6H8O6} en la tableta. Utilizando la masa molar del \ce{C6H8O6}, podemos convertir moles de \ce{C6H8O6} a gramos de \ce{C6H8O6}:
    \[0.0109\text{mol \ce{C6H8O6}} \times \frac{176.12\text{g \ce{C6H8O6}}}{1\text{mol \ce{C6H8O6}}} = 1.92 \text{g \ce{C6H8O6}}\]
    Posteriormente, utilizando la masa calculada de \ce{C6H8O6} y la masa total de la tableta, podemos calcular el porcentaje de masa de \ce{C6H8O6} en la tableta:
    \[1.92\text{g \ce{C6H8O6}} \times \frac{100\%}{2.70\text{g tableta}} = 71\%\]
    El porcentaje de masa de \ce{C6H8O6} en la tableta es 71\%.
\end{solutionbox}