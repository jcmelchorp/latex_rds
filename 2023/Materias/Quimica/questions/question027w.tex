La masa molar del bismuto (Bi) es 208,98 g/mol.
\textbf{Calcula la masa en gramos de una muestra de bismuto (\ce{Bi}) que contiene $7.35 \times 10^{23}$ átomos.}\\
\emph{Escribe tu respuesta usando sólo números enteros.}


\begin{solutionbox}{6cm}
    Podemos usar el número de Avogadro, $6.023\times 10^{23}$, para convertir de moles a partículas representativas (como átomos, moléculas o unidades de fórmula) a moles de una sustancia. Después podemos usar la masa molar de la sustancia para convertir de moles a gramos.
    Primero, usemos el número de Avogadro para convertir átomos de \ce{Bi} a moles de \ce{Bi}:
    \[7.35\times 10^{23} \text{átomos Bi}\times\frac{1\,\text{mol\,Bi}}{6.023\times 10^{23}\,\text{átomos\,Bi}}=1.22\,\text{mol\,Bi} \]
    Después, usemos la masa molar de \ce{Bi} para convertir de moles de \ce{Bi} a gramos de \ce{Bi}:
    \[1.22\,\text{mol\,Bi}\times\frac{208.98\,\text{g\,Bi}}{1\,\text{mol\,Bi}}=255\,\text{g\,Bi} \]
    Por lo tanto, una muestra de 7.35$\times 10^{23}$ átomos de \ce{Bi} tiene una masa de 255 g.
\end{solutionbox}