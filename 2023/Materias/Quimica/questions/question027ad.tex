Un estudiante determina que 1.5 g de una mezcla de \ce{CaCO3}(s) y \ce{NaHCO3}(s) contiene 0.010 mol de \ce{CaCO3}(s).\\
\textbf{Con base en los resultados del estudiante, ¿cuál es el porcentaje de masa de Na en la mezcla?}\\

\begin{oneparchoices}
    \choice  10\%
    \CorrectChoice  15\%
    \choice  20\%
    \choice  25\%
\end{oneparchoices}

\begin{solutionbox}{6cm}
    El porcentaje de masa de una sustancia en una mezcla se puede determinar por la comparación de la masa de la sustancia en la mezcla contra la masa total de la mezcla.
    Primero, determinemos la masa de Na en la mezcla. Utilizando los resultados del estudiante y la masa molar del Na, podemos encontrar el número de moles de Na en la muestra y después convertirlos a gramos:
    \[0.010\text{mol \ce{NaHCO3}(s)} \times \frac{1\text{mol Na}}{1\text{mol \ce{NaHCO3}(s)}} \times \frac{22.99\text{g Na}}{1\text{mol Na}} = 0.230\text{g Na}\]
    Posteriormente, usando la masa calculada del Na y la masa total de la mezcla, podemos encontrar el porcentaje de masa del Na en la mezcla:
\[0.230\text{g Na} \times \frac{100}{1.5\text{g mezcla}} = 15\text{\% Na}\]
    El porcentaje de masa de Na en la mezcla es 15\%.
\end{solutionbox}
