\question[5] Señala en cada uno de los enunciados si la sentencia es falsa o verdadera.

\begin{multicols}{2}
    \begin{parts}
        \part La tabla periódica se encuentra
        constituida por filas (períodos) y
        columnas (grupos).

        \begin{oneparchoices}
            \CorrectChoice Verdadero
            \choice Falso
        \end{oneparchoices}

        \part  A los elementos del subgrupo A,
        se les denomina representativos.
        Los elementos de transición se
        ubican en el subgrupo B.

        \begin{oneparchoices}
            \CorrectChoice Verdadero
            \choice Falso
        \end{oneparchoices}

        % \part  La tabla periódica actual se puede
        % dividir en 5 bloques; s, p, d, f y g.

        % \begin{oneparchoices}
        %     \choice Verdadero
        %     \choice Falso
        % \end{oneparchoices}

        % \part  El bloque s está constituido por los grupos I y II A.

        % \begin{oneparchoices}
        %     \choice Verdadero
        %     \choice Falso
        % \end{oneparchoices}

        \part  Los electrones de valencia se encuentran siempre en el penúltimo
        nivel de energía.

        \begin{oneparchoices}
            \choice Verdadero
            \CorrectChoice Falso
        \end{oneparchoices}

        % \part  Si la configuración electrónica de
        % un elemento termina en s o en p,
        % pertenece al subgrupo B.

        % \begin{oneparchoices}
        %     \choice Verdadero
        %     \choice Falso
        % \end{oneparchoices}

        \part  El oxígeno y el nitrógeno son dos gases nobles de gran importancia.

        \begin{oneparchoices}
            \choice Verdadero
            \CorrectChoice Falso
        \end{oneparchoices}
        {
        \printanswers
        \part  Los metales se ubican a la derecha y al centro de la tabla periódica.

        \begin{oneparchoices}
            \choice Verdadero
            \CorrectChoice Falso
        \end{oneparchoices}
        }
        \columnbreak
        \part  Las mejores fuentes de hierro
        son las visceras, el hígado, los
        quelites, acelgas y espinacas.

        \begin{oneparchoices}
            \CorrectChoice Verdadero
            \choice Falso
        \end{oneparchoices}
        {
        \printanswers
        \part  El mercurio es el único elemento
        líquido.

        \begin{oneparchoices}
            \choice Verdadero
            \CorrectChoice Falso
        \end{oneparchoices}
        }
        \part   El magnesio es el constituyente
        esencial de la clorofila en las plantas verdes.

        \begin{oneparchoices}
            \CorrectChoice Verdadero
            \choice Falso
        \end{oneparchoices}

        \part   La deficiencia de yodo es la causa
        del bocio en los humanos.

        \begin{oneparchoices}
            \CorrectChoice Verdadero
            \choice Falso
        \end{oneparchoices}
        {
        \printanswers
        \part   Los no metales son maleables,
        dúctiles y buenos conductores del
        calor y la electricidad.

        \begin{oneparchoices}
            \choice Verdadero
            \CorrectChoice Falso
        \end{oneparchoices}
        }
        \part   Los metaloides se ubican arriba y
        abajo de la línea diagonal que divde a los metales de los no metales.

        \begin{oneparchoices}
            \choice Verdadero
            \CorrectChoice Falso
        \end{oneparchoices}
    \end{parts}
\end{multicols}

