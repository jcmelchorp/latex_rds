\question Señala en cada uno de los enunciados si la sentencia es falsa o verdadera.

\begin{parts}
    \part[2] La tabla periódica se encuentra
    constituida por filas (perìodos) y
    columnas (grupos).
    \part[2] A los elementos del subgrupo A,
    se les denomina representativos.
    Los elementos de transición se
    ubican en el subgrupo B.
    \part[2] La tabla periódica actual se puede
    dividir en 5 bloques; s, p, d, f y g.
    \part[2] El bloque s está constituido por los
    grupos I y II A.
    \part[2] Los electrones de valencia se en-
    cuentran siempre en el penúltimo
    nivel de energía.
    \part[2] Si la configuración electrónica de
    un elemento termina en s o en p,
    pertenece al subgrupo B.
    \part[2] El oxígeno y el nitrógeno son dos
    gases nobles de gran importancia.
    \part[2] Los metales se ubican a la dere-
    cha y al centro de la tabla periódi-
    ca.
    \part[2] Las mejores fuentes de hierro
    son las visceras, el hígado, los
    quelites, acelgas y espinacas.
    \part[2] El mercurio es el único elemento
    líquido.
    \part[2]  El magnesio es el constituyente
    esencial de la clorofila en las plan-
    tas verdes.
    \part[2]  La deficiencia de yodo es la causa
    del bocio en los humanos.
    \part[2]  Los no metales son maleables,
    dúctiles y buenos conductores del
    calor y la electricidad.
    \part[2]  Los metaloides se ubican arriba y
    abajo de la línea diagonal que divi-
    de a los metales de los no metales.
\end{parts}

