Contesta a las siguientes preguntas, argumentando ampliamente tu respuesta.

\begin{parts}
    \part Explica bajo qué condiciones el número atómico permite deducir el número de electrones presentes en un
    átomo.
    \begin{solutionbox}{2cm}
        El número atómico Z se relaciona con la cantidad de protones en un átomo. Si consideramos un átomo eléctricamente neutro, la cantidad de electrones deberá ser la misma.
    \end{solutionbox}
    % \part ¿Por qué todos los átomos de un mismo elemento tienen el mismo número atómico, a pesar
    % de que pueden tener diferente número de masa?
    % \begin{solutionbox}{3cm}
    % \end{solutionbox}
    % \part ¿Cuál es el número de masa de un átomo de cobalto que tiene 30 neutrones?
    % \begin{solutionbox}{3cm}
    % \end{solutionbox}
    \part En términos generales, el radio de un átomo es aproximadamente 10,000 veces mayor que su
    núcleo. Si un átomo pudiera amplificarse de manera que el radio de su núcleo midiera 2 mm (lo que mide un grano de sal), ¿cuál sería el radio del átomo en metros?
    \begin{solutionbox}{2cm}
        \[ 10,000 \times 2 \text{ mm}=20,000 \text{ mm} = 20 m \]
    \end{solutionbox}
\end{parts}
