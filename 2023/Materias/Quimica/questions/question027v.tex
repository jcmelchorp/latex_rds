La masa molar de la plata (Ag) es 107,87 g/mol.
\textbf{Calcula la masa en gramos de una muestra de plata (\ce{Ag}) que contiene $1.97 \times 10^{22}$ átomos.}\\
\emph{Escribe tu respuesta usando tres cifras significativas.}

\begin{solutionbox}{6cm}
    Podemos usar el número de Avogadro, $6.023\times 10^{23}$, para convertir de moles a partículas representativas (como átomos, moléculas o unidades de fórmula) a moles de una sustancia. Después podemos usar la masa molar de la sustancia para convertir de moles a gramos.
    Primero, usemos el número de Avogadro para convertir átomos de \ce{Ag} a moles de \ce{Ag}:
    \[1.97\times 10^{22} \text{átomos Ag}\times\frac{1\,\text{mol\,Ag}}{6.023\times 10^{23}\,\text{átomos\,Ag}}=0.0327\,\text{mol\,Ag} \]
    Después, usemos la masa molar de \ce{Ag} para convertir de moles de \ce{Ag} a gramos de \ce{Ag}:
    \[0.0327\,\text{mol\,Ag}\times\frac{107.87\,\text{g\,Ag}}{1\,\text{mol\,Ag}}=3.53\,\text{g\,Ag} \]
    Por lo tanto, una muestra de 1.97$\times 10^{22}$ átomos de \ce{Ag} tiene una masa de 3.53 g.
\end{solutionbox}