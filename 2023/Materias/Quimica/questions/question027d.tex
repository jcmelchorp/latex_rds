Una muestra pura de un compuesto contiene 80\% de carbono y 20\% de hidrógeno por masa.
\textbf{¿Cuál es la fórmula que corresponde a este compuesto?}

\begin{oneparchoices}
    \CorrectChoice \ce{CH3 }
    \choice \ce{CH4 }
    \choice \ce{C3H6}
    \choice \ce{C2H8}
\end{oneparchoices}

\begin{solutionbox}{8cm}
    La fórmula empírica de un compuesto es la proporción en números enteros más simple de elementos en el compuesto.
    Si un compuesto es 80 \% carbono y 20\% hidrógeno por masa, entonces 100 g del compuesto contendrían 80 g de carbono y 20 g de hidrógeno. Utilizando estos valores y las masas molares del carbono y del hidrógeno podemos determinar el número de moles de cada elemento en 100 g del compuesto:

    \[80 \text{ g \ce{C}} \times \frac{1 \text{ mol \ce{C}}}{12.01 \text{ g \ce{C}}} = 6.66 \text{ mol \ce{C}} \]
    \[20 \text{ g \ce{H}} \times \frac{1 \text{ mol \ce{H}}}{1.01 \text{ g \ce{H}}} = 19.8 \text{ mol \ce{H}} \]

    Por lo tanto, la muestra contienen aproximadamente 6.66 moles de átomos de carbono y 19.8 moles de átomos de hidrógeno.
    A continuación, podemos encontrar la relación en números enteros más simple de átomos en el compuesto dividiendo cada uno de los valores molares del azufre y del oxígeno entre el menor de estos dos valores:
    \[ \frac{6.66 \text{ mol \ce{C}}}{6.66 \text{ mol \ce{C}}} = 1 \qquad \frac{19.8 \text{ mol \ce{H}}}{6.66 \text{ mol \ce{C}}} = 3 \]
    Entonces, hay 1 átomo de carbono por cada 3 átomos de hidrógeno en el compuesto.
    La fórmula empírica del compuesto es \ce{CH3}.
\end{solutionbox}