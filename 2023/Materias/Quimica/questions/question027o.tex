La masa molar del Galio (Ga) es 69,72 g/mol.
\textbf{Calcula el número de átomos en una muestra de 27.2 mg de Ga.}\\
\emph{Escribe tu respuesta en notación científica usando tres cifras significativas.}

\begin{solutionbox}{6cm}
    Podemos usar la masa molar de la sustancia para convertir gramos a moles de sustancia. Después, podemos usar el número de Avogadro, $6.023 \times 10^{23}$, para convertir moles a  partículas representativas (como átomos, moléculas o unidades de fórmula).
    Primero, usemos la masa molar de \ce{Ga} para convertir de gramos de \ce{Ga} a moles de \ce{Ga}:
    \[27.2 \text{ mg \ce{Ga}} \times \frac{1 \text{ g \ce{Ga}}}{1000 \text{ mg \ce{Ga}}} \times \frac{1 \text{ mol \ce{Ga}}}{69.72 \text{ g \ce{Ga}}} \approx 3.90 \times 10^{-4} \text{ mol \ce{Ga}} \]
    Después, usemos el número de Avogadro para convertir de moles de \ce{Ga} a átomos de \ce{Ga}:
    \[3.90 \times 10^{-4} \text{ mol \ce{Ga}} \times \frac{6.023 \times 10^{23} \text{ átomos \ce{Ga}}}{1 \text{ mol \ce{Ga}}} \approx 2.35 \times 10^{20} \text{ átomos \ce{Ga}} \]
    Por lo tanto, una muestra de 27.2 mg de \ce{Ga} tiene 2.35$\times 10^{20}$ átomos.
\end{solutionbox}