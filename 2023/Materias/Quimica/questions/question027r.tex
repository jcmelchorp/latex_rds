\begin{multicols}{2}
    Usando la información de la tabla \ref{tab:q001},
    \textbf{Calcula el número de moles en una muestra de 5.73 kg de ácido láctico (\ce{C3H6O3}).}\\
    \emph{Escribe tu respuesta usando tres cifras significativas.}

    \begin{table}[H]
        \centering
        \caption{Masa molar de algunos elementos.}
        \label{tab:q001}
        \begin{tabular}{c|p{2.2cm}}
            \textbf{Elemento} & \textbf{Masa molar (g/mol)} \\\midrule
            H                 & 1.008                       \\\hline
            C                 & 12.01                       \\\hline
            O                 & 16.00                       \\\hline
            \bottomrule
        \end{tabular}
    \end{table}
\end{multicols}

\begin{solutionbox}{8cm}
    Podemos usar la masa molar de una sustancia para convertir gramos a moles de sustancia. Con esto en mente, primero calculemos la masa molar de \ce{C3H6O3}, usando la tabla \ref{tab:q001}:
    \[3 \text{ mol de C} \times 12.01 \text{g/mol} = 36.03 \text{g}\]
    \[6 \text{ mol de H} \times 1.008 \text{g/mol} = 6.048 \text{g}\]
    \[3 \text{ mol de O} \times 16.00 \text{g/mol} = 48.00 \text{g}\]
    \[\text{Masa molar} = 36.03 + 6.048 + 48.00 = 90.08 \text{g/mol}\]
    Por lo tanto, 1 mol de \ce{C3H6O3} tiene una masa molar de 90.08 g/mol.
    Después, usemos la masa molar de \ce{C3H6O3} para convertir gramos a moles de \ce{C3H6O3}.
    Ya que nos estan dando la masa de \ce{C3H6O3} en kilogramos, también necesitaremos incluir el factor de conversión de kilogramos a gramos:
    \[5.73 \text{kg} \times \frac{1000 \text{g}}{1 \text{kg}} \times \frac{1 \text{mol}}{90.08 \text{g}} = 63.6 \text{mol}\]
    Por lo tanto, 5.73 kg de \ce{C3H6O3} es equivalente a 63.6 moles de \ce{C3H6O3}.
\end{solutionbox}