\begin{multicols}{2}
    Usando la información de la tabla \ref{tab:q005},
\textbf{Calcula el número de unidades fórmula en una muestra de 27.4 g de cloruro de calcio (\ce{CaCl2}).}\\
\emph{Escribe tu respuesta usando tres cifras significativas.}

    \begin{table}[H]
        \centering
        \caption{Masa molar de algunos elementos.}
        \label{tab:q005}
        \begin{tabular}{c|p{2.2cm}}
            \textbf{Elemento} & \textbf{Masa molar (g/mol)} \\\midrule
            Cl                & 35.45                       \\\hline
            Ca                & 40.08                       \\\hline
            \bottomrule
        \end{tabular}
    \end{table}

\end{multicols}

\begin{solutionbox}{10cm}
    Podemos usar la masa molar de la sustancia para convertir gramos a moles de sustancia. Después, podemos usar el número de Avogadro, $6.023 \times 10^{23}$, para convertir moles a partículas representativas (como átomos, moléculas o unidades de fórmula). Con esto en mente, calculemos primero la masa molar de \ce{CaCl2}, usando la tabla \ref{tab:q005}:
    \[1 \text{ mol de Ca} \times 40.08 \text{g/mol} = 40.08 \text{g}\]
    \[2 \text{ mol de Cl} \times 35.45 \text{g/mol} = 70.90 \text{g}\]
    \[\text{Masa molar} = 40.08 + 70.90 = 110.0 \text{g/mol}\]
    Por lo tanto, 1 mol de \ce{CaCl2} tiene una masa molar de 110.0 g/mol.
    Después, usemos la masa molar de \ce{CaCl2} para convertir gramos a moles de \ce{CaCl2}.
\[27.4 \text{g} \times \frac{1 \text{mol}}{110.0 \text{g}} = 0.249 \text{mol}\]
    Por lo tanto, hay 0.249 moles de \ce{CaCl2} en 27.4 g de \ce{CaCl2}.
    Finalmente, usemos el número de Avogadro para convertir moles a unidades de fórmula:
    \[0.249 \text{mol} \times \frac{6.023 \times 10^{23} \text{ unidades de fórmula}}{1 \text{mol}} = 1.50 \times 10^{23} \text{ unidades de fórmula}\]
    Por lo tanto, hay $1.50 \times 10^{23}$ unidades de fórmula en 27.4 g de \ce{CaCl2}.
\end{solutionbox}