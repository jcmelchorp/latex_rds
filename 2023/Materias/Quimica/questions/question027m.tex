La masa molar del estroncio (Sr) es 87.62 g/mol.
\textbf{Calcula el número de átomos en una muestra de 67.5 mg de Sr.}\\
\emph{Escribe tu respuesta en notación científica usando tres cifras significativas.}

\begin{solutionbox}{6cm}
    Podemos usar la masa molar de la sustancia para convertir gramos a moles de sustancia. Después, podemos usar el número de Avogadro, $6.023 \times 10^{23}$, para convertir moles a partículas representativas (como átomos, moléculas o unidades de fórmula).
    Primero, usemos la masa molar del \ce{Sr} para convertir gramos de \ce{Sr} a moles de \ce{Sr}:

    \[ 67.2 \text{ mg} \times \frac{1 \text{ g}}{1000 \text{ mg}} \times \frac{1 \text{ mol}}{87.62 \text{ g \ce{Sr}}} = 7.70 \times 10^{-4} \text{ mol} \]

    Después, usemos el número de Avogadro para convertir moles de \ce{Sr} a átomos de \ce{Sr}:

    \[7.7 \times 10^{-4} \text{ mol} \times \frac{6.023 \times 10^{23} \text{ átomos}}{1 \text{ mol}} = 4.64 \times 10^{20} \text{ átomos} \]
    Por lo tanto, una muestra de 67.2 mg de \ce{Sr} contiene $4.64 \times 10^{20}$ átomos de \ce{Sr}.
\end{solutionbox}