\question Relaciona cada elemento con las características que le corresponden.

\begin{minipage}{0.6\textwidth}
    \begin{parts}
        \part[2] Elemento del grupo VII, con siete electrones en la última capa de valencia, ubicado en el tercer periodo de la tabla periódica.
        \part[2] Elemento metaloide, ubicado en el tercer periodo de la tabla periódica.
        \part[2] Elemento conocido como gas noble.
        \part[2] Elemento metálico con Z = 11.
        \part[2] Elemento que se ubica en el grupo 15 y en el periodo 2 de la tabla periódica.
        \part[2] Elemento con 12 protones y 12 electrones.
        \part[2] Elemento no metálico con Z =34
        \part[2] Ejemplo de gas inerte.
    \end{parts}
\end{minipage}\hfill
\begin{minipage}{0.25\textwidth}
    \begin{itemize}
        \item[\rule{1cm}{0.2mm}] Selenio
        \item[\rule{1cm}{0.2mm}] Neón
        \item[\rule{1cm}{0.2mm}] Magnesio
        \item[\rule{1cm}{0.2mm}] Xenón
        \item[\rule{1cm}{0.2mm}] Sodio
        \item[\rule{1cm}{0.2mm}] Cloro
        \item[\rule{1cm}{0.2mm}] Nitrógeno
        \item[\rule{1cm}{0.2mm}] Silicio
    \end{itemize}
\end{minipage}