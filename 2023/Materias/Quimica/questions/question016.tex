Señala la opción que responde correctamente a la pregunta de cada uno de los siguientes incisos:

\begin{multicols}{2}
    \begin{parts}
        \part  ¿Qué propiedades periódicas aumentan al recorrer un grupo de arriba hacia abajo en la tabla
        periódica?
        \begin{choices}
            \CorrectChoice El carácter metálico y la electronegatividad
            \choice El potencial de Ionización y el carácter metálico
            \choice El carácter no metálico y el potencial de ionización
            \choice La electronegatividad y la afinidad electrónica
            \choice Ninguna de las anteriores
        \end{choices}

        \part  ¿Qué propiedades periódicas aumentan al desplazarnos en un período de izquierda a dere-
        cha en la tabla periódica?
        \begin{choices}
            \CorrectChoice La electronegatividad y el tamaño atómico
            \choice El radio atómico y el radio iónico
            \choice El carácter metálico y la afinidad electrónica
            \choice Potencial de ionización y electronegatividad
            \choice Ninguna de las anteriores
        \end{choices}

        \part En la tabla periódica, el tamaño atómico tiende a aumentar hacia la:

        \begin{choices}
            \choice  Derecha y hacia arriba
            \CorrectChoice  Derecha y hacia abajo
            \choice  Izquierda y hacia arriba
            \choice  Izquierda y hacia abajo
        \end{choices}

        \columnbreak
        \part El tamaño de los átomos aumenta cuando:

        \begin{choices}
            \choice Se incrementa el número de período
            \choice  Disminuye el número de período
            \choice  Se incrementa el número de grupo
            \choice  Disminuye el número de bloque
            \choice  Ninguna de las anteriores
        \end{choices}

        \part El radio atómico es la distancia que hay del núcleo de un átomo a su electrón más lejano
        ¿Cómo varía esta propiedad atómica en los elementos de la tabla periódica?

        \begin{choices}
            \choice  Disminuye conforme nos desplazamos de izquierda a derecha a lo largo de un período
            \choice  Aumenta conforme nos desplazamos de arriba hacia abajo a lo largo de un grupo
            \choice  Aumenta conforme nos desplazamos de derecha a izquierda a lo largo de un período
            \choice  Todos son correctos
        \end{choices}

    \end{parts}
\end{multicols}
