Balancea la siguiente ecuación química:

\[
    \ce{NH4NO3 -> N2 + H2O + O2}
\]

\begin{solutionbox}{5.5cm}
    Hay 4 H en el reactivo y 2 en el producto, por lo que el coeficiente de H2O es 2.
    \[
        \ce{NH4NO3 -> N2 + 2H2O + O2}
    \]
    Hay 3 O en los reactivos y 4 los productos, por lo que si intentamos dar al \ce{O2} un coeficiente de 1/2, nos da 3 oxígenos en ambos lados.
    \[
        \ce{NH4NO3 -> N2 + 2H2O + 1/2O2}
    \]
    Dado que usualmente no se usan fracciones como coeficientes, multiplicamos todo por 2 para deshacernos de la fracción, y la ecuación balanceada es:
    \[
        \ce{2NH4NO3 -> 2N2 + 4H2O + O2}
    \]
\end{solutionbox}