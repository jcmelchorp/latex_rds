\question Señala en cada uno de los enunciados si la sentencia es falsa o verdadera.

\begin{parts}
    \part[2] El número de oxidación de un elemento combinado con otro, es cero.
    \part[2]La fórmula H 2 O expresa que la molécula de agua está constituida por dos átomos
    de hidrógeno y uno de oxígeno.
    \part[2]En la fórmula 3 NH 4 Cl el coeficiente es el 4.
    \part[2]Al número entero positivo, negativo o cero que se asigna a cada elemento en
    un compuesto, se denomina número de oxidación.
    \part[2]En la construcción de una fórmula química se escribe primero la parte negativa y enseguida la positiva.
    \part[2]Los subíndices expresan el número de átomos de los elementos presentes en
    una molécula o unidad fórmula.
    \part[2]El símbolo Cl - indica que el átomo de cloro ha tenido una reducción o pérdida de
    electrones.
    \part[2]Una fórmula química sólo expresa la composición cualitativa de una sustancia.
    \part[2]En una fórmula química, los coeficientes indican el número de moléculas o unidades fórmula; así como también el número de moles presentes de la sustancia.
    \part[2] En una fórmula química la suma de cargas negativas y positivas siempre es
    mayor de cero.
    El electrón es una partícula subatómica que se encuentra ubicada en el núcleo atómico.
    \part[2] La carga eléctrica del electrón es -1.602 × 10 -19 Culombios
    \part[2] El electrón tiene una masa que es aproximadamente 1836 veces menor con respecto a la del neutrón.
    \part[2] El neutrón es una partícula subatómica que se encuentra girando alrededor del núcleo atómico.
    \part[2] La masa de un neutrón es similar a la del protón.
    \part[2] Las únicas partículas elementales en el núcleo, son los protones y neutrones.
    \part[2] El número de masa representa la suma de protones y neutrones.
    \part[2] El número de electrones determina el número atómico de un átomo.
    \part[2]  Los protones y neutrones son partículas constituidas por quarks.
\end{parts}
