\question[5] Señala en cada uno de los enunciados si la sentencia es falsa o verdadera.

\begin{multicols}{2}
    \begin{parts}

        \part  El número de oxidación de un elemento combinado con otro, es cero.

        \begin{oneparchoices}
            \choice Verdadero
            \CorrectChoice Falso
        \end{oneparchoices}
        {
        \printanswers
        \part La fórmula H$_2$O expresa que la molécula de agua está constituida por dos átomos
        de hidrógeno y uno de oxígeno.

        \begin{oneparchoices}
            \CorrectChoice Verdadero
            \choice Falso
        \end{oneparchoices}
        }
        \part En la fórmula 3NH$_4$Cl el coeficiente es el 4.

        \begin{oneparchoices}
            \choice Verdadero
            \CorrectChoice Falso
        \end{oneparchoices}

        \part Al número entero positivo, negativo o cero que se asigna a cada elemento en
        un compuesto, se denomina número de oxidación.

        \begin{oneparchoices}
            \choice Verdadero
            \CorrectChoice Falso
        \end{oneparchoices}


        \part En la construcción de una fórmula química se escribe primero la parte negativa y enseguida la positiva.

        \begin{oneparchoices}
            \choice Verdadero
            \CorrectChoice Falso
        \end{oneparchoices}

        \part Los subíndices expresan el número de átomos de los elementos presentes en
        una molécula o unidad fórmula.

        \begin{oneparchoices}
            \CorrectChoice Verdadero
            \choice Falso
        \end{oneparchoices}

        \part El símbolo Cl$^-$ indica que el átomo de cloro ha tenido una reducción o pérdida de
        electrones.

        \begin{oneparchoices}
            \choice Verdadero
            \CorrectChoice Falso
        \end{oneparchoices}

        \part Una fórmula química sólo expresa la composición cualitativa de una sustancia.

        \begin{oneparchoices}
            \choice Verdadero
            \CorrectChoice Falso
        \end{oneparchoices}
        {
        \printanswers
        \part En una fórmula química, los coeficientes indican el número de moléculas o unidades fórmula; así como también el número de moles presentes de la sustancia.

        \begin{oneparchoices}
            \CorrectChoice Verdadero
            \choice Falso
        \end{oneparchoices}
        }
        \part  En una fórmula química la suma de cargas negativas y positivas siempre es
        mayor de cero.
        El electrón es una partícula subatómica que se encuentra ubicada en el núcleo atómico.

        \begin{oneparchoices}
            \choice Verdadero
            \CorrectChoice Falso
        \end{oneparchoices}
        {
        \printanswers
        \part  La carga eléctrica del electrón es -1.602$\times$10$^{-19}$ Coulombs.

        \begin{oneparchoices}
            \CorrectChoice Verdadero
            \choice Falso
        \end{oneparchoices}
        }
        \part  El electrón tiene una masa que es aproximadamente 1836 veces menor con respecto a la del neutrón.

        \begin{oneparchoices}
            \CorrectChoice Verdadero
            \choice Falso
        \end{oneparchoices}
        {
        \printanswers
        \part  El neutrón es una partícula subatómica que se encuentra girando alrededor del núcleo atómico.

        \begin{oneparchoices}
            \choice Verdadero
            \CorrectChoice Falso
        \end{oneparchoices}
        }
        \part  La masa de un neutrón es similar a la del protón.

        \begin{oneparchoices}
            \CorrectChoice Verdadero
            \choice Falso
        \end{oneparchoices}

        \part  Las únicas partículas elementales en el núcleo, son los protones y neutrones.

        \begin{oneparchoices}
            \choice Verdadero
            \CorrectChoice Falso
        \end{oneparchoices}

        \part  El número de masa representa la suma de protones y neutrones.

        \begin{oneparchoices}
            \CorrectChoice Verdadero
            \choice Falso
        \end{oneparchoices}

        \part  El número de electrones determina el número atómico de un átomo.

        \begin{oneparchoices}
            \choice Verdadero
            \CorrectChoice Falso
        \end{oneparchoices}
        \part   Los protones y neutrones son partículas constituidas por quarks.

        \begin{oneparchoices}
            \CorrectChoice Verdadero
            \choice Falso
        \end{oneparchoices}
    \end{parts}
\end{multicols}
