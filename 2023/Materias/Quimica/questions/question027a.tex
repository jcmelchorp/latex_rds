Con base en la información de la tabla \ref{tab:q01}, \textbf{¿cuál de los siguientes compuestos contiene el mayor porcentaje de potasio por masa?}

\begin{multicols}{2}
    \begin{oneparchoices}
        \choice \ce{KNO3}
        \CorrectChoice \ce{KF}
        \choice \ce{KClO}
        \choice \ce{KBr}
    \end{oneparchoices}

    \begin{table}[H]
        \centering
        \caption{Compuestos que contienen potasio}
        \label{tab:q01}
        \begin{tabular}{r|p{2.2cm}|p{2.4cm}}
            \textbf{Compuesto} & \textbf{Masa molar (g/mol)} & \textbf{Porcentaje de potasio (\%)} \\ \midrule
            \ce{KNO3}          & 101.1                       & \ifprintanswers{38.67\%}\fi         \\ \hline
            \ce{KF}            & 58.1                        & \ifprintanswers{67.3\% }\fi         \\ \hline
            \ce{KClO}          & 90.6                        & \ifprintanswers{43.1\% }\fi         \\ \hline
            \ce{KBr}           & 119.0                       & \ifprintanswers{33.1\% }\fi         \\ \hline
            \bottomrule
        \end{tabular}
    \end{table}

    \columnbreak

    \begin{solutionbox}{7cm}
        Ya que el peso atómico del potasio es 39.1, el porcentaje de potasio en cada compuesto se puede calcular como:
        \[100\% \times \dfrac{\ce{K}}{\ce{KNO3}}=100\% \times \dfrac{39.1}{101.1}=38.67\%\]
        \[100\% \times \dfrac{\ce{K}}{\ce{KF}}  =100\% \times \dfrac{39.1}{58.1} =67.3\% \]
        \[100\% \times \dfrac{\ce{K}}{\ce{KClO}}=100\% \times \dfrac{39.1}{90.6} =43.1\% \]
        \[100\% \times \dfrac{\ce{K}}{\ce{KBr}} =100\% \times \dfrac{39.1}{119.0}=33.1\% \]
    \end{solutionbox}
\end{multicols}


