Una muestra de un compuesto que contiene únicamente átomos de carbono e hidrógeno se combustiona completamente produciendo 11.0 g de \ce{CO2} y 4.5 g de \ce{H2O}.
\textbf{¿Cuál es la fórmula que corresponde a este compuesto?}

\begin{oneparchoices}
    \choice  \ce{CH}
    \CorrectChoice  \ce{CH2}
    \choice  \ce{C2H3}
    \choice  \ce{C2H4}
\end{oneparchoices}

\begin{solutionbox}{8cm}
    Cuando un compuesto que contiene únicamente carbono e hidrógeno se combustiona completamente, todos los átomos de carbono terminan formando \ce{CO2} y todos los átomos de hidrógeno terminan formando \ce{H2O}.
    Primero, usemos las masas molares del \ce{CO2} y \ce{H2O} para determinar cuantos moles de carbono e hidrógeno había en la muestra del compuesto antes de su combustión:
    
    \[ 11.0 g \text{ \ce{CO2}} \times \frac{1 \text{ mol \ce{CO2}}}{44.01 \text{ g \ce{CO2}}} \times \dfrac{ 1 \text{ mol \ce{C}}}{1 \text{ mol \ce{CO2}}} \approx 0.25 \text{ mol \ce{CO2}}  \] 
    \[ 4.5 g \text{ \ce{H2O}} \times \frac{1 \text{ mol \ce{H2O}}}{18.02 \text{ g \ce{H2O}}}  \times \dfrac{ 1 \text{ mol \ce{H}}}{1 \text{ mol \ce{H2O}}} \approx 0.5 \text{ mol \ce{H}} \]
    
    Por lo tanto, la muestra contienen aproximadamente 0.25 moles de átomos de carbono y 0.5 moles de átomos de hidrógeno.
    A continuación, encontremos la relación en números enteros más simple de átomos en el compuesto dividiendo cada uno de los valores molares del carbono e hidrógeno entre el menor de estos dos valores:
    
    \[ \frac{0.25 \text{ mol \ce{C}}}{0.25 \text{ mol \ce{C}}} = 1 \qquad \frac{0.5 \text{ mol \ce{H}}}{0.25 \text{ mol \ce{C}}} = 2 \]
    
    Entonces, por cada átomo de carbono en el compuesto hay 2 átomos de hidrógeno.
    La fórmula empírica del compuesto es \ce{CH2}.
\end{solutionbox}