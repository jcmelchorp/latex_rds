\question[10] Relaciona la especie química con la cantidad de \textbf{protones} y \textbf{electrones de valencia}.\\

\begin{minipage}{0.4\textwidth}
    \large
    \begin{choices}
        \choice Ión oxígeno (O$_2^-$)
        \choice Nitrógeno (N)
        \choice Silicio (Si)
        \choice Calcio (Ca)
        \choice Oxígeno (O$_2$)
        \choice Neón (Ne)
        \CorrectChoice Ión Litio (Li$^+$)
        \choice Fósforo (P)
        \choice Selenio (Se)
    \end{choices}
\end{minipage}\hfill
\begin{minipage}{0.6\textwidth}
    \begin{parts}
        {
            \printanswers
            \part \fillin[D][1cm] 20 protones y 2 electrones de valencia.
        }
        \part \fillin[H][1cm] 15 protones y 5 electrones de valencia.
        \part \fillin[J][1cm] 47 protones y 1 electrón de valencia.
        \part \fillin[A][1cm] 8 protones y 7 electrones de valencia.
        \part \fillin[I][1cm] 34 protones y 6 electrones de valencia.
        \part \fillin[C][1cm] 14 protones y 4 electrones de valencia.
            {
                \printanswers
                \part \fillin[B][1cm] 7 protones y 5 electrones de valencia.
            }
        \part \fillin[G][1cm] 3 protones y 2 electrones de valencia.
        \part \fillin[E][1cm] 8 protones y 6 electrones de valencia.
        \part \fillin[F][1cm] 10 protones y 8 electrones de valencia.
    \end{parts}
\end{minipage}
% \bohr{47}{Ag}