Relaciona los siguientes cambios químicos con el factor de rapidez que interviene en su realización.
\begin{multicols}{2}
    \begin{choices}
        \choice \adjustbox{valign=t}{
            \begin{minipage}[t]{5cm}
                Los procesos químicos que involucran reactivos sólidos pulverizados tienen mayor rapidez.
            \end{minipage}
        }
        $\square$\\
        \choice \adjustbox{valign=t}{
            \begin{minipage}[t]{5cm}
                La reacción entre los gases de H2 y O2 dura más tiempo si se realiza al nivel del mar.
            \end{minipage}
        }
        $\square$\\
        \choice \adjustbox{valign=t}{
            \begin{minipage}[t]{5cm}
                Los vegetales y otros alimentos se conservan más tiempo si se almacenan en lugares frescos.
            \end{minipage}
        }$\square$\\
        \choice \adjustbox{valign=t}{
            \begin{minipage}[t]{5cm}
                La oxidación de los metales tarda más tiempo si no están expuestos al oxígeno o al agua.
            \end{minipage}
        }$\square$\\
    \end{choices}
    \begin{checkboxes}
        \choice Concentración de los reactivos.      \vspace{1cm}
        \choice Presión.\vspace{1cm}
        \choice Temperatura.\vspace{1cm}
        \choice Estado de agregación.
    \end{checkboxes}
    \vspace{0.5cm}
\end{multicols}