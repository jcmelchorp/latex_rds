Un suplemento de calcio de 1.60 g contiene 37.8\% Ca por masa. El calcio está presente en el suplemento en forma de \ce{CaCO3}(s) (masa molar 100.09 g/mol).
\textbf{¿Cuántos gramos de \ce{CaCO3}(s) hay en el suplemento de calcio?}
\emph{Escribe tu respuesta usando tres cifras significativas.}


\begin{solutionbox}{8cm}
    El porcentaje de masa de un elemento en una mezcla puede decirnos cuántos gramos hay y por ende cuántos moles hay del elemento en la mezcla. Sí el elemento es parte de un compuesto, también podemos determinar cuántos gramos y moles del compuesto hay en la mezcla.
    Primero, utilicemos el porcentaje de masa de Ca en el suplemento:
    \[1.60 \text{g suplemento} \times \frac{37.8\text{\% Ca}}{100} = 0.605\text{g Ca}\]
    Posteriormente, usando la masa calculada y la masa molar del Ca podemos calcular cuántos moles de Ca hay y, entonces, cuántos moles de \ce{CaCO3}(s) hay en el suplemento:
    \[0.605\text{g Ca} \times \frac{1\text{mol Ca}}{40.1\text{g Ca}} \times \frac{1\text{mol \ce{CaCO3}(s)}}{1\text{mol Ca}} = 0.0151\text{mol \ce{CaCO3}(s)}\]
    Por último, usaremos la masa molar del \ce{CaCO3} para convertir moles de \ce{CaCO3} a gramos de \ce{CaCO3}:
    \[0.0151\text{mol \ce{CaCO3}(s)} \times \frac{100.09\text{g \ce{CaCO3}(s)}}{1\text{mol \ce{CaCO3}(s)}} = 1.51\text{g \ce{CaCO3}(s)}\]
    El suplemento de calcio contiene 1.51 g de \ce{CaCO3}(s).
\end{solutionbox}