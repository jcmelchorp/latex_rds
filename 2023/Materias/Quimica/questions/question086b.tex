Elige la o las palabras que completan las afirmaciones.
\begin{center}
    \fbox{impacto ambiental} \quad \fbox{temperatura} \quad \fbox{disponibilidad} \quad
    \fbox{costo} \quad \fbox{cero} \quad \fbox{beneficio} \quad
    \fbox{movilidad} \quad \fbox{servicio} \quad \fbox{densidad de energía} \quad
    \fbox{magia} \quad \fbox{aplicabilidad} \quad \fbox{renovabilidad} \quad
\end{center}
El aumento en la demanda de energía ha impulsado el desarrollo de combustibles más eficientes, de bajo
\rule{3cm}{0.2mm} y con mayor \rule{3cm}{0.2mm} generada por litro consumido. Durante su diseño,
se busca que los nuevos combustibles tengan menor \rule{3cm}{0.2mm}
tras su combustión, que sean de \rule{3cm}{0.2mm} amplia —para que cualquier persona pueda adquirirlos— y que su
\rule{3cm}{0.2mm} sea variada, es decir, que se puedan usar en diferentes dispositivos. Por otro lado, se deben considerar su
\rule{3cm}{0.2mm} y \rule{3cm}{0.2mm}, para transportarlos de manera segura de un sitio a otro, y que su \rule{3cm}{0.2mm}
se lleve a cabo en el menor tiempo posible.