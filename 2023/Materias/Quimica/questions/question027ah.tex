Un suplemento de hierro de 1.05 g contiene 11.8\% Fe por masa. El hierro está presente en el suplemento en forma de \ce{C4H2FeO4}(s) (masa molar 169.9 g/mol).
\textbf{¿Cuántos gramos de \ce{C4H2FeO4}(s) hay en el suplemento de hierro?}
\emph{Escribe tu respuesta usando tres cifras significativas.}

\begin{solutionbox}{9cm}
    El porcentaje de masa de un elemento en una mezcla puede decirnos cuántos gramos hay y por ende cuántos moles hay del elemento en la mezcla. Sí el elemento es parte de un compuesto, también podemos determinar cuántos gramos y moles del compuesto hay en la mezcla.
    Primero, utilicemos el porcentaje de masa de Fe para calcular la masa de Fe en el suplemento:
 \[1.05 \text{g suplemento} \times \frac{11.8\text{\% Fe}}{100} = 0.124\text{g Fe}\]
    Posteriormente, usando la masa calculada y la masa molar del Fe podemos calcular cuántos moles de Fe hay y, entonces, cuántos moles de \ce{C4H2FeO4}(s) hay en el suplemento:
    \[0.124\text{g Fe} \times \frac{1\text{mol Fe}}{55.8\text{g Fe}} \times \frac{1\text{mol \ce{C4H2FeO4}(s)}}{1\text{mol Fe}} = 0.00222\text{mol \ce{C4H2FeO4}(s)}\]
    Por último, usaremos la masa molar del \ce{C4H2FeO4} para convertir moles de \ce{C4H2FeO4} a gramos de \ce{C4H2FeO4}:
    \[0.00222\text{mol \ce{C4H2FeO4}(s)} \times \frac{169.9\text{g \ce{C4H2FeO4}(s)}}{1\text{mol \ce{C4H2FeO4}(s)}} = 0.377\text{g \ce{C4H2FeO4}(s)}\]
    El suplemento de hierro contiene 0.377 g de \ce{C4H2FeO4}(s).
\end{solutionbox}