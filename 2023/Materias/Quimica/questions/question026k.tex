Balancea la siguiente ecuación química:

\[
    \ce{N2 + H2 -> NH3}
\]

\begin{solutionbox}{8cm}
    \begin{multicols}{2}
        Si representamos la ecuación química con átomos de distintos colores para cada elemento, tenemos:
        \begin{table}[H]
            \centering
            \begin{tabular}{cccc}
                \ce{N2}                                                  & + \ce{H2}                                                & \ce{->} & \ce{NH3 }                                                \\
                \includegraphics[height=0.5cm]{../images/20230415002053} & \includegraphics[height=0.5cm]{../images/20230415002057} &         & \includegraphics[height=0.5cm]{../images/20230415002102} \\
            \end{tabular}
        \end{table}
        Hay 2 N  en los reactivos y 1 N en los productos, por lo que hay que multiplicar por 2 al \ce{NH3}.

        \begin{table}[H]
            \centering
            \begin{tabular}{cccc}
                \ce{N2}                                                  & + \ce{H2}                                                & \ce{->} & \ce{2NH3 }                                               \\
                \includegraphics[height=0.5cm]{../images/20230415002053} & \includegraphics[height=0.5cm]{../images/20230415002057} &         & \includegraphics[height=0.5cm]{../images/20230415002102} \\[-0.5em]
                                                                         &                                                          &         & \includegraphics[height=0.5cm]{../images/20230415002102}
            \end{tabular}
        \end{table}

        Ahora, hay 6 H en los productos, por lo que hay que multiplicar por 3 al \ce{H2}.
        \begin{table}[H]
            \centering
            \begin{tabular}{cccc}
                \ce{N2}                                                  & + \ce{3H2}                                               & \ce{->} & \ce{2NH3 }                                               \\
                \includegraphics[height=0.5cm]{../images/20230415002053} & \includegraphics[height=0.5cm]{../images/20230415002057} &         & \includegraphics[height=0.5cm]{../images/20230415002102} \\[-0.5em]
                                                                         & \includegraphics[height=0.5cm]{../images/20230415002057} &         & \includegraphics[height=0.5cm]{../images/20230415002102} \\[-0.5em]
                                                                         & \includegraphics[height=0.5cm]{../images/20230415002057} &         &
            \end{tabular}
        \end{table}
        Por lo tanto, la ecuación química balanceada es:
        \[
            \ce{N2 + 3H2 -> 2NH3}
        \]
    \end{multicols}
\end{solutionbox}