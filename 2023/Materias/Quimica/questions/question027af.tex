Un estudiante determina que 3.0 g de una mezcla de \ce{ZnO}(s) y \ce{Ag2O}(s) contiene 0.010 mol de \ce{ZnO}(s).\\
\textbf{Con base en los resultados del estudiante, ¿cuál es el porcentaje de masa de Zn en la mezcla?}\\

\begin{oneparchoices}
    \choice  11\%
    \CorrectChoice  22\%
    \choice  44\%
    \choice  88\%
\end{oneparchoices}

\begin{solutionbox}{6cm}
    El porcentaje de masa de una sustancia en una mezcla se puede determinar por la comparación de la masa de la sustancia en la mezcla contra la masa total de la mezcla.
    Primero, determinemos la masa de  \ce{Zn} en la mezcla. Utilizando los resultados del estudiante y la masa molar del \ce{ZnO}, podemos encontrar el número de moles de \ce{ZnO} en la mezcla y después convertir ese valor a gramos:
    \[0.010 \text{mol ZnO}\times\frac{1 \text{mol Zn}}{1 \text{mol ZnO}}\times\frac{65.38 \text{g Zn}}{1 \text{mol Zn}}=0.65 \text{g Zn} \]
    Posteriormente, usando la masa calculada de \ce{Zn} y la masa total de la mezcla, podemos calcular el porcentaje de masa de \ce{Zn} en la mezcla:
    \[\frac{0.65 \text{g Zn}}{3.0 \text{g mezcla}}\times 100\%=22\% \]
    Por lo tanto, el porcentaje de masa de \ce{Zn} en la mezcla es 22\%.
    \end{solutionbox}