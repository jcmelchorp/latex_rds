Usando la información de la tabla \ref{tab:q002},
\textbf{Calcula el número de moles en una muestra de 7.89 kg de aspirina (ácido acetilsalicílico) (\ce{C9H8O4}).}\\
\emph{Escribe tu respuesta usando tres cifras significativas.}

\begin{multicols}{2}
    \begin{table}[H]
        \centering
        \caption{Masa molar de algunos elementos.}
        \label{tab:q002}
        \begin{tabular}{c|p{2.2cm}}
            \textbf{Elemento} & \textbf{Masa molar (g/mol)} \\\midrule
            H                 & 1.008                       \\\hline
            C                 & 12.01                       \\\hline
            O                 & 16.00                       \\\hline
            \bottomrule
        \end{tabular}
    \end{table}

    \columnbreak

    \begin{solutionbox}{7cm}
    \end{solutionbox}
\end{multicols}