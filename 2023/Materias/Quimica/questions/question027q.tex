\begin{multicols}{2}
    Usando la información de la tabla \ref{tab:q002},
    \textbf{Calcula el número de moles en una muestra de 7.89 kg de aspirina (ácido acetilsalicílico) (\ce{C9H8O4}).}\\
    \emph{Escribe tu respuesta usando tres cifras significativas.}

    \begin{table}[H]
        \centering
        \caption{Masa molar de algunos elementos.}
        \label{tab:q002}
        \begin{tabular}{c|p{2.2cm}}
            \textbf{Elemento} & \textbf{Masa molar (g/mol)} \\\midrule
            H                 & 1.008                       \\\hline
            C                 & 12.01                       \\\hline
            O                 & 16.00                       \\\hline
            \bottomrule
        \end{tabular}
    \end{table}

\end{multicols}

\begin{solutionbox}{8cm}
    Podemos usar la masa molar de una sustancia para convertir gramos a moles de sustancia. Con esto en mente, primero calculemos la masa molar de \ce{C9H8O4}, usando la tabla \ref{tab:q002}:
    \[9 \text{ mol de C} \times 12.01 \text{g/mol} = 108.1 \text{g}\]
    \[8 \text{ mol de H} \times 1.008 \text{g/mol} = 8.064 \text{g}\]
    \[4 \text{ mol de O} \times 16.00 \text{g/mol} = 64.00 \text{g}\]
    \[\text{Masa molar} = 108.1 + 8.064 + 64.00 = 180.2 \text{g/mol}\]

    Por lo tanto, 1 mol de \ce{C9H8O4} tiene una masa molar de 180.2 g/mol.
    Después, usemos la masa molar de \ce{C9H8O4} para convertir gramos a moles de \ce{C9H8O4}.
    Ya que nos estan dando la masa de \ce{C9H8O4} en kilogramos, también necesitaremos incluir el factor de conversión de kilogramos a gramos:
    \[7.89 \text{kg} \times \frac{1000 \text{g}}{1 \text{kg}} \times \frac{1 \text{mol}}{180.2 \text{g}} = 43.8 \text{mol}\]
    Por lo tanto, hay 43.8 moles de \ce{C9H8O4} en 7.89 kg de \ce{C9H8O4}.
\end{solutionbox}