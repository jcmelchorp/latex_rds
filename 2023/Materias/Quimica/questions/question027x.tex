La masa molar del azufre (\ce{S}) es 32,06 g/mol.
\textbf{Calcula la masa en gramos de una muestra de azufre (\ce{S}) que contiene $2.01 \times 10^{24}$ átomos.}\\
\emph{Escribe tu respuesta usando tres cifras significativas.}

\begin{solutionbox}{6cm}
    Podemos usar el número de Avogadro, $6.023\times 10^{23}$, para convertir de moles a partículas representativas (como átomos, moléculas o unidades de fórmula) a moles de una sustancia. Después podemos usar la masa molar de la sustancia para convertir de moles a gramos.
    Primero, usemos el número de Avogadro para convertir átomos de \ce{S} a moles de \ce{S}:
    \[2.01\times 10^{24} \text{átomos S}\times\frac{1\,\text{mol\,S}}{6.023\times 10^{23}\,\text{átomos\,S}}=3.34\,\text{mol\,S} \]
    Después, usemos la masa molar de \ce{S} para convertir de moles de \ce{S} a gramos de \ce{S}:
    \[3.34\,\text{mol\,S}\times\frac{32.06\,\text{g\,S}}{1\,\text{mol\,S}}=107\,\text{g\,S} \]
    Por lo tanto, una muestra de 2.01$\times 10^{24}$ átomos de \ce{S} tiene una masa de 107 g.
\end{solutionbox}