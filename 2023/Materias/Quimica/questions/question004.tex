\question[10] Relaciona la especie química con la cantidad de \textbf{protones} y \textbf{electrones de valencia}.

\begin{multicols}{3}
    \setbohr{atom-style={\small}}
    \begin{choices}
        \choice Carbono C       \\ \bohr{6}{C}
        \choice Ión azúfre    \\ \bohr{14}{S$^{2+}$}
        \choice Hierro  \\ \bohr{26}{Fe}
        \choice Ión flúor     \\ \bohr{10}{F$^-$}
        \columnbreak
        \choice Ión potasio   \\ \bohr{18}{K$^+$}
        \choice Ión calcio Ca$^{2+}$   \\ \bohr{18}{Ca$^{2+}$}
        \choice Ión aluminio  \\ \bohr{10}{Al$^{3+}$}
        \choice Ión berilio  \\ \bohr{4}{Be$^{+}$}
        \choice Ión Yodo  \\ \bohr{54}{I$^{-1}$}
        \choice Magnesio      \\ \bohr{12}{Mg}
    \end{choices}
\end{multicols}
\begin{multicols}{2}
    \begin{parts}
        \part \fillin[F][1cm] 20 protones y 8 electrones de valencia.
        \part \fillin[D][1cm] 9 protones y 8 electrones de valencia.
        \part \fillin[H][1cm] 4 protones y 1 electron de valencia.
        \part \fillin[B][1cm] 16 protones y 4 electrones de valencia.
        \part \fillin[A][1cm] 8 protones y 4 electrones de valencia.
        \part \fillin[I][1cm] 53 protones y 8 electrones de valencia.
        \part \fillin[G][1cm] 13 protones y 8 electrones de valencia.
        \part \fillin[E][1cm] 19 protones y 8 electrones de valencia.
        \part \fillin[C][1cm] 26 protones y 2 electrones de valencia.
        \part \fillin[J][1cm] 12 protones y 2 electrones de valencia.
    \end{parts}
    % \end{minipage}
\end{multicols}