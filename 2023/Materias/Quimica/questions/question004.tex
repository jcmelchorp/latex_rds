Relaciona la especie química con la cantidad de \textbf{protones} y \textbf{electrones de valencia}.

\begin{multicols}{2}
    \setbohr{atom-style={\small}}
    \begin{choices}
        \choice Ión de Nitrógeno (N$^{3-}$) \\  \bohr{10}{N$^{3-}$}
        \choice Ión de Berilio (Be$^{-}$) \\  \bohr{5}{Be$^{-}$}
        \choice Ión de Flúor   (F$^-$) \\  \bohr{10}{F$^-$}
        \choice Ión de Hierro (Fe$^{3+}$)  \\  \bohr{23}{Fe$^{3+}$}
        \choice Ión de Potasio (K$^+$) \\ \bohr{18}{K$^+$}
        \columnbreak
        \choice Ión de Aluminio  (Al$^{3+}$) \\ \bohr{10}{Al$^{3+}$}
        \choice Ión de Yodo (I$^{-1}$) \\ \bohr{54}{I$^{-1}$}
        \choice Ión de Azúfre (S$^{2+}$) \\ \bohr{14}{S$^{2+}$}
        \choice Litio (Li) \\ \bohr{3}{Li}
        \choice Fósforo (P)    \\  \bohr{15}{P}
    \end{choices}
\end{multicols}
\begin{multicols}{2}
    \begin{parts}
        \part \fillin[C][1cm] 9 protones y 8 electrones de valencia.
        \part \fillin[J][1cm] 15 protones y 5 electrones de valencia.
        \part \fillin[B][1cm] 4 protones y 3 electrones de valencia.
        \part \fillin[H][1cm] 16 protones y 4 electrones de valencia.
        \part \fillin[A][1cm] 7 protones y 8 electrones de valencia.
        \part \fillin[G][1cm] 53 protones y 8 electrones de valencia.
        \part \fillin[F][1cm] 13 protones y 8 electrones de valencia.
        \part \fillin[E][1cm] 19 protones y 8 electrones de valencia.
        \part \fillin[D][1cm] 26 protones y 2 electrones de valencia.
        \part \fillin[I][1cm] 3 protones y 1 electrón de valencia.
    \end{parts}
    % \end{minipage}
\end{multicols}