Una muestra de 118.0 g de un compuesto contiene 72.0 g de carbono, 18.0 g de hidrógeno y 28.0 g de nitrógeno.
\textbf{¿Cuál es la fórmula que corresponde a este compuesto?}

\begin{oneparchoices}

    \choice \ce{CH3N}
    \choice \ce{CH3N2}
    \CorrectChoice \ce{C3H9N}
    \choice \ce{C6H18N2}
\end{oneparchoices}

\begin{solutionbox}{7.5cm}
    La fórmula empírica de un compuesto es la proporción en números enteros más simple de elementos en el compuesto.
    Primero, usemos las masas molares de \ce{C}, \ce{H} y \ce{N} para determinar cuantos moles de cada elemento están presentes en la muestra:

    \[ 72.0 \text{ g \ce{C}} \times \frac{1 \text{ mol \ce{C}}}{12.01 \text{ g \ce{C}}} \approx 6.0 \text{ mol \ce{C}}  \qquad 18.0 \text{ g \ce{H}} \times \frac{1 \text{ mol \ce{H}}}{1.008 \text{ g \ce{H}}} \approx 18.0 \text{ mol \ce{H}}  \qquad  28.0 \text{ g \ce{N}} \times \frac{1 \text{ mol \ce{N}}}{14.01 \text{ g \ce{N}}} \approx 2.0 \text{ mol \ce{N}} \]

    Por lo tanto, los 118.0 g de muestra contienen aproximadamente 6 moles de átomos de carbono, 18 moles de átomos de hidrógeno y 2 moles de átomos de nitrógeno.
    A continuación, encontremos la relación en números enteros más simple de átomos en el compuesto dividiendo cada uno de los valores molares del carbono, hidrógeno y nitrógeno entre el menor de estos tres valores:

    \[ \frac{6.0 \text{ mol \ce{C}}}{2.0 \text{ mol \ce{N}}} = 3 \qquad \frac{18.0 \text{ mol \ce{H}}}{2.0 \text{ mol \ce{N}}} = 9 \qquad \frac{2.0 \text{ mol \ce{N}}}{2.0 \text{ mol \ce{N}}} = 1 \]

    Entonces, hay 3 átomos de carbono y 9 átomos de hidrógeno por cada 1 átomo de nitrógeno en el compuesto.
    La fórmula empírica del compuesto es \ce{C3H9N}.
\end{solutionbox}