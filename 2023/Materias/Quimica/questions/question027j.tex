Una muestra de un compuesto que contiene únicamente átomos de carbono e hidrógeno se combustiona completamente produciendo 66.0 g de \ce{CO2} y 36.0 g de \ce{H2O}.
\textbf{¿Cuál es la fórmula que corresponde a este compuesto?}

\begin{oneparchoices}
    \choice  \ce{CH2}
    \CorrectChoice  \ce{C3H8}
    \choice  \ce{C3H4}
    \choice  \ce{CH}
\end{oneparchoices}

\begin{solutionbox}{8.5cm}
    Cuando un compuesto que contiene únicamente carbono e hidrógeno se combustiona completamente, todos los átomos de carbono terminan formando \ce{CO2} y todos los átomos de hidrógeno terminan formando \ce{H2O}.
    Primero, usemos las masas molares del \ce{CO2} y \ce{H2O} para determinar cuantos moles de carbono e hidrógeno había en la muestra del compuesto antes de su combustión:
    
    \[ 66.0 g \text{ \ce{CO2}} \times \frac{1 \text{ mol \ce{CO2}}}{44.01 \text{ g \ce{CO2}}} \times \dfrac{ 1 \text{ mol \ce{C}}}{1 \text{ mol \ce{CO2}}} \approx 1.5 \text{ mol \ce{C}}  \]
    \[ 36.0 g \text{ \ce{H2O}} \times \frac{1 \text{ mol \ce{H2O}}}{18.02 \text{ g \ce{H2O}}}  \times \dfrac{ 2 \text{ mol \ce{H}}}{1 \text{ mol \ce{H2O}}} \approx 4.0 \text{ mol \ce{H}}  \]
    
    Por lo tanto, la muestra contenía aproximadamente  1.5 moles de carbono y 4 moles de hidrógeno.
    A continuación, encontremos la relación más simple en números enteros de átomos en el compuesto dividiendo cada uno de los valores molares para carbono e hidrógeno entre el menor de estos dos valores. Para obtener números enteros, también necesitaremos multiplicar por un número entero:
    
    \[ \frac{1.5 \text{ mol \ce{C}}}{1.5 \text{ mol \ce{C}}} = 1 \text{ mol \ce{C}} \times 3 = 3 \text{ mol \ce{C}} \qquad \frac{4.0 \text{ mol \ce{H}}}{1.5 \text{ mol \ce{C}}} = 2.\overline{6} \text{ mol \ce{H}} \times 3 = 8 \text{ mol \ce{H}} \]
    
    Por lo tanto, la relación más simple de átomos en el compuesto es 3 átomos de carbono por 8 átomos de hidrógeno, lo que corresponde a la fórmula empírica \ce{C3H8}.
\end{solutionbox}