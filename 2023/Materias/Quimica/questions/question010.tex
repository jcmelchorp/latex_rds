\question Señala en cada uno de los enunciados si la sentencia es falsa o verdadera.

\begin{parts}
    \part[2] La unión de dos o más átomos
    iguales o diferentes forman
    una molécula.
    \part[2]  Las sustancias covalentes
    se forman por la unión de
    elementos no metálicos.
    \part[2]  Los anhídridos u óxidos áci-
    dos resultan de la combi-
    nación de un metal con el
    oxígeno.
    \part[2]  Un ácido de Arrhenius se
    describe como aquella sus-
    tancia que libera iones hidró-
    geno al disolverse en agua.
    \part[2]  Los óxidos ácidos al reac-
    cionar con el agua forman
    hidróxidos.
    \part[2]  Los hidrácidos son compues-
    tos que resultan de la combi-
    nación del hidrógeno con los
    no metales de los grupos VI A
    (16) y VII A (17).
    \part[2]  El CO 2 y el CO son los óxi-
    dos del cobalto.
    \part[2]  El H 2 SO 4 es el ácido de ma-
    yor importancia industrial
    para un país.
    \part[2]  El NO, es un óxido del nitró-
    geno que biológicamente tie-
    ne un efecto vasodilatador, la
    viagra libera este compuesto.
    \part[2]  El HCl es un hidrácido que el
    estómago secreta para ayu-
    dar a digerir los alimentos.
    \part[2]  La nomenclatura que más
    debe ser utilizada por los
    científicos, es la común.

    \part[2] Los óxidos son compues-
    tos iónicos que resultan
    de la unión de dos no
    metales.
    \part[2] Cuando un metal se
    combina químicamente
    con el oxígeno, se for-
    man óxidos.
\end{parts}
