Señala en cada uno de los enunciados si la sentencia es falsa o verdadera.

\begin{multicols}{2}
    \begin{parts}
        La unión de dos o más átomos
        iguales o diferentes forman
        una molécula.

        \begin{oneparchoices}
            \choice Verdadero
            \choice Falso
        \end{oneparchoices}

        Las sustancias covalentes
        se forman por la unión de
        elementos no metálicos.

        \begin{oneparchoices}
            \choice Verdadero
            \choice Falso
        \end{oneparchoices}

        Los anhídridos u óxidos ácidos resultan de la combinación de un metal con el
        oxígeno.

        \begin{oneparchoices}
            \choice Verdadero
            \choice Falso
        \end{oneparchoices}

        Un ácido de Arrhenius se
        describe como aquella sustancia que libera iones hidrógeno al disolverse en agua.

        \begin{oneparchoices}
            \choice Verdadero
            \choice Falso
        \end{oneparchoices}

        Los óxidos ácidos al reaccionar con el agua forman
        hidróxidos.

        \begin{oneparchoices}
            \choice Verdadero
            \choice Falso
        \end{oneparchoices}

        \columnbreak
        Los hidrácidos son compuestos que resultan de la combinación del hidrógeno con los
        no metales de los grupos VI A
        (16) y VII A (17).

        \begin{oneparchoices}
            \choice Verdadero
            \choice Falso
        \end{oneparchoices}

        El CO$_2$ y el CO son los óxidos del cobalto.

        \begin{oneparchoices}
            \choice Verdadero
            \choice Falso
        \end{oneparchoices}

        El H$_2$ SO$_4$ es el ácido de mayor importancia industrial
        para un país.

        \begin{oneparchoices}
            \choice Verdadero
            \choice Falso
        \end{oneparchoices}

        El NO, es un óxido del nitrógeno que biológicamente tiene un efecto vasodilatador, la
        viagra libera este compuesto.

        \begin{oneparchoices}
            \choice Verdadero
            \choice Falso
        \end{oneparchoices}

        El HCl es un hidrácido que el
        estómago secreta para ayudar a digerir los alimentos.

        \begin{oneparchoices}
            \choice Verdadero
            \choice Falso
        \end{oneparchoices}

        La nomenclatura que más
        debe ser utilizada por los
        científicos, es la común.


        \begin{oneparchoices}
            \choice Verdadero
            \choice Falso
        \end{oneparchoices}

        Los óxidos son compuestos iónicos que resultan
        de la unión de dos no
        metales.

        \begin{oneparchoices}
            \choice Verdadero
            \choice Falso
        \end{oneparchoices}

        Cuando un metal se
        combina químicamente
        con el oxígeno, se forman óxidos.

        \begin{oneparchoices}
            \choice Verdadero
            \choice Falso
        \end{oneparchoices}
    \end{parts}

\end{multicols}