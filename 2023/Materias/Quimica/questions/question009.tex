\question[10] Relaciona cada elemento con las características que le
corresponden.

\begin{minipage}{0.2\textwidth}
    \begin{parts}
        \part \fillin[I][1cm] Radón \\[-1em]
        \part \fillin[D][1cm] Helio\\[-1em]
        \part \fillin[B][1cm] Galio\\[-1em]
        \part \fillin[F][1cm] Yodo\\[-1em]
        \part \fillin[H][1cm] Bismuto\\[-1em]
        \part \fillin[G][1cm] Radio\\[-1em]
        \part \fillin[C][1cm] Silicio\\[-1em]
        \part \fillin[J][1cm] Oro\\[-1em]
        \part \fillin[E][1cm] Titanio\\[-1em]
        \part \fillin[A][1cm] Boro\\[-1em]
        % \part \fillin[][1cm] Carbono\\[-1em]
    \end{parts}
\end{minipage}\hfill
\begin{minipage}{0.7\textwidth}
    \begin{choices}
        \choice Elemento metaloide del grupo III, subgrupo A de la tabla periódica.
        \choice Elemento metálico con Z = 31.
        \choice Elemento metaloide, ubicado en el tercer período de la tabla periódica.
        \choice Elemento conocido como gas noble y se encuentra en el período 1 de la tabla periódica.
        \choice Elemento con 22 protones y 22 electrones.
        % \choice Elemento con cuatro electrones en su última capa de valencia y es la base de la vida.
        \choice Elemento de la familia de los Halógenos con 74 neutrones.
        \choice Elemento de la familia de metales alcalino-terreos con 138 neutrones.
        \choice Elemento no metálico con Z =83.
        \choice Gas inerte (gas noble) que se encuentra en el período 6 de la tabla periódica.
        \choice Metal brillante utilizado en joyería.
        % \choice Líquido rojo oscuro.
        % \choice Gas incoloro que arde en presencia de oxígeno.
        % \choice Metal reactivo que reacciona fácilmente con agua.
    \end{choices}
\end{minipage}