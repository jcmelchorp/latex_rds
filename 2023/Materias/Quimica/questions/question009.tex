\question[10] Relaciona cada elemento con las características que le
corresponden.

\begin{minipage}{0.6\textwidth}
    \begin{choices}
        \choice Elemento con cuatro electrones en su última capa de
        valencia y es la base de la vida.
        \choice Gas noble que se encuentra en el período 4 de la tabla
        periódica.
        \choice Elemento de la familia de metales alcalinos.
        \choice Elemento del grupo VII, con siete electrones en la
        última capa
        de valencia, ubicado en el tercer periodo de la tabla
        periódica.
        \choice Elemento metaloide, ubicado en el tercer periodo de la
        tabla
        periódica.
        \choice Elemento conocido como gas noble y se encuentra en el
        período 2 de la tabla periódica.
        \choice Elemento metálico con Z = 11.
        \choice Elemento que se ubica en el grupo 15 y en el periodo 2
        de la
        tabla periódica.
        \choice Elemento con 12 protones y 12 electrones.
        \choice Elemento no metálico con Z =34
        \choice Gas inerte (gas noble) que se encuentra en el período 5
        de la tabla periódica..
        \choice Metal brillante utilizado en joyería.
        \choice Líquido rojo oscuro.
        \choice Gas incoloro que arde en presencia de oxígeno.
        \choice Metal reactivo que reacciona fácilmente con agua.
    \end{choices}
\end{minipage}\hfill
\begin{minipage}{0.25\textwidth}
    \begin{parts}
        \large
        \part \fillin[B][1cm] Kriptón
        \part \fillin[M][1cm] Bromo
        \part \fillin[Ñ][1cm] Calcio
            {
                \printanswers
                \part \fillin[J][1cm] Selenio
            }
        \part \fillin[F][1cm] Neón
        {
        \printanswers
        \part \fillin[I][1cm] Magnesio
        }
        \part \fillin[K][1cm] Xenón
        \part \fillin[G][1cm] Sodio
        \part \fillin[D][1cm] Cloro
        \part \fillin[H][1cm] Nitrógeno
        \part \fillin[E][1cm] Silicio
        \part \fillin[L][1cm] Oro
        \part \fillin[C][1cm] Rubidio
        \part \fillin[N][1cm] Hidrógeno
        {\printanswers
        \part \fillin[A][1cm] Carbono
        }
    \end{parts}
\end{minipage}