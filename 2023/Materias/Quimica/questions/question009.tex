\question[10] Relaciona cada elemento con las características que le
corresponden.

\begin{minipage}{0.35\textwidth}
    \begin{parts}
        \part \fillin[][1cm] Selenio
        \part \fillin[][1cm] Neón
        \part \fillin[][1cm] Magnesio
        \part \fillin[][1cm] Criptón
        \part \fillin[][1cm] Xenón
        \part \fillin[][1cm] Sodio
        \part \fillin[][1cm] Cloro
        \part \fillin[][1cm] Nitrógeno
        \part \fillin[][1cm] Silicio
        \part \fillin[][1cm] Oro
        \part \fillin[][1cm] Rubidio
        \part \fillin[][1cm] Carbono
    \end{parts}
\end{minipage}
\begin{minipage}{0.6\textwidth}\hfill
    \begin{choices}
        % \choice Elemento con cuatro electrones en su última capa de valencia y es la base de la vida.
        \choice Gas noble que se encuentra en el período 4 de la tabla periódica.
        \choice Elemento de la familia de metales alcalinos.
        \choice Elemento del grupo VIIA, con siete electrones en la última capa de valencia, ubicado en el tercer periodo de la tabla periódica.
        \choice Elemento metaloide, ubicado en el tercer periodo de la tabla periódica.
        \choice Elemento conocido como gas noble y se encuentra en el período 2 de la tabla periódica.
        \choice Elemento metálico con Z = 11.
        \choice Elemento con 12 protones y 12 electrones.
        \choice Elemento no metálico con Z =34.
        \choice Gas inerte (gas noble) que se encuentra en el período 5 de la tabla periódica.
        \choice Metal brillante utilizado en joyería.
        % \choice Líquido rojo oscuro.
        % \choice Gas incoloro que arde en presencia de oxígeno.
        % \choice Metal reactivo que reacciona fácilmente con agua.
    \end{choices}
\end{minipage}