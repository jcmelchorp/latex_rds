% --------------------------------------------------------------------------------------------------
% subfile of pgf-PeriodicTable package ----------------------------------------------------------
% --------------------------------------------------------------------------------------------------
% Macros for building the cells of all elements --------------------------------------------------
% --------------------------------------------------------------------------------------------------
%% USER MACRO
% --------------------------------------------------------------------------------------------------
% auxiliary macro to construct the layout of each cell in the Periodic Table of Elements
% \pgfPTbuildcell(<lines>,<columns>)[<entries>]
% or
% \pgfPTbuildcellstyle{name}(<lines>,<columns>)[<entries>]
% each entry is constructed according to the structure <line>|<line_i>-<line_f>;<column>|<column_i>-<column_f>;<what>
% OUTPUT:
% \pgfPT@cellcontents -> a comma separated list.
% --------------------------------------------------------------------------------------------------
% Each entry stores the following data: (contents::csstring::coordinate+wd+ht)
% The coordinate corresponds to the right upper corner "x+y"
% \global\pgfPT@cellcontentssize -> size of the list
% Example of use (the default layout for each cell in the Periodic table):
%   -------------
%   | 1 | 2 | 3 |  line 1
%   -------------
%   | 1 | 2 | 3 |  line 2
%   -------------
%   | 1 | 2 | 3 |  line 3
%   -------------
%   | 1 | 2 | 3 |  line 4
%   -------------
%   | 1 | 2 | 3 |  line 5
% \pgfPTbuildcell(5,3)[% 5 lines x 3 columns
%        (1;1-2;z),(1;3;radio),% line 1; the atomic number spreads over column 1 and 2
%        (2-3;2;cs),(4;2;name),% lines 2 and 3, line 4
%        (5;2;ar)% line 5
%        ]%
% -------------
%%%%%%%%%%%%%%%%%%%%%%%%%%%%%%%%%%%%%%%%%%%%%
\newif\ifpgfPT@to\pgfPT@tofalse%
\newif\ifpgfPT@dot\pgfPT@dotfalse%
\newif\ifpgfPT@unnamedcell\pgfPT@unnamedcelltrue%
\newdimen\pgfPT@cell@collumnwd\pgfPT@cell@collumnwd=0pt%
\newdimen\pgfPT@cell@lineht\pgfPT@cell@lineht=0pt%
\newdimen\pgfPT@tmpx\pgfPT@tmpx=0pt%
\newdimen\pgfPT@tmpy\pgfPT@tmpy=0pt%
\newdimen\pgfPT@tmpwd\pgfPT@tmpwd=0pt%
\newdimen\pgfPT@tmpht\pgfPT@tmpht=0pt%
\newcount\pgfPT@cellcontentssize\pgfPT@cellcontentssize=0%
%%%%%%%%%%%%%%%%%%%%%%%%%%%%%%%%%%%%%%%%%%%%%
\def\pgfPTbuildcellstyle#1(#2,#3)[#4]{\pgfPT@unnamedcellfalse\relax%
\pgfPTbuildcell(#2,#3)[#4]%
\expandafter\edef\csname pgfPT@cellcontents@#1\endcsname{\pgfPT@cellcontents}%
\expandafter\edef\csname ppfPT@thebuildcellcom@#1\endcsname{\textbackslash pgfPTbuildcell(\detokenize{#2},\detokenize{#3})\%\par[\detokenize{#4}]}%
\expandafter\edef\csname pgfPT@nlinhas@#1\endcsname{#2}%
\expandafter\edef\csname pgfPT@ncolunas@#1\endcsname{#3}%
\pgfPT@unnamedcelltrue\relax%
}%
\def\pgfPT@builtincellstyle#1(#2,#3)[#4]{\pgfPT@unnamedcellfalse\relax%
\pgfPTbuildcell(#2,#3)[#4]%
\expandafter\edef\csname pgfPT@cellcontents@#1\endcsname{\pgfPT@cellcontents}%
\expandafter\edef\csname ppfPT@thebuildcellcom@#1\endcsname{\textbackslash pgfPTbuildcell(\detokenize{#2},\detokenize{#3})\%\par[\detokenize{#4}]}%
\expandafter\edef\csname pgfPT@nlinhas@#1\endcsname{#2}%
\expandafter\edef\csname pgfPT@ncolunas@#1\endcsname{#3}%
\pgfPT@unnamedcelltrue\relax%
\global\let\pgfPT@cellcontents\undefined\relax
}%
% the macro code
%        \pgfPTbuildcell(<lines>,<columns>)[<entries>]
%
\def\pgfPTbuildcell(#1,#2)[#3]{%
\ifpgfPT@unnamedcell\gdef\ppfPT@thebuildcellcom{\textbackslash pgfPTbuildcell(\detokenize{#1},\detokenize{#2})\%\par[\detokenize{#3}]}\fi%
\gdef\pgfPT@numlinhas{#1}\gdef\pgfPT@numcolunas{#2}%
\pgfPT@cellcontentssize=0%
\xdef\pgfPT@cellcontents{}% vanishes list contents
\pgfmathsetlength{\pgfPT@cell@collumnwd}{\pgfPTcellwd/#2}%
\pgfmathsetlength{\pgfPT@cell@lineht}{\pgfPTcellht/#1}%
\global\pgfPT@cell@lineht=\pgfPT@cell@lineht%
\global\pgfPT@cell@collumnwd=\pgfPT@cell@collumnwd%
\expandafter\pgfPTg@bblespaces#3\relax%
\@for\@tmp:=\pgfPT@listn@spaces\do{\expandafter\pgfPT@buildcellcontents\@tmp\relax}%
\ifpgfPT@unnamedcell\global\let\pgfPT@cellcontents@unnamed\pgfPT@cellcontents\relax\fi%
}%
% --------------------------------------------------------------------------------------------------
% \pgfPT@buildcellcontents(<line>|<line_i>-<line_f>;<column>|<column_i>-<column_f>;<what>)
% verifies if 'what' is valid and calls addcellentry
% NOTE: in validation the \pgfPT@csstring is defined and stores the 'csname' string corresponding to <what>
% --------------------------------------------------------------------------------------------------
\def\pgfPT@buildcellcontents(#1;#2;#3){%
\pgfPT@validate@cell@content{#3}%
\ifpgfPT@found% valid entry
\global\advance\pgfPT@cellcontentssize by1\relax%
\pgfPT@addcellentry#1;#2;#3.\relax%
\fi% \ifpgfPT@found
}%
% --------------------------------------------------------------------------------------------------
% \pgfPT@addcellentry<line>|<line_i>-<line_f>;<column>|<column_i>-<column_f>;<what>.
% checks if the 1st entry is a single line (<line>) or multiple line (<line_i>-<line_f>)
% in both cases computes the height of the entry (<what>)
% checks if the 1st entry is a single column (<column>) or multiple column (<column_i>-<column_f>)
% in both cases computes the width of the entry (<what>)
% OUTPUT:
% appends the entry to the \pgfPT@cellcontents with
%                               what::x+y+wd+ht
% --------------------------------------------------------------------------------------------------
\def\pgfPT@addcellentry#1;#2;#3.{%
\pgfPT@find@to#1\relax%
\ifpgfPT@to%
\pgfPT@decomposeentry(#1)%
\pgfmathsetlength{\pgfPT@tmpy}{(\pgfPT@firstentry-1)*\pgfPT@cell@lineht}%
\expandafter\pgfPT@find@dot\pgfPT@secondentry\relax%
\ifpgfPT@dot%
\pgfmathsetlength{\pgfPT@tmpht}{(\pgfPT@secondentry-\pgfPT@firstentry)*\pgfPT@cell@lineht}%
\else%
\pgfmathsetlength{\pgfPT@tmpht}{(\pgfPT@secondentry-\pgfPT@firstentry+1)*\pgfPT@cell@lineht}%
\fi%
\else%
\pgfmathsetlength{\pgfPT@tmpy}{(#1-1)*\pgfPT@cell@lineht}%
\pgfmathsetlength{\pgfPT@tmpht}{\pgfPT@cell@lineht}%
\fi\relax%
\pgfPT@find@to#2\relax%
\ifpgfPT@to%
\pgfPT@decomposeentry(#2)%
\pgfmathsetlength{\pgfPT@tmpx}{(\pgfPT@firstentry-1)*\pgfPT@cell@collumnwd}%
\expandafter\pgfPT@find@dot\pgfPT@secondentry\relax%
\ifpgfPT@dot%
\pgfmathsetlength{\pgfPT@tmpwd}{(\pgfPT@secondentry-\pgfPT@firstentry)*\pgfPT@cell@collumnwd}%
\else%
\pgfmathsetlength{\pgfPT@tmpwd}{(\pgfPT@secondentry-\pgfPT@firstentry+1)*\pgfPT@cell@collumnwd}%
\fi%
\else%
\pgfmathsetlength{\pgfPT@tmpx}{(#2-1)*\pgfPT@cell@collumnwd}%
\pgfmathsetlength{\pgfPT@tmpwd}{\pgfPT@cell@collumnwd}%
\fi\relax%
% add to list:
\edef\pgfPT@testa{#3}%
\ifx\pgfPT@acronym@Arstar\pgfPT@testa\relax\edef\pgfPT@testa{Arstar}\fi%
\ifnum\pgfPT@cellcontentssize=1\relax%
\xdef\pgfPT@cellcontents{\pgfPT@testa::\the\pgfPT@tmpx+\the\pgfPT@tmpy+\the\pgfPT@tmpwd+\the\pgfPT@tmpht}\relax%
\else%
\xdef\pgfPT@cellcontents{\pgfPT@cellcontents,\pgfPT@testa::\the\pgfPT@tmpx+\the\pgfPT@tmpy+\the\pgfPT@tmpwd+\the\pgfPT@tmpht}\relax%
\fi%
}%
% --------------------------------------------------------------------------------------------------
% \pgfPT@decomposeentry(i-f)
% OUTPUT:
%       \pgfPT@firstentry(i) & \pgfPT@secondentry(f)
% --------------------------------------------------------------------------------------------------
\def\pgfPT@decomposeentry(#1-#2){%
\edef\pgfPT@firstentry{#1}\edef\pgfPT@secondentry{#2}\relax%
}%
% --------------------------------------------------------------------------------------------------
% \pgfPT@find@to{token list}
% inspects if the token list contains the character '-'
% OUTPUT:
%       true or false (via \ifpgfPT@to)
% --------------------------------------------------------------------------------------------------
\def\pgfPT@find@to#1#2\relax{%
\edef\pgfPT@teststr{#1}\edef\pgfPT@teststrX{#2}%
\ifx\pgfPT@teststr\pgfPT@to\global\pgfPT@totrue\else\global\pgfPT@tofalse\fi\relax%
\ifpgfPT@to\relax\else\ifx\pgfPT@teststrX\@empty\relax\else\pgfPT@find@to#2\relax\fi\fi%
}%
\def\pgfPT@to{-}%
% --------------------------------------------------------------------------------------------------
% \pgfPT@find@dot{token list}
% inspects if the token list contains the character '.'
% OUTPUT:
%       true or false (via \ifpgfPT@dot)
% --------------------------------------------------------------------------------------------------
\def\pgfPT@find@dot#1#2\relax{%
\edef\pgfPT@teststr{#1}\edef\pgfPT@teststrX{#2}%
\ifx\pgfPT@teststr\pgfPT@dot\global\pgfPT@dottrue\else\global\pgfPT@dotfalse\fi\relax%
\ifpgfPT@dot\relax\else\ifx\pgfPT@teststrX\@empty\relax\else\pgfPT@find@dot#2\relax\fi\fi%
}%
\def\pgfPT@dot{.}%
% --------------------------------------------------------------------------------------------------
% USER MACRO -----------------------------------------------------------------------------------
% auxiliary macro to graphical preview the layout of each cell
% in the Periodic Table of Elements
% --------------------------------------------------------------------------------------------------
% \pgfPT@
% OUTPUT:
%       the graphical cell with the <what> labels
% --------------------------------------------------------------------------------------------------
\def\pgfPTpreviewcellstyle{\@ifnextchar[\pgfPT@previewcellstyle{\pgfPT@previewcellstyle[1]}}%
\def\pgfPT@previewcellstyle[#1]#2{%
\edef\pgfPT@preview@cellcontents{\csname pgfPT@cellcontents@#2\endcsname}%
\def\pgfPT@builtin{0}\edef\pgfPT@tempa{#2}\relax%
\@for\pgfPT@tmp:=\pgfPT@builtincells@names\do{\edef\pgfPT@tempb{\pgfPT@tmp}\ifx\pgfPT@tempa\pgfPT@tempb\relax\def\pgfPT@builtin{1}\fi}%
\ifnum\pgfPT@builtin=1\relax\textit{Built-in style}\else\textit{User style}\fi\ \textbf{#2}\relax%
\if\pgfPT@preview@cellcontents\@empty\relax%
\textit{ doesn't exist!}%
\else%
\edef\pgfPT@numlinhas{\csname pgfPT@nlinhas@#2\endcsname}%
\edef\pgfPT@numcolunas{\csname pgfPT@ncolunas@#2\endcsname}%
\pgfmathsetlength{\pgfPT@cell@lineht}{\pgfPTcellht/\pgfPT@numlinhas}%
\global\pgfPT@cell@lineht=\pgfPT@cell@lineht%
\\ [2pt]The build command:\\ \textcolor{green!70!black}{\footnotesize\csname ppfPT@thebuildcellcom@#2\endcsname}\\ \pgfPTpreviewcell[#1]%
\fi%
}%
\def\pgfPTpreviewcell{\@ifnextchar[\pgfPT@previewcell{\pgfPT@previewcell[1]}}%
\def\pgfPT@previewcell[#1]{%
\ifx\undefined\pgfPT@preview@cellcontents\relax%
\ifx\undefined\pgfPT@cellcontents\relax\textbf{Using default cell}%
\pgfPT@make@defaultcell\relax%
\edef\pgfPT@preview@cellcontents{\csname pgfPT@cellcontents@unnamed\endcsname}%
\\ [2pt]The build command:\\ \textcolor{green!70!black}{\footnotesize\ppfPT@thebuildcellcom}\\ %
\else\textbf{Using the last cell built}%
\edef\pgfPT@preview@cellcontents{\csname pgfPT@cellcontents@unnamed\endcsname}%
\\ [2pt]The build command:\\ \textcolor{green!70!black}{\footnotesize\ppfPT@thebuildcellcom}\\ %
\fi\fi%
\makebox[\linewidth][c]{%
\begin{tikzpicture}[scale=#1]%
\pgfmathsetlength{\pgfPT@tmpht}{.5*#1*\pgfPT@cell@lineht}%
\draw[fill=black!5] (0,0) rectangle (\pgfPTcellwd,-\pgfPTcellht) node[below left,font=\fontsize{\pgfPT@tmpht}{\pgfPT@tmpht}\selectfont,inner xsep=0pt] {scale #1:1};%
\foreach \data in \pgfPT@preview@cellcontents{%
\expandafter\pgfPT@previewcell@getdata\data\relax%
\draw[blue,fill=blue!10] (\pgfPT@previewcell@data@x,-\pgfPT@previewcell@data@y) rectangle ++(\pgfPT@previewcell@data@wd,-\pgfPT@previewcell@data@ht);%
\node[font=\fontsize{\pgfPT@tmpht}{\pgfPT@tmpht}\selectfont,anchor=north,below,minimum height=#1*\pgfPT@previewcell@data@ht,inner sep=0pt] %
at (\pgfPT@previewcell@data@x+.5*\pgfPT@previewcell@data@wd,-\pgfPT@previewcell@data@y)%-.5*\pgfPT@previewcell@data@ht) %
{\pgfPT@previewcell@data@what};%
}%
\draw[xstep=\pgfPT@cell@collumnwd,ystep=\pgfPT@cell@lineht,orange,densely dotted,very thin] (0,0) grid (\pgfPTcellwd,-\pgfPTcellht);
\foreach \linha in {1,...,\pgfPT@numlinhas}{\node[left,font=\fontsize{\pgfPT@tmpht}{\pgfPT@tmpht}\selectfont,blue] %
at (0,.5*\pgfPT@cell@lineht-\pgfPT@cell@lineht*\linha) {\linha};}
\foreach \coluna in {1,...,\pgfPT@numcolunas}{\node[below,font=\fontsize{\pgfPT@tmpht}{\pgfPT@tmpht}\selectfont,red] %
at (-.5*\pgfPT@cell@collumnwd+\pgfPT@cell@collumnwd*\coluna,\pgfPT@cell@lineht) {\coluna};}
\end{tikzpicture}%
}%\fi
\let\pgfPT@preview@cellcontents\undefined\relax%
}%
\def\pgfPT@previewcell@getdata#1::#2+#3+#4+#5\relax{%
\xdef\pgfPT@previewcell@data@what{#1}%
\xdef\pgfPT@previewcell@data@x{#2}%
\xdef\pgfPT@previewcell@data@y{#3}%
\xdef\pgfPT@previewcell@data@wd{#4}%
\xdef\pgfPT@previewcell@data@ht{#5}%
}%
% internal macro -> pgfPT@make@defaultcell --------------------------------------------------
% if the user doesn�t set the layout of each cell then the default layout is build --------------
\def\pgfPT@make@defaultcell{%
%\pgfPT@defaultcelltrue%
\pgfPTbuildcell(5,3)[(1;1-2;Z),(1;3;radio),(2-3;1.5-3.5;CS),(4;1-3;name),(5;1-3;Ar)]\relax%
}%
% --------------------------------------------------------------------------------------------------
\def\pgfPT@validate@cell@content#1{%
\global\pgfPT@foundfalse%
\foreach \pgfPTacronym in \pgfPT@contents@acronyms{%
\edef\pgfPT@testa{#1}\edef\pgfPT@testb{\pgfPTacronym}%
\ifx\pgfPT@acronym@Arstar\pgfPT@testa\relax\edef\pgfPT@testa{Arstar}\fi%
\ifx\pgfPT@testa\pgfPT@testb\relax\global\pgfPT@foundtrue%
\breakforeach\fi}%
}%
% --------------------------------------------------------------------------------------------------
% validates the contents of the cell to buid and stores the csname of that content.
% -------------- Valid �acronyms� are: --->
% Z -> Atomic Number
% name -> Element Name
% CS -> Chemical Symbol
% Ar -> Relative Atomic Mass
% Arstar -> Standard Relative Atomic Mass
% radio -> Radioactivity
% R -> Atomic Radius (Empirycal)
% Rcov -> Covalente Radius
% Rion -> Ionic Radius
% Ei -> First Ionization Energy
% eneg -> Electronegativity (Pauling)
% eaff -> Electroaffinity
% O -> Oxidation States
% Tmelt -> Melting Point (Kelvin)
% TmeltC -> Melting Point (degrees Celsius)
% Tboil -> Boiling Point (Kelvin)
% TboilC -> Boiling Point (degrees Celsius)
% eDist -> Electron Distribuition
% eConfign -> Electronic Configuration (increasing n)
% eConfignl -> Electronic Configuration (increasing n+l)
% d -> Density
% Cp -> Specific heat capacity (25�C, 100kPa)
% kT -> Thermal Condutivity (25�C)
% ls -> Lattice Structure
% lsa -> Lattice constant: a
% lsb -> Lattice constant: b
% lsc -> Lattice constant: c
% lsca -> Lattice c/a ratio:
% DiscY -> Discover Year
% DiscC -> Discover Country
% spectra: Visible range spectral lines
\def\pgfPT@acronym@Arstar{Ar*}%
\def\pgfPT@contents@acronyms{Z,name,CS,Ar,Arstar,radio,R,Rcov,Rion,Ei,eneg,eaff,O,Tmelt,TmeltC,Tboil,TboilC,eDist,eConfign,eConfignl,d,Cp,kT,ls,lsa,lsb,lsc,lsca,DiscY,DiscC,spectra}%
% --------------------------------------------------------------------------------------------------
% internal macro -> \pgfPT@loadcell[cell style name]
% INPUT:
%       value of key 'cell style', built via \pgfPTbuildcellstyle[name]
%       if 'cell style' is empty last cell builded via \pgfPTbuildcell is used or if none cell is yet
%       builded then the default cell is called
% --------------------------------------------------------------------------------------------------
% necessary if's to use the data in the cell, e.g., \ifpgfPT@name (is true)
% loads the contents to \pgfPT@name@x \pgfPT@name@ \pgfPT@name@wd \pgfPT@name@ht{} (via \let...)
\newif\ifpgfPT@Z\pgfPT@Zfalse%
\newif\ifpgfPT@name\pgfPT@namefalse%
\newif\ifpgfPT@CS\pgfPT@CSfalse%
\newif\ifpgfPT@Ar\pgfPT@Arfalse%
\newif\ifpgfPT@Arstar\pgfPT@Arstarfalse%
\newif\ifpgfPT@radio\pgfPT@radiofalse%
\newif\ifpgfPT@R\pgfPT@Rfalse%
\newif\ifpgfPT@Rcov\pgfPT@Rcovfalse%
\newif\ifpgfPT@Rion\pgfPT@Rionfalse%
\newif\ifpgfPT@Ei\pgfPT@Eifalse%
\newif\ifpgfPT@eneg\pgfPT@enegfalse%
\newif\ifpgfPT@eaff\pgfPT@eafffalse%
\newif\ifpgfPT@O\pgfPT@Ofalse%
\newif\ifpgfPT@Tmelt\pgfPT@Tmeltfalse%
\newif\ifpgfPT@TmeltC\pgfPT@TmeltCfalse%
\newif\ifpgfPT@Tboil\pgfPT@Tboilfalse%
\newif\ifpgfPT@TboilC\pgfPT@TboilCfalse%
\newif\ifpgfPT@eDist\pgfPT@eDistfalse%
\newif\ifpgfPT@eConfign\pgfPT@eConfignfalse%
\newif\ifpgfPT@eConfignl\pgfPT@eConfignlfalse%
\newif\ifpgfPT@d\pgfPT@dfalse%
\newif\ifpgfPT@Cp\pgfPT@Cpfalse%
\newif\ifpgfPT@kT\pgfPT@kTfalse%
\newif\ifpgfPT@ls\pgfPT@lsfalse%
\newif\ifpgfPT@lsa\pgfPT@lsafalse%
\newif\ifpgfPT@lsb\pgfPT@lsbfalse%
\newif\ifpgfPT@lsc\pgfPT@lscfalse%
\newif\ifpgfPT@lsca\pgfPT@lscafalse%
\newif\ifpgfPT@DiscY\pgfPT@DiscYfalse%
\newif\ifpgfPT@DiscC\pgfPT@DiscCfalse%
\newif\ifpgfPT@spectra\pgfPT@spectrafalse%
%\newif\ifpgfPT@DiscBy\pgfPT@DiscByfalse%
% --------------------------------------------------------------------------------------------------
\def\loadcell#1{Loading cell data\\ --\ #1\ -- CONTENTS\pgfPT@loadcell[#1]% DEBUG
\@for\@tmp:=\pgfPT@load@cellcontents\do{\\ \@tmp}}% DEBUG
% --------------------------------------------------------------------------------------------------
\def\pgfPT@loadcell[#1]{\ignorespaces%
\edef\pgfPT@testa{#1}%
\ifx\pgfPT@testa\empty\relax%
    \ifx\undefined\pgfPT@cellcontents\relax\pgfPT@make@defaultcell\relax%
    \else%
        \ifx\undefined\pgfPT@cellcontents@unnamed\relax\pgfPT@make@defaultcell\relax\fi%
    \fi%
    \edef\pgfPT@load@cellcontents{\pgfPT@cellcontents@unnamed}%
\else% cell style is provided
    \expandafter\ifx\csname pgfPT@cellcontents@#1\endcsname\relax% Testing if macro pgfPT@cellcontents@<name> is defined (if not expands to \relax..)
        \PackageError{pgfPT}{Invalid name '#1' for 'cell style'. Last known style will be used...}%
        \edef\pgfPT@load@cellcontents{\pgfPT@cellcontents}%
    \else% cell style name is OK...
        \edef\pgfPT@load@cellcontents{\csname pgfPT@cellcontents@#1\endcsname}%
    \fi%
\fi%
% Setting all if<what>'s to false:
\@for\@pgfPT@tmp:=\pgfPT@contents@acronyms\do{%
\expandafter\csname pgfPT@\@pgfPT@tmp false\endcsname\relax}%
% Setting the data to the macros:
% x -> \pgfPT@<what>@x
% y -> \pgfPT@<what>@y
% wd -> \pgfPT@<what>@wd
% ht -> \pgfPT@<what>@ht
% and turning the if<what> to true
% what -> \@pgfPT@<what>rtue
    \ifdim\pgfPTcellwd=34pt\relax%
        \xdef\pgfPT@loadcell@scaleX{1}%
    \else%
        \pgfmathparse{\pgfPTcellwd/34pt}\xdef\pgfPT@loadcell@scaleX{\pgfmathresult}%
    \fi%
    \xdef\pgfPT@loadcell@wd{\pgfPTcellwd}%
    \ifdim\pgfPTcellwd=38.25pt\relax%
        \xdef\pgfPT@loadcell@scaleY{1}%
    \else%
        \pgfmathparse{\pgfPTcellht/38.25pt}\xdef\pgfPT@loadcell@scaleY{\pgfmathresult}%
    \fi%
    \xdef\pgfPT@loadcell@ht{\pgfPTcellht}%
\xdef\pgfPT@legend@content{}\relax%
\@for\@pgfPT@tmp:=\pgfPT@load@cellcontents\do{%
% loading the current entry of cell contents to get <what>,x,y,wd,ht
\expandafter\pgfPT@previewcell@getdata\@pgfPT@tmp\relax%
\xdef\pgfPT@legend@content{\pgfPT@legend@content,\pgfPT@previewcell@data@what}\relax%
\pgfmathparse{\pgfPT@loadcell@scaleX*\pgfPT@previewcell@data@x}%
\expandafter\xdef\csname pgfPT@data@\pgfPT@previewcell@data@what @x\endcsname{\pgfmathresult pt}%
\pgfmathparse{\pgfPT@loadcell@scaleY*\pgfPT@previewcell@data@y}%
\expandafter\xdef\csname pgfPT@data@\pgfPT@previewcell@data@what @y\endcsname{\pgfmathresult pt}%
\pgfmathparse{\pgfPT@loadcell@scaleX*\pgfPT@previewcell@data@wd}%
\expandafter\xdef\csname pgfPT@data@\pgfPT@previewcell@data@what @wd\endcsname{\pgfmathresult pt}%
\pgfmathparse{\pgfPT@loadcell@scaleY*\pgfPT@previewcell@data@ht}%
\expandafter\xdef\csname pgfPT@data@\pgfPT@previewcell@data@what @ht\endcsname{\pgfmathresult pt}%
\expandafter\csname pgfPT@\pgfPT@previewcell@data@what true\endcsname\relax%
}%end do
\expandafter\@pgfPT@rem@initcomma\pgfPT@legend@content\relax%
}% end load cell
\def\@pgfPT@rem@initcomma,#1\relax{\xdef\pgfPT@legend@content{#1}\relax}%
% --------------------------------------------------------------------------------------------------
\def\pgfPTresetcell{\global\let\pgfPT@cellcontents\undefined\relax}%
\endinput
