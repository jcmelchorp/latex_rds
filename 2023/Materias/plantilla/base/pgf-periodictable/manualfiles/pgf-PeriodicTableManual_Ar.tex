\vfill%
\subsubsection{\texorpdfstring{\ding{224} The atomic weight}{The atomic weight}}
%%%%%%%%%%%%%%%%%%%%%%%%%%%%%%%%%%%%%%%%%%%%%%%%%%%%%%%%%%%%
% Ar color
\pgfPTMoption{4}{Ar color}{black}{%
Sets the relative atomic mass color.
}
\\ [5pt]\pgfPTMmacrobox{pgfPT}[Z list={1,...,36},Ar color=red]%
\\ [5pt]\makebox[\linewidth][c]{\scalebox{.6}{\pgfPT[Z list={1,...,36},Ar color=red]}}%
\\ [0pt]\pgfPTendoption%
% Ar font
\newpage\ \\ [-32pt]%
\pgfPTMoption{4}{Ar font}{\string\tiny\string\bfseries}{%
Sets the relative atomic mass font.
}
\\ [5pt]\pgfPTMmacrobox{pgfPT}[Z list={1,...,36},Ar font=\string\scriptsize\string\bfseries]%
\\ [5pt]\makebox[\linewidth][c]{\scalebox{.6}{\pgfPT[Z list={1,...,36},Ar font=\scriptsize\bfseries]}}%
\\ [0pt]\pgfPTendoption%
% Ar label
\vfill%
\pgfPTMoption{4}{Ar label}{m}{%
Sets the label to be used within the relative atomic mass description. When set to \sq{m} the term \green{mass} is used and when set to \sq{w} the term \green{weight} is used, resulting in \textit{Relative Atomic Mass} and \textit{Atomic Weight} labels respectively.
}
\\ [5pt]\pgfPTMmacrobox{pgfPT}[Z list={1,...,36}]%
\\ [5pt]\makebox[\linewidth][c]{\scalebox{.6}{\pgfPT[Z list={1,...,36}]}}%
\\ [10pt]\pgfPTMmacrobox{pgfPT}[Z list={1,...,36},Ar label=w]%
\\ [5pt]\makebox[\linewidth][c]{\scalebox{.6}{\pgfPT[Z list={1,...,36},Ar label=w]}}%
\\ [0pt]\pgfPTendoption%
% Ar precision
\vfill%
\pgfPTMoption{4}{Ar precision}{-1}{%
Sets the relative atomic mass precision, \ie, the decimal places displayed in the relative atomic mass value, performing the respective rounding, without zero padding the value.
\\ [10pt]\tikz{\node[text width=\linewidth-.6666em,fill=orange!5!white,draw=orange,rounded corners=2pt] {\textbf{\red{NOTE}}:\\ Rounding is performed over the relative atomic mass data values witch actually have a maximum of 4 decimal places. So giving this key a value of -1 (the value of relative atomic mass as-is) or 4 has the same effect.
\\ \textit{Therefore the values provided to this key should be any integer between -1 and 3, \ie, -1, 0, 1, 2 or 3. Any other integer provided will be processed as -1.}};}
}
\vfill\newpage%\\ [5pt]
\pgfPTMmacrobox{pgfPT}[Z list={1,...,36}]%
\\ [5pt]\makebox[\linewidth][c]{\scalebox{.6}{\pgfPT[Z list={1,...,36}]}}%
%\vfill%
\\ [5pt]\pgfPTMmacrobox{pgfPT}[Z list={1,...,36},Ar precision=2]%
\\ [5pt]\makebox[\linewidth][c]{\scalebox{.6}{\pgfPT[Z list={1,...,36},Ar precision=2]}}%
\\ [5pt]\pgfPTMmacrobox{pgfPT}[Z list={1,...,36},Ar precision=1]%
\\ [5pt]\makebox[\linewidth][c]{\scalebox{.6}{\pgfPT[Z list={1,...,36},Ar precision=1]}}%
\\ [0pt]\pgfPTendoption%
\vfill%
% Ar={c=??,f=??,p=??,l=??}
%       Ar/.default={c=black,f=\tiny\bfseries,l=m,p=-1}
\pgfPTMstyle{4}{Ar}{\{c=black,f=\string\tiny\string\bfseries,l=m,p=-1\}}%
{\ \\ [-3pt]\textit{Pseudo style} to set the keys: Ar \textbf{c}olor, Ar \textbf{f}ont, Ar \textbf{l}abel and/or Ar \textbf{p}recision.
None of the \textit{keys} -- c, f, l and \mbox{p -- are} mandatory.
\\ [3pt]\makebox[\linewidth][c]{\use{Ar=\{c=<color>,f=<font commands>,l=<m|w>p=<integer value>\}}}%
}%
\\ [5pt]\pgfPTMmacrobox{pgfPT}[Z list={1,...,36},Ar={c=red!50!black,p=2}]%
\\ [5pt]\makebox[\linewidth][c]{\scalebox{.6}{\pgfPT[Z list={1,...,36},Ar={c=red!50!black,p=2}]}}%
\vfill\newpage%\\ [10pt]
\pgfPTMmacrobox{pgfPT}[Z list={1,...,36},Ar={c=red!50!black,p=1,l=w}]%
\\ [5pt]\makebox[\linewidth][c]{\scalebox{.6}{\pgfPT[Z list={1,...,36},Ar={c=red!50!black,p=1,l=w}]}}%
\\ [0pt]\pgfPTendstyle%
\endinput
