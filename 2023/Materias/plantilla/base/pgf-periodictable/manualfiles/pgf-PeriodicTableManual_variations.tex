\subsubsection{\texorpdfstring{\ding{224} Periodic variations}{Periodic variations}}
%%%%%%%%%%%%%%%%%%%%%%%%%%%%%%%%%%%%%%%%%%%%%%%%%%%%%%%%%%%%
% show periodic variations=false
\pgfPTMoption{4}{show periodic variations}{false}%
{When set to \red{true} the periodic variations -- for atomic radius, ionization energy and/or electron affinity -- are shown with two \textit{arrows}. One horizontal arrow is placed at the top of the Periodic Table for the variation over the period and the other vertically to the left of the Periodic Table for the variation over the group.
\\ [5pt]\tikz{\node[text width=\linewidth-.6666em,fill=orange!5!white,draw=orange,rounded corners=2pt] {\textbf{\orange{NOTE}}:\\ The variations are only shown when the \textit{base cell} of the Periodic Table contains the atomic radius, the ionization energy and/or the electron affinity. If none of them is present setting this key (\red{show periodic variations}) has no effect.};}
}%
\vfill%\\ [5pt]
\pgfPTMmacrobox{pgfPTstyle}[Z list=spd,show title=false]%
\pgfPTstyle[Z list=spd,show title=false]%
\\ [-1pt]\pgfPTMmacrobox{pgfPT}[show periodic variations]%
\\ [0pt]\makebox[\linewidth][c]{\scalebox{.6}{\pgfPT[show periodic variations]}}%
\\ [10pt]\pgfPTMmacrobox{pgfPT}[show periodic variations,cell style=pgfPTR]%
\\ [5pt]\makebox[\linewidth][c]{\scalebox{.6}{\pgfPT[show periodic variations,cell style=pgfPTR]}}%
\vfill\newpage%\\ [10pt]
\pgfPTMmacrobox{pgfPT}[show periodic variations,cell style=pgfPTREi]%
\\ [5pt]\makebox[\linewidth][c]{\scalebox{.6}{\pgfPT[show periodic variations,cell style=pgfPTREi]}}%
\\ [5pt]\pgfPTendoption%
% varR color
\pgfPTMoption{4}{varR color}{\pgfPTMcolorDemo{colorvariations}{RGB: 128,191,191}}%
{Sets the color used in the filling of the \textit{arrows} for the atomic radius variations.
\\ [10pt]\tikz{\node[text width=\linewidth-.6666em,fill=orange!5!white,draw=orange,rounded corners=2pt] {\textbf{\orange{NOTE}}:\\ The color provided to \red{varR color} could be any defined color via the command \texttt{\normalsize\textbackslash definecolor} or by \textit{mixing} colors, using, for instance, the syntax \texttt{\normalsize color1!value!color2}, as explained in the \href{https://ctan.org/pkg/xcolor}{xcolor} package documentation.};}
}%
\\ [5pt]\pgfPTMmacrobox{pgfPT}[show periodic variations,cell style=pgfPTR,varR color=teal,R color=purple]%
\\ [5pt]\makebox[\linewidth][c]{\scalebox{.6}{\pgfPT[show periodic variations,cell style=pgfPTR,varR color=purple!50!white, R color=purple]}}%
\\ [5pt]\pgfPTendoption%
% varR font
\vfill%
\pgfPTMoption{4}{varR font}{\string\footnotesize\string\bfseries}%
{Sets the font for the text displayed inside the arrow, describing the variation of the atomic radius.}%
\vfill\newpage%\\ [5pt]
\pgfPTMmacrobox{pgfPT}[show periodic variations,cell style=pgfPTR,varR font=\string\small\string\itshape]%
\\ [5pt]\makebox[\linewidth][c]{\scalebox{.6}{\pgfPT[show periodic variations,cell style=pgfPTR,varR font=\small\itshape]}}%
\\ [5pt]\pgfPTendoption%
% varR font color
\pgfPTMoption{4}{varR font color}{\pgfPTMcolorDemo{colorvariations!50!black}{(value of varR color)!50!black}}%
{Sets the color of the text showing the atomic radius variations displayed inside the corresponding arrows.
\\ \textit{See the note in \red{varR color}}.}%
% varEi color
\pgfPTMoption{4}{varEi color}{\pgfPTMcolorDemo{colorvariations}{RGB: 128,191,191}}%
{Sets the color used in the filling of the \textit{arrows} for the ionization energy variations.
\\ \textit{See the note in \red{varR color}}.}%
% varEi font
\pgfPTMoption{4}{varEi font}{\string\footnotesize\string\bfseries}%
{Sets the font for the text displayed inside the arrow, describing the variation of the ionization energy.}%
% varEi font color
\pgfPTMoption{4}{varEi font color}{\pgfPTMcolorDemo{colorvariations!50!black}{(value of varEi color)!50!black}}%
{Sets the color of the text showing the ionization energy variations displayed inside the corresponding arrows.
\\ \textit{See the note in \red{varR color}}.}%
% vareaff color
\pgfPTMoption{4}{vareaff color}{\pgfPTMcolorDemo{colorvariations}{RGB: 128,191,191}}%
{Sets the color used in the filling of the \textit{arrows} for the electron affinity variations.
\\ \textit{See the note in \red{varR color}}.}%
% vareaff font
\pgfPTMoption{4}{vareaff font}{\string\footnotesize\string\bfseries}%
{Sets the font for the text displayed inside the arrow, describing the variation of the electron affinity.}%
% vareaff font color
\pgfPTMoption{4}{vareaff font color}{\pgfPTMcolorDemo{colorvariations!50!black}{(value of vareaff color)!50!black}}%
{Sets the color of the text showing the electron affinity variations displayed inside the corresponding arrows.
\\ \textit{See the note in \red{varR color}}.}%
\vfill%
% var font (style)
\pgfPTMstyle{4}{var font}{\string\footnotesize\string\bfseries}%
{Style to set a common font for the variations along the Periodic Table.
\\ Setting \red{var font=<font commands>} is equivalent to setting \red{\{varR font=<font commands>,\\varEi font=<font commands>,vareaff font=<font commands>\}}.
}%
\vfill\newpage%\\ [5pt]
\pgfPTMmacrobox{pgfPT}[show periodic variations,cell style=pgfPTREi,var font=\string\small\string\itshape]%
\\ [5pt]\makebox[\linewidth][c]{\scalebox{.6}{\pgfPT[show periodic variations,cell style=pgfPTREi,var font=\small\itshape]}}%
\\ [5pt]\pgfPTendstyle%
% var color (style)
\pgfPTMstyle{4}{var color}{\pgfPTMcolorDemo{colorvariations}{RGB: 128,191,191}}%
{Style to set a common color for the variations along the Periodic Table.
\\ Setting \red{var color=<color>} is equivalent to setting \red{\{varR color=<color>,varEi color=<color>,\\vareaff color=<color>\}}.
\\ [10pt]\tikz{\node[text width=\linewidth-.6666em,fill=cyan!5!white,draw=cyan,rounded corners=2pt] {\textbf{\blue{NOTE}}:\\ The color provided to \red{var color} could be any defined color via the command \texttt{\normalsize\textbackslash definecolor} or by \textit{mixing} colors, using, for instance, the syntax \texttt{\normalsize color1!value!color2}, as explained in the \href{https://ctan.org/pkg/xcolor}{xcolor} package documentation.
\\ [2pt]\green{\textit{Keep in mind that setting the variations colors also changes the default text colors for them.}}};}
}%
\\ [5pt]\pgfPTMmacrobox{pgfPT}[show periodic variations,cell style=pgfPTREi,var color=blue!50!white]%
\\ [5pt]\makebox[\linewidth][c]{\scalebox{.6}{\pgfPT[show periodic variations,cell style=pgfPTREi,var color=blue!50!white]}}%
\\ [5pt]\pgfPTendstyle%
% varR={c=??,f=??,fc=??}
%       varR/.default={c=colorvariations,f=\footnotesize\bfseries}
\pgfPTMstyle{4}{varR}{\{c=colorvariations,f=\string\footnotesize\string\bfseries\}}%
{\textit{Pseudo style} to set the keys: varR \textbf{c}olor, varR \textbf{f}ont and/or varR \textbf{f}ont \textbf{c}olor. None of the \textit{keys} -- c, f and fc -- are mandatory.
\\ [5pt]\tikz{\node[text width=\linewidth-.6666em,fill=cyan!5!white,draw=cyan,rounded corners=2pt] {\textbf{\blue{NOTE}}:\\ The color provided to \red{varR color} could be any defined color via the command \texttt{\normalsize\textbackslash definecolor} or by \textit{mixing} colors, using, for instance, the syntax \texttt{\normalsize color1!value!color2}, as explained in the \href{https://ctan.org/pkg/xcolor}{xcolor} package documentation.};}
\\ [5pt]\makebox[\linewidth][c]{\use{varR=\{c=<color>,f=<font commands>,fc=<color>\}}}%
}%
\\ [5pt]\pgfPTMmacrobox{pgfPT}[show periodic variations,cell style=pgfPTREi, varR={c=green!70!black,f=\string\small\string\bfseries}]%
\\ [5pt]\makebox[\linewidth][c]{\scalebox{.6}{\pgfPT[show periodic variations,cell style=pgfPTREi,varR={c=green!70!black,f=\small\bfseries}]}}%
\\ [0pt]\pgfPTendstyle%
\vfill%
% varEi={c=??,f=??,fc=??}
%       varEi/.default={c=colorvariations,f=\footnotesize\bfseries}
\pgfPTMstyle{4}{varEi}{\{c=colorvariations,f=\string\footnotesize\string\bfseries\}}%
{\textit{Pseudo style} to set the keys: varEi \textbf{c}olor, varEi \textbf{f}ont and/or varEi \textbf{f}ont \textbf{c}olor. None of the \textit{keys} -- c, f and fc -- are mandatory.
\\ [10pt]\tikz{\node[text width=\linewidth-.6666em,fill=cyan!5!white,draw=cyan,rounded corners=2pt] {\textbf{\blue{NOTE}}:\\ The color provided to \red{varEi color} could be any defined color via the command \texttt{\normalsize\textbackslash definecolor} or by \textit{mixing} colors, using, for instance, the syntax \texttt{\normalsize color1!value!color2}, as explained in the \href{https://ctan.org/pkg/xcolor}{xcolor} package documentation.};}
\\ [10pt]\makebox[\linewidth][c]{\use{varEi=\{c=<color>,f=<font commands>,fc=<color>\}}}%
}%
\\ [5pt]\pgfPTMmacrobox{pgfPT}[show periodic variations,cell style=pgfPTREi, varR={c=green!70!black,f=\string\small\string\bfseries}, %
varEi={c=lime!70!black,f=\string\small\string\bfseries}]%
\vfill\newpage%\\ [5pt]
\makebox[\linewidth][c]{\scalebox{.6}{\pgfPT[show periodic variations,cell style=pgfPTREi,varR={c=green!70!black,f=\small\bfseries},%
varEi={c=lime!70!black,f=\small\itshape}]}}%
\\ [0pt]\pgfPTendstyle%
%\vfill%
% vareaff={c=??,f=??,fc=??}
%       vareaff/.default={c=colorvariations,f=\footnotesize\bfseries}
\pgfPTMstyle{4}{vareaff}{\{c=colorvariations,f=\string\footnotesize\string\bfseries\}}%
{\textit{Pseudo style} to set the keys: vareaff \textbf{c}olor, vareaff \textbf{f}ont and/or vareaff \textbf{f}ont \textbf{c}olor. None of the \textit{keys} -- c, f and fc -- are mandatory.
\\ [10pt]\tikz{\node[text width=\linewidth-.6666em,fill=cyan!5!white,draw=cyan,rounded corners=2pt] {\textbf{\blue{NOTE}}:\\ The color provided to \red{vareaff color} could be any defined color via the command \texttt{\normalsize\textbackslash definecolor} or by \textit{mixing} colors, using, for instance, the syntax \texttt{\normalsize color1!value!color2}, as explained in the \href{https://ctan.org/pkg/xcolor}{xcolor} package documentation.};}
\\ [10pt]\makebox[\linewidth][c]{\use{vareaff=\{c=<color>,f=<font commands>,fc=<color>\}}}%
}%
\\ [5pt]\pgfPTMmacrobox{pgfPT}[show periodic variations,cell style=pgfPTeaff, vareaff={c=purple!70!white,f=\string\small\string\bfseries,fc=white}]%
\\ [10pt]\makebox[\linewidth][c]{\scalebox{.6}{\pgfPT[show periodic variations,cell style=pgfPTeaff,vareaff={c=purple!70!white,f=\small\bfseries,fc=white}]}}%
\\ [0pt]\pgfPTendstyle%
\\ [5pt]\pgfPTMmacrobox{pgfPTresetstyle}[]%
\pgfPTresetstyle%
\endinput
