\vfill%
\subsubsection{\texorpdfstring{\ding{224} \itshape The other contents}{The other contents}}
%%%%%%%%%%%%%%%%%%%%%%%%%%%%%%%%%%%%%%%%%%%%%%%%%%%%%%%%%%%%
For all the \textit{other contents} available for the cells of the periodic table, two keys can be set: \red{<content name> color} and \red{<content name> font}.
\\ [10pt]\tikz{\node[text width=\linewidth-.6666em,fill=cItemList!5!white,draw=cItemList,rounded corners=2pt] {\textbf{The \red{<content name>}'s list}:
\\ [5pt]\begin{minipage}{.45\linewidth}\footnotesize
\begin{itemlist}
%\item\red{\textbf{name}}:\hspace{.5ex}element name
\item\red{\textbf{R}}:\hspace{.5ex}atomic radius
\item\red{\textbf{Rcov}}:\hspace{.5ex}covalente radius
\item\red{\textbf{Rion}}:\hspace{.5ex}ionic radius
\item\red{\textbf{Ei}}:\hspace{.5ex}first ionization energy
\item\red{\textbf{eneg}}:\hspace{.5ex}electronegativity (Pauling)
\item\red{\textbf{eaff}}:\hspace{.5ex}electroaffnity
\item\red{\textbf{O}}:\hspace{.5ex}oxidation states
\item\red{\textbf{Tmelt}}:\hspace{.5ex}melting point (Kelvin)
\item\red{\textbf{TmeltC}}:\hspace{.5ex}melting point (Celsius degrees)
\item\red{\textbf{Tboil}}:\hspace{.5ex}boiling point (Kelvin)
\item\red{\textbf{TboilC}}:\hspace{.5ex}boiling point (Celsius degrees)
\end{itemlist}
\end{minipage}\begin{minipage}{.55\linewidth}\footnotesize
\begin{itemlist}
\item\red{\textbf{eConfign}}:\hspace{.5ex}electronic configuration (increasing n)
\item\red{\textbf{eConfignl}}:\hspace{.5ex}electronic configuration (increasing n+$\ell$)
%\item\red{\textbf{d}}:\hspace{.5ex}density
\item\red{\textbf{Cp}}:\hspace{.5ex}specific heat capacity
\item\red{\textbf{kT}}:\hspace{.5ex}thermal conductivity
\item\red{\textbf{lsa}}:\hspace{.5ex}lattice constant -- a
\item\red{\textbf{lsb}}:\hspace{.5ex}lattice constant -- b
\item\red{\textbf{lsc}}:\hspace{.5ex}lattice constant -- c
\item\red{\textbf{lsca}}:\hspace{.5ex}lattice c/a ratio
\item\red{\textbf{DiscC}}:\hspace{.5ex}discover country
\item\red{\textbf{spectra}}:\hspace{.5ex}visible range spectral lines
\end{itemlist}
\end{minipage}
};}\vfill%
% <content name> color
\pgfPTMoption{4}{<content name> color}{black}{%
Sets the \red{<content name>} color.
}
\\ [5pt]\pgfPTMmacrobox{pgfPT}[Z list={1,...,36},name color=blue]%
\\ [10pt]\makebox[\linewidth][c]{\scalebox{.6}{\pgfPT[Z list={1,...,36},name color=blue]}}%
\\ [5pt]\pgfPTendoption%
\vfill%
% <content name> font
\newpage\ \\ [-32pt]%
\pgfPTMoption{4}{<content name> font}{\string\tiny\string\bfseries}{%
Sets the \red{<content name>} font.
}
\\ [5pt]\pgfPTMmacrobox{pgfPT}[Z list={1,...,36},name font=\string\tiny\string\itshape]%
\\ [10pt]\makebox[\linewidth][c]{\scalebox{.6}{\pgfPT[Z list={1,...,36},name font=\tiny\itshape]}}%
\\ [5pt]\pgfPTendoption%
% cell font (style) all fonts except Z, CS
\pgfPTMstyle{4}{cell font}{\string\bfseries\string\tiny}%
{Style to set the font for all cell contents, except for the Z and Chemical Symbol fonts.}%
\\ [5pt]\pgfPTMmacrobox{pgfPT}[Z list={1,...,36},cell font=\string\tiny\string\itshape]%
\\ [10pt]\makebox[\linewidth][c]{\scalebox{.6}{\pgfPT[Z list={1,...,36},cell font=\tiny\itshape]}}%
\\ [5pt]\pgfPTendstyle%
% cell color (style) all colors except Z, CS
\pgfPTMstyle{4}{cell color}{black}%
{Style to set the color for all cell contents, except for the Z and Chemical Symbol colors.}%
\\ [5pt]\pgfPTMmacrobox{pgfPT}[Z list={1,...,36},cell color=blue]%
\\ [10pt]\makebox[\linewidth][c]{\scalebox{.6}{\pgfPT[Z list={1,...,36},cell color=blue]}}%
\\ [5pt]\pgfPTendstyle%
\vfill%\\ [10pt]
The precision of the \textit{other contents}, which have numerical values, can also be set by a key. \textit{Atomic radius, covalent radius, and ionic radius all have integer values, so precision does not apply to them}.
\vfill%
\newpage\ \\ [-32pt]%
% E precision
\pgfPTMoption{4}{E precision}{-1}{%
Sets the first ionization energy and the electroaffnity precision, \ie, the decimal places displayed in their value, performing the respective rounding, without zero padding the value.
\\ [10pt]\tikz{\node[text width=\linewidth-.6666em,fill=orange!5!white,draw=orange,rounded corners=2pt] {\textbf{\red{NOTE}}:\\ Rounding is performed over energy values witch actually have a maximum of 3 decimal places. So giving this key a value of -1 (the value of the energy as-is) or 3 has the same effect.
\\ \textit{Therefore the values provided to this key should be any integer between -1 and 2, \ie, -1, 0, 1 or 2. Any other integer provided will be processed as -1.}};}
}
\vfill%\\ [10pt]
\pgfPTresetstyle\pgfPTMbuildcellstyle{myE}(5,3)[(1;1-2;Z),(1;3;radio),(2-3;1.5-3.5;CS),(4;1-3;name),(5;1-2.5;Ei), %
(5;2.5-3;eaff)]%
\pgfPTbuildcellstyle{myE}(5,3)[(1;1-2;Z),(1;3;radio),(2-3;1.5-3.5;CS),(4;1-3;name),(5;1-2.5;Ei),(5;2.5-3;eaff)]%
\\ [-4pt]\pgfPTMmacrobox[l]{pgfPTstyle}[Z list={1,...,54},cell style=myE,show title=false]%
\pgfPTstyle[Z list={1,...,54},cell style=myE,show title=false]%
\\ [-4pt]\pgfPTMmacrobox{pgfPT}[]%
\\ [10pt]\makebox[\linewidth][c]{\scalebox{.6}{\pgfPT}}%
\\ [10pt]\pgfPTMmacrobox{pgfPT}[E precision=0]%
\\ [10pt]\makebox[\linewidth][c]{\scalebox{.6}{\pgfPT[E precision=0]}}%
\\ [10pt]\pgfPTMmacrobox{pgfPT}[E precision=1]%
\\ [10pt]\makebox[\linewidth][c]{\scalebox{.6}{\pgfPT[E precision=1]}}%
\vfill%
\newpage%\\ [10pt]
\pgfPTMmacrobox{pgfPT}[E precision=2]%
\\ [10pt]\makebox[\linewidth][c]{\scalebox{.6}{\pgfPT[E precision=2]}}%
\\ [10pt]\pgfPTMmacrobox{pgfPT}[E precision=3]%
\\ [10pt]\makebox[\linewidth][c]{\scalebox{.6}{\pgfPT[E precision=3]}}%
\\ [0pt]\pgfPTendoption%
% T precision
\pgfPTMoption{4}{T precision}{-1}{%
Sets the melting point an boiling point precision, \ie, the decimal places displayed in their value, performing the respective rounding, without zero padding the value.
\\ [10pt]\tikz{\node[text width=\linewidth-.6666em,fill=orange!5!white,draw=orange,rounded corners=2pt] {\textbf{\red{NOTE}}:\\ Rounding is performed over melting or boiling point values witch actually have a maximum, res\-pectively, of 4 or 2 decimal places. So giving this key a value of -1 (the value of the melting or boiling point as-is) or, respectively, 4 or 2 has the same effect.
\\ \textit{Therefore the values provided to this key should be any integer between -1 and 3 or 2. Any other integer provided will be processed as -1.}};}
}
\\ [10pt]\pgfPTMbuildcellstyle{myT}(6,3)[(1;1-2;Z),(1;3;radio),(2-3;1.5-3.5;CS),(4;1-3;name),(5;1-2.5;Tmelt), %
(5;2.5-3;Tboil),(6;1-2.5;TmeltC),(6;2.5-3;TboilC)]%
\pgfPTbuildcellstyle{myT}(6,3)[(1;1-2;Z),(1;3;radio),(2-3;1.5-3.5;CS),(4;1-3;name),(5;1-2.5;Tmelt),(5;2.5-3;Tboil),(6;1-2.5;TmeltC),(6;2.5-3;TboilC)]%
\\ [-4pt]\pgfPTMmacrobox[l]{pgfPTstyle}[Z list={1,...,36},cell style=myT,Tmelt color=blue!50!black,TmeltC color=blue,Tboil color=red!50!black,TboilC color=red,show title=false]%
\pgfPTstyle[Z list={1,...,36},cell style=myT,Tmelt color=blue!50!black,TmeltC color=blue,Tboil color=red!50!black,TboilC color=red,show title=false]%
\\ [-4pt]\pgfPTMmacrobox{pgfPT}[]%
\\ [10pt]\makebox[\linewidth][c]{\scalebox{.6}{\pgfPT}}%
\\ [5pt]\pgfPTMmacrobox{pgfPT}[T precision=0]%
\\ [10pt]\makebox[\linewidth][c]{\scalebox{.6}{\pgfPT[T precision=0]}}%
\\ [5pt]\pgfPTMmacrobox{pgfPT}[T precision=1]%
\\ [10pt]\makebox[\linewidth][c]{\scalebox{.6}{\pgfPT[T precision=1]}}%
\\ [5pt]\pgfPTMmacrobox{pgfPT}[T precision=2]%
\\ [10pt]\makebox[\linewidth][c]{\scalebox{.6}{\pgfPT[T precision=2]}}%
\\ [5pt]\pgfPTMmacrobox{pgfPT}[T precision=3]%
\\ [10pt]\makebox[\linewidth][c]{\scalebox{.6}{\pgfPT[T precision=3]}}%
\\ [5pt]\pgfPTMmacrobox{pgfPT}[T precision=4]%
\\ [10pt]\makebox[\linewidth][c]{\scalebox{.6}{\pgfPT[T precision=4]}}%
\\ [0pt]\pgfPTendoption%
\newpage\ \\ [-32pt]%
% Cp precision
\pgfPTMoption{4}{Cp precision}{-1}{%
Sets the specific heat capacity precision, \ie, the decimal places displayed in their value, performing the respective rounding, without zero padding the value.
\\ [10pt]\tikz{\node[text width=\linewidth-.6666em,fill=orange!5!white,draw=orange,rounded corners=2pt] {\textbf{\red{NOTE}}:\\ Rounding is performed over density values witch actually have a maximum 3 decimal places. So giving this key a value of -1 (the value of the melting or boiling point as-is) or 3 has the same effect.
\\ \textit{Therefore the values provided to this key should be any integer between -1 and 2. Any other integer provided will be processed as -1.}};}
}
\vfill%\\ [10pt]
\pgfPTMbuildcellstyle{myCp}(5,3)[(1;1-2;Z),(1;3;radio),(2-3;1.5-3.5;CS),(4;1-3;name),(5;1-3;Cp)]%
\pgfPTbuildcellstyle{myCp}(5,3)[(1;1-2;Z),(1;3;radio),(2-3;1.5-3.5;CS),(4;1-3;name),(5;1-3;Cp)]%
\\ [-4pt]\pgfPTMmacrobox[l]{pgfPTstyle}[Z list={1,...,36},cell style=myCp]%
\pgfPTstyle[Z list={1,...,36},cell style=myCp]%
\\ [-4pt]\pgfPTMmacrobox{pgfPT}[]%
\\ [10pt]\makebox[\linewidth][c]{\scalebox{.6}{\pgfPT}}%
\\ [10pt]\pgfPTMmacrobox{pgfPT}[Cp precision=0]%
\\ [10pt]\makebox[\linewidth][c]{\scalebox{.6}{\pgfPT[Cp precision=0]}}%
\\ [10pt]\pgfPTMmacrobox{pgfPT}[Cp precision=1]%
\\ [10pt]\makebox[\linewidth][c]{\scalebox{.6}{\pgfPT[Cp precision=1]}}%
\vfill%
\newpage%\\ [10pt]
\pgfPTMmacrobox{pgfPT}[Cp precision=2]%
\\ [10pt]\makebox[\linewidth][c]{\scalebox{.6}{\pgfPT[Cp precision=2]}}%
\\ [10pt]\pgfPTMmacrobox{pgfPT}[Cp precision=3]%
\\ [10pt]\makebox[\linewidth][c]{\scalebox{.6}{\pgfPT[Cp precision=3]}}%
\\ [0pt]\pgfPTendoption%
% kT precision
\vfill\vfill%
\pgfPTMoption{4}{kT precision}{-1}{%
Sets the thermal condutivity precision, \ie, the decimal places displayed in their value, performing the respective rounding, without zero padding the value.
\\ [10pt]\tikz{\node[text width=\linewidth-.6666em,fill=orange!5!white,draw=orange,rounded corners=2pt] {\textbf{\red{NOTE}}:\\ Rounding is performed over density values witch actually have a maximum 5 decimal places. So giving this key a value of -1 (the value of the melting or boiling point as-is) or 5 has the same effect.
\\ \textit{Therefore the values provided to this key should be any integer between -1 and 4. Any other integer provided will be processed as -1.}};}
}
\\ [10pt]\pgfPTMbuildcellstyle{mykT}(5,3)[(1;1-2;Z),(1;3;radio),(2-3;1.5-3.5;CS),(4;1-3;name),(5;1-3;kT)]%
\pgfPTbuildcellstyle{mykT}(5,3)[(1;1-2;Z),(1;3;radio),(2-3;1.5-3.5;CS),(4;1-3;name),(5;1-3;kT)]%
\\ [-4pt]\pgfPTMmacrobox[l]{pgfPTstyle}[Z list={1,...,36},cell style=mykT,show title=false]%
\pgfPTstyle[Z list={1,...,36},cell style=mykT,show title=false]%
\\ [-4pt]\pgfPTMmacrobox{pgfPT}[]%
\\ [10pt]\makebox[\linewidth][c]{\scalebox{.6}{\pgfPT}}%
\vfill%
\newpage%\\ [10pt]
\pgfPTMmacrobox{pgfPT}[kT precision=0]%
\\ [10pt]\makebox[\linewidth][c]{\scalebox{.6}{\pgfPT[kT precision=0]}}%
\\ [10pt]\pgfPTMmacrobox{pgfPT}[kT precision=1]%
\\ [10pt]\makebox[\linewidth][c]{\scalebox{.6}{\pgfPT[kT precision=1]}}%
\\ [10pt]\pgfPTMmacrobox{pgfPT}[kT precision=2]%
\\ [10pt]\makebox[\linewidth][c]{\scalebox{.6}{\pgfPT[kT precision=2]}}%
\\ [10pt]\pgfPTMmacrobox{pgfPT}[kT precision=3]%
\\ [10pt]\makebox[\linewidth][c]{\scalebox{.6}{\pgfPT[kT precision=3]}}%
\\ [10pt]\pgfPTMmacrobox{pgfPT}[kT precision=4]%
\\ [0pt]\vbox to 0pt{\makebox[\linewidth][c]{\scalebox{.6}{\pgfPT[kT precision=4]}}}%
\newpage%\\ [10pt]
\pgfPTMmacrobox{pgfPT}[kT precision=5]%
\\ [10pt]\makebox[\linewidth][c]{\scalebox{.6}{\pgfPT[kT precision=5]}}%
\\ [0pt]\pgfPTendoption%
\\ [10pt]\pgfPTMmacrobox[l]{pgfPTresetstyle}[]%
\pgfPTresetstyle%
\endinput
