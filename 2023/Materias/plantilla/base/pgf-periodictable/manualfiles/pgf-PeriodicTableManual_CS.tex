\vfill%
\subsubsection{\texorpdfstring{\ding{224} The chemical symbol}{The chemical symbol}}
%%%%%%%%%%%%%%%%%%%%%%%%%%%%%%%%%%%%%%%%%%%%%%%%%%%%%%%%%%%%
% CS solid
\pgfPTMoption{4}{CS solid}{\pgfPTMcolorDemo{solido}{RGB: 255,166,51}}{%
Sets the color of the chemical symbol for elements that are in the solid state at normal temperature and pressure (NTP).
}
\\ [5pt]\pgfPTMmacrobox{pgfPT}[Z list={1,...,54},CS solid=red]%
\\ [10pt]\makebox[\linewidth][c]{\scalebox{.6}{\pgfPT[Z list={1,...,54},CS solid=red]}}%
\\ [5pt]\pgfPTendoption%
% CS liquid
\pgfPTMoption{4}{CS liquid}{\pgfPTMcolorDemo{liquido}{RGB: 0,204,204}}{%
Sets the color of the chemical symbol for elements that are in a liquid state at normal temperature and pressure (NTP).
}
\\ [5pt]\pgfPTMmacrobox{pgfPT}[Z list={1,...,54},CS liquid=red]%
\\ [10pt]\makebox[\linewidth][c]{\scalebox{.6}{\pgfPT[Z list={1,...,54},CS liquid=red]}}%
\\ [5pt]\pgfPTendoption%
\newpage\ \\ [-32pt]%
% CS gas
\pgfPTMoption{4}{CS gas}{\pgfPTMcolorDemo{gasoso}{RGB: 0,102,153}}{%
Sets the color of the chemical symbol for elements that are in a gaseous state at normal temperature and pressure (NTP).
}
\\ [5pt]\pgfPTMmacrobox{pgfPT}[Z list={1,...,54},CS gas=red]%
\\ [10pt]\makebox[\linewidth][c]{\scalebox{.6}{\pgfPT[Z list={1,...,54},CS gas=red]}}%
\\ [5pt]\pgfPTendoption%
% CS synt
\pgfPTMoption{4}{CS synt}{\pgfPTMcolorDemo{sintetico}{RGB: 236,236,236}}{%
Sets the color of the chemical symbol for elements that are synthetic.
}
\\ [5pt]\pgfPTMmacrobox{pgfPT}[CS synt=red]%
\\ [10pt]\makebox[\linewidth][c]{\scalebox{.6}{\pgfPT[CS synt=red]}}%
\\ [5pt]\pgfPTendoption%
% CS all (style)%
\vfill
\pgfPTMstyle{4}{CS all}{black}{%
Style to set a common color to the chemical symbols, equivalent to \red{CS solid=<color>,\\ CS liquid=<color>,CS gas=<color>,CS synt=<color>}.
}%
\vfill
\newpage%\\ [5pt]
\pgfPTMmacrobox{pgfPT}[CS all=red]%
\\ [5pt]\makebox[\linewidth][c]{\scalebox{.6}{\pgfPT[CS all=red]}}%
\\ [0pt]\pgfPTendstyle%
% CS font
\vfill%
\pgfPTMoption{4}{CS font}{\string\small\string\bfseries}{%
Sets the font for the chemical symbol.
}
\\ [5pt]\pgfPTMmacrobox{pgfPT}[Z list={1,...,36},CS font=\string\small\string\fontfamily{fmm}\string\selectfont]%
\\ [5pt]\makebox[\linewidth][c]{\scalebox{.6}{\pgfPT[Z list={1,...,36},CS font=\small\fontfamily{fmm}\selectfont]}}%
\\ [0pt]\pgfPTendoption%
% CS render mode
\vfill%
\pgfPTMoption{4}{CS render mode}{fill and outline}{%
Sets the chemical symbol render mode. Available modes are \red{fill}, \red{outline} or \red{fill and outline}.
}
\\ [5pt]\pgfPTMmacrobox{pgfPT}[Z list={1,...,36}]%
\\ [5pt]\makebox[\linewidth][c]{\scalebox{.6}{\pgfPT[Z list={1,...,36}]}}%
\newpage%\\ [5pt]
\pgfPTMmacrobox{pgfPT}[Z list={1,...,36},CS render mode=fill]%
\\ [5pt]\makebox[\linewidth][c]{\scalebox{.6}{\pgfPT[Z list={1,...,36},CS render mode=fill]}}%
\\ [5pt]\pgfPTMmacrobox{pgfPT}[Z list={1,...,36},CS render mode=outline]%
\\ [5pt]\makebox[\linewidth][c]{\scalebox{.6}{\pgfPT[Z list={1,...,36},CS render mode=outline]}}%
\\ [-5pt]\pgfPTendoption%
% CS outline color
\pgfPTMoption{4}{CS outline color}{black}{%
Sets the outline color for the chemical symbol.
}
\\ [5pt]\pgfPTMmacrobox{pgfPT}[Z list={1,...,36},CS outline color=red]%
\\ [5pt]\makebox[\linewidth][c]{\scalebox{.6}{\pgfPT[Z list={1,...,36},CS outline color=red]}}%
\\ [-5pt]\pgfPTendoption%
% CS outline width
\pgfPTMoption{4}{CS outline width}{0.05}{%
Sets the outline width of the chemical symbol. It is any positive numerical value \textbf{without dimensions} (1.0 is roughly 1.0pt).
}
\\ [5pt]\pgfPTMmacrobox{pgfPT}[Z list={1,...,36},CS outline width=.2]%
\\ [5pt]\makebox[\linewidth][c]{\scalebox{.6}{\pgfPT[Z list={1,...,36},CS outline width=.2]}}%
\\ [-5pt]\pgfPTendoption%
% CS={r=??,olc=??,olw=??,f=??,s=??,l=??,g=??,sy=??} (pseudo style)%
%       CS/.default={r=fill and outline,c=black,w=.05,f=\small\bfseries,s=solido,l=liquido,g=gasoso,sy=sintetico}%
\newpage\ \\ [-25pt]%
\pgfPTMstyle{4}{CS}{\vbox to 7pt{\hbox{\{r=fill and outline,c=black,w=.05,f=\string\small\string\bfseries,s=solido,l=liquido,}\vskip0pt\hbox{\ g=gasoso,sy=sintetico\}}}}{%
\ \\ [6pt]\textit{Pseudo style} to set the keys: CS \textbf{r}ender mode, CS \textbf{o}ut\textbf{l}ine \textbf{c}olor, CS \textbf{o}ut\textbf{l}ine \textbf{w}idth, CS font, CS \textbf{s}olid, CS \textbf{l}iquid, CS \textbf{g}as and/or CS \textbf{sy}nt and/or the style CS \textbf{all}.
None of the \textit{keys} -- r, olc, olw , f, s, l, g, sy and all -- are mandatory.
\\ [5pt]\makebox[\linewidth][c]{\use{\tikz{\node[text width=10.8cm] {CS=\{r=<fill|outline|fill and outline>,olc=<color>,olw=<positive numerical value>\\ %
\textcolor{cyan!10!white}{CS=\{}f=<font commands>,s=<color>,l=<color>,g=<color>,sy=<color>,all=<color>\}};}}}%
}%
\\ [5pt]\pgfPTMmacrobox{pgfPT}[Z list={1,...,36},CS={r=outline,olc=red,olw=.4},show legend pins=false]%
\\ [5pt]\makebox[\linewidth][c]{\scalebox{.6}{\pgfPT[Z list={1,...,36},CS={r=outline,olc=red,olw=.4},show legend pins=false]}}%
\\ [0pt]\pgfPTendstyle%
\endinput
