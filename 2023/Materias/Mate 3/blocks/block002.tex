\begin{infocard}{Factiorización de una ecuación cuadrática}
    Factorizar una ecuación cuadrática significa escribirla como una multiplicación (expresiones algebraicas separadas por paréntesis), y sirve para encontrar las soluciones a una ecuación cuadrática de forma rápida:
    \begin{enumerate}
        \item Verifica si existe un factor en común para los coeficientes $a$, $b$ y $c$ y divide la ecuación entre el factor común (obtendras una ecuación cuadrática de la forma $x^2+bx+c=0$).
        \item Escribe dos paréntesis, de esta forma:
              \[ x^2+bx+c = \left(x \dashedbox{$-x_1$}\right)\cdot\left(x \dashedbox{$-x_2$}\right)                  \]
        \item Coloca en los espacios dos números que al sumarlos tengan el valor de $b$ y al multiplicarlos el valor de $c$.
              \[
                  \begin{array}{rl}
                      b & = x_1 + x_2     \\
                      c & = x_1 \cdot x_2
                  \end{array}
              \]
        \item Verifica el signo de los coeficientes $a$ y $b$.
    \end{enumerate}
\end{infocard}
%----------------------------------------------------------------------------------------
