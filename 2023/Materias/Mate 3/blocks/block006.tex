\begin{infocard}{Funciones trigonométricas}
    Para un triángulo rectangulo, existen 3 funciones trigonométricas:



    \newcommand{\pythagwidth}{2.2cm}
    \newcommand{\pythagheight}{1.8cm}
    \colorlet{anglecolor}{green!50!black}

    \centering
    \begin{tikzpicture}

        \coordinate  (A) at (0, 0);
        \coordinate  (B) at (0, \pythagheight);
        \coordinate (C) at (-\pythagwidth, 0);

        \draw [thick] (A) -- node [below] {CA} (C) -- node [above left] {H} (B) -- node [above right] {CO} (A);
        \draw[fill=lightgray, thick] (C) -- ++(0:0.8cm) arc (0:90-atan2(\pythagwidth,\pythagheight):0.8cm) node at ($(20:0.5cm)+(C)$) {$\theta$} -- cycle;

        \newcommand{\ranglesize}{0.3cm}
        \draw (A) -- ++ (0, \ranglesize) -- ++ (-\ranglesize, 0) -- ++ (0, -\ranglesize);


    \end{tikzpicture}
    \[
        \sin(\theta)  = \frac{\text{CO}}{\text{H}} \quad
        \cos(\theta)  = \frac{\text{CA}}{\text{H}} \quad
        \tan(\theta)  = \frac{\text{CO}}{\text{CA}}
    \]

\end{infocard}
