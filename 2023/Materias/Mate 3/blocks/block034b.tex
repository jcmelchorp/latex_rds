\begin{infocard}{Teorema de Pitágoras}
    % \begin{minipage}{0.65\textwidth}
    El \textbf{teorema de Pitágoras} es una relación en geometría euclidiana entre los tres lados de un triángulo rectángulo.
    Afirma que el área del cuadrado cuyo lado es la hipotenusa $c$ (el lado opuesto al ángulo recto) es igual a la suma de las áreas de los cuadrados cuyos lados son los catetos $a$ y $b$ (los otros dos lados que no son la hipotenusa), como se muestra a continuación:
    \[a^2+b^2=c^2\]
    % \end{minipage}\hfill
    % \begin{minipage}{0.3\textwidth}
    \begin{figure}[H]
        \centering
        \includegraphics[width=0.55\linewidth]{../images/pythagorean_right_angle}
        \caption{}
        \label{fig:pythagorean_right_angle}
    \end{figure}
    % \end{minipage}
    % \newcommand{\pythagwidth}{3cm}
    % \newcommand{\pythagheight}{2cm}

    % \begin{tikzpicture}

    %     \coordinate [label={below right:$A$}] (A) at (0, 0);
    %     \coordinate [label={above right:$B$}] (B) at (0, \pythagheight);
    %     \coordinate [label={below left:$C$}] (C) at (-\pythagwidth, 0);

    %     \coordinate (D1) at (-\pythagheight, \pythagheight + \pythagwidth);
    %     \coordinate (D2) at (-\pythagheight - \pythagwidth, \pythagwidth);

    %     \draw [very thick] (A) -- (C) -- (B) -- (A);

    %     \newcommand{\ranglesize}{0.3cm}
    %     \draw (A) -- ++ (0, \ranglesize) -- ++ (-\ranglesize, 0) -- ++ (0, -\ranglesize);

    %     \draw [dashed] (A) -- node [below] {$b$} ++ (-\pythagwidth, 0)
    %     -- node [right] {$b$} ++ (0, -\pythagwidth)
    %     -- node [above] {$b$} ++ (\pythagwidth, 0)
    %     -- node [left] {$b$} ++ (0, \pythagwidth);

    %     \draw [dashed] (A) -- node [right] {$c$} ++ (0, \pythagheight)
    %     -- node [below] {$c$} ++ (\pythagheight, 0)
    %     -- node [left] {$c$} ++ (0, -\pythagheight)
    %     -- node [above] {$c$} ++ (-\pythagheight, 0);

    %     \draw [dashed] (C) -- node [above left] {$a$} (B)
    %     -- node [below left] {$a$} (D1)
    %     -- node [below right] {$a$} (D2)
    %     -- node [above right] {$a$} (C);

    % \end{tikzpicture}
\end{infocard}