%----------------------------------------------------------------------------------------
%	Síntesis
%----------------------------------------------------------------------------------------
\begin{tcolorbox}[
        colback=colorrds!5!white,
        colframe=colorrds!35!white,
        coltitle=black,
        fonttitle=\bfseries,
        center title,
        title=Ecuación cuadrática
    ]
    Una \textbf{ecuación cuadrática} completa en una variable es una ecuación del tipo
    \begin{equation}
        ax^2 + bx + c = 0
    \end{equation}
    donde $a$, $b$ y $c$ son enteros, decimales o fraccionarios y $a$ no es igual a 0. Como el
    mayor exponente de la variable es 2 también se
    le conoce como \textbf{ecuación de segundo grado}.
\end{tcolorbox}

\begin{tcolorbox}[
        colback=colorrds!5!white,
        colframe=colorrds!35!white,
        coltitle=black,
        fonttitle=\bfseries,
        center title,
        title=Formas de una ecuación cuadrática
    ]
    \begin{tabular}{lr}
        $ ax^2+bx+c = 0$    & Forma \textbf{general o estándar} \\
        $a(x-x_1)(x-x_2)=0$ & Forma \textbf{factorizada}        \\
        $a(x-h)^2+k=0$      & Forma \textbf{canónica}
    \end{tabular}
\end{tcolorbox}

\begin{tcolorbox}[
        colback=colorrds!5!white,
        colframe=colorrds!35!white,
        coltitle=black,
        fonttitle=\bfseries,
        center title,
        title=Discriminante $\delta$]
    El discriminante $\delta$ es un parámetro que indica cuantas soluciones tiene una ecuación cuadrática:
    \[\text{Número de soluciones}=
        \begin{cases}
            2 & \text{si } \delta > 0 \\
            1 & \text{si } \delta = 0 \\
            0 & \text{si } \delta < 0
        \end{cases}
    \]
\end{tcolorbox}

\begin{tcolorbox}[
        colback=colorrds!5!white,
        colframe=colorrds!35!white,
        coltitle=black,
        fonttitle=\bfseries,
        center title,
        title=Fórmula para las soluciones de una ecuación cuadrática]
    \[
        x = \frac{-b\pm \sqrt{\delta}}{2a} \quad \text{ donde, }\delta=b^2-4ac
    \]
    que se pueden escribir en una sola expresión:
    \[x= \dfrac{-b\pm\sqrt{b^2-4ac}}{2a}\]
\end{tcolorbox}

\begin{tcolorbox}[
        enhanced,
        colback=orange!5!white,
        colframe=orange!35!white,
        coltitle=black,
        fonttitle=\bfseries,
        center title,
        title=Vocabulario]
    \begin{itemize}[leftmargin=*]
        \item[(s.)] \textbf{signo} $\rightarrow$ característica $+$ o $-$ de una cantidad.
        \item[(s.)] \textbf{ecuación} $\rightarrow$ expresiones algebráicas con un signo '$=$'.
        \item[(s.)] \textbf{factor} $\rightarrow$ aquello que se multiplica.
        \item[(v.)] \textbf{factorizar} $\rightarrow$ convertir una expresión algebráica en un producto.
        \item[(s.)] \textbf{coeficiente} $\rightarrow$ número que multiplica a una literal; ejemplo: $a$, $b$, $c$ son coeficientes de $ax^2+bx+c$
        \item[(s.)] \textbf{ecuación cuadrática} $\rightarrow$ $0 = ax^2+bx+c$
        \item[(s.)] \textbf{raíces} $\rightarrow$ soluciones de una ecuación cuadrática.
        \item[(s.)] \textbf{formula} $\rightarrow$ ecuación
    \end{itemize}
\end{tcolorbox}
