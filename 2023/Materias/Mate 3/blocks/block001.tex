\begin{infocard}{Ecuación cuadrática}
    Una \textbf{ecuación cuadrática} completa en una variable es una ecuación del tipo
    \begin{equation}
        ax^2 + bx + c = 0
    \end{equation}
    donde $a$, $b$ y $c$ son enteros, decimales o fraccionarios y $a$ no es igual a 0. Como el
    mayor exponente de la variable es 2 también se
    le conoce como \textbf{ecuación de segundo grado}.
\end{infocard}


\begin{infocard}{Formas de una ecuación cuadrática}
    \begin{tabular}{lr}
        $ ax^2+bx+c = 0$    & Forma \textbf{general o estándar} \\
        $a(x-x_1)(x-x_2)=0$ & Forma \textbf{factorizada}        \\
        $a(x-h)^2+k=0$      & Forma \textbf{canónica}
    \end{tabular}
\end{infocard}

\begin{infocard}{Discriminante $\delta$}
    El discriminante $\delta$ es un parámetro que indica cuantas soluciones tiene una ecuación cuadrática:
    \[\text{Número de soluciones}=
        \begin{cases}
            2 & \text{si } \delta > 0 \\
            1 & \text{si } \delta = 0 \\
            0 & \text{si } \delta < 0
        \end{cases}
    \]
\end{infocard}

\begin{infocard}{Fórmula para las soluciones de una ecuación cuadrática}
    \[
        x = \frac{-b\pm \sqrt{\delta}}{2a} \quad \text{ donde, }\delta=b^2-4ac
    \]
    que se pueden escribir en una sola expresión:
    \[x= \dfrac{-b\pm\sqrt{b^2-4ac}}{2a}\]
\end{infocard}