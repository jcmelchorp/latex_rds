\documentclass[12pt,addpoints,answers]{repaso}
\grado{3}
\nivel{Secundaria}
\cicloescolar{2023-2024}
\materia{Matemáticas}
\unidad{1}
\title{Repaso para el examen de la Unidad}
\aprendizajes{
      \item 
}
\author{Melchor Pinto, J.C.}
\begin{document}
\INFO%
\begin{questions}
      \section*{\ifprintanswers{Cálculos numéricos}\else{}\fi}

      \question[10] Realiza las siguientes operaciones de \textit{cálculo numérico}:
      \begin{parts}
            \begin{multicols}{2}
                  \subsection*{\ifprintanswers{Suma de números}\else{}\fi}
                  \part $\dfrac{5}{6}+\dfrac{3}{8}=$ \fillin[$1\dfrac{5}{24}$][0in]
                  \subsection*{\ifprintanswers{Multiplicación de números}\else{}\fi}
                  \part $9.27\times 5.4=$ \fillin[$50.058$][0in]
                  \subsection*{\ifprintanswers{Resta de números}\else{}\fi}
                  \part $\dfrac{1}{2}-\dfrac{2}{5}=$ \fillin[$\dfrac{1}{10}$][0in]
                  \subsection*{\ifprintanswers{División de números}\else{}\fi}
                  \part $622.21\divisionsymbol 115=$ \fillin[$5.41$][0in]
            \end{multicols}
            \subsection*{\ifprintanswers{Resolución de problemas}\else{}\fi}
            \part Si un dólar equivale a 19 pesos. ¿Cuántos dólares serán 1634 pesos? \fillin[$1634\divisionsymbol 19 =$ 86 dólares][0in]
      \end{parts}

      \newpage
      \section*{\ifprintanswers{Factorización}\else{}\fi}
      \subsection*{\ifprintanswers{Término común}\else{}\fi}
      \subsection*{\ifprintanswers{Diferencia de cuadrados}\else{}\fi}
      \subsection*{\ifprintanswers{Trinomio cuadrado perfecto}\else{}\fi}
      \subsection*{\ifprintanswers{Trinomios de la forma x²+bx+c}\else{}\fi}
      \subsection*{\ifprintanswers{Trinomios de la forma ax²+bx+c}\else{}\fi}
      \section*{\ifprintanswers{Leyes de los exponentes}\else{}\fi}
      \subsection*{\ifprintanswers{Suma de exponentes}\else{}\fi}
      \subsection*{\ifprintanswers{Resta de exponentes}\else{}\fi}
      \subsection*{\ifprintanswers{Multiplicación de exponentes}\else{}\fi}
      \subsection*{\ifprintanswers{División de exponentes}\else{}\fi}
      \subsection*{\ifprintanswers{Exponentes negativos}\else{}\fi}
      \section*{\ifprintanswers{Números negativos}\else{}\fi}
      \subsection*{\ifprintanswers{Ubicación en la recta numérica}\else{}\fi}
      \subsection*{\ifprintanswers{Comparación de negativos}\else{}\fi}
      \subsection*{\ifprintanswers{Suma y resta con negativos}\else{}\fi}
      \subsection*{\ifprintanswers{Multiplicación y división con negativos}\else{}\fi}
      \subsection*{\ifprintanswers{Jerarquía de operaciones}\else{}\fi}
      \section*{\ifprintanswers{Sucesiones aritméticas}\else{}\fi}
      \subsection*{\ifprintanswers{Completando la sucesión}\else{}\fi}
      \subsection*{\ifprintanswers{Diferencia de una sucesión}\else{}\fi}
      \subsection*{\ifprintanswers{Término general}\else{}\fi}
      \subsection*{\ifprintanswers{Término enésimo}\else{}\fi}
      \subsection*{\ifprintanswers{Suma de una sucesión aritmética}\else{}\fi}
\end{questions}
\end{document}