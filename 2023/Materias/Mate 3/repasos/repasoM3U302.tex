\documentclass[12pt,addpoints]{repaso}
\grado{3}
\nivel{Secundaria}
\cicloescolar{2023-2024}
\materia{Matemáticas}
\unidad{2}
\title{Practica la Unidad}
\aprendizajes{
    \item Comprende las series y sucesiones cuadraticas y geométricas y sus respectivas formulaciones algebraicas.
    \item Reconoce y aplica los principales productos notables y su interpretación geométrica.
    \item Resuelve problemas mediante la formulación y la solución algebraica de ecuaciones cuadráticas. 
    \item Formula, justifica y usa el teorema de Pitágoras para resolver problemas.
    \item Usa las funciones trigonométricas para resolver problemas geométricos con aplicación en la vida diaria.
    }
\author{Melchor Pinto, J.C.}
\begin{document}
\INFO%
\section*   {Sucesiones cuadráticas y geométricas}
\subsection*{Sucesión cuadrática}

\begin{questions}
    \subsection*{Completando la sucesión cuadrática}
    \questionboxed[6]{
        \begin{multicols}{3}
            \begin{parts}
                \part
                \begin{solutionbox}{2cm}
                \end{solutionbox}
                \part
                \begin{solutionbox}{2cm}
                \end{solutionbox}
                \part
                \begin{solutionbox}{2cm}
                \end{solutionbox}
            \end{parts}
        \end{multicols}
    }

    \subsection*{Término general}
    \subsection*{Sucesión geométrica}
    \subsection*{Razón de una sucesión geométrica}
    \section*{   Productos notables}
    \subsection*{Binomios conjugados}
    \subsection*{Binomios con término común}
    \subsection*{Binomio al cuadrado}
    \subsection*{Binomios de la forma (mx+a)(nx+b)}
    \subsection*{Binomio al cubo}
    \section*{   Ecuaciones cuadráticas}
    \subsection*{Clasificación de ecuaciones cuadráticas}
    \subsection*{Discriminante}
    \subsection*{Ecuaciones cuadráticas incompletas}
    \subsection*{Ecuaciones cuadráticas completas 1}
    \subsection*{Ecuaciones cuadráticas completas 2}
    \section*{   Teorema de Pitágoras}
    \subsection*{Identificación de lados}
    \subsection*{Hallando la hipotenusa}
    \subsection*{Hallando el cateto}
    \subsection*{Áreas y perímetros}
    \subsection*{Resolución de problemas}
    \section*{   Trigonometría}
    \subsection*{Identificando lados}
    \subsection*{Identificando funciones}
    \subsection*{Encontrando lados}
    \subsection*{Encontrando ángulos}
    \subsection*{Resolución de problemas}
    \questionboxed[10]{}
\end{questions}
\end{document}