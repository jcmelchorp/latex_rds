\documentclass[12pt,addpoints,answers]{guia}
\grado{3$^\circ$ de Secundaria}
\cicloescolar{2022-2023}
\materia{Matemáticas 3}
\guia{39}
\unidad{3}
\title{Utiliza áreas de cuadrados para visualizar el teorema de Pitágoras}
%\unidad{3}
\title{El título de la guía}
\aprendizajes{
    \begin{itemize}[leftmargin=*,label=\small\color{colorrds}\faIcon{user-graduate}]
        \item Formula, justifica y usa el teorema de Pitágoras.
    \end{itemize}
}
\author{J. C. Melchor Pinto}
\begin{document}
\pagestyle{headandfoot}

\INFO
\printanswers
\vspace{-0.5cm}
\begin{multicols}{2}
    \include*{../blocks/block034a}
    \include*{../blocks/block034c}
    \columnbreak
    \include*{../blocks/block034b}
\end{multicols}
\begin{questions}
    \questionboxed[10]{\include*{../questions/question043a}}
    \questionboxed[10]{\include*{../questions/question043b}}
    \questionboxed[10]{\include*{../questions/question043c}}
    \questionboxed[10]{\include*{../questions/question043d}}
    \questionboxed[10]{\include*{../questions/question043e}}
    \questionboxed[10]{\include*{../questions/question043f}}
    \questionboxed[10]{\include*{../questions/question043g}}
    \questionboxed[10]{\include*{../questions/question043h}}
    \questionboxed[10]{\include*{../questions/question043i}}
\end{questions}

\end{document}