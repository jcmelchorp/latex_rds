\documentclass[12pt,addpoints,answers]{guia}
\grado{3$^\circ$ de Secundaria}
\cicloescolar{2022-2023}
\materia{Matemáticas 3}
\guia{34}
\unidad{3}
\title{Utiliza el teorema de Pitágoras para obtener las \\longitudes de lados de un trángulo isóceles}
\aprendizajes{
    \begin{itemize}[leftmargin=*,label=\small\color{colorrds}\faIcon{user-graduate}]
        \item Formula, justifica y usa el teorema de Pitágoras.
    \end{itemize}
}
\author{JC Melchor Pinto}
\begin{document}
\pagestyle{headandfoot}
\INFO%
\begin{multicols}{2}
    \include*{../blocks/block034a}
    \include*{../blocks/block034c}
    \include*{../blocks/block034d}
    \columnbreak
    \include*{../blocks/block034b}%
\end{multicols}%
\begin{questions}
    \questionboxed[10]{\include*{../questions/question038a}}
    \questionboxed[10]{\include*{../questions/question038b}}
    \questionboxed[10]{\include*{../questions/question038c}}
    \questionboxed[10]{\include*{../questions/question038d}}
    \questionboxed[10]{\include*{../questions/question038e}}
    \questionboxed[10]{\include*{../questions/question038f}}
    \questionboxed[10]{\include*{../questions/question038g}}
    \questionboxed[10]{\include*{../questions/question038h}}
    \questionboxed[10]{\include*{../questions/question038i}}
    \questionboxed[10]{\include*{../questions/question038j}}
    \questionboxed[10]{\include*{../questions/question038k}}
    \questionboxed[10]{\include*{../questions/question038l}}
    \questionboxed[10]{\include*{../questions/question038m}}
    \questionboxed[10]{\include*{../questions/question038n}}
\end{questions}
\end{document}