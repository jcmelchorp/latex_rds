\documentclass[12pt]{guia}
\grade{3$^\circ$ de Secundaria}
\cycle{2022-2023}
\subject{Matemáticas 3}
\guide{35}
\title{Longitudes de lados de un triángulo rectángulo}
%\title{El título de la guía}
\aprendizajes{
    \begin{itemize}[leftmargin=*,label=\small\color{colorrds}\faIcon{user-graduate}]
        \item Formula, justifica y usa el teorema de Pitágoras.
    \end{itemize}
}
\requisitos{
    \begin{itemize}
        \item Requisito 1
        \item Requisito 2
    \end{itemize}
}
\author{J. C. Melchor Pinto}

\begin{document}
\pagestyle{headandfoot}
\addpoints
\INFO
%\printanswers
\begin{multicols}{2}
    \include*{../blocks/block034a}
    \include*{../blocks/block034c}
    % \include*{../blocks/block034d}
    \columnbreak
    \include*{../blocks/block034b}
\end{multicols}
\begin{questions}
    %\include*{../questions/question000}
    \questionboxed[10]{\include*{../questions/question037a}}
    \questionboxed[10]{\include*{../questions/question037b}}
    \questionboxed[10]{\include*{../questions/question037c}}
    \questionboxed[10]{\include*{../questions/question037d}}
    \questionboxed[10]{\include*{../questions/question037e}}
    \questionboxed[10]{\include*{../questions/question037f}}
    \questionboxed[10]{\include*{../questions/question037g}}
    \questionboxed[10]{\include*{../questions/question037h}}
    \questionboxed[10]{\include*{../questions/question037i}}
    \questionboxed[10]{\include*{../questions/question037j}}
\end{questions}

\end{document}