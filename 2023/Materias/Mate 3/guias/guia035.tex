\documentclass[12pt,addpoints,answers]{guia}
\grado{3$^\circ$ de Secundaria}
\cicloescolar{2022-2023}
\materia{Matemáticas 3}
\guia{35}
\unidad{3}
\title{Longitudes de lados de un triángulo rectángulo}
\aprendizajes{\item Formula, justifica y usa el teorema de Pitágoras.}
\author{JC Melchor Pinto}
\begin{document}
\INFO%
\begin{multicols}{2}
    \include*{../blocks/block034b}
    \include*{../blocks/block034a}
    \include*{../blocks/block034c}
\end{multicols}
\ejemplosboxed[\include*{../questions/question037a}]
\begin{questions}
    \questionboxed[10]{\include*{../questions/question037b}}
    \questionboxed[10]{\include*{../questions/question037c}}
    \questionboxed[10]{\include*{../questions/question037d}}
    \questionboxed[10]{\include*{../questions/question037e}}
    \questionboxed[10]{\include*{../questions/question037f}}
    \questionboxed[10]{\include*{../questions/question037g}}
    \questionboxed[10]{\include*{../questions/question037h}}
    \questionboxed[15]{\include*{../questions/question037i}}
    \questionboxed[15]{\include*{../questions/question037j}}
\end{questions}
\end{document}