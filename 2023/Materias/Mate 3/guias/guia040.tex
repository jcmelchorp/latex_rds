\documentclass[12pt]{guia}
\grade{3$^\circ$ de Secundaria}
\cycle{2022-2023}
\subject{Matemáticas 3}
\guide{40}
\title{Usa el teorema de Pitágoras para calcular el área de figuras compuestas}
%\title{El título de la guía}
\aprendizajes{
    \begin{itemize}[leftmargin=*,label=\small\color{colorrds}\faIcon{user-graduate}]
        \item Formula, justifica y usa el teorema de Pitágoras.
    \end{itemize}
}
\requisitos{
    \begin{itemize}
        \item Requisito 1
        \item Requisito 2
    \end{itemize}
}
\author{J. C. Melchor Pinto}

\begin{document}
\pagestyle{headandfoot}
\addpoints
\INFO
\printanswers
\include*{../blocks/block000}
\newpage
\begin{questions}
    %\include*{../questions/question000}
    \questionboxed[10] \include*{../questions/question042a}
    \questionboxed[10] \include*{../questions/question042b}
    \questionboxed[10] \include*{../questions/question042c}
    \questionboxed[10] \include*{../questions/question042d}
    \questionboxed[10] \include*{../questions/question042e}
\end{questions}

\end{document}