\documentclass[12pt,addpoints,answers]{guia}
\grado{3$^\circ$ de Secundaria}
\cicloescolar{2022-2023}
\materia{Matemáticas 3}
\guia{45}
\unidad{3}
\title{Problemas verbales de trigonometría de triángulos rectángulos}
%\unidad{3}
\title{El título de la guía}
\aprendizajes{\item Resuelve problemas utilizando las razones trigonométricas seno, coseno y tangente.
    }
\author{JC Melchor Pinto}
\begin{document}
\pagestyle{headandfoot}

\INFO
\printanswers
\vspace{-0.9cm}
\begin{multicols}{2}
    %   \include*{../blocks/block030a}
    %    \include*{../blocks/block030b}
    \columnbreak
    %    \include*{../blocks/block030e}
\end{multicols}
%\include*{../blocks/block030c}
% \include*{../blocks/block030f}
% \include*{../blocks/block030d}
\begin{questions}
    \questionboxed[10]{\include*{../questions/question048a}}
    \questionboxed[10]{\include*{../questions/question048b}}
    \questionboxed[10]{\include*{../questions/question048c}}
    \questionboxed[10]{\include*{../questions/question048d}}
    \questionboxed[10]{\include*{../questions/question048e}}
    \questionboxed[10]{\include*{../questions/question048f}}
    \questionboxed[10]{\include*{../questions/question048g}}
\end{questions}

\end{document}