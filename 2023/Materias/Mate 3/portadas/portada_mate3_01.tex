\documentclass[letterpaper,12pt]{article}
\usepackage[utf8]{inputenc}
\usepackage[spanish]{babel}
\usepackage{tikz}
\usetikzlibrary{calc,intersections,through,backgrounds,positioning}
\usepackage[lua]{tkz-euclide}
\usepackage{adjustbox}
\usepackage[
    letterpaper,
includehead,
left=15mm,
right=15mm,
top=5mm,
bottom=25mm,
headheight=10mm,% Set \headheight to 10mm
]{geometry}
\usepackage{wrapfig}
\definecolor{colorrds}{HTML}{0060A0}
\definecolor{subjectColor}{HTML}{98780F}
\definecolor{gold}{RGB}{239,173,91}
% \definecolor{subjectColor}{RGB}{192,192,192}
\newcommand{\thesubject}{}
\newcommand{\subject}[1]{\renewcommand{\thesubject}{#1}}
\newcommand{\thegrade}{}
\newcommand{\grade}[1]{\renewcommand{\thegrade}{#1}}
\makeatletter
\renewcommand{\maketitle}{
    \thispagestyle{empty}
    \begin{center}
        % \vspace{4cm}}
        % \begin{figure*}[!t]
        %     \centering
        % \rotatebox{270}{
        %     \begin{tikzpicture}[scale=0.5]
        %         \tkzDefPoint(0,0){O}
        %         \tkzDefPoint(1,0){A}
        %         \foreach \ang in {10,20,...,360}{%
        %                 \pgfmathsetmacro\k{\ang/360*100}
        %                 \tkzDefPoint(\ang:2){M}
        %                 \tkzDrawCircle[color=subjectColor!\k!white](M,A)
        %         };
        %     \end{tikzpicture}
        % }
        % \end{figure*}
        {\Huge \thesubject } \\
        \vspace{1cm} \normalsize
        {\bfseries \large {\@title}} \\
        para los alumnos de {\thegrade} \\
        {en el curso durante el ciclo escolar} \\
        {\bfseries 2022-2023} \\
        {\large \bfseries Unidad 3}\\
        \vspace{2cm}
        {\small POR} \\
        \Large {\@author} \\[0.5em]
        \normalsize Profesor de asignatura en \\
        \vspace{1cm}
        \includegraphics[width=4cm]{../../../resources/logo.png} \\
        %  \resizebox{0.25\textwidth}{!}{%
        %   }
        % \vspace{2.75cm}
        % \resizebox{\textwidth}{!}{%
        %     \begin{tikzpicture}[remember picture,thin]
        %         \tkzInit[xmin=-6,ymin=-4,xmax=6,ymax=6]
        %         \tkzClip
        %         \tkzSetUpLine[thin,color=colorrds]
        %         \tkzDefPoint(0,0){O}
        %         \tkzDefPoint(132:5){A}
        %         \tkzDefPoint(4,0){B}
        %         \foreach \ang in {5,10,...,360}{%
        %         \tkzDefPoint(\ang:4){M}
        %         \tkzDefLine[mediator](A,M)
        %         \tkzGetPoints{x}{y}
        %         \tkzDrawLine[add= 3 and 3](x,y)}
        %         \end{tikzpicture}
        % }
        % \begin{figure*}[!b]
        %     \centering
        %         \rotatebox{270}{
        %             \resizebox{\textwidth}{!}{%
        %             \begin{tikzpicture}[thin,scale=1]
        %                 \tkzDefPoint(0,0){O}
        %                 \tkzDefPoint(1,0){A}
        %                 \foreach \ang in
        %                     {10,20,...,360}{%
        %                         \pgfmathsetmacro\k{\ang/360*100}
        %                         \tkzDefPoint(\ang:2){M}
        %                         \tkzDrawCircle[color=subjectColor!\k!white](M,A)
        %                     };
        %             \end{tikzpicture}
        %         }
        %     }
        % \end{figure*}
    \end{center}
}
\makeatother
\title{Cuaderno de Trabajo}
\subject{Matemáticas}
\grade{3$^\circ$ de Secundaria}
\author{Melchor Pinto, JC}
\begin{document}
\pagestyle{empty}
\maketitle%% Crea la portada
\newpage
\begin{figure}[!b]
    \centering
    \begin{tikzpicture}[scale=1,ultra thin]
        \foreach \i in {0,1,2,...,11}{
                \begin{scope}[rotate=\i*30]
                    \draw[colorrds!10!,fill=colorrds!10!](0,0) to (8,0) to [bend right=45]  (30:8cm) --
                    cycle;
                    \foreach \j in {0,2,4,...,28} {
                            \draw[thin, colorrds] (0,0) -- (\j:\j*0.03cm+7cm);
                        }
                    \draw[line width=10mm, colorrds!10!] (7.5,0) to [bend right=45] (30:7.5cm);
                    \draw[fill=colorrds!10!, colorrds!10!, ultra thick] (0,0) -- (2.2,0) to [bend
                        right=60] (30:2.2cm) -- cycle;
                \end{scope}
            }
        \foreach \i in {0,1,2,...,11}{
                \begin{scope}[rotate=\i*30]
                    \foreach \k in {0.4,0.6,...,1.6} {
                            \draw[thick, colorrds!80!] (\k,0) to [bend right=60] (30:\k cm);
                        }
                    \draw[ultra thick, colorrds] (2,0) to [bend right=60] (30:2cm);
                    \draw[ultra thick, colorrds] (2.1,0) to [bend right=60] (30:2.1cm);
                    \draw[ultra thick, colorrds] (2.2,0) to [bend right=60] (30:2.2cm);
                    \draw[colorrds, decorate, decoration={shape
                                backgrounds,shape=circle,shape size=0.5mm,shape sep=1mm}] (1.7,0) to [bend
                        right=60] (30:1.7cm);
                    \draw[colorrds, decorate, decoration={shape
                                backgrounds,shape=circle,shape size=0.5mm,shape sep=1mm}] (1.8,0) to [bend
                        right=60] (30:1.8cm);
                    \draw[very thick, colorrds!80!] (0,0) --(2.2,0);
                    \draw[line width=3mm, colorrds] (7,0) to [bend right=45] (30:7cm);
                    \draw[colorrds, thick, decorate, decoration={shape
                                backgrounds,shape=circle,shape size=1.8mm,shape sep=3.05mm}] (7.4,0) to [bend
                        right=45] (30:7.4cm);
                    \draw[ultra thick, colorrds!80!, fill=colorrds!10!] (2.2,0) .. controls
                    (3,-0.2) and (5.8,3.1) .. (6,0);
                    \draw[ultra thick, colorrds!80!, fill=colorrds!10!] (2.2,0) .. controls (3,0.2)
                    and (5.8,-3.1) .. (6,0);
                    \draw[ultra thick, colorrds!80!] (4,0.92) .. controls (4.5,1) and
                    (4.7,0) .. (5,0);
                    \draw[ultra thick, colorrds!80!] (4,-0.92) .. controls (4.5,-1) and
                    (4.7,0) .. (5,0);
                    \draw[ultra thick, colorrds!80!] (2.4,0) .. controls (4.5,1) and
                    (4.5,0) .. (2.4,0);
                    \draw[ultra thick, colorrds!80!] (2.4,0) .. controls (4.5,-1) and
                    (4.5,0) .. (2.4,0);
                    \draw[very thick, colorrds] (3.8,0) circle (0.2cm);
                    \draw[very thick, colorrds] (4.15,0) circle (0.15cm);
                    \draw[very thick, colorrds] (4.4,0) circle (0.1cm);
                    \draw[very thick, colorrds] (4.55,0) circle (0.05cm);
                    \draw[very thick, colorrds!80!] (5.5,0) circle (0.5cm);
                    \filldraw[colorrds!80!, inner color=colorrds!10!,outer color=colorrds!80!] (5.5,0)
                    circle (0.5cm);
                    \draw[very thick, colorrds!80!] (5,0.65) circle (0.3cm);
                    \draw[very thick, colorrds!80!] (5,-0.65) circle (0.3cm);
                    \draw[very thick, colorrds!80!] (4.55,0.8) circle (0.15cm);
                    \draw[very thick, colorrds!80!] (4.55,-0.8) circle (0.15cm);
                    \draw[ultra thick, colorrds!80!, fill=colorrds!10!] (6,0) .. controls (6.4,1.2)
                    and (6.9,0) .. (7,0);
                    \draw[ultra thick, colorrds!80!, fill=colorrds!10!] (6,0) .. controls
                    (6.4,-1.2) and (6.9,0) .. (7,0);
                    \foreach \j in {0,0.2,0.4,0.6,0.8,1}{
                            \draw[very thick, colorrds!80!, dotted] (6,0) .. controls (6.4,1-\j)
                            and (6.9,0) .. (7,0);
                            \draw[very thick, colorrds!80!, dotted] (6,0) .. controls (6.4,-1+\j)
                            and (6.9,0) .. (7,0);
                        }
                \end{scope}
            }
    \end{tikzpicture}
\end{figure}
\end{document}