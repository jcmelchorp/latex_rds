\part El área de un rectángulo es 528 cm$^2$. Su altura es 1 cm más que el doble del ancho. Sea $z$ el ancho del rectángulo.

\begin{subparts}
    \subpart \textbf{¿Cuál de las siguientes ecuaciones cuadráticas satisface $z$?}

    \begin{oneparchoices}
        \choice $2z^2+z+528=0$
        \CorrectChoice $2z^2+z-528=0$
        \choice $2z^2-z-528=0$
        \choice $2z^2-z+528=0$
    \end{oneparchoices}

    \begin{solutionbox}{3cm}
        Sea $z$ el ancho del rectángulo, entonces su altura esta dada por $2z+1$, y su área es:
        \[
            \begin{array}{rl}
                z(2z+1)    & =528 \\
                2z^2+z     & =528 \\
                2z^2+z-528 & =0   \\
            \end{array}
        \]
    \end{solutionbox}

    \subpart \textbf{Determina el ancho del rectángulo $z$.}

    \begin{solutionbox}{13.8cm}
        Para encontrar $z$, se debe resolver la ecuación:
        \[ 2z^2+z-528  =0 \]
        De acuerdo con la forma estándar $az^2 + bz + c = 0$ de una ecuación cuadrática, sus coeficientes son:
        \[            a=2, \quad b=1 \quad y \quad c=-528\]
        Sustituyendo los coeficientes en la fórmula
        cuadrática:
        \[
            \begin{array}{rl}
                z              & = \dfrac{-b\pm\sqrt{b^2-4ac}}{2a}          \\[1.2em]
                z              & = \dfrac{-1\pm\sqrt{1^2-4(2)(-528)}}{2(2)} \\[1.2em]
                z              & = \dfrac{-1\pm\sqrt{4225}}{4}              \\[1.2em]
                z              & = \dfrac{-1\pm 65}{4}                      \\[1.2em]
                \therefore z_1 & = \dfrac{-1+65}{4}=\dfrac{64}{4}=16        \\[1.2em]
                z_2            & = \dfrac{-1-65}{4}=\dfrac{-66}{4}=-16.5
            \end{array}
        \]
        Ya que las distancias no pueden ser negativas, el ancho es:
        \[\therefore z=16\]
        %\end{multicols}
    \end{solutionbox}
\end{subparts}