\part[10] El producto de dos enteros pares positivos consecutivos es 80. Sea $n$ el menor entero.

\begin{subparts}
    \subpart \textbf{¿Cuál de las siguientes ecuaciones cuadráticas satisface $n$?}

    \begin{oneparchoices}
        \choice $n^2+2n+80=0$
        \choice $n^2-2n-80=0$
        \choice $n^2-2n+80=0$
        \CorrectChoice $n^2+2n-80=0$
    \end{oneparchoices}

    \vspace{0.5cm}
    \begin{solutionbox}{3.5cm}
        Sea $n$ el menor entero par positivo, y su consecutivo par $n+2$. Entonces, el producto es:
        \[
            \begin{array}{rl}
                n(n+2)     & =80 \\
                n^2+2n     & =80 \\
                n^2+2n -80 & =0
            \end{array}
        \]
    \end{solutionbox}

    \subpart \textbf{Encuentra el número $n$.}

    \begin{solutionbox}{15cm}
        %\begin{multicols}{3}
        Para encontrar $n$, se debe resolver la ecuación:
        \[
            n^2+2n -80 =0
        \]
        De acuerdo con la Forma estándar $at^2 + bt + c = 0$ de una ecuación cuadrática, sus coeficientes son:
        \[
            \begin{array}{rl}
                a & =1   \\
                b & =2   \\
                c & =-80
            \end{array}
        \]
        Sustituyendo los coeficientes en la fórmula cuadrática:
        \[
            \begin{array}{rl}
                n   & = \dfrac{-b\pm\sqrt{b^2-4ac}}{2a}            \\[1.5em]
                n   & = \dfrac{-2\pm\sqrt{(-2)^2-4(1)(-80)}}{2(1)} \\[1.5em]
                n   & = \dfrac{-2\pm\sqrt{324}}{2}                 \\[1.5em]
                n   & = \dfrac{-2\pm 18}{2}                        \\[1.5em]
                n_1 & = \dfrac{-2-18}{2}=\dfrac{-20}{2}=-10        \\[1.5em]
                n_2 & = \dfrac{-2+18}{2}=\dfrac{16}{2}=8
            \end{array}
        \]
        Ya que sólo hablamos de enteros positivos:
        \[\therefore n=8\]
        % \end{multicols}
    \end{solutionbox}
\end{subparts}