\part $y=(x+2)^2-64$


\begin{solutionbox}{5cm}
    \begin{multicols}{2}
        Por despeje:
        \[
            \begin{array}{rl}
                y=         & (x+2)^2-64                          \\
                           & \text{Ya que $y=0$, entonces }      \\
                0=         & (x+2)^2-64                          \\
                64=        & (x+2)^2                             \\
                \pm 8=     & x+2                                 \\
                -2 \pm 8=  & x                                   \\
                \therefore & x_1=-2+8 =6 \text{ y } x_2=-2-8=-10
            \end{array}
        \]
    \end{multicols}
\end{solutionbox}
