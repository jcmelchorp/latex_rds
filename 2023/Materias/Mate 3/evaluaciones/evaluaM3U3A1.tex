\documentclass[12pt,addpoints,answers]{evalua}
\grado{3$^\circ$ de Secundaria}
\cicloescolar{2022-2023}
\materia{Matemáticas 3}
\unidad{3}
\title{Examen con Adecuación Curricular de la Unidad}
\aprendizajes{
    \item Calcula valores faltantes en problemas de proporcionalidad directa, con constante natural, fracción o decimal (incluyendo tablas de variación).
    \item Resuelve problemas mediante la formulación y solución algebraica de ecuaciones lineales.
    \item Formula expresiones de primer grado para representar propiedades
    (perímetros y áreas)
    de figuras geométricas y verifica equivalencia de expresiones, tanto
    algebraica como geométricamente (análisis de las figuras).
    \item Calcula el volumen de prismas y cilindros rectos.
}
\author{Prof.: Julio César Melchor Pinto}
\graphicspath{{../../Mate 2/images/}{../images/}}
\begin{document}
\begin{multicols}{2}%
    \include*{../../Mate 2/blocks/block002}
    \include*{../../Mate 2/blocks/block035c}
    \include*{../../Mate 2/blocks/block035b}
\end{multicols}%
\begin{questions}
    \question[20] \include*{../../Mate 1/questions/question067}
    \newpage
    \question[10] \include*{../../Mate 1/questions/question050c}
    \question[10] \include*{../../Mate 1/questions/question050a}
    \question[10] \include*{../../Mate 2/questions/question103i}
    \question[10] \include*{../../Mate 2/questions/question104c}
    \newpage
    \question[10] \include*{../../Mate 2/questions/question102d}
\end{questions}
\end{document}