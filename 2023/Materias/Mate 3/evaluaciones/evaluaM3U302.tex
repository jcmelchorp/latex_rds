\documentclass[12pt,addpoints]{evalua}
\grado{3$^\circ$ de Secundaria}
\cicloescolar{2023-2024}
\materia{Matemáticas 3}
\unidad{3}
\title{Examen de la Unidad}
\aprendizajes{
      \item 
      }
\author{Prof.: Julio César Melchor Pinto}
\begin{document}
\begin{questions}
      \section*{\ifprintanswers   {Sucesiones cuadráticas y geométricas}\else{}\fi}
      \subsection*{\ifprintanswers{Sucesión cuadrática}\else{}\fi}
      \subsection*{\ifprintanswers{Completando la sucesión cuadrática}\else{}\fi}
      \subsection*{\ifprintanswers{Término general}\else{}\fi}
      \subsection*{\ifprintanswers{Sucesión geométrica}\else{}\fi}
      \subsection*{\ifprintanswers{Razón de una sucesión geométrica}\else{}\fi}
      \section*{\ifprintanswers{   Productos notables}\else{}\fi}
      \subsection*{\ifprintanswers{Binomios conjugados}\else{}\fi}
      \subsection*{\ifprintanswers{Binomios con término común}\else{}\fi}
      \subsection*{\ifprintanswers{Binomio al cuadrado}\else{}\fi}
      \subsection*{\ifprintanswers{Binomios de la forma (mx+a)(nx+b)}\else{}\fi}
      \subsection*{\ifprintanswers{Binomio al cubo}\else{}\fi}
      \section*{\ifprintanswers{   Ecuaciones cuadráticas}\else{}\fi}
      \subsection*{\ifprintanswers{Clasificación de ecuaciones cuadráticas}\else{}\fi}
      \subsection*{\ifprintanswers{Discriminante}\else{}\fi}
      \subsection*{\ifprintanswers{Ecuaciones cuadráticas incompletas}\else{}\fi}
      \subsection*{\ifprintanswers{Ecuaciones cuadráticas completas 1}\else{}\fi}
      \subsection*{\ifprintanswers{Ecuaciones cuadráticas completas 2}\else{}\fi}
      \section*{\ifprintanswers{   Teorema de Pitágoras}\else{}\fi}
      \subsection*{\ifprintanswers{Identificación de lados}\else{}\fi}
      \subsection*{\ifprintanswers{Hallando la hipotenusa}\else{}\fi}
      \subsection*{\ifprintanswers{Hallando el cateto}\else{}\fi}
      \subsection*{\ifprintanswers{Áreas y perímetros}\else{}\fi}
      \subsection*{\ifprintanswers{Resolución de problemas}\else{}\fi}
      \section*{\ifprintanswers{   Trigonometría}\else{}\fi}
      \subsection*{\ifprintanswers{Identificando lados}\else{}\fi}
      \subsection*{\ifprintanswers{Identificando funciones}\else{}\fi}
      \subsection*{\ifprintanswers{Encontrando lados}\else{}\fi}
      \subsection*{\ifprintanswers{Encontrando ángulos}\else{}\fi}
      \subsection*{\ifprintanswers{Resolución de problemas}\else{}\fi}
      \question[10]
\end{questions}
\end{document}