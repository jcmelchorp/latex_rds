
% \begin{minipage}[t][][t]{0.6\textwidth}
\label{ques:longitud_emb}
La hermana de Ernesto está embarazada y ha ido a consulta con su médico obstetra.
Éste le ha dicho que su embarazo va bien y que lleva 23 semanas de gestación.
Ella se lo contó a Ernesto y además le preguntó cuánto medirá (altura) su bebé.
Como él estudia medicina sabe que la longitud (altura) de un feto de más
de 12 semanas de gestación se puede aproximar mediante la fórmula
\begin{equation}\label{eq:longitud_emb}
    L = 1.53t - 6.7
\end{equation}
en donde $L$ es la longitud del feto en cm y $t$ la edad en semanas.

\begin{parts}
    \part El tiempo de un embarazo humano dura entre 38 y 40 semanas.

    \begin{solutionbox}{1.2cm}

    \end{solutionbox}

    \begin{subparts}
        \subpart ¿Qué longitud se obtendría si se aplica la fórmula para una $t$ de 1 o 2 semanas?

        \begin{solutionbox}{1.2cm}

        \end{solutionbox}

        \subpart ¿Tiene sentido esa respuesta? ¿Por qué?

        \begin{solutionbox}{1.2cm}

        \end{solutionbox}

    \end{subparts}

    \part ¿Para qué tiempo $t$ en semanas la longitud es cero?

    \begin{solutionbox}{1.2cm}

    \end{solutionbox}

    \begin{subparts}
        \subpart ¿Tiene sentido hablar de 0 cm de crecimiento en un tiempo $t$?

        \begin{solutionbox}{1.2cm}

        \end{solutionbox}

    \end{subparts}

    \part ¿Qué valores puede tomar la variable t? ¿Por qué?

    \begin{solutionbox}{1.2cm}

    \end{solutionbox}

    \part ¿Qué valores puede tener la variable L? Expliquen.

    \begin{solutionbox}{1.2cm}

    \end{solutionbox}

    \part ¿Cuál es la longitud del bebé de la hermana de Ernesto?\\
    \emph{Describe detalladamente las operaciones para obtener el valor de L.}

    \begin{solutionbox}{1.2cm}

    \end{solutionbox}

\end{parts}