La hermana de Ernesto está embarazada y ha ido a consulta con su médico obstetra.
Éste le ha dicho que su embarazo va bien y que lleva 23 semanas de gestación.
Ella se lo contó a Ernesto y además le preguntó cuánto medirá (altura) su bebé.
Como él estudia medicina sabe que la longitud (altura) de un feto de más
de 12 semanas de gestación se puede aproximar mediante la fórmula
\label{ques:longitud_emb}
\begin{equation}\label{eq:longitud_emb}
    L = 1.53t - 6.7
\end{equation}
en donde $L$ es la longitud del feto en cm y $t$ la edad en semanas.

\begin{multicols}{2}
    \begin{parts}
        \part El tiempo de un embarazo humano dura entre 38 y 40 semanas.

        \begin{subparts}
            \subpart ¿Qué longitud se obtendría si se aplica la fórmula para una $t$ de 1 o 2 semanas?

            \begin{solutionbox}{3.5em}
                $L = -5.17$ es un número negativo
            \end{solutionbox}

            \subpart ¿Tiene sentido esa respuesta? ¿Por qué?

            \begin{solutionbox}{7em}
                No, no tiene sentido en este contexto porque
                no hay altura negativa; por ello se aclara que es arriba de las 12 semanas de
                gestación cuando se puede utilizar la fórmula.
            \end{solutionbox}

        \end{subparts}

        \part ¿Qué valores puede tener la variable $L$? Expliquen.

        \begin{solutionbox}{6em}
            Desde 11.6 cm hasta 54.5 cm; son las evaluaciones de la fórmula en 12 y 40
            semanas, respectivamente.
        \end{solutionbox}
        
        \part ¿Para qué tiempo $t$ en semanas la longitud es cero?

        \begin{solutionbox}{14em}
            \[L=1.53t-6.7\]
            Si $L=0$, entonces:
            \begin{align*}
                0=                 & 1.53t-6.7 \\
                6.7=               & 1.53t     \\
                \dfrac{6.7}{1.53}= & t         \\
                t \approx          & 4.38      \\
            \end{align*}

        \end{solutionbox}

        \part ¿Qué valores puede tomar la variable $t$? ¿Por qué?

        \begin{solutionbox}{5em}
            Técnicamente, para $t > 5.17$, pero por el contexto se especifica que debe ser
            $t > 12$.
        \end{solutionbox}

        \begin{subparts}
            \subpart ¿Tiene sentido hablar de 0 cm de crecimiento en un tiempo $t$?

            \begin{solutionbox}{6em}
                No, no tiene sentido porque ya pasó tiempo de gestación y el tamaño del
                feto no puede ser de 0 cm a las cinco semanas.
            \end{solutionbox}

        \end{subparts}

        \part ¿Cuál es la longitud del bebé de la hermana de Ernesto?
        \emph{Describe detalladamente las operaciones para obtener el valor de L.}

        \begin{solutionbox}{11em}
            \[L=3t-6.7\]
            Si $t=23$, entonces:
            \begin{align*}
                L=  & 1.53(23)-6.7 \\
                L=  & 35.19-6.7    \\
                L = & 28.59 \text{ cm}   
            \end{align*}

        \end{solutionbox}
    \end{parts}
\end{multicols}