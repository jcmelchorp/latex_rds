Elige todas las respuestas adecuadas:
\begin{parts}
    \part ¿Cuáles longitudes de lados forman un triángulo rectángulo?
    \begin{checkboxes}
        \CorrectChoice 4.5, 6, 7.5
        \choice 5, $\sqrt{8}$, 3
        \CorrectChoice 12, 9, 15
        \CorrectChoice $\sqrt{2}$, $\sqrt{2}$, 2
    \end{checkboxes}
\end{parts}
\begin{solutionbox}{6cm}
    Para verificar si las opciones contienen o no las longitudes que corresponden a un triángulo rectángulo, es necesario sustituir estos valores en el teorema de Pitágoras, considerando la hipotenusa como el lado más largo en un triángulo rectángulo. Si se cumple la igualdad, entonces se trata de un triángulo rectángulo.

    \begin{multicols}{4}
        \begin{align*}
            c^2=   & a^2+b^2   \\
            7.5^2= & 4.5^2+6^2 \\
            56.25= & 20.25+36  \\
            56.25= & 56.25
        \end{align*}

        \begin{align*}
            c^2=  & a^2+b^2            \\
            5^2 = & (\sqrt{8})^2 + 3^2 \\
            25=   & 8+9                \\
            25=   & 17
        \end{align*}

        \begin{align*}
            c^2=  & a^2+b^2  \\
            15^2= & 9^2+12^2 \\
            225=  & 81+144   \\
            225=  & 225
        \end{align*}

        \begin{align*}
            c^2= & a^2+b^2                    \\
            2^2= & (\sqrt{2})^2+ (\sqrt{2})^2 \\
            4=   & 2+2                        \\
            4=   & 4
        \end{align*}
    \end{multicols}
\end{solutionbox}