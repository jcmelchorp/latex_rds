\question[5] Considera los dos triángulos que se muestran abajo en la Figura \ref{fig:20230323153510} (los triángulos no están dibujados a escala).

\begin{figure}[H]
    \includegraphics[width=0.5\linewidth]{../images/20230323153510}
    \caption{}
    \label{fig:20230323153510}
\end{figure}

\textbf{¿Los dos triángulos son congruentes?}
\emph{Escoge 1 respuesta:}

\begin{choices}
    \choice Sí.
    \CorrectChoice No.
    \choice No hay suficiente información para decidir.
\end{choices}

\begin{solutionbox}{5cm}
    Dos triángulos son congruentes si tienen la misma forma y tamaño. En otras palabras, dos triángulos son congruentes si todos los lados y ángulos correspondientes son congruentes.

    Puesto que nos dan cuatro ángulos distintos, no hay manera de que los tres ángulos del primer triángulo sean congruentes a los ángulos del segundo triángulo.
    De hecho, como los ángulos de un triángulo suman 180$^\circ$, podemos calcular estos ángulos para verificarlo. Los ángulos del primer triángulo serían 45$^\circ$, 108$^\circ$ y 27$^\circ$, y los ángulos del segundo triángulo serían 73$^\circ$, 67$^\circ$ y 40$^\circ$.
    Los ángulos correspondientes no son congruentes. Por lo tanto los triángulos no pueden ser congruentes.

    No, los triángulos no son congruentes.
\end{solutionbox}