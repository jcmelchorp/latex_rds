¿Cuál es el \'area del paralelogramo de la figura \ref{fig:area_isoseles_03}?

\begin{figure}[H]
    \begin{center}
        \includegraphics[width=0.3\linewidth]{../images/area_isoseles_03.png}
    \end{center}
    \caption{}
    \label{fig:area_isoseles_03}
\end{figure}
\begin{solutionbox}{14cm}

    \begin{minipage}{0.6\textwidth}
        Para determinar el área del triángulo debemos saber la base y la altura. Llamemos $x$ a la longitud (ver Figura \ref{fig:area_isoseles_03a}).
        Cuando tenemos un triángulo rectángulo, podemos usar el teorema de Pitágoras para obtener la longitud del cateto.
        La ecuación para el teorema de Pitágoras es:
        \[c^2=a^2+b^2\]
        En este caso, $a=4$, $b=x$ y $c=5$, Entonces,
        \begin{align*}
            4^2+x^2 & =5^2   \\
            16+x^2  & =25    \\
            x^2     & =25-16 \\
            x^2     & =9     \\
            x       & =3
        \end{align*}
        La altura del triángulo es 2. El área del triángulo es:

        \begin{align*}
            A & =\frac{1}{2}bx             \\
            A & =\frac{1}{2}\cdot 8\cdot 3 \\
            A & =12 \text{ u}^2
        \end{align*}
    \end{minipage}\hfill
    \begin{minipage}{0.35\textwidth}
        \begin{figure}[H]
            \centering
            \includegraphics[width=0.5\linewidth]{../images/area_isoseles_03a.png}
            \caption{}
            \label{fig:area_isoseles_03a}
        \end{figure}
    \end{minipage}

\end{solutionbox}

