Considera los dos triángulos que se muestran abajo en la Figura \ref{fig:20230323154930} (los triángulos no están dibujados a escala).

\begin{minipage}{0.6\textwidth}
    \textbf{¿Los dos triángulos son congruentes?}\\
    \emph{Escoge 1 respuesta:}\\

    \begin{choices}
        \CorrectChoice Sí.
        \choice No.
        \choice No hay suficiente información para decidir.
    \end{choices}

\end{minipage}%
\begin{minipage}{0.35\textwidth}
    \begin{figure}[H]
        \centering
        \includegraphics[width=\linewidth]{../images/20230323154930}
        \caption{}
        \label{fig:20230323154930}
    \end{figure}
\end{minipage}

\begin{solutionbox}{4.5cm}\footnotesize
    \begin{minipage}{0.65\textwidth}
        Dos triángulos son congruentes si tienen la misma forma y tamaño. En otras palabras, dos triángulos son congruentes si todos los lados y ángulos correspondientes son congruentes.
        Sin embargo, no necesitamos mostrar la congruencia de todos los lados y ángulos correspondientes para demostrar que dos triángulos son congruentes. Los criterios de congruencia (LLL, LAL, ALA) y el teorema AAL son atajos útiles para determinar congruencia de triángulos.
        En cada triángulo, observa que los dos lados dados son congruentes y que el ángulo entre ellos también es congruente.
        Por lo tanto, los dos triángulos son congruentes por LAL.

        \textbf{Sí, los triángulos son congruentes.}
    \end{minipage}\hfill
    \begin{minipage}{0.3\textwidth}
        \begin{figure}[H]
            \centering
            \includegraphics[width=0.9\linewidth]{../images/20230323155107}
            \caption{}
            \label{fig:20230323155107}
        \end{figure}
    \end{minipage}

\end{solutionbox}