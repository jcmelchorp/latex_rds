Convierte las siguientes expresiones gramaticales en expresiones algebraicas.

\begin{parts}
    \part Escriban una expresión algebraica que describa cada oración.

    \begin{subparts}
        \subpart La diferencia entre el doble de un número y su sexta parte.

        \begin{solutionbox}{1.6cm}
            \[2x-\dfrac{x}{6}\]
        \end{solutionbox}

        \subpart El cociente de un número más uno entre ese número menos dos.

        \begin{solutionbox}{1.6cm}
            \[\dfrac{x+1}{x-2}\]
        \end{solutionbox}

        \subpart La diferencia entre el doble de un número y su sexta parte es 1.

        \begin{solutionbox}{1.6cm}
            \[2x-\dfrac{x}{6}=1\]
        \end{solutionbox}

        \subpart El cociente de un número más 1 entre ese número menos 2 es igual a 3.

        \begin{solutionbox}{1.6cm}
            \[\dfrac{x+1}{x-2}=3\]
        \end{solutionbox}
    \end{subparts}

    \part ¿Qué característica tienen en común las expresiones que escribieron?

    \begin{solutionbox}{2.5cm}
        Algunas de las expresiones son de una sola variable o literal representando,
        en cada caso, el total de elementos de un grupo o un número desconocido;
        y otras, representan una igualdad con la que es posible obtener los valores de la variable o literal.
    \end{solutionbox}
\end{parts}