\documentclass{article}
\usepackage{jsonparse} % Load the json package

% \begin{filecontents}[noheader]{recipe.json}
% {
%     "recipe": {
%         "title":"First recipe",
%         "source":"My first cookbook",
%         "carbs":"1 oz",
%         "fat":"1 oz",
%         "protein":"1 oz",
%         "cal":"100 kcal",
%         "ingredients": [
%             {"item":"Eggs"},
%             {"item":"Oil"},
%             {"item":"Nuts"}
%         ],
%         "cooking": [
%             {"step":"Mix eggs and oil"},
%             {"step":"Add nuts"}
%         ]
%     }
% }
% \end{filecontents}

\title{High School Student CV Profiles}
\author{Your Name}



% Load the JSON file

%\JSONParseFromFile{\recipedata}{recipe.json}
\JSONParseFromFile{\studentsdata}{obj.json}




\begin{document}
\maketitle
\JSONParseArrayMapInline{\studentsdata}{students}{%
\noindent
   Nombre: \JSONParseValue{\studentsdata}{students[#1].name} \\
   Grado: \JSONParseValue{\studentsdata}{students[#1].grade} \\
   Edad: \JSONParseValue{\studentsdata}{students[#1].personal_info.age} \\
   Género: \JSONParseValue{\studentsdata}{students[#1].personal_info.gender} \\
   Dirección: \JSONParseValue{\studentsdata}{students[#1].personal_info.address} \\
   Contacto: \JSONParseValue{\studentsdata}{students[#1].personal_info.contact}   \\  

    } 

\JSONParseArrayUse{\studentsdata}{students[#1].courses}{
    \JSONParseValue{\studentsdata}{students[#1].courses[#2].course_name}
    \JSONParseValue{\studentsdata}{students[#1].courses.teacher}
    \JSONParseValue{\studentsdata}{students[#1].courses.grade}
}
    

\end{document}